\begin{center}
     \Large{\textbf{Chhatrapati Shivaji Maharaj Jeevan Rahasya \\ Narhar Kurundkar }}  % Change Title
\end{center}

{\em Well-known scholar Prof. Kurundkar, instead of mulling over widely popular-dramatic events in Shivaji’s life, has tried to fathom the very inspiration behind his achievements and his deification in the society at large.}

%\section{Salient Points}
\bigskip

\begin{itemize}[noitemsep,nolistsep]

\item There have been many rulers who were good guardians,  ensured welfare of the progeny but there is only one who has been revered as God,  not just by the contemporaries but also by the later generations, and that’s – Shivaji!!

\item The dramatic-eventful period in Shivaji’s life-span is very short (‘Killing of Afzalkhan’ to ‘Escape from Agra’, about 7 years). The period before (about 29 years) and the one after (about 14 years) were almost event-less. It is these periods that depict real achievements of Shivaji and justify his God-like status in the society. Nations are built not by dramatic events but by small, persistent progressive deeds.

\item Shivaji’s battles should not be seen as Hindu vs. Muslim religious conflicts. Shivaji had quite a few Muslim soldiers and commanders. He was on friendly terms with Kutubashah. Instead, enemies he had to fight most of the times were not just Hindus but were from his own community-his relatives. 

\item Shivaji had always thought of his empire as that of the people and by wish of God. That’s why even people thought of it as being their own and fought for it even if, at times, nobody commanded them.

\item 1642-47: Got hold of 12 territories in Maval and established proper administration under guidance of Dadoji Konddev. Landlords (Suzerains) at that time were harassing people, looting them. Shivaji slayed this menace, thereby winning a place in people’s heart.

\item 1645-49: Mutiny
	\begin{itemize}[noitemsep,nolistsep]
	\item Conquered forts like Torna-Rohida and put administration in place. Penalised offenders. Every such punishments enforced faith people had in him and got him more supporters.
	\item Through small wins he gathered confidence. In a seemingly odd move he even relinquished Sinhagad and made friends with Vijapur to avoid a big battle. He had to lose a few times, retreat, but people did not lose faith in him. Anyone hardly revolted against him. People trusted him fully.
	\end{itemize}

\item 1649-56: Quieter times. Experimented with new paradigms of administration. Strengthened military force and confidence of people.

\item 1656-59: Challenge to Adilshahi
	\begin{itemize}[noitemsep,nolistsep]
	\item Captured  Javali at a lightning speed.  Getting Sinhagad as well was like an open challenge to Adilshahi. Expanded his territory up to the sea-shores. He had even planned to surrender few of his regions in case he was forced into a pact. In case of certain defeat, it has been Shivaji’s strategy to retreat, thereby protecting his strength-army. He could always get them back later anyway!
	\item Killing of Afzalkhan: Although there have been debates, the question of ‘who betrayed, attacked or deceived first?’ is irrelevant. Afzalkhan had to be killed anyway. His killing was just a part of a bigger plan, the plan to annihilate whole army of 12000 or so.
	\item After Afzalkhan’s killing, Shivaji looted quite a few towns and with that built a formidable army. Doubled the cavalry. Before Adilshahi could recover from the setback, Shivaji captured Wai, Panhalgad and neighbouring regions. Hindustan now took notice of this new star on the horizon.
	\end{itemize}

\item 1659-70: Bigger challenges
	\begin{itemize}[noitemsep,nolistsep]
	\item Adilshah sent Siddi Johar to take the revenge. He besieged and stranded Shivaji at Panhalgad. This time the enemy chose the time and place of the battle, and it was hard to even slip out.
	\item Inspiration for martyrdom:
		\begin{itemize}[noitemsep,nolistsep]
		\item In the escape-plan of Panhalgad, Shiva- the barber embraced death, knowingly.  Bajiprabhu with his small force of 200-300 fought valiantly…. till death.
		\item Inspiration for martyrdom comes from strong ideals. And if the leader of such a cause is equally devoted then followers are ready to lay down their lives for him. This is peculiarity of the human race. At that time that leader represents society’s own aspirations. He just articulates what the society actually wants.
		\end{itemize}
	\item In 1665, within just 3 months Mirza Raje Jaisingh overpowered Shivaji and got him to surrender. Shivaji then came back with unwavering mental determination-strength. His motto was ‘Accepting defeat against a bigger army can be justified, only to buy time to gather strength and then attack with more power’.
	\end{itemize}

\item 1670-:  Fighting back
	\begin{itemize}[noitemsep,nolistsep]
	\item Looted Mugal regions. Got back the forts he had surrendered in the pact. Had to bear loss of Tanaji-Nilopant-Annaji. Defeated Moguls in Konkan. Looted Surat.
	\item Coronation. With this he sent message to Hindustan that he is there to support the cause of freedom.
	\item Aurangzeb arrived in Deccan. Even after demise of Shijavi, people continued to fight for the same cause.
	\end{itemize}

\item Navy:
	\begin{itemize}[noitemsep,nolistsep]
	\item Mughals ignored threats at the sea. But Shivaji did not and built his own Navy.
	\item Shivaji’s Navy was not at all comparable with that of Europeans. Shivaji could never fully defeat the foreigners but was at least successful in putting a firm check on them.
	\item Later in the times of Peshvas Navy was destroyed completely. Europeans could establish their single-handed control over the sea-shores, which helped them do the same , controlling India.
	\end{itemize}

\item New Administrative Practices: One of the prominent reasons for acceptance of Shivaji, at that time was due to the administrative practices he introduced.
	\begin{itemize}[noitemsep,nolistsep]
	\item That time, Land-Authority was passed on to generations. Landlords would resort to any means to protect the control. Rivalries were rampant.
	\item Landlords would cheat the king they represented. They would not give a second thought of joining the enemy to protect their control. In spite of the tax being just 1/6 th of the profits, they would harass and collect more from people, even in droughts. People were left in a hapless situation as there was nobody to complain to and ask for support. Their nuisance did not end there. They would abduct girls, dishonour them. Military would also add to the misery by devastating the region they would pass through.
	\item Key to Shivaji’s success can be attributed to getting rid of the menace of Landlords. He enforced new system. Re-calibrated the land. Estimated the yield. Put 2/5 as tax. The new tax was more than what was before but people happily paid it as it was certain that even after paying the tax, they would be left with enough for their sustenance.  Full protection and flawless execution. In case of droughts, he would not only forego the taxes, but help people with seeds, food, farm-equipment, etc. In setting up this system he was ably helped by his trusted confidant – Annaji Datto (Kulkarni). 
	\item Shivaji snatched away the armies these Landlords had. Destroyed their fortresses. Removed their authority to collect taxes and decision making. Slayed the offenders. Gave security to women. Gave food to people. Penalized military that tried to loot the land (once he had cut off hands-feet of 300 soldiers as part of such a punishment). Shivaji brought ‘Civil Authority’. That’s why Ramdas called him ‘Shivkalyan Raja (The Benevolent King)’.
	\end{itemize}

\item Ideal:
	\begin{itemize}[noitemsep,nolistsep]
	\item Shivaji ensured well-being and freedom of the progeny. He had firm moral standing. He never dishonoured women, places of worship, holy texts of other religions (even that of enemy’s).
	\item Shivaji lived his life detached from materialism. Like an ascetic saint. That’s why Ramdas called him ‘Shirmant Yogi (The Monk amidst riches)’.
	\end{itemize}

\end{itemize}

{\em [Disclaimer: English translation may or may not articulate the exact viewpoint Prof Kurundkar had put forth. I would urge people to read the book in original.]}

