\documentstyle[11pt]{book}
%\setlength{\evensidemargin}{0.25in}
%\setlength{\textwidth}{7.0in}
%\setlength{\textheight}{8.9in}
%\topmargin -0.5in
\begin{document}
\begin{titlepage}

\title{The Graduate Student's Guide to Indian Recipes}
\author{edited by\\Somesh Rao\\}


\maketitle
\thispagestyle{empty}

\end{titlepage}

\tableofcontents
\newpage
\chapter{Preface}
\section{Somesh Rao}
I have collected these recipes over the network. I am just responsible for 
typesetting it and running it through a spell checker. In a later edition I 
hope to add my own recipes. 
At this point I would like to apologize to those whose recipes I have used but 
whose names I have not mentioned. During the editing process I lost most of 
the addresses.
Finally, I am not responsible for any of the recipes as I have been so busy 
trying to put this book together that I have had no time for experimentation.
Please send comments, suggestions and other recipes to
\begin{verbatim}
              somesh@maxine.wpi.edu
              somesh@wpi.wpi.edu

or
               Somesh Rao,
               Computer Science Dept.,
               Worcester Polytechnic Institute,
               Worcester, MA 01609.
               U.S.A

This book is available through anonymous FTP from

                    wpi.wpi.edu

in recipes/indian.recipes.*
\end{verbatim}
I would like to thank Greg Leichtman and Jamille Hetke at The
University of Michigan who took the trouble to correct my \LaTeX
errors. 


By the way, I would prefer recipe contributions in electronic form.\\
Bon Appetit. \\
\section{Sanjiv Singh}
  The  problem  with  the cooks I grew up with, like my mother for instance, is
that they never wrote any of their craft down.  So here I am looking  for  ways
to  cook  the wonderful things I have been downing in large quantities and all
these people can tell me is ``Well, you add a `little'  of  this  and  wait  'til
`that'  happens  and  then  serve.''  Having not had most of the experience
firsthand, I have had to blunder through many ``adventures'' with  the  meals  I  have
cooked for the first couple of times.

  Through  my  efforts  to  write  down  some means of replicating what I would
undoubtedly  forget,  this  collection  of  recipes  has  come  together  as  a
collection  of  concoctions,  hopefully  exotic  enough  that  cannot  be found
together elsewhere.  Credit goes to Sriram for getting me started on this; indeed
most of the recipes listed here came via him.

  At  this  writing,  the  recipes are disproportionally Indian.  The reason is
simply that I can find other recipes in abundance without having to  transcribe
them.    The reality is that most of the Indian recipes assume some familiarity
with Indian spices and  more  so  with  cooking  in  general.    They  are  not
recommended  for  complete  novices.    The exceptions to this are the first two
recipes for chick peas and for cauliflower.

  In the latest update, a couple of recipes have been added and  the  book  has
been separated into chapters.

  Make sure you brown them onions right.

                                                                   Sanjiv Singh

                                                       Pittsburgh, January 1989

\section{Acknowledgement}
Also this book contains the recipes taught by Mrs. Pawan Datta in her 
Indian Cooking class of Fall 1981.  This compilation was done by Marc Meyer
(ingvax:marc).

Minor editing was also done by Gregg Leichtman and Jamille Hetke at The
University of Michigan.

\chapter{Ingredients}
\section{Guide to Ingredients}

Most of the ingredients are available at the grocery stores or
supermarkets. Some, though, are special and have to be obtained from
the Indian stores.
Substitutes may change the character of the dish. It is better to
omit an ingredient if not available than to substitute for it.
If whole spice is not available, you may use the ground form, but the
ground form is less pungent.

\begin{tabbing}
\={\bf Name}   \hspace{2.0cm} \={\bf Indian Name} \hspace{2.0cm} \={\bf
Description}\\

\> Asafoetida   \> Hing \> Dried gum resin from the root of various\\
\>              \>      \> Iranian and East Indian plants.  Has a strong \\
\>              \>      \> fetid odor---definitely an acquired taste.\\
\>              \>      \> May be obtained.\\

\> Besan        \> Besan        \> Flour of dried chick peas.\\

\> Cardamom     \> Elaichi\> Dried fruit of a plant. \\
\>              \>      \> Mostly the seeds are used.\\
\>              \>      \> Seeds of 4 pods measure approximately $\frac{1}{4}$ t.\\
                
\> Coriander    \> dhania\> Aromatic herb of the parsley family. Sold as\\
\>              \>      \> cilantro or chinese parsley.  Also sold as\\
\>              \>      \> seed or dry powder.\\

\> Cumin        \> Jeera\> Very aromatic and reminiscent. Sold whole or\\
\>              \>      \> ground.\\

\> Dals          \> Dal  \> Hindi name for all members of the legume\\
\>               \>      \> or pulse family.  Commonly used are: Arhar, \\
\>               \>      \> Channa, Masur, Mung, Labia (black-eyed peas),\\
\>               \>      \> Rajma (red kidney beans).\\

\> Fennel Seed   \> Sauf\> Has an agreeable odor and licorice flavor.\\
\>               \>      \> Available whole or ground.\\

\> Fenugreek     \> Methi\> Has a pleasant bitter flavor and sweetish odor.\\

\> Garam Masala  \>Garam Masala\>A mixture of spices; details come later.\\
\> Chat Masala  \>Chat Masala\>A variation of Garam masala\\
\>  		\>	     \>Available in Indian stores.\\

\> Ghee           \> Fat for frying.\>Pure ghee is clarified butter.\\

\> Mustard oil    \> Larson \> Pungent oil made from black mustard seeds.\\

\> Mint            \> Pudina\> Aromatic herb.  Fresh and dried leaves are\\
\>                 \>      \> used in the preparation of chutneys.  Dried\\
\>                 \>      \> leaves are much less fragrant than the fresh\\
\>                 \>      \> ones.\\

\> Pomegranate      \> Anar dana\> A flavoring agent.  Has some scent.\\

\> Saffron          \> Kesar\> Made of stigmas of a flower grown in\\
\>                  \>      \> Kashmir and Spain.  It is aromatic and yields\\
\>                  \>      \> a yellow color.\\

\> Turmeric          \> Haldi\> An aromatic powdered root.  Used as a flavoring\\
\>                   \>      \>and for flavoring curries.\\
\end{tabbing}

To make Garam Masala: (approximately 1 $\frac{1}{2}$ cups)
\begin{tabbing}
\hspace{1.0cm}  \={\bf Quantity}   \hspace{3.0cm} \={\bf Ingredient}\\
\>                      3 - 5'' pieces        \> Cinnamon stick\\
\>                      1 c (???)        \> Green cardamom pods \\
\>                      $\frac{1}{2}$ c  \> Cumin seed \\
\>                      $\frac{1}{2}$ c  \> Black pepper corns \\
\>                      $\frac{1}{2}$ c  \> Cloves \\
\>                      $\frac{1}{2}$ c  \> Coriander seeds \\
\end{tabbing}

Procedure:\newline
Dry the ingredients in an oven.  Do not let them turn brown.  Remove
the seeds from the cardamom pods. Pound cinnamon sticks into smaller
pieces.  Combine ingredients until they are well mixed and blend at
high speed for 2-3 minutes until completely pulverized
(LINE MISSING, The recipe seems to be complete, but as the original
had this I am letting it remain.)

\chapter{This and That}

\section{Onion and Tomato Raita}
{\bf 4-6 Servings}
\begin{tabbing}
\hspace{1.0cm}  \={\bf Quantity}   \hspace{3.0cm} \={\bf Ingredient}\\
\>  8 oz. \> Yogurt (plain)\\
\> 1 small    \> Onion\\
\> $\frac{1}{2}$ t \> Salt\\
\> 1 small    \> Tomato\\
\> $\frac{1}{2}$ t \> Chat Masala (optional)\\
\> $\frac{1}{2}$ t \> Black pepper (ground)\\
\> $\frac{1}{4}$ c \> Milk\\
\end{tabbing}

\begin{enumerate}
\item  Beat yogurt and milk until smooth.
\item  Chop onion and tomatoes and add to yogurt.
\item  Add salt and pepper and sprinkle the Chat Masala over, and serve.
\end{enumerate}

\section{Boondhi Raita}

\begin{tabbing}

\hspace{1.0cm}  \={\bf Quantity}   \hspace{3.0cm} \={\bf Ingredient}\\
\> $\frac{1}{4}$ c \> Besan\\
\> $\frac{1}{2}$ c \> Water\\
\> \> Ghee for frying\\
\> to taste \> Salt\\
\> to taste\> Pepper\\
\> to taste\> Chat Masala \\
\> 1 $\frac{1}{2}$ c \>  Yogurt\\
\> $\frac{1}{4}$ c \> Milk\\
\end{tabbing}

\begin{enumerate}
\item  Make a pouring paste of the besan and water.
\item Heat ghee and drop paste into it through a
slotted spoon to get little drops that fall one at a time (these are boondhi).
\item Remove the drops when golden brown and dry on a paper towel to
remove extra oil.
\item Soak the drops in warm water.
\item Add milk, salt, pepper, and add Chat Masala to yogurt.
\item Squeeze water out of boondhi and add to yogurt.
\end{enumerate}

\section{Mint and Coriander Chutney}

\begin{tabbing}
\hspace{1.0cm}  \={\bf Quantity}   \hspace{3.0cm} \={\bf Ingredient}\\
\> 1 bunch \> Coriander leaves\\
\> 1 bunch \> Mint leaves\\
\>   1 \> Green chili\\
\> 1 oz. \> Seedless tamarind\\
\> 1 tsp \> Salt\\
\> 4 T \>  Water\\
\>   1 medium \> Onion\\
\end{tabbing}

\begin{enumerate}
\item  Wash and soak tamarind in water for $\frac{1}{2}$ hour.
\item Clean, pick and wash the coriander and mint.
\item Separate pulp from the tamarind and squeeze out the pulp.
\item Grind coriander, mint, green chili and onion into a fine paste.
\item Add the tamarind pulp and salt.
\item Blend well.
In an airtight jar this can be refrigerated for up to one week.
\end{enumerate}

\chapter{Breads}

\section{Chapati (Phulka)}
{\bf for 4, serving 1 or 2}

\begin{tabbing}
\hspace{1.0cm}  \={\bf Quantity}   \hspace{3.0cm} \={\bf Ingredient}\\
\> 1 c \> Whole wheat flour (or $\frac{1}{3}$ white + $\frac{2}{3}$ whole wheat)\\
\> $\frac{1}{2}$ c \> Water\\
\end{tabbing}

\begin{enumerate}
\item  Put flour in a large bowl with half the water.
\item Blend the two together until it holds.
\item Beat and knead well until it forms a compact ball.
\item Knead dough until it is smooth and elastic.
\item Set aside for 30 minutes.
\item  Knead and divide dough into 4 to 6 parts.
\item Roll each ball into a tortilla like flat, about $\frac{1}{8}$'' thick.
\item Heat an ungreased skillet.
\item Put phulka on it, and let it cook for about 1 minute (The top should just start to 
look dry and small bubbles should just start to form).
\item Turn and cook the second side for $\frac{2}{3}$ minutes until small 
    bubbles form.
\item  Turn again and cook the first side pressed lightly with a towel.
It should puff. Serve warm (maybe slightly buttered).
\end{enumerate}

Note: Since the rolled out chappati's will dry out if they are left to stand while
cooking the others, it is advantageous to roll them out individually before 
cooking them.

\section{Paratha}
\begin{tabbing}
\hspace{1.0cm}  \={\bf Quantity}   \hspace{3.0cm} \={\bf Ingredient}\\

\> 1 c \> Whole wheat flour\\
\> \> Ghee\\
\> \>  Water \\
\end{tabbing}

\begin{enumerate}
\item  Make chappati dough.
\item Divide into 6 parts and make balls.
\item Flatten and roll each. 
\item Spread ghee over them and fold.
\item Roll again.
\item Heat the paratha on a griddle like you would a chappati, but spread
some ghee over the top side.  Turn and spread ghee on the other side.  Fry until
the bottom is crisp and golden, then turn and fry the remaining side.
\item Repeat with all six.
\item  Serve at once, since they lose crispness if stored.
\end{enumerate}

\section{Stuffed Parathas}

Make dough for regular chappati's.

Fillings
\begin{enumerate}
\item {\bf Potato - }\\ Boil potatoes, mash, add salt and chili to taste. Add Garam Masala and
mango powder.

\item {\bf Radish - }\\ Grate one large Diakon Radish, add salt and leave for $\frac{1}{2}$ hour.  Squeeze
out  all the water, add grated ginger, chili, and pomegranate seeds.

\item {\bf Cauliflower - }\\Grate cauliflower, add salt, pepper, garlic, and Garam Masala.
\end{enumerate}

{\bf Method}
\begin{enumerate}
\item  Roll out 2 small chappati's. Place filling on one, cover with the second,
seal edges and cook as for parathas.
\end{enumerate}

\chapter{Snacks}

\section{Bhel}

  This  is  a  concoction that I often bought from street vendors in India.  My
mouth still waters whenever I think of Bhel.  The  recipe  presented  here  was
taken off the net, and I haven't had a chance to try it yet.  I
include the note from the contributors:

\begin{quote}
    {\em Warning: This recipe is directed at those  who  know  what  Bhelpuri
    tastes  like, quantities mentioned are approximate, proportions are left to
    the reader's taste. Purists will  have  to  go  to  an  Indian
    grocery  shop.  Deviationists  may  use substitutes. The most important
    thing is to keep the puffed rice-sev mixture crisp by  not  adding  the
    other ingredients to it until it is served.  This should be done on the plate.}
\end{quote}


\begin{tabbing}
\hspace{1.0cm}  \={\bf Quantity}   \hspace{3.0cm} \={\bf Ingredient}\\
\>   \>Puffed Rice (1 carton of Rice Krispies may be used)\\
\>1 packet \>Bhel  mix or Sev \\
\>2 cups \>Mashed boiled potatoes (mashed coarsely and then salted)\\
\>$\frac{1}{2}$ cup \>Chopped fresh coriander leaves (a.k.a Chinese parsley)\\
\>3 Tbsp \>Freshly roasted and ground cumin\\
\> to taste   \>Green chilies \\
\>1-2 Tbsp \>Freshly ground black pepper\\
\> to taste   \>Tamarind \\
\> to taste   \>Jaggery (or Brown Sugar)\\
\>1 cup \>Chopped onions.\\
\end{tabbing}

{\bf Method}
\begin{enumerate}
   \item First boil the potatoes, mash them, salt  them,  and add  pepper  to
     taste.  Add some coriander leaves too.
   \item Roast the cumin and grind it.
   \item Dissolve  about 4 Tbsp of tamarind concentrate in 1 cup of hot water,
     and let it simmer and thicken  gradually.  Dissolve  the  jaggery  (or
     sugar)  until  the sauce becomes tart and slightly sweet. (You may add
     some salt and ground red paprika, if you want to.) The  sauce  should
     be  of  a  consistency slightly thinner than maple syrup. Pour into a
     serving container (like a creamer).  Mix the puffed rice and sev/bhel
     mix in a large bowl.
   \item On  a  plate,  serve  the  rice-bhel  mixture,  add  the
     potatoes, then the onions, chilies, and then dust the cumin powder  over  it.
     Next pour on the sauce and top with the coriander garnish. (Add salt/pepper
     to taste).
   \item Mix the ingredients on the plate and eat.
\end{enumerate}

\section{Bonda With Instant Mashed Potato}

  This recipe was taken from Saranya Mandava's book on Indian cuisine.

\begin{tabbing}
\hspace{1.0cm}  \={\bf Quantity}   \hspace{3.0cm} \={\bf Ingredient}\\
\>1 medium   \>Onion \\
\>2 cups   \>Potato buds \\
\>1 and $\frac{1}{2}$ cup   \>Peas and carrots \\
\>2 big ones   \>Green chilies \\
\>1 teaspoon   \>Lemon juice \\
\>1 cup   \>Gram flour \\
\>$\frac{1}{2}$ tea spoon   \>Mustard seed \\
\>pinch   \>Turmeric \\
\>$\frac{1}{2}$ teaspoon   \>Baking powder \\
\>$\frac{1}{4}$'' piece  \>Ginger \\
\>1 small bunch   \>Coriander leaves \\
\>2 teaspoons   \>Salt \\
\>   \>Oil \\
\end{tabbing}

  {\bf Method - The filling is prepared as follows:}
\begin{enumerate}
   \item  Mix potato buds and $\frac{1}{2}$ teaspoon of salt with 1 cup of hot water.
   \item Finely chop ginger, chilies, coriander leaves, and onion.
   \item Heat oil (about 5 Tbsp) and add mustard seeds.
   \item Add chopped ingredients and fry until onions are brown.
   \item Add carrots, peas, turmeric and 1 tsp of salt and cook  on  low
      heat for about 10 minutes.
   \item Add potato (now mashed) and fry for 5 minutes.
   \item Remove from heat, add lemon juice and let cool.
\end{enumerate}

{\bf Method - The batter is prepared as follows:}

\begin{enumerate}
   \item  Combine  gram  flour,  3  Tbsp  of  oil,  $\frac{3}{4}$  cup water, $\frac{1}{2}$
      tsp salt, the baking powder and mix thoroughly
\item  After the batter is prepared, make small balls out of the filling and  roll  them
in the batter. Next fry it in hot oil. You will get about 20-30 small bondas.
\end{enumerate}


\section{Pakoras (Savory Fritters)}

\begin{tabbing}

\hspace{1.0cm}  \={\bf Quantity}   \hspace{3.0cm} \={\bf Ingredient}\\
{\em Batter:}\\

\>  $\frac{1}{2}$ c\> Besan\\
\> 5 oz. \>Warm water\\
\> $\frac{1}{4}$ t \>Red pepper\\
\> $\frac{3}{4}$ t \>Salt\\
\> $\frac{1}{2}$ t \>Garam Masala\\
\> \>paprika (optional)\\

{\em Vegetables:}\\

\> 1 \>Small onion\\
\> 1 \>Potato\\
\>\> A few spinach leaves \\
\>\> Oil for deep frying\\
\end{tabbing}

{\bf Method}
\begin{enumerate}
\item  In a bowl put the besan and half the water, and stir until it becomes
a thick batter.  Beat hard for 5 minutes. gradually add the rest of the water,
and leave to swell for 30 minutes.  Add salt, pepper and Garam Masala and
beat again.
\item Wash peel and slice the onion and potatoes.
\item Wash and pat dry the spinach leaves.
\item Heat oil until smoking hot, dip the vegetables in the batter and deep
fry until golden brown.
\item Serve hot.
\end{enumerate}


\section{Dahi Vada (Savory Balls in Yogurt)}

\begin{tabbing}
\hspace{1.0cm}  \={\bf Quantity}   \hspace{3.0cm} \={\bf Ingredient}\\
\>  $\frac{1}{2}$ c \>Urad dal\\
\>  $\frac{1}{2}$ \>Moong dal\\
\>  1 c \>Yogurt\\     
\>\>  Spice to taste (cumin and paprika)\\
\>\>  Oil for frying\\
\>  $\frac{1}{4}$ c\> Milk\\
\end{tabbing}

{\bf Method}
\begin{enumerate}
\item  Clean dal, wash and soak in water for 4 hours.
\item Blend in blender at medium speed using the minimum amount
 of water required to keep paste moving freely.
\item Add salt and start heating oil.
\item Drop spoonfuls of batter in the hot oil using a large tablespoon. Fry until golden
brown.
\item Drop in water. Let it soak till ready to serve.
\item Beat yogurt with milk. Add spice according to taste. Squeeze out water from
the vada and add yogurt. Serve.
\end{enumerate}


\section{Upma}
  Recipe from Sriram, 1985

  This is a breakfast dish in the southern part of India.  Ingredients

\begin{tabbing}
\hspace{1.0cm}  \={\bf Quantity}   \hspace{3.0cm} \={\bf Ingredient}\\
\>1 cup   \>Cream of wheat \\
\>1   \>Onion  cut lengthwise\\
\>1 teaspoon   \>Salt \\
\>$\frac{1}{4}$ teaspoon   \>Mustard seeds \\
 \>$\frac{1}{8}$ teaspoon   \>Urad dal\\
\>1   \>Cashew-nut \\
\>$\frac{1}{4}$   \>Lemon \\
\>$\frac{1}{2}$ cup   \>Peas \\
\>1 clove   \>Garlic \\
\>1/10'' piece \>Ginger\\
\>1 \> Green chili cut into small pieces.\\
\end{tabbing}

{\bf Method}
\begin{enumerate}
   \item  Fry cream of wheat on a dry pan for 5 minutes and set aside.
   \item Put two tablespoons of oil in a pan and heat.
   \item Add mustard seeds, Urad dal, cashewnut, and garlic clove.  Wait  till
      mustard seeds stop splitting.
   \item Add  the  onion,  chili,  and  ginger  and fry till the onion turns
      brown.
   \item Add cream of wheat and fry for 3-5 minutes.
   \item Add salt and peas.
   \item Add two cups of boiling water and stir for 2 minutes.  (Switch  off  the
      stove as soon as the water is poured.)
   \item Cover the vessel for 4 minutes. Add lime if needed.
\end{enumerate}

\section{Vegetable Puff}
{\bf 12 puffs\\}
This is a local Karnataka speciality.
\begin{tabbing}
\hspace{1.0cm}  \={\bf Quantity}   \hspace{3.0cm} \={\bf Ingredient}\\
\>1 \>Ready to use pastry roll (Pepperidge farms)\\
\>  \> Mixed Vegetables. ( potatoes, peas may be added)\\
\>to taste  \> Masala\\
\>\>Green Chilies\\
\>to taste\>Salt\\
\end{tabbing}

{\bf Method}
\begin{enumerate}
\item Cook a vegetable mix with potatoes, peas, green chilies and
lots of Masala. Check for salt, before you stuff it in the pastry roll; no way to rectify it later.
\item Thaw the roll for about 10 minutes before unfolding.
\item           After the pastry roll has thawed, open it out on a flat sheet and
roll it with a pin to make it a little thinner.
The pastry sheet would now be about 12'' x  12''. 
\item Cut the sheet into 6 pieces.
\item Place about 2-3 Tbsp of the cooked
vegetable onto the sheet and fold it around it. Seal all the corners, by
pressing the sheets together and applying a little water.
\item Stick it into a pre-heated oven (350 $F^\circ$) for about 20-30
minutes or until it browns. Make sure that you flip it around every
5-10 minutes.
\end{enumerate}
                

\chapter{Vegetables}

\section{Gobi Aloo (Cauliflower and Potatoes)}
  This is my own recipe.

  This  recipe  livens  up ordinary cauliflower and potatoes into something
quite different.  The recipe shown here has spices that are not necessary (like
cloves and cardamom) or at least that's not how mom made it at home, however it
adds a nice touch.

\begin{tabbing}
\hspace{1.0cm}  \={\bf Quantity}   \hspace{3.0cm} \={\bf Ingredient}\\
\>1  \>Large  cauliflower \\
\>3     \>Medium sized potatoes \\
\>$\frac{1}{2}$ large   \>Onion   sliced thinly in long slices\\
\>1 teaspoon   \>Mustard seeds \\
\>2 or 3 pods   \>Cardamom \\
 \>1 teaspoon   \>Coriander\\
\>1 teaspoon   \>Cumin seeds \\
\>$\frac{1}{2}$ teaspoon   \>Turmeric \\
\>1   \>Bayleaf \\
\>3   \>Cloves \\
\>3 tablespoons   \>Vegetable Oil \\
\end{tabbing}

{\bf Method}
\begin{enumerate}
\item  Start boiling the potatoes in a saucepan.   Let  them  boil  for  at
      least  15  minutes.   After they are done, turn off the heat and let
      them stand in the water.
\item Cut the cauliflower into small bite  sized  pieces  (roughly  1''
      cubes),  throwing  away  most of the stem pieces.  Wash and drain in a
      collander.
\item While the potatoes are cooking, heat the oil in a wide skillet  until
      it  is  very  hot.    Add the mustard seeds and wait until they start
      popping.  Add bay leaves, cardamom and cloves.
\item Mix around for a while and then add onions.  Wait until the onion  starts
      to turn before adding the rest of the spices (except for turmeric).
   \item Put the cauliflower in the skillet and  fry in the oil and spices
      for 2 minutes.  While the cauliflower is frying, cut up the potatoes
      into  bite  sized  pieces  and  add to the skillet.  Add turmeric and
      stir.
   \item Continue stirring the  vegetables  under  medium  heat  for  another
      couple  of minutes.  Add $\frac{1}{2}$ cup of water and reduce heat to low.
      Cover skillet and let cook for 5 minutes.
   \item Check tenderness of vegetables.  If they are  still  too  hard,  add
      another $\frac{1}{4}$ cup of water and cover again for 5 minutes.
   \item Salt to taste and serve.

\end{enumerate}

\section{Chole ``Bill and Jim'' (Chick Peas)}
  This  recipe  is  named after Bill Chiles and Jim Muller both of whom liked my
concoctions well enough that I started cooking this a lot.    This  is  a  real
simple  way of making chick peas.  It barely takes 15 minutes and the result is
quite delicious.  Ask Jim and Bill.

\begin{tabbing}
\hspace{1.0cm}  \={\bf Quantity}   \hspace{3.0cm} \={\bf Ingredient}\\

\>1 can   \>\parbox[t]{2in}{Chick peas (also called gar\-ban\-zo beans) (Pro\-gres\-so is a good brand.)}\\
\>1 large   \>Onion  chopped finely\\
\>2 medium sized   \>Potatoes (optional)\\
\>1 teaspoon   \>Mustard seeds \\
\>2 or 3 pods   \>Cardamom \\
\>1 teaspoon   \>Coriander \\
\>1 teaspoon   \>Cumin seeds \\
\>1 tablespoon   \>Garam Masala \\
\> \>Vegetable Oil \\
\end{tabbing}

{\bf Method}
\begin{enumerate}
   \item  If  you  are  using  the potatoes, start boiling them in a saucepan.
      Let them boil for at least 15 minutes.  After they  are  done,  turn
      off the heat and let them stand in the water.
   \item While  the potatoes are cooking, heat the oil in a wide skillet until
      it is very hot.  Add the mustard seeds  and  wait  until  they  start
      popping.  Add bay leaves, cardamom and cloves.
   \item Mix  around  for a while and then add onions.  Waituntill the onion starts
      to turn golden before adding the rest of the spices (except for the
      Garam Masala).
   \item Add  chick-peas  with  all the liquid.  Cut up the potatoes into bite
      sized pieces and add to the skillet.  Add Garam Masala.
   \item Continue stirring the chick-peas under medium heat  for  5-7  minutes
      without covering.
   \item Check  the tenderness  of the potatoes.    If  they are still too hard, add
      another $\frac{1}{4}$ cup of water and cook for another couple of minutes.
   \item Salt to taste and serve.
\end{enumerate}

\section{Masaledar Sem (Spicy Green Beans)}
{\bf Serves 6}\\
  This recipe is mostly Madhur Jaffrey's, although  I  don't  follow  it  to  the
letter  when I cook anymore.  I like to serve the beans a little crunchier than
you would find in an indian home, so I don't cook it as much in  the  end.    It
definitely  warrants  experimentation.  This recipe is guaranteed to spice up an
ordinary meal.  It also goes well with plain rice and meat or  chicken  that  has  been
prepared simply. 

\begin{tabbing}
\hspace{1.0cm}  \={\bf Quantity}   \hspace{3.0cm} \={\bf Ingredient}\\
   \>1  $\frac{1}{2}$  lb. \> \parbox[t]{2in}{Green beans  (Trim the ends and then cut the  beans in half
     cross\-wise.)}\\
   \>\parbox[t]{2.5in}{1 $\frac{1}{2}$'' long and} \> Fresh ginger  (Peel and chop coarsely.)\\
   \>1'' thick piece \> \\
   \>10 cloves \>Garlic peeled\\
   \>1 cup \>Water\\
   \>4 tablespoons \>Vegetable oil\\
   \>3 teaspoons \>Whole cumin seeds\\
   \>2 teaspoons \>Ground coriander seeds\\
   \>2 medium \> \parbox[t]{2in}{To\-ma\-toes, peeled (put to\-ma\-toes in very hot water  for  a  few
     sec\-onds, peel off the skin and finely chop.)}\\
   \>to taste   \>Salt \\
   \>  \>Freshly ground pepper\\
   \>3 tablespoons \>Lemon juice (or to taste)\\
\end{tabbing}

{\bf Method}
\begin{enumerate}
   \item  Put  ginger  and garlic into a food processor and add $\frac{1}{2}$ cup water.
      Blend until fairly smooth.
   \item Heat the oil in a wide, heavy saucepan over a medium  flame.    When
      hot, put in the cumin seeds.  Stir for half a minute.
   \item Pour  in  the  ginger-garlic  paste.    Stir  and cook for about two
      minutes.  Put in the coriander and stir a few times.
   \item Put in the chopped tomatoes.  Stir and cook for  2  minutes while mashing
      the tomato pieces with the back of a slotted spoon.
   \item Put in the beans and salt and one cup of water and simmer them.
   \item Cover,  turn heat  to low and cook for 8-10 minutes or until the
      beans are tender.
   \item Remove the cover.  Add the lemon juice  and  a  generous  amount  of
      freshly ground pepper.
   \item Turn  up  the  heat and boil away the remaining liquid, stirring the
      beans gently as you do so.
\end{enumerate}

\section{Vegetable Kurma}
  This Recipe from Sriram

\begin{tabbing}
\hspace{1.0cm}  \={\bf Quantity}   \hspace{3.0cm} \={\bf Ingredient}\\
 \>2 cups   \>Vegetables\\
\>2   \>Onions  cut length-wise\\
\>2   \>Green chilies cut length-wise\\
\>1 teaspoon   \>Coriander powder \\
\>1 and $\frac{1}{4}$ teaspoon   \>Salt \\
 \>one pinch   \>Turmeric powder\\
\>$\frac{1}{2}$''   \>Cinnamon stick\\
\>2   \>Cloves \\
\>2   \>Cardamom \\
\>2 tablespoons   \>Coconut powder \\
\>1 teaspoon   \>Khus-Khus (poppy seeds) \\
\>$\frac{1}{4}$ teaspoon (3 cloves)   \>Garlic \\
\>$\frac{1}{4}$ teaspoon powder (or $\frac{1}{2}$'' fresh)   \>Ginger \\
\end{tabbing}

{\bf Method}
\begin{enumerate}
   \item  Put a reasonable sized vessel on the range and heat oil.
   \item Add cinnamon, cloves and cardamom and fry for 2-3 minutes.
   \item Add onions and green chilies and fry till onions turn brown.
   \item Add garlic + ginger paste and fry for a minute or so.
   \item Add vegetables and fry for about 3 minutes.
   \item Add Water (about a cup or two).
   \item Let the vegetables + turmeric powder cook.
   \item If you are using canned or frozen vegetables skip the above step.
   \item Add coconut paste, khus-khus, salt and wait  until  cooked.    (note:
      Cook on low heat.)
\end{enumerate}

\section{Vegetable Curry}
  Recipe from Sriram, 1985

\begin{tabbing}
 \hspace{1.0cm} \={\bf Quantity}   \hspace{3.0cm} \={\bf Ingredient}\\
\>1 cup   \>Vegetables \\
\>$\frac{1}{4}$-$\frac{1}{2}$ teaspoon   \>Coriander powder \\
\>$\frac{1}{8}$-$\frac{1}{4}$ teaspoon   \>Chili powder \\
\>$\frac{1}{8}$-$\frac{1}{4}$ teaspoon   \>Garlic powder \\
\>1 teaspoon   \>Salt \\
\>1   \> large Onion \\
\>$\frac{1}{4}$-$\frac{1}{2}$ teaspoon   \>Mustard seeds \\
\>$\frac{1}{8}$ teaspoon   \>Urad Dal \\
\>$\frac{1}{4}$-$\frac{1}{2}$ cup   \>Tomatoes-crushed \\
\end{tabbing}

{\bf Method}
\begin{enumerate}
\item  Mix  the  garlic, coriander and the chili powder along with salt and
      place it aside.
\item Pour about 2 tablespoons of oil in a pan and heat.
\item Add mustard seeds and urad dal. The mustard seeds  will  split  and
      the  oil  may  spill.  Be careful when you are doing this. Wait until
      the mustard seeds stop making any noise.
\item Add onions and fry until the onions turn brown.
\item Add the vegetables, the mixture of step 1 and the crushed tomatoes.
\item Fry for about 5 minutes, if you are using  canned vegetables.  Otherwise
      cover the pan and let the vegetables cook. (This might take about
      10-15 min.)
\end{enumerate}

\section{Green Pepper Curry}
  This recipe from Sriram

\begin{minipage}{7in}
\begin{tabbing}
\hspace{1.0cm}  \={\bf Quantity}   \hspace{3.0cm} \={\bf Ingredient}\\
 \>2 large ones    \>Green Pepper\\
 \>$\frac{1}{4}$ teaspoon   \>Chili powder\\
 \>$\frac{1}{8}$ teaspoon   \>Turmeric powder\\
 \>$\frac{1}{2}$ teaspoon   \>Dhania powder\\
 \>1 tablespoon   \>Coconut flakes\\
 \>1 teaspoon     \>Khus Khus (poppy seeds) \\
 \>1 small bunch  \>Fresh Coriander leaves\\
 \>2 small        \>Tomatoes \\
 \>2              \>Onions\\
 \>2 tablespoons  \>Oil\\
\>1 small piece   \>Vadium\footnote{Vadium is a combination of various spices.} \\
\>1 $\frac{1}{4}$ teaspoon   \>Salt \\
\end{tabbing}
\end{minipage}

\vspace*{.25in}
{\bf Method}
\begin{enumerate}
   \item  Cut the green peppers, onion and tomatoes lengthwise.
   \item Grind chili-powder, turmeric, dhania powder, coconut and
poppy seeds.
   \item Heat oil and add vadium.
   \item When vadium turns brown, add onions and fry for 4 minutes.
   \item Add tomatoes and fry for 2 minutes.
   \item Add green pepper and Masala.
   \item Add coriander leaves.
   \item Cook on low heat (should take around 15 minutes).
\end{enumerate}

\section{Dry Potatoes (Sookha Aloo)}
{\bf 4-6 Servings}
\begin{tabbing}
\hspace{1.0cm}  \={\bf Quantity}   \hspace{3.0cm} \={\bf Ingredient}\\
\>4 medium size \>Potatoes\\
\>2 t \>Cumin seeds\\
\>1 t \>Salt\\
\>2 t \>Mango powder\\
\>$\frac{1}{4}$ t\> Hot pepper\\
\>2 t \>Garam Masala\\
\> \>Oil (to fill pan to 2'')\\
\end{tabbing}


{\bf Method}
\begin{enumerate}
\item  Boil potatoes until cooked but not overdone.
\item  Peel and cut into $\frac{1}{2}$'' cubes.
\item  Heat oil very hot, add and brown cumin seeds.
\item  Add potatoes and fry until they are golden brown.  Add the remaining
ingredients, and fry for 2-3 minutes or more.  Remove from oil with a slotted
spoon.
\item  Serve hot.
\end{enumerate}
Tips:  Use enough oil so that the potatoes will not need to be stirred
often.  This avoids breaking them up.

\section{Okra (Bhindi)}
{\bf Serves 6}

\begin{tabbing}
\hspace{1.0cm}  \={\bf Quantity}   \hspace{3.0cm} \={\bf Ingredient}\\
\>1 lb\> okra \\
\> 2 small \>Onions \\
\>2 small \>Tomatoes    \\
\>$\frac{1}{4}$ t \>Turmeric \\
\>to taste\>Salt \\
\>\>Red pepper (optional) \\
\>\>Oil for frying\\
\end{tabbing}

{\bf Method}
\begin{enumerate}
\item  Wash the okra and dry it thoroughly.
\item Cut off the heads and cut into small circles.
\item Chop the onions and tomatoes separately.
\item Deep fry the okra until very brown.
\item  Remove from heat and set aside.  Pour out some oil.
\item Add turmeric to hot oil.  Add the onions and fry until golden brown.
\item Add the fried okra, salt, pepper, and tomatoes.
\item Cover and bake at 250 $F^\circ$ for 15 minutes.
\end{enumerate}

\section{Paneer (Cheese)}

\begin{tabbing}
\hspace{1.0cm}  \={\bf Quantity}   \hspace{3.0cm} \={\bf Ingredient}\\
\>1 $\frac{1}{2}$ pints \>Milk\\
\>$\frac{1}{2}$ t       \>White vinegar {\bf AND}\\
\>$\frac{1}{2}$ t       \>Lemon juice {\bf OR}\\
\>1 c                   \>Yogurt\\
\end{tabbing}

{\bf Method}
\begin{enumerate}
\item Heat milk and stir constantly to prevent a layer of cream from forming on the top.
\item Remove from heat when it boils and slowly add white vinegar and lemon
juice or yogurt.  This sours the milk.
\item Strain through a muslin cloth or a double layer of cheese cloth and
squeeze out the whey (liquid). 
\item Hang to drip dry for 2-3 hours (or overnight).  Then lay out the
cheese in a rectangle in a tray and place a weight (the more the better,
but at least 10 lbs) on it for $\frac{3}{4}$ hour.  
\item Cut it into whatever shape you like.
\end{enumerate}

\section{Mattar Paneer (Peas \& Cheese)}

\begin{tabbing}
\hspace{1.0cm}  \={\bf Quantity}   \hspace{3.0cm} \={\bf Ingredient}\\
\>2 \>Onions    \\
\>2 \>Tomatoes\\
\>4 cloves \>Garlic             \\
\>$\frac{1}{2}$ packet frozen \>Peas\\
\>1'' cube\> Ginger              \\
\>$\frac{1}{4}$ t \>Turmeric\\
\>to taste\>Salt \\
\>to taste\>Pepper              \\
\>$\frac{1}{2}$ t \>Garam Masala\\
\>2 c \>Water\\
\end{tabbing}

{\bf Method}
\begin{enumerate}
\item Cut paneer in 1'' cubes and deep fry.
\item Make Masala with onion, garlic, ginger, and tomatoes.
\item Season and add turmeric.
\item Add peas and paneer.
\end{enumerate}

\section{Navrathna Kurma}
{\bf Serves  6}
\begin{tabbing}
\hspace{1.0cm}  \={\bf Quantity}   \hspace{3.0cm} \={\bf Ingredient}\\
\>100 grams \>Paneer (or cottage cheese)\\
\>2 teacups \>Mixed boiled vegetables\\
\>\>(carrots, french beans, green peas, potatoes)\\
\>3 \>Tomatoes\\
\>2 \>Onions\\
\>1 t\> Ginger and garlic paste\\
\>1$\frac{1}{2}$ t \>Chili powder\\
\>$\frac{1}{2}$ t \>Turmeric powder\\
\>2 t \>Coriander powder\\
\>1 t \>Garam Masala\\
\>1 teacup \>Milk\\
\>3 t \>Fresh cream\\
\>3 t \>Ghee \\
\>to taste\>Salt\\
\>\>Ghee for deep frying\\
\end{tabbing}

{\bf Method}
\begin{enumerate}
\item Grate the onions.
\item Put the tomatoes in hot water. After 10 minutes take off the skin and chop.
\item Cut the paneer into small pieces and deep fry in ghee.
\item Heat oil in a vessel and fry the onions for a few minutes.
\item Add the ginger and garlic paste, and fry for $\frac{1}{2}$ minute.
\item Add the chopped tomatoes, turmeric powder, coriander powder and chili powder, 
Garam Masala and salt. Fry for at least 3-4 minutes.
\item Add the boiled vegetables, milk, cream and fried paneer pieces.
\item Cook for a few minutes.
\item Serve hot decorated with silver foil.
\end{enumerate}

%____________  THANKS TO THE DELIGHTS OF VEGETARIAN COOKING-- TARLA DALAL.

\section{Cauliflower and Potatoes (Aloo Gobi)}

\begin{tabbing}
\hspace{1.0cm}  \={\bf Quantity}   \hspace{3.0cm} \={\bf Ingredient}\\

\>  1 medium \>Cauliflower        \\
\>2 medium \>Potatoes\\
\>  1 \>Onion   \\
\>1 \>Tomato\\
\>1 clove\>  Garlic\\
\>1'' piece \>Ginger\\
\>pinch\>  Turmeric\\ 
\>to taste\>Salt\\
\>to taste\> Pepper\\
\>to taste\>Garam Masala\\
\end{tabbing}

{\bf Method}
\begin{enumerate}
\item Make Masala with onion, garlic, ginger, and tomatoes.
\item Add turmeric and spices.
\item Break the cauliflower in flowerettes and cut the potatoes
into cubes (8 pieces each).
\item Add both to Masala and lower heat to simmer. Cover the pot until the
cauliflower and potatoes are coated.
\end{enumerate}

\section{Curried eggplant (Bhartha)}
{\bf Serves 4 to 6}

\begin{tabbing}
\hspace{1.0cm}  \={\bf Quantity}   \hspace{3.0cm} \={\bf Ingredient}\\
\> 2 lb. \>Eggplant      \\
\>4 medium \>Tomatoes\\
\> 3 t \>Fresh chopped coriander\\
\>$\frac{1}{2}$ c \>Ghee\\
\> $\frac{1}{2}$ c \>Finely chopped onion\\
\end{tabbing}

{\bf Method}
\begin{enumerate}
\item Preheat oven to 450 $F^\circ$.
\item Bake in the middle level of the oven for 1 hour or until very tender.
\item While they are still warm, peel and crush the eggplants.
\item Heat oil and fry onions until soft and clear. Do not brown.
\item Add the tomatoes and fry for 2 minutes.
\item Add the eggplant and stir until almost all liquid disappears and the 
mixture leaves the side.
\item place into a bowl, sprinkle on coriander and serve at once.
\end{enumerate}


\section{Curried Mushrooms}

\begin{tabbing}
\hspace{1.0cm}  \={\bf Quantity}   \hspace{3.0cm} \={\bf Ingredient}\\
\>$\frac{1}{2}$ lb. \>Mushrooms\\
\>1 large \>Onion\\
\>$\frac{1}{4}$ t \>Turmeric\\
\>to taste\>Salt \\
\>to taste\> Chili \\
\> 2 large \>Tomatoes\\
\>\>Oil\\
\end{tabbing}

{\bf Method}
\begin{enumerate}
\item Wash and finely slice mushrooms. Slice onion fine.
\item Heat oil and add turmeric and onions.  Fry until soft.
\item Add the sliced tomatoes and cook for 3 minutes, stirring all the time.
\item Add the mushrooms, cover and simmer for 15-20 minutes.
\item Remove cover and dry out all the water. 
\item Serve hot.
\end{enumerate}


\chapter{Lentils}
  This chapter needs more recipes about Dals.  For now, I only have a
recipe for 
'sambar' a thin dal recipe from South India.  This  is  typically  served  with
idlis  or plain rice.  Idlis are hard to make and I haven't been able to find a
satisfactory recipe for them to date.

\section{Sambhar}
  This recipe from Sriram


\begin{tabbing}
\hspace{1.0cm}  \={\bf Quantity}   \hspace{3.0cm} \={\bf Ingredient}\\
 \>1 cup   \>Toor Dal\\
 \>1 teaspoon   \>Tamarind\\
 \>3 teaspoons   \>Salt\\
 \>a pinch   \>Turmeric\\
 \>2 teaspoons   \>Channa Dal\\
 \>3 teaspoons   \>Dhania seeds\\
 \>1 pinch   \>Hing\\
 \>3   \>Red chilies\\
 \>$\frac{1}{4}$ cup   \>Grated coconut \\
 \>1 teaspoon   \>Mustard\\
\> 10  \>Coriander leaves\\
 \>1    \>Green peppers cut into pieces\\
 \>1    \>Onion chopped\\
 \>1    \>Tomato cut into pieces\\
\end{tabbing}

{\bf Method}
\begin{enumerate}
   \item Boil the toor dal with 3 cups of water.
   \item Fry channa dal, dhania seeds, hing, and red chilies for a few  minutes,
      and then fry them with the grated coconut.
   \item Grind the above mixture with water.
   \item Fry the green pepper in oil for a few minutes.
   \item Boil the tamarind paste, water, salt, turmeric, tomato and vegetables.
   \item Add ??? 3 ??? and cook for about 5 minutes.
   \item Add boiled dal and bring it to a boil
   \item In the meantime fry the mustard seeds and onion.
   \item Add the above ingredients \& coriander leaves to the mixture.
\end{enumerate}


\section{Masur Dal (Lentils)}
{\bf Serves 4}

\begin{tabbing}
\hspace{1.0cm}  \={\bf Quantity}   \hspace{3.0cm} \={\bf Ingredient}\\
\>1 c \>Dal (Moong - yellow, or masur - pink)\\ 
\>3 $\frac{1}{2}$ c \>Water\\
\>to taste\>Salt \\
\>to taste\>Pepper      \\
\>$\frac{1}{4}$ teaspoon \>Turmeric\\
\>2 cloves \>Garlic\\
\>1'' piece\>Ginger\\
\>1 small \>Onion\\
\>2-3 Tablespoons\>Ghee (Can be replaced by butter)\\
\>1 t \>Cumin seed\\
\end{tabbing}

{\bf Method}
\begin{enumerate}
\item Wash the dal and drain it.
\item Boil water and add the dal, salt, pepper, turmeric, finely chopped
ginger, and garlic.  Cover the pot and simmer for 20 minutes.
\item When done, heat the ghee, add the cumin and fry until golden brown. Add
thinly sliced onions.  Fry until crisp and brown.
\end{enumerate}
   You may add paprika and finely chopped tomatoes to the above for color (Pour
over the dal and serve).

\section{Mah Ki Dal (Whole Black Beans)}

\begin{tabbing}
\hspace{1.0cm}  \={\bf Quantity}   \hspace{3.0cm} \={\bf Ingredient}\\
\>1 c \>Urad or Mah dal\\
\>1'' piece\>Ginger\\
\>12 cloves \>Garlic\\
\>\>Water\\
\>to taste\>Salt        \\
\>2-3 Tablespoons\>Ghee (Can be replaced by butter)\\
\>to taste\>Green chili (optional)\\
\>to taste\>Garam Masala\\
\>$frac{1}{4}$ t\>  Turmeric\\
\end{tabbing}

{\bf Method}
\begin{enumerate}
\item Clean, wash and add the dal to boiling water.
\item Add turmeric, salt, half of the finely chopped ginger, garlic and cook 
on medium heat for 2-3 hours or pressure cook at 15psi for $\frac{1}{2}$ hour.
\item Uncover, cook further in same pan on low heat for $\frac{1}{2}$
hour, stir and mash every now and then until a creamy consistency is achieved.
\item Heat the ghee, add remaining ginger, stir, add sliced onions, chili, and
fry until golden brown.
\item Add cumin or coriander (optional). Pour over dal just before serving.  
\end{enumerate}


\section{Red Kidney Beans (Rajma)}
{\bf Serves 6 - 8}

\begin{tabbing}
\hspace{1.0cm}  \={\bf Quantity}   \hspace{3.0cm} \={\bf Ingredient}\\
\>  2 c\> Red kidney beans\\
\>3 qt.\>Water\\
\>1  t \>Turmeric\\
\>1 T \>Salt\\
\>  $\frac{1}{4}$ c\> Oil\\
\>1 c \>Onion, Chopped\\
\> 1'' piece \>Ginger, chopped \\
\>1 t \>Garam Masala\\
\>  3 \>Chopped tomatoes\\
\>\>  Coriander leaves for garnish\\
\end{tabbing}

{\bf Method}
\begin{enumerate}
\item Wash beans and boil for 2-3 hours or $\frac{1}{2}$ hour in a pressure cooker.
\item In the meantime make Masala of onions, garlic, ginger and tomato as in
chicken curry.
\item Add to the beans and cook again until most of the liquid dries up and the
beans are soft and thoroughly cooked.
\item Garnish with coriander leaves and serve.
\end{enumerate}

\section{Curried Garbanzo Beans}

\begin{tabbing}
\hspace{1.0cm}  \={\bf Quantity}   \hspace{3.0cm} \={\bf Ingredient}\\
\>8 oz. \>Garbanzo beans                \\
\>to taste\>Green chili (optional)\\
\>  12 cloves \>Garlic\\
\>2'' piece\>Ginger\\
\>to taste\>  Pepper\\
\>1 large \>Cardamom\\
\> 6 \>Cloves\\
\>1'' \>Cinnamon stick\\
\>to taste\>  Salt\\
\>\> Paprika    \\
\>4 oz. \>Oil\\
\>\>  Mango powder\\
\>3 \>Onions\\
\>\>  Dried pomegranate seeds\\
\>4 \>Tomatoes\\
\end{tabbing}

{\bf Method}
\begin{enumerate}
\item Clean, wash and soak the beans overnight.
\item Boil them in the same water with salt, 1 small finely chopped onion, 4
cloves garlic, 4 large cardamom, a 1'' piece of ginger and 6 cloves.
\item Simmer in pan about an hour or until tender, or pressure cook for 7 minutes at
15 psi.
\item Heat oil. Fry thinly sliced remaining onions and cloves of garlic.
Cook until mixture browns and dries up. Add finely chopped tomatoes and cook
4-5 minutes more.
\item Add the beans and cook for 10 minutes more.  Add the mango powder and
pomegranate seeds, grated ginger and simmer over low heat for 15-20
minutes.
\item Before serving, pour 1 oz. (2 T) sizzling ghee over the beans.
\end{enumerate}


\chapter{Rice}

\section{Chicken Pullao}
  Recipe from Sriram, 1985

\begin{tabbing}
\hspace{1.0cm}  \={\bf Quantity}   \hspace{3.0cm} \={\bf Ingredient}\\
\>2 large \>Onions cut lengthwise\\
\>2 large \>Chilies cut lengthwise\\
\>2 c 	  \>Basmathi rice (about $\frac{1}{2}$ kg. (1 kilogram=2.2lbs)\\
\>1 large  \>Tomato (cut into small pieces)\\
\>10-15   \>Coriander leaves\\
\> 5      \>Mint leaves\\
\>1 clove \>Garlic \\
\>1'' piece\> Ginger \\
\>$\frac{1}{2}$ cup   \>Coconut powder\\
\>3 teaspoons   \>Salt\\
\>3-4  	\>Cloves\\
\>2    	\>Cardamom\\
\>1   	\>Bay leaf\\
\>1''   \>Cinnamon stick\\
\>1 c   \>Yogurt\\
\>2 tablespoons   \>Butter\\
\>1 lb.	\> Boneless chicken\\
\end{tabbing}

{\bf Method}
\begin{enumerate}
   \item Heat vessel with butter.
   \item Fry bay leaves, cloves, cardamom and cinnamon.
   \item Put  onions  and  chilies in vessel and fry on low heat until onions turn
      brown.
   \item Add ginger + garlic paste and fry until oil separates.
   \item Add tomato and fry for 1 minute.
   \item Add chicken + salt + yogurt and fry for one minute.
   \item Add mint + coriander leaves.
   \item Cover and cook until the gravy becomes semi-solid.
   \item Cook rice in a separate vessel.
  \item Put rice into chicken and mix (It is advisable to  cook  rice  about
      $\frac{3}{4}$ths and then let it cook with the chicken).
  \item Remove and serve (Will serve about 4 hungry grad. students.)
\end{enumerate}


\section{South Indian Pullav (Rice)}
  From Sriram, 1985

\begin{tabbing}
\hspace{1.0cm}  \={\bf Quantity}   \hspace{3.0cm} \={\bf Ingredient}\\
\>1 c   \>Rice (Preferably Basmathi rice)\\
 \>$\frac{1}{3}$ c   \>Tomato puree\\
 \>1 large   \>Onion\\
 \>1 c   \>Vegetables (preferably peas and carrots)\\
 \>$\frac{1}{4}$-$\frac{1}{2}$ teaspoon   \>Coriander powder\\
\> \>(also called dhania powder) \\
\>$\frac{1}{8}$-$\frac{1}{4}$ teaspoon    \>Garlic powder or \\
\> 1 and a half cloves garlic \\
\>$\frac{1}{8}$-$\frac{1}{4}$ teaspoon \>Ginger powder\\
\>$\frac{1}{4}$'' piece\>Ginger \\
\>$\frac{1}{8}$-$\frac{1}{4}$ teaspoon \>Chili  powder\\
\> 1 \>Green chili cut into small pieces\\
 \>1-2 pieces\>Cardamom\\
 \>1 piece   \>Cloves\\
 \>$\frac{1}{4}$'' stick or $\frac{1}{8}$ teaspoon powder\>Cinnamon\\
 \>1   \>Bay leaf\\
 \>1 teaspoon   \>Salt\\
 \> 1 teaspoon   \>Coriander leaves (if needed)\\
\end{tabbing}

{\bf Method}
\begin{enumerate}
   \item Clean the rice with water and set aside.
   \item Cut the onions length wise.
   \item Fry the onions and cardamom in butter for about 4 minutes.  If  you  are
      using  green  chili, then  add  the chili.
   \item Add bay leaf, cloves, cinnamon and fry until the onions  turn  golden
      brown (This will probably take another 4-5 minutes).
   \item Add  the  garlic  and  ginger  paste (preferably prepared from fresh
      ginger and garlic).
   \item Add dhania powder and chili powder (if green chili was  not  added
      before).
   \item Add  the  tomato  paste and one cup of water (you have to experiment
      with the quantity of water needed. I found 1-$\frac{1}{2}$ cups to be  optimal)
      and bring the mixture to boil.
   \item Add the vegetables, rice and salt.
   \item If you like coconut, add $\frac{1}{4}$ cup of coconut flakes.
  \item Reduce the flame and cover the vessel.
  \item After about 4 minutes, stir the mixture.
  \item Cover  the  lid  again  and wait until cooked (might take about 10-15
      minutes).
  \item Sprinkle on the coriander leaves in the end.
\end{enumerate}

\section{Vegetable Pullav 2}
  Recipe from Sriram, 1985
This recipe is slightly spicer than the previous one.  

\begin{tabbing}
\hspace{1.0cm}  \={\bf Quantity}   \hspace{3.0cm} \={\bf Ingredient}\\
 \>1 c   \>Rice\\
 \>1 $\frac{1}{2}$ c   \>Water\\
 \>1 c   \>Vegetables\\
 \>$\frac{1}{2}$''   \>Cinnamon stick\\
 \>2   \>Cloves\\
 \>2   \>Cardamom\\
 \>1 $\frac{1}{4}$ teaspoon   \>Salt\\
 \>$\frac{1}{8}$ teaspoon   \>Turmeric powder\\
 \>1 teaspoon   \>Dhania powder\\
 \>2    \>Chilies or \\
\> $\frac{1}{4}$ teaspoon powder\>\\
 \>$\frac{1}{2}$ can or 1 lb. (16 oz)   \>Tomatoes\\
 \>$\frac{1}{2}$ cup   \>Coconut\\
 \>1 bunch   \>Coriander leaves\\
\> 4 cloves    \>Garlic\\
\> $\frac{1}{2}$'' piece\>Ginger (made into a paste)\\
\>2 tablespoons   \>Butter\\
 \>1    \>Onion cut lengthwise\\
\end{tabbing}

{\bf Method}
\begin{enumerate}
   \item Wash the rice and drain the water.
   \item Extract one cup of water from tomatoes.
   \item Pour the butter into a vessel and heat.
   \item Add cinnamon, cardamom and cloves.
   \item Add onions and chilies and fry until onions turn golden brown.
   \item Add ginger + garlic paste and turmeric powder paste and fry until you
      get a nice smell.
   \item Now pour in the tomato water + 1 cup water.
   \item Add coconut, coriander powder (Dhania powder), salt and let boil
  \item Add rice + coriander leaves + vegetables.
  \item Reduce to low heat and let the rice cook.
\end{enumerate}


\section{Saffron Rice (Kesar Chawal)}

\begin{tabbing}
\hspace{1.0cm}  \={\bf Quantity}   \hspace{3.0cm} \={\bf Ingredient}\\
\>  2 c\> Rice\\
\>4 c \>Water\\
\>  6 T \>Ghee\\
\>1 t \>Saffron threads (or less)\\
\>  2 T \>Hot water\\
\>1 c \>Sliced onion\\
\>  1 small\>Cinnamon stick - splintered\\
\>  4 \>Bay leaves\\
\>4 large \>Black cardamoms\\
\>1  T \>Cumin seed\\
\>4 \>Cloves\\
\>  2 t \>Salt\\
\end{tabbing}

{\bf Method}
\begin{enumerate}
\item Soak saffron in hot water.
\item Wash and soak rice in 3 c water (optional).
\item Heat ghee and fry onions and then remove and keep aside.
\item Add cinnamon, cumin seeds, cardamoms, cloves and salt.  Wait 1 minute and then
add the bay leaves and $\frac{1}{2}$ the onions.  Drain the rice and reserve the water.
\item Add the rice grains and stir for 4-5 minutes until all the water
evaporates and the grains of rice are coated with oil.
\item Add the water and bring to a boil.  
\item Add saffron and its water and pressure
cook at 15 psi.
\item Remove from the heat and allow the pressure to drop by itself.
\end{enumerate}


\section{Navrattan Pullao  (Nine-Jeweled Rice)}

\begin{tabbing}
\hspace{1.0cm}  \={\bf Quantity}   \hspace{3.0cm} \={\bf Ingredient}\\

\>  1 c\> Basmathi rice\\
\>1 $\frac{3}{4}$ c \>Water\\
\>  $\frac{1}{4}$ c \>Oil\\
\>1 small \>Finely sliced onion\\
\>  6 \>Cloves\\
\>1'' \>Cinnamon stick\\
\>  1 t \>Salt \\
\>$\frac{1}{2}$ t \>Cumin seed\\
\end{tabbing}

For Mixing with rice before serving:

A  $\frac{1}{4}$ c frozen peas (boiled), salt, 6 drops green food coloring mixed with
   1 t water.

B  $\frac{1}{4}$ c diced tomatoes, $\frac{1}{4}$ t red pepper, salt, Garam Masala, 6 drops red
   food coloring mixed in 2 t water.
\begin{tabbing}
\hspace{1.0cm}  \={\bf Quantity}   \hspace{3.0cm} \={\bf Ingredient}\\

\>1 \>Thinly sliced onion\\
\>2 $\frac{1}{4}$ c \>Ghee\\
\>1 oz. \>Almonds\\
\>1 oz. \>Cashew nuts\\
\>1 $\frac{1}{2}$ oz. \>Golden raisins\\
\>1 oz. \>Pistachio nuts\\
\>1'' piece \>Ginger thinly sliced \\
\>1 \>Green chili (optional)\\
\>1 \>Hard boiled egg\\
\end{tabbing}

{\bf Method}
\begin{enumerate}
\item Clean, wash, and soak rice in 1 $\frac{3}{4}$ c water for pressure cooking, or in
2 cups of water for pan cooking.
\item Heat oil and fry the onions. Add whole spices.  Fry 1 minute.  Add rice and
fry for $\frac{2}{3}$ minutes to coat the rice grains with oil. Add the water which the
rice was soaking in.  Pressure cook, building up the pressure to 15 psi and
let drop, OR cook in a pan bringing to full boil then down to a simmer followed by
20 minutes of cooking until the rice is done.
\item Divide rice into three parts. Thoroughly mix A with one part and B with
another.  Leave the last part plain.  Keep rice separate and warm in the oven.
\item (C)  Meanwhile fry the onion until it is crisp.  Remove and fry almonds,
cashew nuts, pistachios, raisins, ginger and chili.  Chop egg and sprinkle with
salt.  Keep warm until required.
\item To serve, place onions, nuts, chopped egg and all of (C) in a dish
and spread the three batches of rice in clumps above it aesthetically.
\end{enumerate}


\chapter{Fish}

\section{Prawn (Shrimp) Curry}
  This Recipe from Sriram, 1985

\begin{tabbing}
\hspace{1.0cm}  \={\bf Quantity}   \hspace{3.0cm} \={\bf Ingredient}\\

 \>$\frac{1}{2}$ kg=1.1 lb.    \>Prawns\\
 \>2    \>Onions diced into small pieces\\
 \>$\frac{1}{4}$''   \>Cinnamon stick\\
 \>$\frac{1}{4}$ teaspoon   \>Chili powder\\
 \>$\frac{1}{2}$ teaspoon   \>Dhania powder\\
 \>$\frac{1}{4}$ teaspoon   \>Garlic powder\\
 \>$\frac{1}{2}$ teaspoon   \>ginger powder\\
 \>1 bunch   \>Fresh coriander\\
 \>1 teaspoon   \>Salt\\
 \>$\frac{1}{4}$ teaspoon   \>Turmeric powder\\
 \>1 tablespoon   \>Oil\\
\end{tabbing}

{\bf Method}
\begin{enumerate}
   \item Clean the prawns and squeeze out the water.
   \item Add chili, dhania, garlic, ginger, turmeric powder,  salt  and  mix
      well.
   \item Boil prawns on low heat.
   \item Add 1 teaspoon of oil to the boiling prawns.
   \item When water evaporates and the prawns are dry remove from the stove.
   \item Heat the oil and put in the cinnamon.
   \item Add prawns and fry for 2 minutes.
   \item Add onions and fry until they turn brown.
   \item Sprinkle on coriander leaves, remove from the heat and serve.
\end{enumerate}

\section{Coriander Fish (Bharia Machli)}

\begin{tabbing}
\hspace{1.0cm}  \={\bf Quantity}   \hspace{3.0cm} \={\bf Ingredient}\\

\>  4 lb. \>Sole, flounder, rock cod, or any other white whole fish\\
\>  8 cloves \>Garlic\\
\>3 \>Hot chilies (optional) (or cayenne)\\
\>  1'' piece\>Ginger\\
\>1 medium bunch \>Coriander\\
\>1  T \>Coriander seeds\\
\>1 t \>Brown sugar\\
\>1  t \>Turmeric\\
\>$\frac{1}{2}$ t \>Black mustard\\
\>  $\frac{1}{2}$ t \>Fenugreek seeds\\
\>1 T \>Salt\\
\>  $\frac{1}{2}$ c \>Lemon juice\\
\>$\frac{1}{2}$ c \>Vegetable oil\\
\>  2 c \>Chopped onion\\
\>1 c \>Chopped tomato\\
\>  $\frac{1}{2}$ t \>Garam Masala\\
\end{tabbing}

{\bf Method}
\begin{enumerate}
\item Preheat oven to 400 $F^\circ$.
\item Wash and pat fish dry. Sprinkle 1 t salt inside and set aside.
\item Blend garlic, chili, ginger, $\frac{1}{2}$ the coriander, coriander seeds, brown
sugar, turmeric, mustard seeds, fenugreek seeds, salt and lemon juice until
it all becomes a smooth paste (Add some water if needed).
\item Fry onions until they are soft and golden brown.
\item Add the blended Masala and cook until most of the liquid is gone, and it
starts to leave the sides of the pan.
\item Add the tomatoes and Garam Masala.  Fry for 2 minutes more and remove.
\item Coat one side of fish, stuff 1 $\frac{1}{2}$ cups inside.  Close opening, spread
the rest of the Masala over it.  Cover tightly and bake for about 25 minutes.
\item Grill for 1-2 minutes in the broiler, and sprinkle on the remaining coriander.
\item Serve.
\end{enumerate}


\chapter{Chicken}

\section{Mughlai Chicken with Almonds}
  This recipe is taken from Madhur Jaffrey's book

\begin{tabbing}
\hspace{1.0cm}  \={\bf Quantity}   \hspace{3.0cm} \={\bf Ingredient}\\
 \>1'' piece  \>Ginger\\
 \>8 to 9 cloves   \>Garlic\\
 \>6 tablespoons   \>Blanched Almonds\\
 \>7 tablespoons   \>Vegetable oil\\
 \>1''   \>Cinnamon stick\\
 \>2   \>Bay leaves\\
 \>5   \>Cloves\\
 \>10 pods   \>Cardamom\\
 \>2  medium   \>Onions  (cut into small pieces)\\
 \>2 teaspoons   \>Ground cumin seeds\\
 \>$\frac{1}{8}$-$\frac{1}{2}$ teaspoon   \>Red pepper\\
 \>7 tablespoons   \>Yogurt\\
 \>1 small carton   \>Whipped Cream\\
 \>$\frac{1}{4}$ teaspoon   \>Garam Masala\\
 \>2-2$\frac{1}{2}$ lbs.    \>Chicken boneless (2 trays of holy farms)\\
 \>2 $\frac{1}{2}$ teaspoons   \>Salt\\
 \>one bunch   \>Coriander leaves\\
\end{tabbing}

{\bf Method}
\begin{enumerate}
   \item Grind the ginger, garlic, and almonds with water.
   \item Heat oil in a non-stick pan, and fry the chicken until it turns golden
      brown. Keep it aside and drain the oil.
   \item Heat some oil and add the cardamom, bay leaves and  cloves  and  fry
      until the bay leaves turn brown.
   \item Add the onions and fry for a few minutes.
   \item Pour the paste from the blender and fry for a couple of minutes until
      the oil separates.
   \item Add 1 tablespoon of yogurt and  fry  for  30  seconds.  Keep  adding
      tablespoons of yogurt and fry until you get a consistent mixture.
   \item Add  the  chicken, whipped cream and salt and cook gently (low heat)
      for 20 minutes.
   \item Add Garam Masala and coriander  leaves  and  cook  for  another  10
      minutes.
\end{enumerate}

\section{Malai Chicken}
  Recipe from Sriram, 1985

\begin{tabbing}
\hspace{1.0cm}  \={\bf Quantity}   \hspace{3.0cm} \={\bf Ingredient}\\
 \>1   \>Tray chicken\\
 \>1   \>Chopped onion\\
\>1 small can   \>Tomato paste \\
 \>1   \>Red Chili\\
 \>2   \>Cloves\\
 \>2   \>Cardamom\\
 \>$\frac{1}{2}$'' \>Cinnamon stick\\
 \>1   \>Bay leaf\\
 \>1 carton   \>Light whipping cream\\
 \>1 tsp   \>Dhania powder\\
 \>$\frac{1}{2}$ tsp   \>Cumin powder\\
 \>1 tsp   \>Garam Masala\\
 \>1-1 $\frac{1}{4}$ tsp   \>Chili powder\\
\>$\frac{1}{2}$''   piece\>Ginger made into a paste\\
\>6 cloves     \>Garlic  (made into paste)\\
\> \>Fresh coriander\\
\> to taste \>Salt\\
\>   \>Turmeric\\
\end{tabbing}

{\bf Method}
\begin{enumerate}
   \item Heat oil.
   \item Add  red chili, cloves, cardamom, cinnamon stick, and bay leaf, and cook until the
      bay leaf turns golden brown.
   \item Add the onion and fry for two minutes.
   \item Add the ginger garlic paste and fry for 4 - 6 minute.
   \item Add the chicken and fry for 5 minutes.
   \item Sprinkle in chili, dhania powder, cumin powder and turmeric.
   \item Cover chicken + salt (add water if needed) and cook  for  around  15
      minutes until $\frac{3}{4}$ cooked.
   \item Add can of tomato paste and cook on low heat.
   \item Just before removing, add the whipping cream and cook for a few minutes.
  \item Add Garam Masala and coriander leaves
  \item Remove after a couple of minutes.
\end{enumerate}

\section{Chicken Curry North Indian Style}
  My own, mostly from mom's but synthesized with recipes from other friends.

\begin{tabbing}
\hspace{1.0cm}  \={\bf Quantity}   \hspace{3.0cm} \={\bf Ingredient}\\
 \>1 lb.  \>Chicken-drumsticks, thighs, breast pieces\\
\>1 small carton    \>Plain yogurt\\
\>2 medium    \>Onions very finely chopped\\
 \>4 tbl   \>Vegetable oil\\
 \>2   \>Cloves\\
 \>$\frac{1}{2}$ tsp    \>Mustard powder\\
 \>2 pods    \>Cardamom\\
 \>$\frac{1}{2}$ tsp   \>Cumin powder\\
 \>1 tsp    \>Garam  Masala\\
 \>1 tsp   \>Chili Powder\\
\>$\frac{1}{2}$''  piece\>Ginger \\
\>4 cloves \>Garlic \\
\>$\frac{1}{3}$ tsp   \>Coriander Seeds \\
\>to taste \>Salt \\
\>$\frac{1}{2}$ tsp   \>Freshly ground pepper \\
\end{tabbing}

{\bf Method}
\begin{enumerate}
   \item Remove fat from the chicken and then salt and pepper it.  Sprinkle
      with chili powder.  Add yogurt and mix  well  until  the  chicken  is
      covered  liberally  with  yogurt.   Use your hands.  Set aside for 1
      hour before cooking.  If kept in the refrigerator, set aside for  at  least  4
      hours.
   \item Heat  oil  in a large heavy pan.  When oil is hot, add mustard seeds,
      if you are using them.  Add cloves, cardamom, and coriander  seeds and fry for 30 seconds.
   \item Add  the onion and fry for two minutes until the onion beings to turn brown.  Lower
      heat to medium.
   \item Add the ginger and garlic paste and fry for 4-6 minutes.
   \item Add mustard powder, if using it,  add  Garam  Masala,  and add  cumin
      powder.
   \item Brush  excess  yogurt  off  the chicken and put it in a large pot.  Add
      ingredients from the frying pan.  Cook uncovered over high  heat  for  4
      minutes.
   \item Reduce  heat  to  low and cover. Cook for 25 minutes or until the
      chicken is tender, stirring every 5 minutes.
\end{enumerate}

{\bf Important note:}  When chicken is cooked with a cover on the pot,  it
      releases  water  that  becomes  a  part  of  the sauce.  If after 10
      minutes, there isn't enough sauce in the pot,  add  $\frac{1}{4}$  cup  water.
      Conversely,  if  there  is too much liquid in the pot, cook uncovered
      until the liquid evaporates.

Variations:  There are several variations to the above recipe:

\begin{enumerate}
\item Leave out the yogurt. Add $\frac{1}{4}$ cup of water just before  turning  the
      heat to low and covering the pot.
   \item Boil  two  potatoes  for 10 minutes before slicing them thinly.  Add sliced
      potatoes to the pot when you start cooking the chicken.
   \item This variation  is  usually  called  ``Malai  Chicken''  or  literally
      ``creamy''  chicken.  Leave  out  the yogurt.  When the chicken is $\frac{3}{4}$
      done, add one small can of tomato paste.   Just  before  removing  add a
      small carton of whipping cream, and cook for a few minutes.
\end{enumerate}

\section{Chicken Curry (Murga)}         
{\bf 4-6 Servings}

\begin{tabbing}
\hspace{1.0cm}  \={\bf Quantity}   \hspace{3.0cm} \={\bf Ingredient}\\

\>2-2 $\frac{1}{2}$ lb. \>Skinned chicken\\
\>1 t \>Garam Masala\\
\>3 t \>Salt\\
\>1 c \>Finely chopped tomato\\
\>$\frac{1}{4}$ c \>Vegetable oil\\
\>$\frac{1}{2}$ c \>Water\\
\>1 $\frac{1}{2}$ c \>Finely chopped onion\\
\>1 $\frac{1}{2}$ t \>Fresh ginger, chopped\\
\>1 t \>Finely chopped garlic\\
\>1 t \>Vinegar\\
\>1 \>Dried chili (optional)\\
\end{tabbing}

{\bf Method}
\begin{enumerate}
\item Cut chicken, separate legs and thighs, back and split breast.
\item Heat oil over high heat. Add onions and fry until golden brown. Take care not
to burn them.
\item Add garlic, ginger and tomatoes.  Fry until a smooth paste is obtained.
\item Add chicken, add water, bring to a boil, and add salt.  Cover pot and lower
heat.  Add chili to make it hot if desired.
\item Stir constantly to avoid burning and coat the chicken pieces evenly.
\item After the chicken is done, add vinegar and cook for another 5 minutes on
very low heat.  Sprinkle on Garam Masala and serve.
\end{enumerate}

\section{Tandoori Chicken}
{\bf 2-3 Servings}

\begin{tabbing}
\hspace{1.0cm}  \={\bf Quantity}   \hspace{3.0cm} \={\bf Ingredient}\\

\>6 pieces 	\>Thawed chicken, skinned \\
\> 2 tsp	\> Ground Coriander \\
\> 2 tsp	\> Masala (Tandoori paste is available) \\
\> to taste	\> Red pepper powder \\
\> Dash		\> Garlic powder\\
\> to taste	\> Salt \\
\> 1 tsp	\> Ground jeera \\
\> 		\> Soy sauce (or yogurt)\\
\>		\> (needed only if tandoori masala is used)\\
\end{tabbing} 

{\bf Method}
\begin{enumerate}
\item If you are using the ready made tandoori paste then life is a
lot easier. Replace all occurences of masala and soy sauce(or yogurt)
with the tandoori paste.
\item
 Take the chicken and make *deep* cuts in it (so that the Masalas seeps in quickly). 
\item
If you are using soy sauce as the base, put some on the chicken pieces and let it seep in the cuts. 
\item Rub in the Masalas as a mixture or one at a time. The idea is to let
the Masalas seep in the cuts with the soy sauce. You can leave it
for little while to seep in.
\item If you are using yogurt, you'll get a more authentic taste since
the original TC is after all marinated in it. In this case, mix the
Masalas in the yogurt first and then rub the stuff into the chicken
cuts as before. The yogurt tends to leave a considerable amount of
water behind. DON'T THROW THIS AWAY. Let it evaporate in the oven
with the chicken. This will keep the pieces from getting dry if
over-cooked. I have not faced the same problem with the soy sauce 
version (of dry chicken). 
\item Cook the chicken until it starts turning brown.
and the cuts you made start ``expanding.''
\end{enumerate} 

{\bf Notes:}
You may use any other interesting sauce as the base. Some previous
experiences of my own are: Teriyaki (my Japaani-tandoori), Oriental
sauce (my supermarket-tandoori).

\chapter{Lamb and Beef}

\section{Lamb Vindaloo}
  From Esquire Magazine, 1986:  
I have made the recipe a couple of times and both times with very good results.  The
finished  dish  is  a  spicy lamb dish that is quite exquisite.  Be warned that
this recipe takes quite a bit of effort to put together.

\begin{tabbing}
\hspace{1.0cm}  \={\bf Quantity}   \hspace{3.0cm} \={\bf Ingredient}\\
   \>3 lbs\> Lean boneless lamb\\
\> 3\> Meaty lamb bones
\end{tabbing}

Marinade made from:
\begin{tabbing}
\hspace{1.0cm}  \={\bf Quantity}   \hspace{3.0cm} \={\bf Ingredient}\\

       \> 4 tablespoons \>Light vegetable oil\\
        \> $\frac{1}{4}$ c \>Cider vinegar\\
        \> 3 tablespoons \>Tamarind pulp\\
        \> to taste      \> Salt\\
\end{tabbing}

Puree made from:
\begin{tabbing}
\hspace{1.0cm}  \={\bf Quantity}   \hspace{3.0cm} \={\bf Ingredient}\\

        \> 2 tablespoon \>Vegetable oil\\
        \> 1 large \>White onion\\
        \> 6 \>Garlic cloves\\
        \> 2 tablespoons \>Fresh ginger root, chopped\\
   \>$\frac{1}{2}$ c \>Vegetable oil\\
   \>3 c \>Onion, thinly sliced\\
   \>1 teaspoon \>Ground cumin\\
   \>1 teaspoon \>Ground mustard\\
   \>3 teaspoons \>Turmeric\\
   \>1 $\frac{1}{2}$ teaspoons \>Red pepper\\
   \>3 teaspoons \>Paprika\\
   \>2 $\frac{1}{2}$ c \>Hot water\\
\end{tabbing}

{\bf Method}
\begin{enumerate}
   \item Cut lamb into $\frac{3}{4}$'' cubes.
   \item Place lamb and the  bones  in  a  non-metallic  bowl  with  the  four
      tablespoons  of  oil,  the  vinegar, tamarind pulp, and salt.  Let it
      marinate at room temperature for eight hours  or,  refrigerated,  for
      24 hours.
   \item Put  two  tablespoons of oil, the onion, garlic, and ginger in
      an electric blender or food processor and  run  the
      machine until a fine pasty puree is formed.
   \item Heat  the  one-half  cup  of  vegetable oil in a large enamel coated
      skillet over medium-high heat.  Add the onions and saute until  they
      are caramel brown, stirring constantly to prevent burning.
   \item Add  the  puree.    Reduce  the  heat  and  add ground cumin, ground
      mustard, turmeric, red pepper, and paprika.  When the spices begin to
      sizzle  and  turn dark, in about 15 seconds, add the lamb and bones.
      Cook until slightly seared (about ten minutes).
   \item Add the water and bring to a boil, then lower the heat  and  simmer,
      partially  covered,  until  the  meat  is  very tender (about thirty
      minutes).
   \item Pick out and discard bones.
   \item Serve over rice.
\end{enumerate}

\section{Egg cooked with Meat and Fried (Nargisi Kofta)}

Meat:
\begin{tabbing}
\hspace{1.0cm}  \={\bf Quantity}   \hspace{3.0cm} \={\bf Ingredient}\\
\>1 lb. 		\>Ground lean meat\\
\>$\frac{1}{4}$ c 	\>Chopped onion\\
\>4 cloves 		\>Chopped garlic\\
\>1'' piece		\>Ginger, chopped\\
\>$\frac{1}{2}$ t 	\>Turmeric\\
\>$\frac{3}{4}$ c 	\>Water\\
\>to taste 		\>Salt and pepper \\
\end{tabbing}

To mix with meat:
\begin{tabbing}
\hspace{1.0cm}  \={\bf Quantity}   \hspace{3.0cm} \={\bf Ingredient}\\
\>4 T 		\>Besan (chick-pea flour) \\
\>1 T 		\>Yogurt\\
\>6 		\>Hard boiled eggs          \\
\>1 		\>Egg for mixing and coating\\
\>$\frac{1}{2}$ t \>Garam Masala         \\
\>		\>Oil for frying\\
\end{tabbing}

For curry (Masala):
\begin{tabbing}
\hspace{1.0cm}  \={\bf Quantity}   \hspace{3.0cm} \={\bf Ingredient}\\
\>		\>Oil or ghee\\
\>2 		\>Chopped onions\\
\>8 		\>Cloves chopped garlic\\
\>2 		\>Tomatoes or equivalent paste\\
\>1'' piece 	\>Ginger, chopped\\
\>$\frac{1}{4}$ c \>Yogurt\\
\>$\frac{1}{4}$ t \>Turmeric\\
\>		\>Green onion\\
\>$\frac{1}{2}$ t \>Garam Masala\\
\>10-15 leaves	\>Coriander\\
\>to taste 	\>Salt and pepper\\
\end{tabbing}

{\bf Method}
\begin{enumerate}
\item Heat the water and add the meat, onions, ginger,  garlic,
salt and pepper.  Pressure cook for 10 minutes at 15 psi (or 25 min over
low heat.  Reduce pressure and drain half the liquid.  
Add the besan (or $\frac{1}{2}$ c soaked lentils) and
cook for 10 minutes. Knead or grind until slightly sticky, mix in egg yolk,
Garam Masala and yogurt and knead well.
\item  Coat the hard boiled eggs with the above and deep
fry.
\item Heat ghee, fry the onions to a golden brown, add garlic, ginger,
tomatoes and yogurt and fry well until the Masala is a paste.  Add
water to the mix if necessary.  Add the green sprigs of onion, 1$\frac{1}{2}$ c
water and cook for 10 minutes covered.
\item When curry is ready, pour into a serving dish, cut the koftas in
half and arrange over the curry.  Cover and bake at 250 $F^\circ$ for 15-20
minutes.
\item Serve garnished with coriander leaves and Garam Masala.
\end{enumerate}


\section{Pork Curry}                    
{\bf Serves 6 or 8}

\begin{tabbing}
\hspace{1.0cm}  \={\bf Quantity}   \hspace{3.0cm} \={\bf Ingredient}\\

\>1 $\frac{1}{2}$ lb. \>Pork\\
\>3 \>Medium onions\\
\>2'' piece\>Ginger\\
\>4 cloves \>Garlic\\
\>2 medium \>Tomatoes\\
\>2 T \>Vinegar\\
\>2 t \>Salt\\
\>$\frac{1}{2}$ t \>Garam Masala\\
\>2 T \>Vindaloo paste (hot)\\
\>1 medium \>Potatoes\\
\>2 c \>Water\\
\end{tabbing}

{\bf Method}
\begin{enumerate}
\item Clean, wash, trim and dry pork.  Cut into 1'' cubes.
\item Make Masala with onions, ginger, and tomatoes as in chicken curry, etc.
\item Add the vinegar and Vindaloo paste and cook for 2 minutes.
\item  Add the pork and pressure cook for 20 minutes, or in a heavy pot for 1 $\frac{1}{2}$
hours, until the pieces are tender.
\item Cube potatoes, add and cook the potato pieces coated in the liquid.
\item Boil down the amount of water (30 minutes) while the potatoes cook.
\end{enumerate}


\section{Shahi Korma (Mutton Curry)}

\begin{tabbing}
\hspace{1.0cm}  \={\bf Quantity}   \hspace{3.0cm} \={\bf Ingredient}\\
\>1 $\frac{1}{2}$ lb. \>Goat mutton\\
\>3 \>Medium onions\\
\>$\frac{3}{4}$ c \>Yogurt\\
\>4 cloves \>Garlic\\
\>pinch \>Saffron\\
\>2 t \>Salt\\
\>$\frac{3}{4}$ c \>Cream\\
\>1 oz. \>Almonds\\
\>$\frac{1}{2}$ c \>Oil\\
\>1 oz. \>Coriander seed\\
\>1 t \>Red pepper (optional)\\
\>$\frac{1}{2}$ t \>Garam Masala\\
\end{tabbing}

{\bf Method}
\begin{enumerate}
\item Clean, wash and dry mutton.
\item Blend coriander seeds, 1 onion, almonds and garlic.
\item Marinade the mutton in above mixture for 2 hrs.
\item Heat oil and fry the remaining thinly sliced onion.  Keep aside.
\item Add the mutton and fry until the liquid dries up.
\item Add $\frac{3}{4}$ c hot water and simmer until the meat is almost done ($\frac{3}{4}$ cooked) OR
pressure cook at 15 psi for 20 minutes.  Reduce pressure.  Add salt and pepper.
\item Uncover and dry the liquid.
\item Add beaten yogurt and fry until it leaves oil.
\item Add fried ground onion.
\item Beat the cream. Add soaked or ground saffron.
\item Mix it with the cooked mutton.
\item Add Garam Masala and bake at 250 $F^\circ$ for $\frac{1}{2}$ hour
\item Serve garnished with chopped coriander leaves.
\end{enumerate}


\chapter{Desserts and Other Goodies}

\section{Kheer (Vermicelli Pudding)}

  Recipe from Dalbir Chadda:  
This has been my all time favorite dessert.  Ever since I was  very  little,  I
can remember asking for seconds and thirds.  What makes this dessert unusual is
that it is not as sweet as most Indian desserts.  It is fairly simple to  make.
Make sure that the vermicelli is very fine (angel hair pasta is ok but the very
fine vermicelli that can be bought at chinese stores is the best).

\begin{tabbing}
\hspace{1.0cm}  \={\bf Quantity}   \hspace{3.0cm} \={\bf Ingredient}\\
   \>1 stick \>Butter\\
   \>2 handfuls \>Very fine vermicelli\\
   \>4 cups \> Milk\\
   \>1 pint \>Whipping cream\\
   \>1 handful \>Raisins\\
   \>3 tablespoons \>Sugar\\
   \>4 \>Almonds (optional) peeled and thinly sliced\\
\end{tabbing}

{\bf Method}
\begin{enumerate}
   \item Melt butter in a 4 qt. pot.
   \item Break vermicelli into 3'' pieces.  Over low heat stir  vermicelli
      into butter until it turns light brown.
   \item Pour in the milk and stir over medium heat until it boils.
   \item Put in the raisins, almonds and sugar.
   \item Continue to cook under low heat for 10 minutes.
   \item Add whipping cream and continue to cook for a couple of minutes.
   \item Remove  from  heat  and,  when cool, chill in the refrigerator before
      serving
\end{enumerate}


\section{Rice Flour Pudding }
{\bf 6 servings}

\begin{tabbing}
\hspace{1.0cm}  \={\bf Quantity}   \hspace{3.0cm} \={\bf Ingredient}\\

\>4 $\frac{1}{2}$ c\> Milk\\
\>$\frac{3}{4}$ c \>Sugar\\
\>2 oz. \>Rice flour\\
\>6-8 drops \>Rose water\\
\>1 oz. \>Almonds\\
\>$\frac{1}{2}$ oz. \>Pistachio nuts\\
\end{tabbing}

{\bf Method}
\begin{enumerate}
\item  Blanch (optional) and shred nuts.
\item  Mix rice flour into the milk and mix until smooth.
\item Cook over medium heat until a creamy consistency is achieved (20-30 minutes?).
\item Simmer and add sugar and stir for 2-3 minutes more.
\item Cool (in refrigerator for 30 minute) add the rose water, almonds and
pistachios  (maybe before it cools).
\item Pour into individual dishes and serve.
\end{enumerate}

\section{Besan Burfi}

\begin{tabbing}
\hspace{1.0cm}  \={\bf Quantity}   \hspace{3.0cm} \={\bf Ingredient}\\
\>1 c \>Besan\\
\>1 c \>Shortening\\
\>1 c \>Sugar\\
\>4 seeds \>Cardamom\\
\>\>Nuts (optional)\\
\end{tabbing}

{\bf Method}
\begin{enumerate}
\item Melt shortening in a pan.
\item Turn down heat and add cardamom and Besan.
\item Fry, stirring constantly to prevent burning until it has changed to a brown color
and smells done.  (Test: a few drops of water sprinkled on it
sputters instantly).
\item Turn off the heat and stir in the sugar.
\item Spread $\frac{1}{2}$'' thick onto a platter.
\item Cut into diamond shapes after it has cooled down.
\end{enumerate}


\section{Kheer}

\begin{tabbing}
\hspace{1.0cm}  \={\bf Quantity}   \hspace{3.0cm} \={\bf Ingredient}\\

\>$\frac{1}{2}$ c \>Rice\\
\>4 c \>Milk\\
\>$\frac{1}{4}$ c \>Raisins\\
\>$\frac{3}{4}$-1 c \>Sugar\\
\>1 t \>Cardamom seeds\\
\>$\frac{1}{4}$ c \>Shredded blanched almonds\\
\>6-8 drops \>Rose water\\
\>$\frac{1}{2}$ c \>Water \\
\end{tabbing}

{\bf Method}
\begin{enumerate}
\item Wash and drain the rice.
\item Soak in $\frac{1}{2}$ c water for $\frac{1}{2}$ hour.
\item Boil the rice in the same water until it is coated and the water dries
up.
\item Add the milk and simmer on low heat for 1 $\frac{1}{2}$ hours.
\item Scrape the sides and bottom frequently to prevent sticking and mash rice while
stirring.
\item When it is creamy, add sugar and stir in well.
\item Remove from heat and add crushed cardamom seeds, rose water and shredded
almonds.
\item Serve hot or cold decorated with silver leaves (optional). [Silver
leaves are VERY FINE, tasteless sheets of silver.]
\end{enumerate}


\section{Gulab Jamuns (Easy Method)}

\begin{tabbing}
\hspace{1.0cm}  \={\bf Quantity}   \hspace{3.0cm} \={\bf Ingredient}\\

\>1 c \>Bisquick\\
\>2 c \>Carnation powder\\
\>2 c \>Water\\
\>1 $\frac{1}{2}$ c \>Sugar\\
\>4 pods \>Cardamom     \\
\>few drops \>Rose water \\
\>$\frac{1}{2}$ stick \>Butter (4 T)\\
\>$\frac{1}{8}$ c \>Yogurt\\
\>\>Milk\\
\>\>Oil for frying\\
\end{tabbing}

{\bf Method}
\begin{enumerate}
\item Heat butter and pour in a bowl.
\item Add Bisquick, carnation powder and yogurt and blend together.
\item Knead well adding milk if necessary.
\item Make a smooth ball, cover and let rest (30 minutes?).
\item Make 12-14 small balls.
\item Heat the water, add sugar, bring to boil, add cardamom seeds and
simmer.
\item Boil, then simmer to reduce the water by half.
\item Heat the oil until hot and fry the balls to a golden brown or until they are
dark brown---almost black.
\item Soak in sugar syrup until they double in size (1 hour or overnight)
\item serve hot or cold.
\end{enumerate}

\section{Suji Halva (Semolina Halva)}

{\bf 4 - 6 servings\\}
\begin{tabbing}
\hspace{1.0cm}  \={\bf Quantity}   \hspace{3.0cm} \={\bf Ingredient}\\
\>$\frac{1}{2}$ c \>Suji (semolina)\\
\>$\frac{1}{2}$ c \>Sugar\\
\>$\frac{1}{2}$ c  \>Ghee\\
\>1 $\frac{1}{2}$ c \>Water\\
\>1 oz. \>Sliced almonds\\
\>1 oz. \>Raisins\\
\>8 \>Green cardamoms\\
\end{tabbing}

{\bf Method}
\begin{enumerate}
\item Boil sugar and water together for 5 minutes.
\item Heat ghee add suji and stir on low heat until mixture becomes light
creamy in color and ghee leaves the side of the pan.
\item Add the syrup and stir briskly until it is absorbed in the semolina.
\item Mix in crushed cardamom seeds, almonds, and raisins.
\item Serve hot.
\end{enumerate}


\section{Sewian (Vermicelli)}

\begin{tabbing}
\hspace{1.0cm}  \={\bf Quantity}   \hspace{3.0cm} \={\bf Ingredient}\\

\>2 c \>Sewian (vermicelli)\\
\>3 $\frac{1}{2}$ c \>Milk\\
\>$\frac{3}{4}$ c \>Sugar\\
\>$\frac{1}{4}$ t \>Rose water (or 6-8 small cardamom seeds)\\
\>\>Ghee\\
\end{tabbing}

{\bf Method}
\begin{enumerate}
\item Fry the sewian in hot oil until golden brown.
\item Heat the milk to boiling and add the sewian. Cook until the milk is
reduced by half.
\item Add sugar and cook on low heat until creamy (about 25 minutes).
\item Remove from the heat. Add in rose water.  
\item Decorate with blanched finely 
shredded almonds and pistachio nuts and silver leaves if desired.
\end{enumerate}



\section{Carrot Halva}

\begin{tabbing}
\hspace{1.0cm}  \={\bf Quantity}   \hspace{3.0cm} \={\bf Ingredient}\\

\>4 lbs. \>Carrots\\
\>$\frac{1}{2}$ gal \>Milk\\
\>2 c \>Sugar\\
\>2 c \>Carnation milk powder\\
\>1 c \>Oil\\
\>to taste \>Nuts \\
\end{tabbing}

{\bf Method}
\begin{enumerate}
\item Clean and grate the carrots.
\item Heat milk to boiling and add the carrots.
\item Cook until liquid is almost gone, stirring to prevent sticking and
burning (3 to 4 hours).
\item Add oil and cook more, stirring often, to roast the carrots well (about
$\frac{1}{2}$ hour).
\item Add the powdered milk and sugar and cook until all the liquid is gone
and the mass does not stick to the sides.
\item Add the nuts and raisins and turn off the heat.
\item Pour in a serving dish and serve warm or cold.  Will keep in the
refrigerator for up to 1 week.
\end{enumerate}


\section{Rasgoola}

\begin{tabbing}
\hspace{1.0cm}  \={\bf Quantity}   \hspace{3.0cm} \={\bf Ingredient}\\
\> 1 L   	\> Homogenized Milk  \\
\> 2 tsp  	\> White Vinegar\\
\> 1$\frac{1}{2}$ C \> Sugar\\
\> 3 C  	\>  Water\\
\end{tabbing}

{\bf Method}
\begin{enumerate}
\item Bring the milk to a boil and add vinegar to the boiling milk to separate
the whey. 
\item Throw away the liquid part by sifting the stuff onto a muslin cloth.
\item Pour some cold water over the curd to cool and wash it. 
Discard the water and hang the cloth for 15-20 minutes to let the excess
water drip off. 
\item Put the curd in a food processor or blender and blend at high speed to
get a smooth consistency. You may add just a little (1 tsp or so)
water while blending, if the curd is too dry and will not blend. Be very
careful so as not to add any extra water.
\item Remove the paste and make small balls (1-2'' in diameter). 
\item Boil water in a wide vessel. Make sure that there is at least
2-3'' of water in the vessel. If not, add more water and increase
the quantity of sugar proportionately. Add sugar to the boiling water to
make a light syrup.  
\item Continue boiling the syrup and gently drop the curd balls in the
boiling syrup. Cook the balls in the boiling syrup for 30-40 minutes. 
\item Remove from the heat and let the stuff cool down. Put the balls
and the syrup in a storage container and refrigerate (don't freeze). 
\item Serve cold.
\end{enumerate}


\section{Mango Ice Cream}
This is a great desert which can be made with very little effort. You
can replace the Mango pulp with any other pureed fruit.
\begin{tabbing}
\hspace{1.0cm}  \={\bf Quantity}   \hspace{3.0cm} \={\bf Ingredient}\\
\> 1 can	\> Condensed Milk\\
\> 12 oz.	\> Whipped cream(Cool whip)\\
\> 1 can  	\> Mango pulp (Alphonso)\\
\end{tabbing}
It is very confusing to describe quantities as  1 can.
Well, I do not remember the exact numbers so let me
describe the sizes. The Mango pulp can is about 6'' high and 3'' in
diameter.I think it is the only size available in an Indian store. The
condensed milk can is about 3'' high and about 2.5 '' in diameter and
should be available in your neighbourhood grocery store.

{\bf Method}
\begin{enumerate}
\item Mix all of the pulp, condensed milk and whipped cream in a bowl.
\item Put in the freezer for about 8 hours.
\end{enumerate}


\end{document}

