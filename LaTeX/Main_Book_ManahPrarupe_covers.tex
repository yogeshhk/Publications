\documentclass{article}
\usepackage[paperwidth=6in, paperheight=9in, margin=0.75in]{geometry}
\usepackage{fontspec}
\usepackage{xcolor}
\usepackage{graphicx}
\usepackage{tikz}

% Set up fonts
\setmainfont[Script=Devanagari] {Tiro Devanagari Marathi}
\newfontfamily\devanagarifont[Scale=MatchUppercase]{Tiro Devanagari Marathi}
%\setmainfont[Script=Devanagari]{Nirmala Text}
%\newfontfamily\devanagarifont[Scale=MatchUppercase]{Nirmala Text}
% \setmainfont[Script=Devanagari]{Noto Serif Devanagari}
% \newfontfamily\devanagarifont[Scale=MatchUppercase]{Noto Serif Devanagari}

\newfontfamily\devtransl[Mapping=DevRom]{Segoe UI}
\graphicspath{{images/}}

% Define colors
\definecolor{titleorange}{RGB}{255,140,0}
\definecolor{subtitleblue}{RGB}{0,51,102}
\definecolor{authorgreen}{RGB}{0,102,51}

\pagestyle{empty}

\begin{document}

% FRONT COVER PAGE
\thispagestyle{empty}
\null\vfill

\begin{center}
% Title - big orange font, 60% page width
{\fontsize{58}{78}\selectfont\color{titleorange}\textbf{मन:प्रारूपे }}

\vspace{1em}

% Subtitle - quarter size, dark blue
{\fontsize{12}{14}\selectfont\color{subtitleblue}(विविध क्षेत्रातील 'मेंटल मॉडेल्स'ची तोंडओळख)}

\vspace{5em}

% Author - dark green
{\fontsize{16}{20}\selectfont\color{authorgreen}\textbf{डॉ. योगेश हरिभाऊ कुलकर्णी}}
\end{center}

\vfill\null
\clearpage

% BACK COVER PAGE
\thispagestyle{empty}
\vspace*{0.5in}

% Book description
\noindent\textbf{पुस्तकाविषयी:} या पुस्तकात विविध क्षेत्रातील 'मेंटल मॉडेल्स'ची तोंडओळख आहे.  मेंटल मॉडेल्स (मन:प्रारूपे) म्हणजे जगाकडे पाहण्याची, निर्णय करण्याची आणि विचारांना आकार देण्याची एक सखोल पद्धत आहे. या पुस्तकात आपण त्या विचारचौकटींच्या जंगलात शिरतो, जिथे प्रत्येक मॉडेल एखाद्या नव्या दृष्टीकोनाची खिडकी उघडते. कृत्रिम बुद्धिमत्ता, मानसशास्त्र, अर्थशास्त्र किंवा दैनंदिन आयुष्य, कोणत्याही क्षेत्रातील गुंतागुंती समजून घेण्यासाठी ही मॉडेल्स आपल्याला अधिक सक्षम बनवतात. चार्ली मंगर यांनी सुचविलेल्या ‘लॅटिसवर्क'च्या संकल्पनेप्रमाणे, विविध विषयांतील मॉडेल्स जोडून विचार करण्याची कला इथे विस्ताराने उलगडली आहे. योग्य प्रसंगी योग्य विचार वापरण्याचे कौशल्य कसे विकसित करायचे, यासाठी हे पुस्तक एक मार्गदर्शक दीप ठरावे. आपल्या अंधबिंदूंची जाणीव करून देत, हे मॉडेल्स विचारांचे बंधन मोडायला शिकवतात. बदलत्या काळात आपली समज सुधारत राहणे का महत्त्वाचे आहे, याचीही या प्रवासात वाचकांना जाणीव होते. अखेरीस, हे पुस्तक आपण नकळत बांधलेल्या मानसिक पिंजऱ्याचे दार उघडण्याचे आमंत्रण देते.

\vspace{1.5em}

% Target audience
\noindent\textbf{वाचकांविषयी:} काही नवीन शिकण्यात रस असणाऱ्या सर्वांसाठी हे पुस्तक योग्य आहे.

\vspace{1.5em}

% Author bio
\noindent\textbf{लेखकाविषयी: डॉ. योगेश हरिभाऊ कुलकर्णी} हे तंत्रज्ञान क्षेत्रातील अनुभवी अभ्यासक आहेत. त्यांना कृत्रिम बुद्धिमत्ता, मशीन लर्निंग आणि डेटा सायन्स या क्षेत्रात अनेक वर्षांचा अनुभव आहे. शैक्षणिक क्षेत्रात तसेच उद्योगात काम करून त्यांनी या विषयावर व्यापक अभ्यास केला आहे. त्यांचे पूर्वीचे लेखन कार्य आणि संशोधन हे मराठी भाषेत तंत्रज्ञान विषयक साहित्याला वाव देण्याच्या दिशेने योगदान आहे.

\vfill

% Author photo and contact info at bottom
\begin{tabular}{@{}p{0.3\textwidth}@{}p{0.2\textwidth}@{}p{0.3\textwidth}@{}}
\centering
\includegraphics[width=\linewidth,keepaspectratio]{myphoto} 

yogeshkulkarni@yahoo.com
&
% blank column &
& 
\centering
\includegraphics[width=\linewidth,keepaspectratio]{mylinkedinqr}

+91 9890251406
\end{tabular}


% Publisher info at bottom
\begin{center}
\textbf{Notion Press}\\
\texttt{www.notionpress.com}\\
ISBN-13 ‏ : ‎ 979-8899613739
\end{center}

\begin{center}
\includegraphics[width=3cm]{isbn_barcode}
\end{center}

\end{document}