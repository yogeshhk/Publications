%%%%%%%%%%%%%%%%%%%%%%%%%%%%%%%%%%%%%%%%%%%%%%%%%%%%%%%%%%%%%%%%%%%%%%%%%%%%%%%%%%
\begin{frame}[fragile]\frametitle{}
\begin{center}
{\Large Notes of ``Essays'' by Paul Graham minus Lisp!!}

{\tiny www.paulgraham.com/articles.html}


\end{center}
\end{frame}


%%%%%%%%%%%%%%%%%%%%%%%%%%%%%%%%%%%%%%%%%%%%%%%%%%%%%%%%%%%
\begin{frame}[fragile]\frametitle{Programming Bottom-up}

\begin{itemize}
\item Keep software units (say, functions, classes) small, else software will be hard to read, hard to test, and hard to debug.
\item Top Down refactoring, upto manageable granularity
\item Lisp: Bottom up. If some basic operators are not there, they are coded first and used up later, like kernel to UI.
\item Once you abstract out the parts which are merely bookkeeping, what's left is much shorter; much desirable.
\end{itemize}

{\tiny http://www.paulgraham.com/progbot.html}

\end{frame}

%%%%%%%%%%%%%%%%%%%%%%%%%%%%%%%%%%%%%%%%%%%%%%%%%%%%%%%%%%%
\begin{frame}[fragile]\frametitle{Beating the Averages}

\begin{itemize}
\item Conventional wisdom: Lisp will make you a better programmer, and yet you won't use it.
\item If you do what average person does, the results will be? ``Average''.
\item If you're running a startup, you had better be doing something odd. If not, you're in trouble.
\item  In business, as in war, surprise is worth as much as force.
\item Programming languages are not just technology, but what programmers think in. They're half technology and half religion. Hard to change.
\end{itemize}

{\tiny http://www.paulgraham.com/avg.html}

\end{frame}

%%%%%%%%%%%%%%%%%%%%%%%%%%%%%%%%%%%%%%%%%%%%%%%%%%%%%%%%%%%
\begin{frame}[fragile]\frametitle{Taste for Makers}

\begin{itemize}
\item Taste: Design (or appreciate) beautiful things.
\item Saying that taste is just personal preference is a good way to prevent disputes. The trouble is, it's not true. There should be universal appeal for the beauty.
\item Your old tastes were not merely different, but worse
\item Good design is simple, iss timeless
\item Good design solves the right problem, is suggestive
\item Good design is often slightly funny (not really, mostly)
\item Good design is hard (but) looks easy
\item Good design uses symmetry, resembles nature
\item Good design is redesign
\end{itemize}

[Yogesh: Pytorch looks beautiful, clean, elegant, (not Tensorflow!!) so does vi editor, Unix OS and the pdf font and also, Python. Minimalist, just enough.]

{\tiny http://www.paulgraham.com/taste.html}

\end{frame}

%%%%%%%%%%%%%%%%%%%%%%%%%%%%%%%%%%%%%%%%%%%%%%%%%%%%%%%%%%%
\begin{frame}[fragile]\frametitle{How to Think for Yourself}

\begin{itemize}
\item To be a successful scientist, your ideas have to be both correct and novel.
\item You have to do something that sounds to most other people like a bad idea, but that you know isn't. 
\item Need to be independent-minded vs conventional conformist.
\item Goal should be not to let anything into your head unexamined
\item Components of independent-mindedness: fastidiousness about truth, resistance to being told what to think, and curiosity.
\end{itemize}



{\tiny http://www.paulgraham.com/think.html}

\end{frame}

%%%%%%%%%%%%%%%%%%%%%%%%%%%%%%%%%%%%%%%%%%%%%%%%%%%%%%%%%%%
\begin{frame}[fragile]\frametitle{What you can't say}

\begin{itemize}
\item Dressing oddly gets you laughed at. Violating moral fashions can get you fired, ostracized, imprisoned, or even killed.
\item Galileo got in big trouble when he said it — that the earth moves. Morally incorrect in that era.
\item Do we believe things that people in the future will find ridiculous??!!
\item Looking for things we can't say that are true, or at least have enough chance of being true that the question should remain open. 
\item Political correctness:  Harvard, it was inappropriate to compliment a colleague or student's clothes. No more ``nice shirt.''.
\end{itemize}



{\tiny http://www.paulgraham.com/say.html}

\end{frame}
