\chapter{तयारी एका मोठ्या स्थित्यंतराची }

{\textbf{दावोस येथे भरलेल्या 'वर्ल्ड इकॉनॉमिक फोरम'च्या बैठकीच्या सुमारास आय-एम-एफ' चा (आंतरराष्ट्रीय नाणेनिधी)कृत्रिम बुद्धिमत्ताविषयक एक अहवाल प्रसिद्ध झाला. त्यात 'एआय'च्या सद्यःस्थितीचे सखोल विश्लेषण करण्यात आले आणि काही भाकिते वर्तविण्यात आली आहेत. ती भविष्यातील वाटचालीच्या दृष्टीने अत्यंत महत्त्वाची आहेत. या अहवालातील ठळक मुद्यांचा ऊहापोह.}}

\vspace{1.5em}

दावोस येथे भरलेल्या 'वर्ल्ड इकॉनॉमिक फोरम'च्या बैठकीच्या सुमारास आंतरराष्ट्रीय नाणेनिधीचा कृत्रिम बुद्धिमत्ताविषयक एक अहवाल प्रसिद्ध झाला. मानवी बुद्धिमत्तेप्रमाणे जर मशीन, म्हणजेच संगणकप्रणाली कामे करू लागली तर त्यास कृत्रिम बुद्धिमत्ता (एआय) म्हणता येईल. त्यात विषयाचे आकलन करून घेणे, काय करायला पाहिजे ते समजणे, त्यानुसार कृती करणे इत्यादी अनेक पैलूंचा समावेश होतो.

नजीकच्या काळात प्रसिद्ध झालेले जननशील कृत्रिम बुद्धिमत्ता (जनरेटिव्ह एआय ' जेन एआय) हे विस्तृत अशा 'एआय'चाच एक भाग आहे. त्यात नवनवीन गोष्टींची निर्मिती केली जाते. सर्वांना परिचित असे 'चॅटजीपीटी' हे त्याचेच एक रूप, ज्यात भाषा निर्माण केली जाते. इतर काही प्रणालींमध्ये प्रतिमा-चित्रे तर काहींत चक्क चलत् चित्रे निर्माण केली जातात.

'एआय'च्या या अफलातून क्षमतेमुळे मानवी जीवनावर, व्यवसायांवर आणि अर्थकारणावर काय काय परिणाम होऊ शकेल, याविषयीची चर्चा आणि भविष्यवेध आंतरराष्ट्रीय नाणेनिधीच्या अहवालात आहे. त्याविषयी अधिक जाणून घेऊयात. खरे तर कोठल्याही गोष्टीविषयी भविष्य वर्तवणे अवघडच असते. विषयाचा पट 'जगावरील परिणाम' इतका व्यापक असेल तर आणखीनच अवघड. पण तरीही गोळा केलेल्या माहितीच्या आधारे, विविध प्रवाहांचा अंदाज ढोबळमानाने काढता येऊ शकतो.

'एआय'च्या प्रभावामुळे इतके सर्वदूर आणि आमूलाग्र बदल होणार आहेत (खरे तर सुरुवात झाली पण आहे) की त्याला पुढील औद्योगिक क्रांतीच म्हणावी लागेल. मागील क्रांत्यांप्रमाणेच त्यावेळेसही कार्यक्षमता वेगाने वाढणार आणि तसेच नोकरी-व्यवसायाची हानी होण्याची पण शक्यता आहे. कोठे सकारात्मक आणि कोठे नकारात्मक प्रभाव पडेल.

या अहवालानुसार जागतिक पातळीवर जवळपास ४० टक्के नोकरी-व्यवसायांवर 'एआय'चा प्रभाव पडणार आहे. विकसित देशात तर तो जरा जास्तच म्हणजे सुमारे ६० टक्के पण असेल, याचे कारण त्यांच्याकडे मानवी बुद्धिमतेसंदर्भातील नोकऱ्या-व्यवसाय जास्त आहेत. त्यामानाने जे अविकसित देश आहेत, त्यात 'एआय'चा प्रभाव जरा कमी, म्हणजे सुमारे २६ टक्के पडण्याची शक्यता आहे.

आता प्रभाव हा सकारात्मक किंवा नकारात्मक असू शकतो. त्याचे प्रमाण ५०-५० टक्के असेल असे वाटते. सकारात्मक म्हणजे उत्पादकता वाढणे, नव्या कल्पनांचे सृजन, क्लिष्ट समस्यांना उत्तरे मिळणे तर नकारात्मक प्रभाव म्हणजे नोकरी-व्यवसायावर गदा, आर्थिक व शेक्षणिक विषमतेत वाढ, इत्यादी.

पूर्वीच्या औद्योगिक क्रांतीने स्वयंचलन (ऑटोमेशन) आणल्याने मध्यम-कुशलतेच्या कामांवर प्रभाव पडला होता; पण 'एआय'मुळे उच्च-कुशलतेच्या कामांवर, जसे की संगणकावरील काम, वैद्यकीय तसेच आर्थिक उलाढाल यासारख्या प्रगत क्षेत्रातही मोठा प्रभाव पडणार आहेत. तोच-तोच-पणा असणारे, सहज शिकता येणारे काम जे असेल ते 'एआय' इमानेइतबारे नक्की करेल.

आता फक्त कामकाजाचे स्वयंचलनच नाही तर विचारांचेही 'स्वयंचलन' केले जाणार आहे. काही कामे पूर्णपणे 'एआय' करेल तर काहींत ते मदतनीस म्हणून काम करेल. काही उदाहरणे पाहूयात. सध्या सॉफ्टवेअर कंपन्यांमध्ये खालच्या  स्तरांवर  सॉफ्टवेअर इंजिनियर १, सॉफ्टवेअर इंजिनियर २ हे हुद्दे असलेले कर्मचारी प्रोग्रामिंगची कामे करतात.

त्यावरील पातळीवर सॉफ्टवेअर लीड (गटप्रमुख) हे त्यांचे प्रोग्रॅम तपासणे, मार्गदर्शन करणे, प्रणाली-प्रक्रिया यांचे नियोजन करणे, अशी जरा प्रगत कामे करतात. सध्या उपलब्ध असलेले 'एआय' पण बऱ्यापैकी प्रोग्रॅमिंग करू लागले आहे. नजीकच्या काळात ते इंजिनीअर १-२ प्रमाणे प्रोग्रॅम लिहायला लागल्यावर मग त्यांचे काय होणार? याकरिता विद्यार्थ्यांना स्वतःला इतके प्रशिक्षित करावे लागेल की, सुरवातीपासूनच 'लीड' सारखे काम हाती घ्यावे लागेल.

'एआय'चा प्रभाव फक्त तंत्रज्ञान क्षेत्रातच नाही, तर अन्य सर्जनशीलतेच्या क्षेत्रातही पडायला लागला आहे. 'एआय' कविता लिहितो, चित्र काढतो, गाणे बनवतो अगदी रंजक कथा पण लिहितो. याचाच अर्थ, जर तुम्ही फक्त 'ट' ला 'ट' जुळवणारी आशयहीन कविता लिहित असाल किंवा एखाद्या आंग्ल गाण्यावरून प्रभावित होऊन संगीत देणार असाल तर तुमचे काही खरे नाही. नाही का?

तुमची सर्जनशीलता, मानवी भावना, आयुष्याचे अस्सल अनुभव, संस्कार, सखोल चिंतन आणावे लागेल, तरच 'एआय'वर मात करता येईल. 'एआय'ला इतक्या वरच्या पातळीवर पोहोचणे होणे अवघड आहे, याचे कारण हे मानवी पैलू त्याला प्रशिक्षित केलेल्या माहिती (डेटा) मध्येच नाहीत (सध्यातरी!!) तर त्याला या गोष्टी कळणारही नाहीत. आढ्यातच नाही तर पोहोऱ्यात कुठून येणार?

सर्वसाधारणपणे 'एआय'मुळे मानवी उत्पादकता तर वाढेलच; पण त्यापेक्षा मानवी क्षमतांचा ही विस्तार होईल. ज्यांची उत्पादनक्षमता जास्त त्यांच्याकडे पैशाचा ओघ नक्कीच वाढणार. अशा पद्धतीने काही मोजक्या लोकांच्या-देशांच्या हातात 'एआय'ची शक्ती एकवटली तर त्याचा परिणाम आर्थिक दरी वाढण्यात होणार. अशी ही कारण-परिणामांची मालिकाच सुरु होण्याची शक्यता आहे.

'एआय' तंत्रज्ञान आत्मसात करण्यावाचून आता गत्यंतर नाही. हे तंत्रज्ञान विद्यार्थ्यांना, तरुणांना शिकणे एकवेळ शक्य होईल; पण ज्येष्ठांचे तसेच वरिष्ठ कर्मचाऱ्यांचे काय? त्यांनाही नवीन कौशल्ये शिकावी लागणार, नवीन विषय शिकणायची तयारी ठेवावी लागणार.

'एआय'ची त्सुनामी रोखणे अवघड आहे, त्यामुळे त्यावर आरूढ होण्यासाठी काय काय करावे लागेल, त्यासाठी आपली किती तयारी आहे हे जोखण्यासाठी या अहवालात एक 'कृत्रिम बुद्धिमत्ता तयारी मानक' (एआय प्रिपेअर्डनेस इंडेक्स) सुचवले आहे. त्यात पुढील चार भाग येतात. 'डिजिटल पायाभूत सुविधा' मध्ये माहिती (डेटा) साठवण, इंटरनेट प्रसार व त्यावर आधारित संगणक प्रणालींचा विकास यासारख्या गोष्टी येतात.

'मानवी भांडवल-कामगार धोरणे' यात हितसंबंध संरक्षण, एआय प्रशिक्षण अशा गोष्टी येतात. 'नवसंशोधन-एकत्रीकरण (इनोव्हेशन-इंटिग्रेशन)' यात नवीन कल्पनांना वाव-प्रोत्साहन देणे, बहुविध प्रणालींचा संवाद घडवून आणणे यांसारखे विषय येतात तर 'नियमन-नैतिकता' यामध्ये माहितीचा भेदभावरहित वापर, वैयक्तिक माहितीची जपणूक, वैधानिक तत्त्वे, पारदर्शकता इत्यादी विषय येतात. या सर्व पैलूंवर आपण कोठे आहोत हे तपासून, पुढील प्रगतीसाठी आपल्याकडे काय आराखडा (रोडमॅप) आहे, हे महत्त्वाचे असेल. त्यावरच आपण 'एआय'ला कसे तोंड देणार हे ठरेल.

एकंदरीत पाहता, आंतरराष्ट्रीय नाणेनिधीचा हा अहवाल जागतिक भविष्यावर 'एआय'च्या परिणामांचे एक सूचक चित्र रेखाटतो आहे. आर्थिक विकासाला चालना देण्यासाठी, अनेक नोकऱ्या-व्यवसाय सुधारण्यासाठी सर्वांचा फायदा होईल आणि असमानता कमी होईल, यासाठी आपली तयारी काय आहे हे तपासण्याची एक पद्धतदेखील या अहवालाने सुचवली आहे. त्याचा किंवा तत्सम मापदंडांचा आधार घेऊन, सजगतेने नियोजन करून आपल्या सर्वांना भविष्य घडवावे लागेल.

