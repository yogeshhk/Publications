\chapter{'कृत्रिम बुद्धिमत्ता' म्हणजे काय?}

२०११ची अमेरिकेतील 'जेपर्डी' नावाची प्रश्नमंजुषा स्पर्धा खरीच अनोखी होती. त्यात भाग घेणाऱ्या ३ स्पर्धकांपैकी एक 'व्यक्ती' बिलकुल वेगळी होती. आश्चर्याची गोष्ट म्हणजे तीच 'व्यक्ती' ही मानाची आणि मोठ्या पुरस्काराची स्पर्धा जिंकली. तसे पाहता, बाकीचे दोघे खेळाडू काही लेचेपेचे नव्हते. त्यातील एकाची एकामागून एक ७२ वेळा अपराजित राहण्याची अखंडित परंपरा होती, तर दुसऱ्याने आतापर्यंतचे सगळ्यात मोठे बक्षीस कमावले होते. पण या दोघा दिग्गजांना हरवणाऱ्या 'व्यक्ती'चे वेगळेपण म्हणजे, ती कोणी मानव नसून, ती एक संगणक प्रणाली (आज्ञावली, प्रोग्रॅम्स ) होती. आय-बी-एम वॉटसन   नावाच्या या प्रणालीने मानवाचा त्याच्याच स्पर्धेत पराभव केला होता!! या घटनेने संगणकात वापरल्या जाणाऱ्या अशा बुद्धिमान प्रणालीला, म्हणजेच त्यामागील 'आर्टिफिशियल इंटेलिजन्स' या विषयाला खऱ्या अर्थाने प्रकाशझोत आणले.

एखादे काम करताना मानवाला जिथे विशेष बुद्धीची गरज पडते, तेच काम जर संगणक प्रणाली करू लागली तर त्याला 'कृत्रिम बुद्धिमत्ता' (आर्टिफिशियल इंटेलिजन्स', एआय ) असे म्हणता येईल. एक्स-रे सारखे स्कॅन पाहून रोगनिदान करणे, भाषांतर करणे, अनेक आर्थिक व्यवहारातून गडबड (फ्रॉड ) शोधणे, तसेच बुद्धिबळ खेळणे, यासारख्या गोष्टी काही सोप्या नाहीत. त्यासाठी उत्तम बुद्धिमत्ता लागते. हीच कामे आता एआयसंगणक प्रणाली करू लागल्या आहेत आणि म्हणूनच या विषयाचे अप्रूप कुतूहल सर्वांना वाटायला लागले आहे. नक्की हे एआयप्रकरण काय आहे? एआयकसं काम करतं? त्यामागे काही अनाकलनीय, गूढ गोष्ट आहे का? या सर्वांची उत्सुकता जनमानसात निर्माण झाल्याचे जाणवते. त्याच्या निरसनासाठी हा लेखप्रपंच.

एआय हा विषय खरंतर तसा जुनाच. अगदी १९५०च्या दशकापासून सुरू झालेला. पण प्रामुख्याने संशोधन क्षेत्रापुरता मर्यादित राहिलेला. पूर्वी त्याला तज्ञ-प्रणाली (एक्स्पर्ट सिस्टिम्स) असेही म्हणत. त्या प्रणालीही थोड्याफार प्रमाणात मानवी बुद्धीप्रमाणे काम करीत असत. त्या प्रणाली ह्या विविध नियम वापरून (रुल -बेस्ड) बनलेल्या होत्या. समजा, 'उद्या पाऊस पडेल का' असे सांगणारी प्रणाली बनवायची असेल तर हवामानतज्ञ अनेक गोष्टींचा विचार करून, कोण-कोणत्या गोष्टी पाऊस पडणे ठरवू शकतात आणि त्यांचा किती प्रमाणात प्रभाव असतो, याचे संशोधन करीत, समीकरणे मांडत आणि त्याची आज्ञावली बनत असे. ढोबळमानाने अशाप्रकारे इतर कोणत्याही विषयातील तज्ञ-प्रणाली बनवण्याची पद्धत होती. पण काळ बदलला आणि पावसाचे भविष्य वर्तवण्यासारख्या प्रणाली अधिक गुंतागुंतीच्या होऊ लागल्या. साधी सोपी समीकरणे बरोबर उत्तर देईनारी झाली. त्याचबरोबर संगणकीय क्षमताही वाढत होत्या. कोणत्या परिस्थितीत, म्हणजेच तापमान, आर्द्रता, वाऱ्याचा वेग, इत्यादींच्या कोणत्या मापांना/संख्याना किती पाऊस पडतो याच्या बऱ्याच नोंदी, म्हणजेच माहिती (डेटा ) तयार होऊ लागली. संशोधनाने समीकरणे शोधण्यापेक्षा ह्या नोंदींचाच वापर करून आपल्याला एखादी प्रगत प्रणाली बनवता येईल का या विचारात, सध्या प्रचलित असलेल्या एआयप्रणालींचा उगम झाला असे म्हणावे लागेल. त्याला लागणारी संगणकीय क्षमता ही आता उपलब्ध आणि आवाक्यात आली होती. अशा, पूर्व माहितीवर (पास्ट डेटा) आधारित प्रणालींना यांत्रिक बुद्धिमत्ता (मशीन लर्निंग, एम-एल) असे म्हणतात. तो एआय चाच एक भाग आहे, कारण त्याच्याकडूनही, मानवी बुद्धीप्रमाणे काम होते, तसे काम केले जाऊ शकते. अशा एआयचे अनेक अविष्कार आता आपल्याला नेहमीच्या जीवनात सुद्धा पाहायला मिळू लागले आहेत.

काही इमेल प्रणाली आता 'एखाद्या संदेशाला काय उत्तर द्यायचे' हे अपोआप सुचवू लागल्या आहेत. हे काही सोपे काम नाही. कारण उत्तर देण्यासाठी तुम्हाला तो संदेश वाचावा लागतो आणि मग विचार करून उत्तर ठरवावे लागते. हे एआय कसे करीत असेल? त्याच्याकडे उत्तर देण्याची बुद्धिमत्ता कशी येते? ढोबळमानाने सांगायचे झाले तर, एआयप्रणाली हे करण्यासाठी तुम्हाला पूर्वी आलेले संदेश आणि त्याला तुम्ही पाठवलेली उत्तरे बघते. कशा प्रकारच्या संदेशांना कशा प्रकारची उत्तरे दिली गेली आहेत हे पाहून आकृतिबंध/आराखडे (पॅटर्न्स) बनवले जातात आणि मग, एखादा नवा संदेश आला की बनवलेल्या आराखड्यानुसार साजेसे उत्तर सुचवले जाते.

अशाच प्रकारची एक प्रणाली मोबाईल  मध्ये आपण टाईप करीत असताना पण आपल्याला दिसते. आपण जसे टाईप करीत जातो, तसे मोबाईल  पुढचा शब्द सुचवत जातो. तो सुचवलेला शब्द पूर्वी बऱ्याच वेळेला तुम्ही आता टाईप करीत असलेल्या शब्दाच्या पुढे वापरलेला असतो. त्यामुळे तो सुचवला जातो.

वाचताना हे सोपे वाटत असले तरी त्याच्या मागे अनेक गणितीय संकल्पनांचा व त्यातील तंत्रांचा, जसे की  संभाव्यता, सांख्यिकी कॅल्क्युलस, इत्यादी विषयांचा वापर,  पूर्व माहिती मधून पॅटर्न्स  शोधण्यासाठी केला जातो. यावरून एक गोष्ट निश्चित लक्षात येते की एआय प्रणाली तयार करण्यासाठी पूर्व माहिती ची नितांत आवश्यकता असते. त्यातूनच तो पूर्वी वारंवार वापरले गेलेले पॅटर्न्स  शोधत असतो. ही पूर्व माहितीची गरज, या प्रणालीची, तंत्राची, मर्यादा सुद्धा आहे. जर तुमच्याकडे तसा माहिती नसेल तर ढोबळमानाने म्हणता येईल की तुम्हाला एआय तंत्र लौकिक अर्थाने वापरता येणार नाही. म्हणजेच एखादी गोष्ट संगणक वाचू शकेल अशा स्वरूपात नसेल तर ती माहिती म्हणून साठवली जाऊ शकणार नाही व त्याचा एआयसाठी उपयोग पण करता येणार नाही. जसे की आपल्या मनात चालणारे विचार जोपर्यंत आपण स्वतः संगणकात टाईप करीत नाही तोपर्यंत त्याचा काही उपयोग नाही. संगणकात माहिती अंक स्वरूपात रूपांतरित केल्यावरच एआयकाम करू शकते. म्हणूनच एआय साठी 'कृत्रिम बुद्धिमत्ता' या कृत्रिम प्रति शब्दापेक्षा 'अंक-ज बुद्धी' (अंकांमधून जन्मलेली बुद्धी) हा शब्द सयुक्तिक वाटतो. असो.

नजीकच्या काळात सर्वव्यापी इंटरनेट-अंतरजालामुळे, उपकरणांमध्ये बसवलेल्या संवेदके (सेन्सर्स)  मुळे खूप माहिती उपलब्ध व्हायला लागला आहे. इतका की तुम्ही ठरवलं की हा माहिती पाहून एक एआय प्रणाली बनवावी, तर तो तुमच्या संगणकात पण मावणार पण नाही इतका. अशा माहितीच्या महापुराला बृहत माहिती असेही म्हणतात. अशाप्रकारे खूप माहिती मिळत असल्याने, संगणक अतिशय वेगवान झाल्याने, इंटरनेट वर माहिती ठेवायला जागा मिळत असल्याने आणि नव नवीन संगणक प्रणालींमुळे एआय वापरणे आता शक्य झाले आहे. यासाठी काही उत्तम व मुक्त-स्त्रोत (ओपन सोर्स ) संगणक प्रणाली सुद्धा उपलब्ध आहेत. वर उल्लेखलेल्या उदाहरणांसाठीच नाही तर इतर अनेक अवघड आणि गुंतागुंतीच्या प्रश्नांसाठी सुद्धा एआय चा वापर जोमाने होत आहे. काही सकारात्मक गोष्टींसाठी, तर काही नकारात्मक गोष्टींसाठी सुद्धा.

संगणक जर एवढा 'हुशार' होऊ शकत असेल तर माझे काय होणार? माझ्या नोकरी-व्यवसायाचे काय होणार? हा प्रश्न पडणे स्वाभाविकच आहे. याचे उत्तर म्हणजे एआय चा आपल्या जीवनावर थोड्याफार प्रमाणात प्रभाव अगदी जरूर पडणार आहे. अगदी उपजीविकेवर सुद्धा. जे सृजनशीलतेची, जसे कथा लिहिणे, चित्रकला, नव-संशोधन अशी कामे करतात त्यांना धोका फारच कमी. उलट त्यांचे क्षेत्र एआय च्या मदतीने अजून बहरेल. पण ज्यांच्या कामात तोचतोचपणा आहे, ज्याला आपण बौद्धिक 'पाट्या टाकणे' असे म्हणतो, त्यांना मात्र पुढे कमी संधी उरणार आहे. ती क्षेत्रे एआय हळूहळू काबीज करेल असे वाटते.

यात आपली भूमिका काय असली पाहिजे? विद्यार्थी असाल तर या विषयाची सखोल माहिती घ्यायला हवी. आपल्या क्षेत्राला पूरक असा एआय चा वापर पण करायला शिकले पाहिजे. सध्या याविषयी अनेक उत्तम कोर्सेस (काही अगदी फुकट  सुद्धा) इंटरनेट व इतरत्र उपलब्ध आहेत. त्याचा उपयोग करून ही अजून एक, नव-प्रचलित होऊ घातलेली, प्रश्न निवारण पद्धती  शिकून घेतली पाहिजे. नोकरी-व्यवसायातील मंडळींनी सुद्धा शक्य तितके या विषयात पारंगत होण्याचा प्रयत्न करावयास हवा. काळाची ती गरजच व्हायला लागली आहे. इतरांनी, अगदी खोलात शिरणे शक्य नसले तरी याविषयाची जुजबी माहिती तरी घ्यावी, जेणेकरून एआय या क्रांतीबरोबर आपण समजून-उमजून मागाक्रमणा करू.



