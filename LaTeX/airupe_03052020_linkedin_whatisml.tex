\chapter{`यंत्र बुद्धिमत्ता' म्हणजे काय?}
कोणती क्रिकेट टीम जिंकणार? कोण पंतप्रधान होणार?  अशा प्रश्नांवर आपल्यातले बरेच (आपापल्या परीने!!) भाकिते करीत असतात. पूर्वी पाहिलेले सामने अथवा निवडणुकांवरून आपल्या डोक्यात काही समीकरणे आपोआप फिट्ट झालेली असतात. विराट खेळणार असेल तर, समोरचा संघ लेचापेचा असेल तर, खेळपट्टी फिरकीला साथ देणारी असेल तर, अशा काही बाबी (डेटा) लक्षात घेऊन आपण ही भविष्यवाणी करतो. ज्याचे भाकीत खरे त्याला आपण या विषयातील तज्ञ अथवा बुद्धीमान पण म्हणतो. हेच डेटा-आधारित  काम जर यंत्र (म्हणजेच संगणक आज्ञावली, कंप्यूटर प्रोग्राम) करू लागले तर त्याला `यंत्र बुद्धिमत्ता' (`मशीन लर्निंग', अथवा `एम-एल') असे म्हणू शकतो.

मशीन लर्निंगला पूर्वीची माहिती, उदाहरणे, जसे की क्रिकेटच्या सामन्यांची आकडेवारी (डेटा) देऊन प्रशिक्षित (ट्रेन) करावे लागते. तो दिलेल्या डेटामध्ये समीकरणे अथवा आकृतिबंध (पॅटर्न्स ) शोधतो. एकदा का समीकरण शोधले की भविष्यवाणी म्हणजे त्या समीकरणात आकडे घालून उत्तर काढणे एवढेच राहते. तज्ञ किंवा ज्येष्ठ लोकांचे उत्तर अचूक येण्याची शक्यता जास्त असते कारण त्यांनी अनेक `पावसाळे' (जसे की सामने) पहिले असतात, हार-जीत कशामुळे होते यासंदर्भातील जास्त बाबी त्यांना माहीत असतात, म्हणून त्यांचे अंदाज जास्त बरोबर येऊ शकतात. तसेच मशीन लर्निंगला सुद्धा खूप डेटा, अनेक बाबींची आकडेवारी दिली तर त्याचेही अंदाज बरोबर येऊ शकतात.

हे नीट समजावून घेण्यासाठी एक सोपे उदाहरण बघू या. समजा तुम्हाला घर घ्यायचे आहे. कुठल्या भागात घ्यायचे आहे ते ठरले आहे. किमतीचा अंदाज घेण्यासाठी तुम्ही एखाद्या तज्ज्ञाला (रियल इस्टेट एजंटला) गाठता, मग त्यांना विचारले की ६०० चौ. फूट क्षेत्रफळाच्या १  बीएचकेच्या घराची किंमत किती असेल? १००० चौ. फूट क्षेत्रफळाचा २ बीएचके कितीला असतो? अशा प्रश्नांची उत्तरे धडधड मिळतात, कारण त्यांच्या डोक्यात ती समीकरणे, पूर्वी पाहिलेल्या उदाहरणांवरून, तयार झालेली असतात. क्षेत्रफळाचा (एरिया, $A$) चा आणि खोल्यांच्या आकड्याचा (रूम्स, $R$) किती प्रमाणात किमतीवर (किंमत, $P$) वर पडतो ते अनुभवातून ठरले गेलेले असते. उदाहरणादाखल एक सोपे समीकरण मांडूया: $P = W_0 + W_1 * A + W_2 * R$. यात $W_1$ हा क्षेत्रफळाचा ($A$) प्रभाव दर्शवतो, तर $W_2$ हा खोल्यांच्या ($R$) आकड्यांचा, तर $W_0$ ला मूळ (बेस) किंमत म्हणू शकतो. या सर्वांची बेरीज म्हणून घराच्या किंमतीचे($P$) ढोबळ समीकरण बनू शकते. $W_0$, $W_1 $आणि $W_2$ या प्रभावांना  भार/वजन  (वेट) म्हणतात. जेवढे वेट जास्त तेवढा प्रभाव जास्त. तज्ज्ञांच्या डोक्यात या वेटचे आकडे तयार झाले की ते कोणत्याही क्षेत्रफळासाठी आणि खोल्यांच्या आकड्यासाठी आपल्याला घराची किंमत सांगू शकतात. हेच काम मशीन लर्निंग आपोआप करू शकते. म्हणजे, त्यालाही पूर्वीच्या घरांची आकडेवारी दिली तर तोही असे समीकरण आपोआप शोधू शकतो. वर पाहिल्याप्रमाणे, तो ही वेटची किंमत काढू शकतो. कशी, ते ढोबळमानाने पाहू. समजा खालीलप्रमाणे आपल्याकडे माहिती उपलब्ध आहे:

\begin{table}[h!]
\centering
\begin{tabular}{|c|c|c|}
\hline
\textit{क्षेत्रफळ (A)} & \textit{खोली (R)} & \textit{घराची किंमत (P)} \\
\hline
८०० & ३ & ₹३३,००,००० \\
१२०० & ३ & ₹३६,१०,००० \\
७०० & २ & ₹२३,१०,००० \\
१५०० & ४ & ₹५४,००,००० \\
\hline
\end{tabular}
% \caption{घरांच्या क्षेत्रफळ, खोली व किंमतीचा तक्ता}
% \label{tab:ghar_kimmat}
\end{table}


वरील आकडे पाहून आपल्याला असे $W_0$, $W_1$ आणि $W_2$ शोधायचे आहेत जे वर दाखवलेल्या समीकरणाला आणि घरांच्या उदाहरणाला बऱ्यापैकी लागू पडतील. कसे शोधायचे याची एक साधी पद्धत अशी की , $W_0$, $W_1$ आणि $W_2$ यांची किंमत काहीतरी धरून सुरुवात करायची, समजा `१' धरली. तर समीकरण काय झालं: $P = 1+ 1 *A + 1 *R$. आपल्याकडे असलेल्या घरांचे आकडे त्यात घातले, जसे पहिले घर, तर किंमत काय येईल? $P = 1+ 1 * 800 + 1*3 = 804$ . अरेच्चा, पण या घराची किंमत तर ३३ लाख आहे. उत्तर चुकतंय, कारण आपण वापरलेले वेट `१' आहेत, जे बरोबर नाहीत. ते बदलावे लागणार हे साहजिकच आहे. पण कसे ते पाहू. वेट `१-१-१', घेतल्याने सर्वच घरांच्या किमती चुकीच्या येणार आहेत. समीकरणाने काढलेली किमत आणि खरी किंमत यात बरीच तफावत असणार आहे. अश्या सर्व घरांची तफावत एकत्र केली (तफावतीच्या आकड्याचा वर्ग करणे संयुक्तिक, ज्याने उणे आकड्यांची काळजी घेतली जाते) तर वेट `१-१-१' घेतल्याने आलेल्या एकत्रित चुकीचा आकडा मिळतो. तो आपल्याला कमीत कमी करायचा आहे. वेटचे आकडे बदलून हे साध्य केले जाते. त्यासाठी `ग्रेडीयंट डीसेंट' नावाच्या पद्धतीचा वापर करून, असे वेट शोधले जातात की ज्याने एकत्रित चूक कमीत कमी होईल. हे चांगले वेट कळाले, की समीकरण तयार!! जसे $P = -678167 + 1272 * A + 1031111 * R$. हे वापरून आपण कोणतेही क्षेत्रफळ ($A$) दिले आणि खोल्यांचा आकडा ($R$) दिला की  घराची किंमत सांगू शकतो, अगदी मानवी तज्ज्ञासारखी.

हे समीकरण सध्या ठीकठाक उत्तरे देईल. अजून काही घरांचे आकडे  घातले  आणि इतर बाबी, ज्या घरांच्या किमती ठरवतात, जसे की बांधकाम व्यावसायिकाची पत, सुविधा, शाळा महाविद्यालयांपासूनचे अंतर, इत्यादी, त्या समीकरणात आणल्या तर उत्तर अधिक अचूक यायला लागेल. हे सर्व पाहून आपल्याला असे लक्षात येईल की, फक्त डेटा देऊन मशीन लर्निंगने त्यातून समीकरण-आकृतिबंध शोधला आहे. हीच याची ताकद. आपण पाहिलेल्या या प्रकाराला `लिनियर रेग्रेशन' असे म्हणतात. हे `मशीन लर्निंग' मधील एक प्रसिद्ध तंत्र आहे. मशीन लर्निंग या विषयाचा बोलबाला आता सुरू झाला असला तरी यातील अनेक तंत्रे खूप जुनी आहेत, सांख्यिकी शास्त्राचा भाग म्हणून.

मशीन लर्निंग हा कृत्रिम बुद्धिमतेचाच (एआय) एक भाग आहे. कारण, जसे मानवी तज्ज्ञ बुद्धी वापरून, अनुभवातून समीकरणे शोधतात तसेच मशीन लर्निंगसुद्धा करते, आणि ते ही कोठलीही स्पष्ट आज्ञावली न देता.

मशीन लर्निंगचे अनेक उपयोग आहेत. अंदाज देणे, जसे की  घराच्या किमती, शेयरबाजारातील भाव, इत्यादी. वर्गीकरण करणे, जसे आपल्याला जीमेलमध्ये किंवा याहू इमेलमध्ये दिसते की आपल्याला आलेल्या संदेशांचे स्पॅम (कचरा संदेश) आणि वाचावे असे संदेश याचे वर्गीकरण करून, फक्त चांगले तेच तो आपल्याला दाखवतो. आपली उत्पादने विशिष्ट प्रकारच्या ग्राहकांनाच डोळ्यासमोर ठेवून केली असतील तर तसे गट (क्लस्टर) शोधण्याचे कामसुद्धा मशीन लर्निंग करते.

मशीन लर्निंगच्या अद्भुत क्षमतेमुळे त्याला अनन्यसाधारण महत्त्व आले आहे. त्याचे उपयोग विविध क्षेत्रात वाढत आहेत. एका मार्केट रिसर्च कंपनीच्या म्हणण्यानुसार या विषयाच्या कामांची किंमत २०२२ पर्यंत ६० ते ७० हजार कोटी रुपये इतकी होणार आहे. या विषयाच्या लोकप्रियतेमुळे, कंपन्यांकडे असलेल्या अनेक प्रोजेक्टसमुळे, या विषयातील तज्ज्ञांची खूप मोठी गरज निर्माण झाली आहे. अनेक इंजिनीअरिंग आणि इतर महाविद्यालयांत, तसेच कंपन्यांमध्ये मशीन लर्निंग शिकवणे सुरू झाले आहे. अनेक चांगले कोर्सेस युट्युबवर फुकट उपलब्ध आहेत. याचा आपण जरूर लाभ घेऊ या आणि मशीन लर्निंग हे तंत्र आपल्या विषयात कसे वापरता येईल व जास्तीत जास्त अवघड आणि क्लिष्ट प्रॉब्लेम कसे सोडवता येतील याचा विचार करू या!!