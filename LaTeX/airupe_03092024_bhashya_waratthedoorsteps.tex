\chapter{युद्ध येई आपुल्या दारी}

{\textbf{सत्ता काबीज करण्यासाठी एका देशाने दुसऱ्या देशाशी केलेले युद्ध हे पूर्वापारपासून चालू आहे. देशांतर्गत सत्ताबदलासाठी आंदोलने पेटवून गृहयुद्ध घडवणे पण आता जोर धरू लागले आहे, तेही परकीय महसत्तांच्या इशाऱ्यावर. भारतासमोरील अशा आव्हानांचा, त्यांच्या आधुनिक तंत्रांचा आणि संभाव्य उपायांचा या लेखात उहापोह केला आहे.}}

आपल्या शेजारील राष्ट्र काय किंवा युरोपात काय, जगात अशांतता ही वाढतानाच दिसत आहे. या सर्वांच्या मागे काही जागतिक-अदृश्य हात असावेत ही चर्चा उघडपणे होत आहे. दोन सैन्यांमधील युद्ध आपण एक वेळ समजू शकतो पण जनतेच्याद्वारे उठाव करवून, लोकशाही मार्गाने आलेले सरकार उलथवून टाकणे हा प्रकार आपण सध्या आपल्या शेजारील राष्ट्रात बघत आहोत. ते एका प्रकारचे छुपे युद्धच आहे, तेही सैन्याशिवाय. असा सत्ताबदलाचा प्रकार काही नवीन नाही. वेगवेगळ्या प्रकारे आंदोलन करवून, जनतेत भ्रांती पसरवून, वेगवेगळ्या क्लुप्त्या लढवून, अनेक मार्गाने जागतिक महासत्ता हे घडवत आल्या आहेत. याचे लोण आपल्या देशात, आपल्या दारी पण येऊ शकते. याची अनेक पदचिन्हे उघडपणे आपल्याला दिसायला लागली आहेत म्हणूनच हा विषय समजणे महत्वाचे.

युद्ध म्हटले की आपल्या डोळ्यासमोर दोन सैन्य एकमेकांसमोर उभी ठाकली आहेत असे दृश्य येते. पण त्याचे ही अनेक प्रकार आहेत. खरेतर युद्धांचीपण उत्क्रांती झालेली आहे. त्यांचाही पिढ्या म्हणता येतील असे टप्पे आहेत. ते थोडक्यात पाहू.

पहिल्या पिढीतील युद्धे मानवी शक्तीवर अवलंबून असणारी होती. लाठ्या-काठ्यांची, ढाल-तलवारीची. दोन्ही सैन्ये एका ठिकाणी येऊन जो जास्त बलशाली व शूर त्याचा विजय अशी पद्धत होती, जसे कि रामायण, महाभारत यात वर्णन केलेली युद्धे.

दुसऱ्या पिढीतील युद्धे औद्योगिक क्रांतीनंतरची. बंदुका, मशिनगन, विमाने सारखी यांत्रिक साधनांची व शस्त्रांची. पहिल्या महायुद्धाला या प्रकारात मोडता येईल.

तिसऱ्या पिढीतली युद्धे जास्त यांत्रिक व बहू-आघाड्यांची होती. दुसऱ्या महायुद्धात हिटलरने ब्लिट्झक्रीग सारखा मार्ग वापरला ज्यात अनेक आघाड्यांवर एकाच वेळी युद्ध सुरु करायचे आणि शत्रूला विविध अंगाने नेस्तनाबूत करायचे.

येथपर्यंतची युद्धे ही प्रामुख्याने सैन्यांमध्ये होती आणि सर्व सामान्य जनतेला जरी नुकसान पोहोचत असले तरी त्यांचा लढण्यात सक्रिय सहभाग नव्हता. यापुढे मात्र ही परिस्थिती बदलली.

चौथ्या पिढीतील युद्धात सैन्य आणि जनता यातील फरक अस्पष्ट झाला. देशभक्तीच्या भावनेनें म्हणा किंवा धार्मिक कट्टरतेमुळे जनता पण युद्धात उतरली. घुसखोरी, अतिरेकी हल्ले हे अधिकृत सैनिक नसलेल्यांकडून होऊ लागले. या प्रकारच्या युद्धातून मात्र बऱ्याच वेळेला जागतिक महासत्तेला सुद्धा पळ काढावा लागला आहे.

पाचव्या पिढीतील युद्ध हे प्रामुख्याने विचारांच्या-मानसिकतेच्या पातळीवर केले जाते आणि त्याचा उद्रेक मात्र जमिनीवरील लढ्यात होतो. समाजातील विविध गटात भांडणे लावणे, धर्मांधतेला खतपाणी घालणे, भ्रष्टाचारातून राजकीय पक्षांना अथवा त्यांच्या नेत्यांना, शासकीय व्यवस्थेला, न्यायव्यवस्थेला, विद्यापीठांना, वृत्तमाध्यमांना विकत घेऊन त्यांच्या तर्फे आपला अजेंडा (स्वार्थ) साधून घेणे असली कामे केली जातात. एखाद्या देशाचे भवित्यव्य प्रामुख्याने मध्यमवर्गावर अवलंबून असल्याने त्यांना लक्ष्य केले जाते. गट-गटात भेद करून, समाजमाध्यमांच्या आधीन करून, मोबाईल-व्यसनाधीन करून मध्यमवर्गाला निरुत्साही आणि शक्तिहीन बनवले जाते. कोठलीही कौशल्ये आत्मसात करण्याची, कष्ट करण्याची शक्ती-सवय गमावल्याने पूर्ण पिढीच हतबल केली जाते. ही पद्धत सध्या जास्त वापरात दिसत आहे आणि ते चित्र आपल्याला आजूबाजूला दिसत आहेच.

यापुढे येऊ घातलेल्या काही अजून युद्ध-पिढ्या पण आहेत आणि त्यांच्या कारभाराच्या चाहूलखुणा आताच दिसायला लागल्या आहेत.

सहाव्या पिढीतील युद्धात सैन्याने जवळ येण्याची आवश्यकताच नाही. दुरूनच, स्वतः:ला काही नुकसान न करून घेता, आधुनिक तंत्रज्ञानाचा आणि कृत्रिम बुद्दीमत्तेचा वापर करून, स्वयंचलित पद्धतीने ते लढले जाते. शत्रू-जनतेचा समाज-माध्यमातील वापरावर नजर ठेवून, त्यांची ठिकाणे बघून त्यावर आपोआप क्षेपणास्त्र टाकली जाऊ शकतात. ड्रोन्सची टोळधाड आपल्यावर सोडण्यात येऊ शकते. गाझा आणि युक्रेन युद्धात याचा वापर सुरु झालेला दिसतो.

सातव्या पिढीतील युद्ध हे आभासी विश्वात जास्त लढले जाणार आहे. त्यात आंतरजालाच्या माध्यमातुन आक्रमण करून तुमच्या वित्तीय संस्था बंद पाडणे, विमान व बस सेवा बंद पाडणे, जल किंवा विद्युत सेवा ठप्प करणे असे मार्ग वापरून तुम्हाला हतबल केले जाते. प्रामुख्याने रक्तरहित पण प्रचंड नुकसान केले जाते. याचे एखाद-दुसरे का होईना नजीकच्या काळातील उदाहरण आपल्याला आठवत असेल.

‘बळी तो कान पिळी’ या तत्वानुसार काही जागतिक आणि आक्रमक महासत्तांमध्ये संपूर्ण जग आपल्या कह्यात खेचण्याची स्पर्धा चालू आहे. का तर, जगातील सर्व संसाधनांवर फक्त आपला अधिकार असावा या महत्वाकांक्षेतून. इतर छोट्या राष्ट्रांचे यात एकतर नुकसान तरी होते किंवा ते प्यादे बनून मांडलिक होतात, आपल्या काही शेजाऱ्यांप्रमाणे. भारत काही अशी आक्रमक संस्कृतीची महासत्ता नसल्याने आपली ठामपणे उभे राहण्याची काय तयारी आहे याचा विचार केला पाहिजे. प्रथम दर्शनी काही उपाय सुचत आहेत त्यातील काही वानगीदाखल देत आहे.

‘आत्मनिर्भर भारत’ हा एक गंभीर आणि आवश्यक विचार आहे. अन्न-सुरक्षा, ऊर्जा-सुरक्षा आणि वित्त-विदा (डेटा ) सुरक्षा हे यातील महत्वाचे घटक/टप्पे आहेत. या सारख्या मूलभूत गरजांवर आपल्याला इतर कोणावरही, अगदी मित्र राष्ट्रावरही अवलंबून राहता कामा नये. नाहीतर आपल्याला कठपुतळीप्रमाणे दुसऱ्यांच्या तालावर नाचावे लागेल.

सर्वांनाच, खास करून तरुण पिढीला, समाज-माध्यमांच्या व्यसनाधीनतेतून बाहेर काढणे. ‘डिजिटल वेलबीइंग’ सारख्या ऍपवर आपला मोबाईल वापर तपासून त्यावर मर्यादा आणणे. जरा जालीम पण एक उपाय असा आहे की, मोबाईल-डेटा कंपन्यांनी ठराविक मर्यादेनंतरच्या डेटा वापरायला अतिशय महाग करून टाकणे, म्हणजे लोक तासंतास रिल्स किंवा व्हिडीओज बघत बसणार नाहीत. इमेल किंवा संदेश बघणे-पाठवणे यांना खूप काही डेटा लागत नसल्याने ती महत्वाची कामे पण थांबणार नाहीत. अजून एक उपाय म्हणजे, शाळेतील मुलांना तर डब्बा फोनच द्यायला पाहिजे. सखोल विचारांचे काम करायला, अभ्यासाला सलग व निर्विघ्न वेळेची गरज असते.

मध्यमवर्ग सक्षम करणे. शैक्षणिक सुधारणा अजून प्रभावी करून, पदवीपेक्षा कौशल्यांवर जास्त भर देणे सुरु केले पाहिजे. सुलभ अर्थ पुरवठा करून लघु-मध्यम उदयोगांना जास्त प्रोत्साहन द्यायला हवे, नोकरीच्या मागे लागण्यापेक्षा व्यवसाय उभारण्यास मदत केली पाहिजे. खास करून सरकारी नोकऱ्यांच्या स्पर्धा परीक्षांच्या मागे मर्यादेबाहेर मागे लागून आयुष्याची उमेदीची वर्षे वाया घालवण्यासारखेच आहे, कारण त्यात बहुतांश लोकांना यश मिळणे केवळ दुरापास्त आहे.

भारताच्या एकात्मतेला धक्का देणाऱ्या आणि अंतर्गत गृह-कलह घडवू पाहणाऱ्या सर्व शक्तींना नाकारले पाहिजे, अगदी कोठल्याही विचारधारेच्या असल्या तरी. ‘केरोसीन छिडकले आहे’, ‘रक्ताचे पाट वाहतील’, ‘हे राज्य जसे जळते आहे तशी इतर राज्ये पण जाळू’ ही कसली भाषा? अशा घोषणांना अजिबात समर्थन द्यायला नाही पाहिजे. जे विधायक आणि विकासाचा सक्षम पर्याय समोर ठेवतील त्यांनाच जनतेने पाठिंबा द्यायला पाहिजे नाहीतर आपणच आपल्या पायावर कुऱ्हाड मारून घेतल्या सारखे होईल.

‘युद्धस्य कथा’ कधीही रम्य नसतात, खासकरून जेंव्हा ते आपल्या दारात उभे ठाकलेले असते. आधी उद्धृत केलेले धोके जर आपल्याला कल्पनाविलास वाटत असतील तरी ठीक आहे, पण ते होतील अशी शक्यता विचारात घेऊन सुरक्षेची तयारी तरी करायला काय हरकत आहे?