\chapter{युद्धभूमीवरील 'कृत्रिम प्रज्ञा'}

संरक्षण क्षेत्र आणि तंत्रज्ञान यांची पूर्वापार घट्ट मैत्री आहे. नवनवीन शोधांचा जसा सकारात्मक-सदुपयोग केला जातो, तसाच त्याचा उपयोग संरक्षणात, युद्धात व संहारक्षमतेत वाढ करण्यासाठीही केला जातो. कृत्रिम बुद्धिमत्ता (आर्टिफिशिअल इंटेलिजन्स, एआय) त्याला कसा अपवाद असेल? देशाच्या पातळीवर वेगवेगळ्या धोक्यांपासून संरक्षण करावे लागते. ते अंतर्गत स्वकीयांकडून किंवा बाहेरील शत्रूंपासून असू शकते. ते जमिनीवर, पाण्यात, हवेत तर करावेच लागते; पण आधुनिक युगात ते आभासी (व्हर्च्युअल, सायबर) विश्वातील धोक्यांपासूनही करावे लागते. कल्पना करा की आपले ड्रोन युद्धभूमीचे निरीक्षण करून देत आहेत, यंत्रमानवी-सैनिक गस्त घालीत आहेत, क्षेपणास्त्रे त्यांना दाखवलेल्या लक्षाचा वेध अचूकपणे घेत आहेत. अशा सर्व कामांमध्ये एआयचा प्रभावी वापर केला जाऊ शकतो, आणि मोठ्या प्रमाणात तो सुरूही झाला आहे. मानवी क्षतीची जोखीम कमी करणे, अचूकता वाढवणे, जेवढे युद्ध लांबून लढता येईल तेवढे पाहणे ' हे सर्व एआयमुळे साध्य होत आहे. युद्धनीतीचा हा सारीपाट आता एआयच्या मदतीने अधिक सखोल, सर्वदूर, विविधांगी आणि कमीत-कमी नुकसान करणाऱ्या धोरणांचा क्रीडामंच झाला आहे. कोणत्या पद्धतीने युद्ध-संरक्षण विषयक कामांमध्ये एआय वापरले जाऊ शकते, ते पाहू.

युद्धमोहिमेच्या आखणीसाठी, उपग्रहाद्वारे गोळा केलेल्या माहितीचा उपयोग करून, आक्रमण करण्याच्या वाटा कोणत्या, रसद पुरवण्याचे रस्ते कोणते, सामुग्री-माणसे किती लागतील, असे सर्व ठरवता येते, गुप्तहेर न पाठवता. जवळच्या अंतरांसाठी ड्रोन्स ही कामे करतात. भूभागाचे प्रारूप (मॉडेल) बनवून त्याद्वारे आक्रमण कोठून व कसे करायचे याची रंगीत तालीमही करता येते. आभासी-संवर्धित वास्तव (व्हर्च्युअल-ऑगमेंटेड रिऍलिटी, एआर-व्हीआर) या तंत्रज्ञानाने हे शक्य झाले आहे. जर-तरच्या गोष्टी तपासून पाहता येतात. नवीन सैनिकांना प्रशिक्षण देण्यासाठीसुद्धा अशा मॉडेल्सचा उपयोग केला जातो.

नवीन युद्धसामुग्री कशी वापरायची, विमान कसे उडवायचे याच्या प्रशिक्षणातसुद्धा एआर-व्हीआरचा वापर होतो. सीमेवर आपले जवान डोळ्यात तेल घालून पहारा देत असतात. आपल्या देशाची शेकडो किलोमीटरची सीमा पाहता थोड्या-थोड्या अंतरावर जवान ठेवणे आपल्याला शक्य नाही. येथे त्यांच्या मदतीला एआय-आधारित कॅमेरे येऊ शकतात. दिवस काय, रात्रीसुद्धा घडणारी संशयास्पद हालचाल टिपून ते सतर्क करू शकतात. सियाचिनच्या बोचऱ्या थंडीत असो की कच्छच्या रणरणत्या उन्हात, दोन्ही ठिकाणी हे प्रभावीपणे काम करू शकतात. अरब स्प्रिंग'सारख्या उठावांमध्ये समाजमाध्यमांचा (सोशल मीडिया) खूप मोठ्या प्रमाणात वापर झाला. देशांतर्गत होणाऱ्या दंगलींमध्येही अफवा व दिशाभूल करणारी माहिती वणव्यासारखी पसरवण्यात त्यांचा हात दिसून येतो. एआयचा वापर करून समाजमाध्यमांवर होणाऱ्या चर्चांवर लक्ष ठेवता येते. काही अनुचित वाटल्यास लगेच ठोस उपाययोजनाही करता येते. अंतरजालाद्वारे (इंटरनेट) केवळ माणसेच जोडलेली नाहीत, तर यंत्रेही जोडलेली असतात. अशा नेटवर्क (जाल) वर आभासी हल्ला करणे शक्य असते. काही दिवसांपूर्वी आपल्याकडे काही महत्त्वाच्या भागांचा वीजपुरवठा खंडित झाला होता. त्यामागे आपल्या शेजारील शत्रूराष्ट्राचा हात असावा, अशी चर्चा झाली. अशा प्रकारचे सायबर हल्ले व्यक्तीला लक्ष्य करूनही केले जाऊ शकतात. बँकेतील पैसे गायब करणे यांसारखे गुन्हे, पारंपरिक युद्धाचे नसले तरी आभासी दुनियेत संपूर्ण समाजाला हतबल करू शकतात. आपल्याला आश्चर्य वाटेल, पण विविध आकर्षक ऍप्सद्वारे तुम्हाला मोबाईल-व्यसनी करणे हा सुद्धा एका युद्धाचा भाग असू शकतो.

भारतही संरक्षण क्षेत्रात एआयचा वापर वाढवत आहे. त्यासाठी नीती आयोग आणि संरक्षण मंत्रालयाने २०१८ मध्येच आखणी केली आहे. मार्गदर्शनासाठी डिफेन्स एआय कौन्सिल (राष्ट्रीय एआय रक्षा परिषद) स्थापन झाली आहे. 'अग्नी-डी', 'डीआरडीओ तरुण शास्त्रज्ञ प्रयोगशाळा' यांसारखे उपक्रम सुरू झाले आहेत.

एआय जरी बहुतेक वेळा प्रभावी काम करत असला, तरी त्याच्याकडून चुका होऊ शकतात. मग प्रश्न उभा राहतो की, त्या चुकीची जबाबदारी कोणाची? शत्रूऐवजी स्वकीयच मारला गेला, तर? माणसे व यंत्रे जोडल्याने फायदे जरी अनेक असले, तरी सायबर हल्ला तुमच्या खिशापर्यंत (इस्रायलने नुकताच घडवलेला पेजर हल्ला आठवतोय?) पोहोचतो आहे. यावर मार्ग काय? असे एक ना अनेक प्रश्न उभे राहतात. एआयचा वापर समजून-उमजून, सजगपणेच केला पाहिजे.