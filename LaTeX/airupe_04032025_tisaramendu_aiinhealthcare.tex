\chapter{माणसाला गवसलेली नवसंजीवनी}

आजही जगातील अर्ध्याहून अधिक लोकांना वैद्यकीय उपचार सहजपणे उपलब्ध नाहीत. विशेषतः ग्रामीण भागातील दवाखान्यांपुढील रांगा आणि रुग्णालयातील गर्दी असे विदारक चित्र दिसते. भारतात तर लोकसंख्येसाठी पुरेसे डॉक्टर्सच नाहीत. याचे कारण म्हणजे सरकारी वैद्यकीय शिक्षणाच्या संधी अपुऱ्या असून खाजगी शिक्षण सर्वसामान्यांच्या आवाक्याबाहेर गेले आहे. या दुर्व्यवस्थेला काही उपाय आहे का? होय, आहे. एक आशेचा किरण म्हणजे एआय (आर्टिफिशिअल इंटेलिजन्स ' कृत्रिम बुद्धिमत्ता). मागणी आणि पुरवठ्यातील दरी भरून काढण्यासाठी एआय विविध रूपांत आणि प्रक्रियांमध्ये कसे उपयोगी पडते ते पाहूया.

आपल्याला बरे वाटत नसेल किंवा काही आजार असेल तर पहिली पायरी असते रोगनिदान. निदान जितके अचूक, तितकी पुढील उपचारयोजना प्रभावी ठरते. एक्स-रे चित्रांवरून ( स्कॅन) एआय अचूकपणे रोग ओळखू लागले आहे. हाड मोडलेले ठिकाण किंवा कर्करोगाच्या गाठी एआय सहज शोधू शकते आणि अनेक वेळा मानवी निदानापेक्षा अधिक अचूकतेने. कारण एआयने मानवी डॉक्टरच्या तुलनेत अनेक पटीने जास्त स्कॅन्स अभ्यासलेले असतात. काही आजार अनुवांशिक असतात आणि त्यांच्या शक्यता गुणसूत्रांच्या अभ्यासावरून ठरवता येतात.

मधुमेह, हृदयरोग यांसारखे आजार आधीच ओळखण्यात एआय प्रभावी ठरते. ऍस्ट्राझेनेका या कंपनीने एआय-मशीन लर्निंग वापरून १००० हून अधिक रोग निदान करण्याचे प्रारूप (मॉडेल) बनवले आहे. व्यक्ती तितक्या प्रकृती असल्याने, जरी तोच आजार असलातरी उपाययोजना सरधोपटपणे तीच ती करून चालत नाही. रुग्णाचा समग्र अभ्यास करून औषधें ठरवणे सर्वात उत्तम. अशी वैयक्तिक उपचारयोजना एआयने शक्य होते. रुग्णाचा वैद्यकीय इतिहास, औषधोपचार, चाचण्या, कुटुंबाचा इतिहास, गुणसूत्रांची माहिती या सर्वांच्या आधारे एआय रुग्णाचे मॉडेल तयार करते आणि कर्करोगासारख्या आजारांसाठी योग्य उपचार सुचवू शकते.

शस्त्रक्रियेच्या क्षेत्रात, जेथे मानवी डॉक्टरचे परमोच्च कौशल्य पणास लागते तेथेही यंत्रमानवाच्या रूपाने एआयचा शिरकाव(!) झाला आहे. विशेषकरून गुडघ्याच्या शस्त्रक्रियेच्या तशा जाहिरातीही तुम्ही पहिल्या असतील . कमीत-कमी छेद घेऊन, आरोग्य घटकांवर सतत लक्ष ठेवून, जोखीम कमी करण्यात एआय मदत करते.

मोठ्या रुग्णायालात अनेक रुग्ण उपचार घेत असल्याने तेथील सर्व विभागांचे व्यवस्थापन मोठे जिकिरीचे काम असते. डॉक्टरांचे वेळापत्रक, मोठ्या-खर्चिक यंत्रांच्या वेळांचे नियोजन, आर्थिक व्यवहार, मामा-मावश्या इत्यादी सहकाऱ्यांचे नियोजन, अशा एक ना अनेक गोष्टी असतात. एकाच रुग्णाच्या विविध चाचण्या, त्याच्यावर विविध तज्ञ (स्पेशालिस्ट) त्यांच्या कार्यक्षेत्राप्रमाणे सुचवत असल्येल्या उपाययोजना हे सर्व सुसंबद्ध रीतीने साठवून आणि जरुर असलेली गोपनीयता पळून वापरावे लागते. या नियोजनास तंत्रज्ञान मदत तर करतेच पण साठवलेल्या डेटा चा आधार घेऊन भविष्यातील उपाय योजना सुचवता येतात.

ग्रामीण भागात मोठी रुग्णालये नसल्याने रुग्णांसाठी दूरस्थ निदान (रिमोट डायग्नोस्टिक) आणि औषधोपचार (टेलीमेडिसिन) उपयोगी पडतात. भारत सरकारच्या 'भारतनेट' योजनेमुळे गावोगावी वेगवान इंटरनेट पोहोचत आहे, ज्यामुळे वैद्यकीय सेवा सूदूर पोहोचवता येतील. संगणकीय आरोग्यसेवा (डिजिटल हेल्थकेअर) झपाट्याने वाढत आहे. यावर्षी या क्षेत्राचा जागतिक बाजार ५०० अब्ज डॉलर्सपर्यंत पोहोचेल असा अंदाज आहे. अगदी वैयक्तिक स्वरूपाच्या, ढोबळमानाच्या अंदाजांसाठी मोबाईल मधील ऍप्स सुध्या काही प्राथमिक निदानाच्या गोष्टी सांगू शकतात. मोठी वैद्यकीय उपकरणे एआयचा वापर करत आहेत. वैद्यकीय संभाषण प्रणाली (मेडिकल चॅटबॉट्स) रुग्णांच्या शंकानिरसनासाठी बाजारात येत आहेत.. चॅटजिपीटी सारखेच पण वैद्यकीय प्रश्नाचे तज्ञ अशी बृहत भाषा प्रारूपे (लार्ज लँग्वेज मॉडेल्स, एल-एल-एम्स) प्रशिक्षित होत आहेत. भारतीय भाषांमध्ये त्यांची प्रारूपे बनवणे हे आपल्यापुढी मोठे आव्हान (आणि फार मोठी संधी) आहे. त्यांचा वापर करून विविध आरोग्यसेवा निर्माण करणे हे नवउद्योगांना (स्टार्टअप्स) शक्य आहे. हे सारे क्षेत्र असे आहे की ज्यातील सेवांना-व्यवसायांना 'मरण' नाही. भारतातील डेटा उपलब्धतेवर आणि त्यावर प्रारूपे प्रशिक्षित करण्याचे तंत्रज्ञान आत्मसात करून या क्षेत्रात आपण मोठी भरारी मारली पाहिजे.

तंत्रज्ञान कितीही प्रगत झाले तरी मानवी डॉक्टरच्या आश्वासक बोलण्याला, कौशल्याला आणि अनुभवसिद्ध निदानाला पर्याय नाही. मात्र, जेथे डॉक्टर उपलब्ध नाहीत किंवा त्यांच्यावरचा ताण कमी करायचा आहे, तिथे एआय उपयुक्त ठरू शकते. त्यामुळे आपण 'एआय'चे स्वागत केले पाहिजे.