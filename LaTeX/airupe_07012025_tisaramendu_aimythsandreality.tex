\chapter{'कृत्रिम प्रज्ञे ' चे वास्तव आणि भ्रम}

मानवी बुद्धिमत्तेचे अनेक पैलू आहेत. ढोबळ मानाने पाहता, माहिती ग्रहण करणे, पृथ:करण करणे, त्यातील महत्वाच्या गोष्ट 'ज्ञान' म्हणून साठवणे आणि नंतर त्याचा उपयोग नवनवीन कामांसाठी करणे, हे सर्व त्यात समाविष्ट होते. सामान्य समजुतीनुसार, जेंव्हा संगणक प्रणालीसुद्धा अशी कामे थोड्याफार फरकाने करते, तेंव्हा तिला ‘कृत्रिम बुद्धिमत्ता’ (आर्टिफिशिअल इंटेलिजन्स, एआय, AI) असे म्हटले जाते. यातून मानवी आणि कृत्रिम बुद्धिमत्ता यांच्यात समानता असल्याचे गृहीत धरले जाते, ज्यामुळे बरेच भ्रम निर्माण होतात. या दोन्ही बुद्धिमत्तांचे प्रकटीकरण जरी समान दिसत असले तरी त्यामागच्या प्रक्रिया, गरजा आणि त्यांचा प्रभावीपणा हा खूप वेगळा आहे. त्यामुळे निर्माण होणारे भ्रम आणि त्यामागील सत्य स्पष्ट करणे महत्वाचे आहे.

एखाद्या लहान बाळाला एकदाच सांगितले की, “हे भू-भू आहे, ही मनी-माऊ आहे”, तर त्याला त्यानंतर ते प्राणी ओळखणे कधीही अवघड जात नाही, अगदी ते वेगळ्या प्रकारचे, जातीचे,आकाराचे व रंगांचे असले तरीही. ही प्रक्रिया आपल्या दैनंदिन जीवनात साधी वाटते, पण ती अत्यंत अद्भुत आहे. मानवी बुद्धिमत्तेला एखाद-दुसऱ्या उदाहरणातून ‘ज्ञान’ मिळते, जे त्याला पूर्वी न पाहिलेल्या गोष्टी ओळखण्यात मदत करते.

याउलट, कृत्रिम बुद्धिमत्तेला अशा कामांसाठी लाखो उदाहरणे (डेटा) जसे की प्राण्यांची चित्रे त्यांचा प्रकारांच्या नावांसकट द्यावी लागतात. मोठ्या संगणकीय शक्तीच्या मदतीने, महागडे प्रारूप (मॉडेल) तयार केले जाते. त्यानंतर प्राणी अगदी अचूक ओळखत नाही, तरी बऱ्यापैकी ओळखू शकते. मानवाला मात्र अगदी कमी उदाहरणे दिली तरी पुरतात. हा एक खूप महत्वाचा फरक आहे. 

अजस्त्र प्रमाणात डेटा देऊन मात्र आता काही कृत्रिम बुद्धिमत्तेची मॉडेल्स मानवी बुद्धिमत्तेशी स्पर्धा जरूर करू लागली आहेत (काही प्रकारात ती पुढेही आहेत) पण जेवढ्या कमी प्रशिक्षणात (डेटा आणि लागणारी ऊर्जा) मानवी बुद्धिमत्ता ते काम करते, ते कृत्रिम बुद्धिमत्तेला अजूनतरी शक्य झालेले नाही. 

चित्रांसोबतच आवाज, गंध अशा विविध माध्यमांचे ज्ञानही माणूस प्रभावीपणे साठवतो. त्यानंतर तत्सम आवाज किंवा गंध सहज ओळखतो. याला विविधांगी (मल्टी-मोडल) मॉडेल म्हणतात, आणि येथेही मानवी बुद्धिमत्ता सरस ठरते.

भाषा शिकताना सुद्धा मानवी मूल भोवताली पाहून, संबंध लावून अंदाजाने का होईना पटापट शिकते. त्याला ना व्याकरणाचे पुस्तक लागत ना प्रशिक्षक. ज्यांच्या सभोवती दोन-चार भाषा बोलल्या जातात तेंव्हा सुद्धा मानवी मूल त्या सर्व आत्मसात करते. हे सर्व करायला कृत्रिम बुद्धिमत्तेला प्रशिक्षणासाठी खूप मोठा डेटा, खूप जास्त प्रमाणात संगणकीय शक्ती आणि त्याच प्रमाणात पैसा लागतो. मानवी मेंदू मात्र हे कामही कमी उदाहरणात व ऊर्जेत करतो. सारांश हाच की  जरी दोन्ही प्रकारच्या बुद्धिमत्ता एकसारखी काम करीत असल्यामुळे आपल्याला भ्रम होऊ शकतो की त्या सारख्याच आहेत, तरी सत्य हे आहे की त्यांच्या मागील तत्वे, प्रशिक्षण आणि कार्यपद्धती पूर्णतः वेगळी आहेत. 

अजून एक मोठा भ्रम म्हणजे, जसे मानवी मेंदूत लाखो मज्जातंतू (न्यूरॉन्स) असतात, त्याच प्रमाणे कृत्रिम बुद्धिमत्तेच्या (एआयच्या) मॉडेलमथ्ये पण असतात. यावर कळस म्हणजे, कृत्रिम बुद्धिमत्तेमधील एका प्रणाली प्रकाराला “न्यूरल-नेटवर्क” म्हणजेच “मज्जातंतूंचे-जाळे” असे नावही दिलेले आहे. पण सत्य हे आहे की,  कृत्रिम बुद्धिमत्तेमधील न्यूरॉन्सचा आणि मानवी मेंदूतील न्यूरॉन्सचा अर्थाअर्थी काहीही संबंध नाही. एआयचे न्यूरॉन्स हे केवळ संकल्पनेच्या पातळीवर मानवी मज्जातंतू सारखे काम करतात पण दोघांच्याही मागील तत्वे व प्रक्रिया फार वेगळी असतात. 

मानवी बुद्धिमत्ता दृश्ये ओळखणे किंवा भाषा बोलणे यापलीकडे जाते. भावना, सृजनशीलता, विवेक अशा अनेक क्षेत्रांमध्ये ती कृत्रिम बुद्धिमत्तेपेक्षा सरस आहे. मग तरीही कृत्रिम बुद्दीमत्तेचा एवढा बोलबाला का? तर काही प्रकारच्या गोष्टींमध्ये कृत्रिम बुद्धिमत्ता खरंच सरस ठरते. मोठ्या प्रमाणावर डेटा साठवणे व त्यातून उत्तर शोधणे, जटिल गणिते सोडवणे, आणि अनेक घटक असलेल्या समस्यांमध्ये उत्तर शोधणे यात कृत्रिम बुद्धिमत्ता अधिक प्रभावी ठरते. शेवटी, मानवी आणि कृत्रिम बुद्धिमत्तेतील संघर्षापेक्षा, त्यांच्या बलस्थानांचा उपयोग करून घेणे अधिक महत्त्वाचे आहे.



