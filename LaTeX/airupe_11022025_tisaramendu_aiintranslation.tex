\chapter{बहुभाषी `कृत्रिम बुद्धिमत्ता'}

पंचवीस-एक वर्षांपूर्वी कामानिमित्त काही महिने जपानमध्ये राहण्याचा योग आला होता. जपानी भाषेचे अगदीच जुजबी ज्ञान असल्याने रोजच्या व्यवहारात फलक वाचताना, स्थानिकांशी बोलताना नुसती भंबेरी उडायची. तेंव्हा मोबाइल अत्यंत प्राथमिक अवस्थेत होते. भाषांतर करणारे, चित्रलिपी वरून अर्थ सांगणारे त्यात काहीही नसायचे. आता मात्र परिस्थिती एकदम बदलली आहे. दुभाष्याचे काम करणाऱ्या संगणक प्रणाल्या (ऍप्स) आता उपलब्ध आहेत. हा चमत्कार एआय मुळे शक्य झाला आहे.  अनादी काळापासून भाषा हा मानवी जीवनाचा अविभाज्य भाग आहे. विविध ठिकाणी असणाऱ्या वेगवेगळ्या संस्कृतींचा आधार तेथील भाषा आहेत. नाते एवढे अतूट की अगदी अस्मितेपर्यंत पोहोचते कधी कधी. सुदूर प्रवासाच्या तसेच संभाषणाच्या सोयी जशा जशा वाढल्या आणि विविध संस्कृतीच्या लोकांशी संपर्क वाढला तशी त्यांची भाषा येण्याची गरज निर्माण झाली. दुभाषांचे महत्व वाढले. बहुभाषी असणे तर विद्वत्तेचे लक्षण समजले जाऊ लागले. पण इच्छा असूनही नवीन भाषा शिकणे हे सर्वसामान्य लोकांना कठीण जाते. त्यावर उपाय म्हणून भाषांतर करणाऱ्या प्रणाल्या (अल्गोरिदम्स) बनवण्यावर संशोधन सुरु झाले.

भाषांतर करणे हे अजिबात सोपे काम नाही. मानवालासुद्धा ते क्लिष्ट वाटते तर संगणकाचे काय घेऊन बसलाय. प्रथम शब्दाला-शब्द, नंतर शब्द-समूहाचा विचार करून नंतर वाक्यच्या-वाक्य यांचा विचार करून भाषांतर करणाऱ्या प्रणाल्या निर्माण झाल्या. एआय'मधील `न्यूरल नेटवर्क्स' ने यात क्रांती घडवली. तेंव्हापासून भाषांतरातील कृत्रिमता कमी कमी होऊन मानवी बोलचालीप्रमाणे भाषांतर होऊ लागले आहे. जसा जसा वापर वाढेल तसे तसे नवीन नवीन भाषांचे आकृतिबंध (पॅटर्न्स) येतील, ते एआय च्या प्रशिक्षणात वापरले जातील तशा त्या प्रणाल्या, भाषा-प्रारूपे (लँग्वेज मॉडेल्स) अधिक प्रबुद्ध व प्रभावशाली होतील. पण संस्कृतीचे सर्व आयाम, मानवी भावना, प्रचलित वाक्प्रचार, पूर्वग्रह, इतिहासातील संदर्भ-वाक्ये, स्थानिक-बोली भाषेचे रंग, विशिष्ठ तांत्रिक  शब्द, हे सर्व एआयच्या प्रशिक्षण माहितीत (डेटा, विदा) मध्ये क्वचितच असल्याने, अजूनही तज्ञ मानवी दुभाषाला पर्याय नाही. भाषा साधी, सोपी, रोजच्या व्यवहारातील असेल तर एआय फार व्यवस्थित काम करताना दिसते.  

सध्या भाषांतराच्या ऍप्स मध्ये गुगल ट्रान्सलेट सारख्या प्रणाल्या १०० हुन अधिक भाषांमध्ये काम करू शकतात. त्यात बहुभाषी भाषांतर तर आहेच, पण चित्रातून लिपी ओळखून अर्थ काढणे, लिहिलेलेच नाही तर, बोललेलेपण भाषांतरित करणे, या गोष्टी ते लीलया करते. चॅटजिपीटी सारख्या संभाषण प्रणाल्या पण बहुभाषी होत आहेत. भारतीय भाषांसाठी अजून खूप प्रभावी नसले तरी,कामचलाऊ मात्र आहे. भारतातही याविषयीही चांगले काम चालू आहे. पंतप्रधान मोदी यांनी `काशी तामिळ संगमम' मध्ये केलेले हिंदीतील मनोगत, `भाषिणी' प्रणालीने तेथल्यातेथें तामिळ मधून ऐकवले.  भाषणांचे केवळ भाषांतरच नाही इतर काही ऍप्स वापरून तुम्ही चित्रफितींना अनुवादित वेगळ्या भाषेत उपशीर्षके (सबटाइटलींग) देणे, संवादाचे वेगळ्या भाषेत रूपांतर करणे (डबिंग), अशा अनेक गोष्टी करू शकता. एकाच मजकुरापासून त्याचे अनेक भाषांमध्ये रूपांतर स्वयंचलित पद्धतीने करता येत असल्याने श्रम, पैसे व वेळेची मोठी बचत होते. तुमचे कार्यक्षेत्र जरा हटके असले तर त्यासंदर्भात अजून उदाहरणे देऊन, शब्द-संग्रह देऊन, एआयच्या प्रणाल्या त्या त्या क्षेत्रासाठी अजून विशेष-समृद्ध करता येतात. या दुभाषी प्रणाल्यांचे असे असंख्य फायदे आहेत. कामाचा जबरदस्त वेग, किफायतशीर, चोवीसतास न थकता न विश्रांती घेता काम करणे आणि एकाच नाही तर अनेक भाषांमध्ये, अनेक कार्यक्षेत्रात काम करणे जमत असल्याने एआय आधारित भाषांतराला तोड नाही.

बदल होत असलेल्या जगात, नोकरी-धंद्यासाठी अथवा प्रवासासाठी, केवळ मराठी-हिंदी-इंग्रजी येऊन भागणार नाहीये. वेगवेगळ्या देशातून भारतीयांना होणारी मागणी पाहता जर्मन, जपानी, कोरियन, अरेबिक आणि हिब्रूसारख्या भाषा आपल्याला वापराव्या लागल्या तर काही नवल नाही. तर भांबावून न जाता, अगदी सुरवातीला का होईना, मोबाईल मधील भाषांतर करणाऱ्या एआयच्या प्रणाल्या तुमच्या नक्की कामास येतील. अगदी स्थानिकांसारखे बोलायचे असल्यास मात्र सध्यातरी तुम्हालाच त्या भाषा शिकाव्या लागतील, हे खरे. {\textit{``वाकारीमास का?''}} %(जपानीतील या शब्दाचा अर्थ 'समजले का?')