\chapter{'कृत्रिम बुद्धिमत्ता' तुमच्या घरात}

पूर्वी कुटुंबे मोठी होती. कामे आपापसात वाटून केली जायची. सध्या छोटी कुटुंबे प्रचलित असल्याने घरकामात मदतनीस-मावश्यांबरोबरच यंत्रांची सुद्धा सर्रास मदत घेतली जाते. ही यंत्रे नेमून दिलेलं काम इमानेइतबारे करत असली तरी कधी काय करायचं किंवा कसं करायचं असं काही डोकं लावत नाहीत. पण यातही बदल होत आहेत. या गोष्टी हुशार ('स्मार्ट') होत चालल्या आहेत. कपड्यांचा प्रकार, पोत (टेक्श्चर) बघून कपडे कसे व किती वेळ धुवायचे हे वॉशिंग मशीन ठरवू लागले आहे. याच्या मागे एआय (आर्टिफिशिअल इंटेलिजन्स, कृत्रिम बुद्धिमत्ता) असते. हेच नाही तर अशा अनेक प्रकारे एआय आपल्या घरात 'शिरले' आहे, त्यातील काही उदाहरणे पाहुयात. ऊर्जेचे (इंधन, वीज) भाव दिवसोंदिवस वाढतच आहेत. जशी जशी नवीन यंत्रांची गरज भासत आहे तशीच विजेची मागणी वाढत आहे. त्यामुळे वाढणाऱ्या खर्चाला आळा घालण्यासाठी म्हणजेच विविध उपकरणांचे ऊर्जा नियोजन करण्यात एआय मदत करते. 'स्मार्ट' मीटर आता विजेची बचत करणे, देखभाल करणे यांसारखी कामे करून आपले पैसे वाचवतात. या मीटर मध्ये, वीज-वापरासंबंधीचा 'डेटा' (माहिती, विदा) अभ्यास करून एआय मॉडेल्स निर्णय घेत असतात. सरकारसुद्धा आपल्या पातळीवर 'स्मार्ट' ग्रीडचा वापर करून वीज नियंत्रणात सुसूत्रता आणत आहेत.

'अलेक्सा', 'गुगल होम' सारखी संभाषण-यंत्रे तुम्ही दिलेल्या आवाजी-आज्ञा समजतात. एखादे गाणे लावणे, प्रश्नांना उत्तरे देणे, हवामानाचा अंदाज सांगणे, घरातील 'स्मार्ट' यंत्रे चालू-बंद करणे यांसारखी कामे ते करतात. याच्यामागे एआय मधील 'नैसर्गिक भाषा प्रक्रिया' (नॅचरल लँग्वेज प्रोसेसिंग, एन-एल-पी) आणि ध्वनीविषयक प्रणाल्यांचा वापर केला जातो. एकच क्रिया आपण वेगवेगळे शब्द वापरून सांगतो ते त्याला कळते. आता भारतीयांनी केलेले इंग्रजी उच्चारच नाही तर काही भारतीय भाषा सुद्धा त्यांना कळायला लागल्या आहेत. संध्याकाळी प्रकाशाचा अंदाज घेत एआय घरातील दिवे लागणीची वेळ अचूक पाळतो. सांजवेळेस पडदे लावून घेणेच नाही तर घरातील बागेतील झाडांना जमिनीतील आर्द्रता बघून यथोचित पाणी देणे पण. आपण घरी येण्याच्या वेळेचा अचूक अंदाज बांधून, एआय घरातील वातानुकूल-यंत्रणा (एअर कंडिशनिंग, ए-सी ) पाच-एक मिनिटे आधीच चालू करून, तुम्हाला आल्याआल्या थंडगार वाटेल अशी व्यवस्था करून ठेऊ शकतो. दिवसभरात नेहमीपेक्षा काही वेगळ्या अथवा विसंगत हालचाली कॅमेरात दिसल्या तर तुम्हाला सतर्क करू शकतो. केवळ घरचंच नाही तर सोसायटीच्या व शहराच्या पातळीवर पण सी-सी-टी-व्ही कॅमेरे हालचालींवर नजर ठेऊन असतात. अशा प्रकारे तुमचे राहणे जास्त सुखकारक आणि सुरक्षित करण्यास एआय मदत करते.

एआयमुळे करमणूक क्षेत्रातही मोठे बदल झाले आहेत. घरात प्रामुख्याने टीव्हीवर किंवा मोबाईलद्वारे आता मनोरंजनाचे कार्यक्रम बघितले जातात. लोकांच्या आवडीनुसार योग्य वेळेला योग्य तो कार्यक्रम सुचवणे एआयला जमते. सभोवतालच्या प्रकाशाचा व आवाजाचा अंदाज घेत 'स्मार्ट' टीव्ही आपले दृक-श्राव्य नियोजन करतात. मोबाइलवरतर करमणुकीच्या ऍप्स ची गणतीच नाही. त्यातही वैयक्तिक सूचना येतात. केवळ करमणूकच नाही तर शैक्षणिक अभ्यासात मदत करण्यात एआय प्रभावी भूमिका घेत आहे. झाडू मारण्यासारखे पाठ-मोडीचे काम आता यंत्रमानवी रोबो करतात. आपले आपण सर्व कानाकोपऱ्यात जाऊन स्वच्छता करतात. जमिनीचा-फरशीचा प्रकार बघून, गालिचाचा पोत बघून ते कसे साफ करायचे हे एआय ठरवू शकतो. हे सर्व काम सांगेल तेंव्हा करतो, न कंटाळता व न खाडा करता.

आपल्या स्वास्थासाठी व्यायामशाळेसारखी (जिम) उपकरणे, आरोग्य घटकांवर लक्ष ठेवणारी यंत्रे घरातच वापरणे सुरु झाले आहे. काही तर परिधानही (वेअरेबल) करता येतात. तुमच्या नाडीचे ठोके बघून, प्राणवायूची पातळी बघून, किंवा झोपेचे प्रकार (पॅटर्न्स) बघून योग्य तो सल्लाही दिला जातो. त्यामुळे घरच्याघरी तंदुरुस्तीच्या मार्गावर राहता येते. अजूनतरी खूप प्रचलित नसले तरी एआय-यंत्रमानव घरातील स्वयंपाकी झाले तर आश्चर्य वाटायला नको. काही रेस्टॉरंट मध्ये याची सुरुवात झाली आहे. घरातील फ्रिज मध्ये ज्या वस्तू दिसत आहेत त्यानुसार कोठले पदार्थ बनवणे शक्य आहे ते एआय सुचवू शकतो. काही गोष्टी कमी पडत असतील तर आपणहून मागवून घेऊ पण शकतो.

स्वप्नवत वाटत असलेतरी एआयच्या या फायद्यांसोबत काही गोष्टींची काळजी पण घ्यावी लागणार आहे. तुमच्या घरातील लोकांविषयी, त्यांचा राहणीमानाविषयी आणि एकंदरीतच आयुष्याविषयी या एआय-आधारित यंत्रांना इतके माहिती होणार आहे कि त्याचा गैरफायदा घेण्याची शक्यता पण निर्माण होऊ शकते. ज्या कंपन्या तुमच्या डेटा गोपनीयतेची खात्री देतात, ज्यांच्यावर तुमचा विश्वास आहे अशाच कंपनीच्या एआय रुपी मदतनिसाला घरात घ्यायला पाहिजे, नाही का?