\chapter{‘लक्ष’ द्यावे उमजून}

{\textbf{मोबाईल न वापरता राहण्याचा एक दिवस प्रयत्न करा. काही न करता बसून राहण्याने बऱ्याच गोष्टी साध्य होतात. मन शांत होते. आयुष्याच्या जमा-खर्चाचा मेळ लावता येतो, नवनवीन कल्पना सुचतात आणि बरेच काही साधते.}}

फार पूर्वीची अर्थव्यवस्था शेतीवर आधारित होती. औद्योगिक क्रांतीनंतर उद्योग व सेवा-क्षेत्र हे अर्थव्यवस्थेचा महत्वाचा भाग होऊ लागले. संगणक आणि इंटरनेट क्रांतीमुळे इन्फॉर्मेशन (माहिती) ची अर्थव्यवस्था बळ धरू लागली आणि सध्या, मोबाईल व समाज माध्यमांमुळे एक नवीनच अर्थव्यवस्था प्रभावी होत चालली आहे. त्याला आपण ‘अटेन्शन इकॉनॉमी’ म्हणजेच ‘लक्ष अर्थव्यवस्था’ म्हणू शकतो. १९९७ सालच्या ‘अटेन्शन शॉपर्स’ या लेखात शास्त्रज्ञ मायकल गोल्डहाबर यांनी ‘लक्ष’ आधारित अर्थव्यस्थेची पहिल्यांदा वाच्यता केली. त्यांच्या म्हणण्यानुसार सध्या स्पर्धा ही फक्त माहिती पुरवण्यात राहिलेलीच नाही तर तुमचे लक्ष राखण्याकडे वळलेली आहे. एखादी वस्तू घ्यायची असेल तर इंटरनेटवर त्याची माहिती असंख्य ठिकाणी मिळू शकते, पण जर एखादी साईट जास्त खिळवून ठेवणारी असेल तर तुम्ही तेथे जास्त वेळ थांबाल तर तुम्हाला जास्त वेळ जाहिराती दाखवता येतील मग त्या साईट-च्या कंपनीला त्याचा जास्त फायदा होईल, नाही का? सध्या माहितीची वानवा-कमतरता नसून तुमच्या ‘लक्षा’ची आहे. सध्याच्या डिजिटल युगातील कंपन्यांचे तुमच्या लक्षावर लक्ष आहे आणि त्यातून मिळणारा अमाप नफा हे त्यांचे ‘लक्ष्य’ आहे.

बऱ्याच जणांना वाटते की हे सारे मला मस्त फुकट वापरायला मिळत आहे, छान छान व्हिडीओ बघायला मिळत आहे, धमाल करमणूक होत आहे, पण नीट विचार केलात तर असे लक्षात येईल की हे बिलकुल फुकट नसून तुम्ही (म्हणजे तुमचे ‘लक्ष’) विकले जात आहात. जेवढे जास्त तुमचे लक्ष ते खेचतील, गोळा करतील, तेवढा जास्त जाहिरातींचा महसूल. ‘लक्ष अर्थव्यवस्था’ तुम्हाला मोबाईलला किंवा स्क्रीनला जखडून ठेवण्याकडे भर देते. एखादा व्हिडीओ अथवा रील पाहात असतानाच दुसरे समोर येते अथवा सुचवले जाते. फोटोंचे-छायाचित्रांचे पण तसेच. आपण एका मागून एक बघायला लागलो की तासंतास कसे गेले हे कळतच नाही. सतत काहीतरी नवीन, आकर्षक समोर येत राहते, ‘डोपामाईन’ मिळते, मग अजून अजून बघावेसे वाटते. ग्रुपवर काही संदेश (मेसेज-नोटिफिकेशन) आला की लगेच वाचावेसे वाटते, उत्तर नाहीतर किमान ईमोजी तात्काळ टाकावीशी वाटते. बॉसच्या इमेलला तर उत्तर पुढच्या मिनिटालाच गेलेच पाहिजे याची धडपड, अगदी विषय खूप तातडीचा नसताना सुद्धा. हे सर्व अगदी व्यसनाधीनतेकडे नेण्यासारखेच आहे. काही कारणांनी मोबाईल-समाज-माध्यम वापरायला नाही मिळाले, जवळ नसेल तर अस्वस्थ व्हायला होते, चिडचिड-त्रागा तर काहीच नाही, वैफल्याकडे वाटचाल सुरु होते.

मोबाईल युगाच्या आधीचे दिवस आठवा! तेंव्हा पण आपण गोष्टी लक्ष देऊनच करायचो. पुस्तक वाचणे, नाटक-सिनेमा पाहणे. त्यामुळे ‘लक्ष’ हे स्वतःहून काही वाईट नाही. पण सतत बदलणारे लक्ष घातक आहे. त्याला कॉन्टेक्सट स्विचिंग म्हणजेच ‘लक्ष वेगवेगळ्या गोष्टींकडे वारंवार बदलत राहणे’ असे म्हणतात. लक्ष सतत विचलित होत राहते. एका ग्रुप वरील मेसेज बघितला की लगेच दुसरा ग्रुपवर नजर गेलीच समजा. पहिल्या रील नंतर दुसरे. ब्राऊझर मध्ये तर असंख्य टॅब्स उघडलेले. याने विचारांची स्थिरता होत नाही. काही नीट समजण्याच्या आताच दुसरे येत असल्याने खोल विचार होत नाही, नीट मनन होत नाही, चिंतन तर दूरच. किंबहुना नीट विचार करावा लागेल असे मग काही बघावेसेच वाटत नाही. जरा मोठा लेख आला तर तो टाळून आपण पुढे जातो, नाही का? कॅलिफोर्निया-अरवाईन विद्यापीठातील ग्लोरिया मार्क यांच्या संशोधनानुसार एखाद्या व्यत्ययानंतर पुन्हा पूर्वीच्याच सखोल विचार स्थितीत यायला सर्वसाधारणपणे २३ मिनिटे लागतात, पण सध्या लगेच दुसऱ्या मिनिटाला अजून एक नोटिफिकेशन-व्यत्यय आलेलाच असतो. यामुळे आपली सखोल-विचार क्षमता हळूहळू क्षीण होत चालली आहे. तुम्हालापण स्वतःमध्ये व इतरांमध्येही जाणवले आहे का, की काही खूप नीट व खोल विचार करण्याच्या कामांना आता जास्त वेळ लागतो, किंवा जमतच नाही. औद्योगिक भाषेत बोलायचे झाले तर आपली ‘उत्पादकता’ कमी होत आहे.

उथळपणाकडे आणि सनसनाटी गोष्टींकडे कल जास्त वाढत चालला आहे. याची स्वतःवर एक चाचणी घेऊ शकता. तुम्ही सध्या कधीही (किंवा अगदी सुट्टीच्या दिवशी) सलग एक तास, शांत चित्ताने एखादे पुस्तक (मासिक-वर्तमानपत्र नव्हे) वाचू शकता का? उत्तर नाही असेल तर, विषय गंभीर आहे असे समजा. अजून एका प्रकारे ‘स्व’ ची ओळख करून घेण्यासाठी तुमच्या मोबाईल मधील ‘डिजिटल वेल बीइंग’ सारखी ऍप बघा. त्यात तुम्ही दिवसभरात कोठल्या ऍप वर किती वेळ घालवता हे दिसते. हे आकडे डोळे उघडणारे ठरू शकतात.

वसाहतवादाचे आधुनिक रूप

मोठ्या समाज माध्यम कंपन्या खरंतर फक्त एक प्लॅटफॉर्म (मंच) उपलब्ध करून देत असतात. तुमच्यातलेच काही (इन्फ्लुएन्सर्स , प्रभावाक) काही गोष्टी (मिम्स, पिक्चर्स, रील्स, पॉडकास्टस ई.) बनवतात आणि आपण सर्व ते पाहत बसतो. नाही म्हणायला या प्रभावकांना (अगदी मोजक्याच आणि खूप जनसंग्रह असणाऱ्या) चांगले-ठीकठाक पैसेही मिळतात. त्यामुळे मागणी-निर्माण-उपभोग-अजून मागणी-अजून निर्माण हे चक्र अव्याहत चालूच राहते. फायदा प्रामुख्याने आंतरराष्ट्रीय बलाढ्य कंपन्यांचा. काहीवेळेला या कंपन्या दादागिरीपण करतात. कुठल्या विचारधारेचा मजकूर ठेवायचा, कोणाला बढावा द्यायचा हे त्यांच्या (म्हणजे त्यांनी बनवल्येल्या प्रणाल्यांच्या, अल्गोरिदम्सच्या) हातात असल्याने ते समाजमन वळवतच नाही तर मोठ्या प्रमाणात नियंत्रित करतात. मागील निवडणुकांमध्ये त्याचा परिचय आपल्याला झालाच असेल. त्यामुळे ज्यांचा समाज माध्यमांवर ताबा त्यांचा (काही प्रमाणात का होईना) जगावर व जगाच्या समाजमनावर ताबा राहतो हे एक ढळढळीत सत्य आहे. मोठी ताकत आहे ही. वसाहतवादाचे हे आधुनिक रूप तेही तुमच्या नकळत. मोबाईल व पर्यायाने समाज माध्यमांचा वापर हा समाजाला व्यसनाधीनतेकडे नेण्यात होतो आहे.

कोठल्याही सार्वजनिक ठिकाणी पाहिले, तर बऱ्याचश्या लोकांचे लक्ष कोठे असते तर मोबाईल मध्ये. कोणी विचारांची तंद्री लावून शून्य आकाशात बघताना दिसतो का? तर, फारच क्वचित. याचा परिणाम वैयक्तिकच नाही, सामाजिकच नाही तर आंतरराष्ट्रीय सुरक्षेवर पण होऊ शकतो. अशी विचार-क्षीण, मोबाईल मध्ये गुरफटलेली जनता कोठल्या शत्रू राष्ट्राला नको आहे?

एवढ्या सगळ्या वाईट गोष्टी ऐकल्यावर आपल्याला वाटेल की मग मोबाईलचा आणि समाजमाध्यमांचा त्यागच करायचा का? तर, नक्कीच नाही. यांचा कितीतरी प्रभावी पद्धतीने वापर करता येतोय. त्यांनीच ज्ञानाची कवाडे सर्वसामान्यांसाठी खुली केली आहेत. जगातील कानाकोपर्यातील नातेवाईकांना, मित्रमंडळीना जवळ आणले आहे. संवाद वाढला आहे. पारदर्शिकता वाढली आहे. लोकशाही जास्त सुदृढ झाली आहे कारण आता जनतेपासून काहीही लपवणे अवघड होऊन बसले आहे. घरबसल्या काम करणे शक्य झाले आहे. आपली आर्थिक-जीवन पद्धती आमूलाग्र बदलली आहे. हे सर्व फायदे नक्कीच आहेत. आक्षेप आहे तो त्यांच्या वायफळ अति-वापराबद्दल आणि कालांतराने होणाऱ्या व्यसनाधीनतेबद्दल.

आपल्या हातात आहे की या सर्वांचा सकारात्मक वापर कसा वाढवता येईल आणि दुष्परिणाम कसे कमी करिता येतील. ज्यांची रोजी-रोटीच मोबाईल-समाज-माध्यम आणि कायम ऑनलाईन राहण्यावर अवलंबून आहे त्यांना काही पर्याय नाही पण ज्यांची तशी परिस्थिती नाही त्यांचासाठी काही उपाय सुचवता येतील. पहिला उपाय म्हणजे ‘डिजिटल मिनिमलिझम’ म्हणजेच या तंत्रज्ञानाचा कमीत कमी वापर. अनावश्यक ऍप्स, अकाउंट्स, सबस्क्रिप्शन्स, डिव्हाइसेस काढून टाकणे, अगदी मन मारून. अजून पुण्य पाहिजे असेल तर ‘डिजिटल फास्टिंग’ म्हणजेच समाज-माध्यम-मोबाइलला न वापरत करण्यात येणारा उपवास, आठवड्यातून १-२ दिवस. शाळेत जाणाऱ्या मुलांच्या हातात फक्त कॉल करता येणारे डब्बा-फोन्स, असे बरेच काही. असे विचारावेसे वाटते की तुम्ही, नजीकच्या काळात कधी ‘बोअर’ (कंटाळा) झाला होतात? मोबाईलमुळे पुढ्यात इतक्या गोष्टी वाढून ठेवल्या आहेत की कंटाळा येणेच दुरापास्त झाले आहे. ‘बोअर’ होणे किंवा काही न करता बसून राहणे याने खरंतर बऱ्याच गोष्टी साध्य होतात. मन शांत व्हायला लागते, बाह्य-उत्तेजनेशिवाय विचार करता यायला लागतो, आयुष्याच्या जमा-खर्चाचा मेळ लावायला वेळ मिळतो, नवीन कल्पना सुचतात आणि अजून बरेच काही. गंमत म्हणून एखादा तास काही न करिता, एकटे बसून बघाच .