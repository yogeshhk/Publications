\chapter{तिसऱ्या मेंदूची `तिसरी घंटा'}

कल्पना करा, कामानिमित्त परगावी गेलेले आहात. कंपनीतून सायंकाळी हॉटेलवर परत आल्यावर घराच्या आठवणी सुरू होतायत. विरंगुळा म्हणून मोबाईलवर गाण्याचे ऍप सुरू करताच शिफारस येते , राग मारवा.  धक्का बसला ना! हा जादूटोणा नसून त्या ऍपमधील एआय तुमचे मन-मूड ओळखतो आहे याचे लक्षण आहे. गाणी सुचवण्यापासून ते गाण्यांना चाल देणे, नाटके-पटकथा लिहिणे, प्रसिद्ध व्यक्तीचा हुबेहूब आवाज तयार करणे अशा अनेक गोष्टींसाठी एआयचा वापर होत आहे. इतर क्षेत्रांप्रमाणे करमणूक विश्वातही एआयने जोरदार प्रवेश केला आहे. त्याची काही उदाहरणे पाहूया.

गाणी, सिनेमे आणि व्हिडीओ सुचवणारे ऍप्स कसे काम करतात? प्रामुख्याने दोन प्रकारे , तुम्ही पूर्वी पाहिलेल्या गोष्टींशी साधर्म्य असणाऱ्या किंवा तुमच्यासारखी इतर माणसे काय पसंत करतात त्यानुसार. जेवढी तुमची माहिती (विदा, डेटा ) त्या ऍप्सकडे जास्त तेवढी त्यांची साधर्म्य ओळखण्याची क्षमता जास्त. याचा फायदा असा की तुम्ही त्या ऍप्सवर जास्त वेळ खिळवून ठेवले जाणार, तेवढेच त्यांचे जाहिरातींचे उत्त्पन्न जास्त. तुमचे `लक्ष' किंवा `अवधान' हे चलनी नाणे झाले आहे.  

पूर्वी व्यंगचित्रांचे चलचित्रपट (कार्टून ऍनिमेशन) हे मोठे जिकरीचे काम असायचे. कोठल्याही कृतीचे अनेक छोट्या छोट्या भागात विभागणीकरून प्रत्येकाचे अचूक चित्र हाताने काढावे लागे, रंगसंगती सांभाळून. एआयमुळे हे काम आता सुलभ झाले आहे. 

द्विमितीयच (टुडी) नाही तर त्रिमितीय (थ्रीडी) ॲनिमेशन अधिक वास्तवदर्शी झाले आहे. वाहती हवा, समुद्राच्या पाण्याचे तुषार, प्राण्यांच्या त्वचेचा पोत इत्यादी बारीक सारीक गोष्टी दाखवणे तंत्रज्ञानाने शक्य झालेले दिसते. डिस्ने चे चित्रपट तर फारच पुढे गेलेले आहेत या बाबतीत.  भारतीय चित्रपटही यात बिलकुल मागे नाहीत. संगणकीय रंग-रेखाटनाने (कॉम्पुटर ग्राफिक्स, सीजी) चित्रपटातील काल्पनिक दृश्ये खरीखुरी वाटायला लागतात. पूर्वीच्या टीव्ही मालिकांतील बाण-युद्ध आता जरा कृत्रिम वाटायला लागते. हजारोंची सेना दाखवणे `सीजी'च्या माध्यमातून शक्य होत आहे. 

याचीच पुढची पायरी म्हणजे `मेटा व्हर्स' सारखी आभासी दुनिया. त्यात शहरे असतात ज्यात तुम्ही घर घेऊ शकता, राहू शकता, खरेदी विक्रीचे व्यवहार करू शकता.  संगणकीय खेळांमध्येही (गेमिंग) तंत्रज्ञानाची प्रगती प्रकर्षाने जाणवते. पूर्वीच्या गेम्स आठवा आणि आजच्या फिफा फुटबॉल किंवा `कॉल ऑफ ड्युटी' सारख्या गेम्स पाहा.. जमीन-अस्मानाचा फरक आहे. आपण क्रीडांगणात उपस्थित आहोत असाच भास होतो. खेळाडू पळताना त्यांच्या क्रिया, त्याप्रमाणे बदलणाऱ्या सावल्या, चेहऱ्यावरचे बदलते हावभाव मानवी वाटायला लागले आहेत.

गीताला चाली देणे, पार्श्वसंगीत, दोन कडव्यांमधील वाद्यवृंदाचे संगीतीय-तुकडे हे बनवण्यास एआय मदत करू शकते. ओपनएआयचे म्यूझनेट किंवा गुगलच्या मॅजेन्टाने असे तुकडे बनवता येतात. वेगवेगळे सांगीतिक प्रयोगही करता येतात. कालवश झालेल्या गायकांचा आवाजही वापरता येऊ शकतो हे प्रसिद्ध संगीतकार ए आर रेहमान यांनी दाखवून दिले आहे (अर्थात त्या गायकांच्या कुटुंबाच्या परवानगी नंतर). बोलण्याचा आवाज बदलणे, वेगवेळ्या भाषेत ध्वनिमुद्रण (डबिंग) करणे शक्य आहे. 

कोणताही चित्रपट फक्त चित्रीकरण संपले म्हणून तयार होत नाही. विविध दृश्यांचे नीट संकलन केले जाते. कोठे कापायचे, कोठे जोडायचे जेणेकरून दृश्य गतिमान वाटेल यांसारखे परिणाम हे कुशल संपादनाने (एडिटिंग) साधता येतात. यातही एआयची मदत होते. `रनवे एम एल', अडोबीचे `सेन्सेई' यांसारखी ऍप्स संपादनातील क्लिष्टता कमी करून दर्जेदार काम करायला मदत करतात.  पूर्वी `चाललेल्या' कथानकानुसार नवीन पटकथा लिहिणे एआयच्या माध्यमातून शक्य झाले आहे. दिलेल्या सूचनेप्रमाणे कथा-दृश्य (सीन) बनवणे, संवाद लिहिणे, इत्यादी करता येते. हॉलिवूडमधील चित्रीकरणगृहे (स्टुडिओ) यासंदर्भात काम करीत आहेत. 

जोड-तोड करून व्हिडिओ अगदी खऱ्यासारखे एवढे हुबेहूब बनवणे शक्य असल्याने त्याचा चुकीचं वापर होताना सुद्धा दिसतो. अशा `डीप फेक' (बनावटी) व्हिडीओंनी प्रसिद्ध तारे-तारकांनाच नाही, नेतेमंडळींनांच नाही तर सामान्य जनतेलाही त्रस्त केले आहे. `दिसतं तसं नसतं म्हणून जग फसतं ' हेच खरे वाटायला लागते. करमणूक क्षेत्रात एआयचा वापर दिवसेंदिवस  वाढतच जाणार आहे. कशावर विश्वास ठेवायचा हा नीर-क्षीर विवेक बाळगावा लागणार आहे. आपल्या जीवनाच्या रंगमंचावर एआयचे आगमन झाले आहे आणि तिसरी घंटा वाजते आहे!