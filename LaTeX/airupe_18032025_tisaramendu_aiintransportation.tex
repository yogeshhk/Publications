\chapter{`वंडर कार' वास्तवात !}

भारतातील मालवाहतूक क्षेत्राला दररोज साठ हजार कोटी रुपयांचे नुकसान होते, असे एका अभ्यासात लक्षात आले. वाहने वाहतूककोंडीत अडकतात, वेळ, इंधन वाया जाते, शिवाय मनस्तापही होतो. आपणही हा त्रास अनुभवलाच असेल. अशा वेळी वाटते, कोणी दुसऱ्यानेच आपली गाडी चालवली तर? गर्दी टाळून लवकर पोहोचवले तर? हे आता एआयमुळे शक्य होत आहे. कार, ट्रक स्वयंचलित होत आहेत. जलद पोहोचवणारा मार्ग एआय सुचवत आहे. आरामदायी प्रवासाचे दिवस सर्वसामान्यांसाठी काही दूर नाहीत. फक्त स्वयंचालनातच नव्हे तर वाहतूकक्षेत्रातील इतर गोष्टींमध्ये पण एआयचा वापर मोठ्या प्रमाणात होतो आहे त्याची काही उदाहरणे पाहुयात.

`वेअमो' कंपनीच्या स्वयंचलित गाड्या सॅन फ्रान्सिस्कोच्या रस्त्यांवर धावत आहेत. या गाड्या `लायडार' कॅमेर्‍यांच्या मदतीने त्रिमितीय चित्रण करतात, सभोवतालचा अंदाज घेतात, रस्ता ओळखतात आणि अडथळा नसेल तर मार्गक्रमणा करतात. वाहतूक दिवे आणि इतर सुरक्षा चिन्हेही त्या ओळखतात व त्यानुसार निर्णय घेतात. पहिल्यांदा प्रवास करणाऱ्यांना स्टिअरिंग व्हील आपोआप हलताना पाहून जरा भुताटकी वाटू शकते पण तेढ्यापुरतेच. गंतव्यापर्यंत आरामात सोडल्यावर तंत्रज्ञानाचे कौतुकच वाटते.

चौकात उभे ठाकलेले वाहतूक नियंत्रक दिवे (सिग्नल) आपल्या संयमाची परीक्षा घेत असतात. गर्दी कमी असुदे वा जास्त ते आपल्या ठरलेल्या वेळेप्रमाणे रंग बदलत असतात. मात्र, एआयच्या मदतीने सिग्नल्स गर्दीनुसार कमी-अधिक वेळ चालवता येतात. लॉस अँजेल्ससारख्या शहरात हे सुरू आहे. पुण्यातही असे एआय-सिग्नल्स वापरण्यात येणार असल्याचे वाचले होते. गुगल मॅप्ससारख्या मार्ग-दर्शक ऍप्समध्ये रंगाद्वारे रहदारीची घनता दाखवली जाते, लागणाऱ्या वेळेचा अंदाज घेता येतो त्यानुसार मार्ग (शक्य असेल तर) बदलता येतो. याने वेळ, इंधन आणि चिडीचिड कमी होऊ शकते. वस्तू-सामान पोहोचवणाऱ्या `युपीएस' सारख्या अमेरिकेतील कंपन्या एआय द्वारे मार्ग-नियोजन करून मोठी बचत करतात. आपल्याकडेही केवळ मालवाहतूकच नाही तर अगदी शहरात चालणाऱ्या बसेस यांचेही मार्ग नियोजन एआयद्वारे केले जाऊ शकते. त्यामुळे अगदीच रिकाम्या किंवा खचाखच भरलेल्या बस आपल्याला पाहाव्या लागणार नाही, नागरिकांचा प्रवासही सुकर होईल.

सध्या जोरात प्रचलीत होत असलेले `ड्रोन्स' ही एआयचा वापर करतात. त्यांच्यातले कॅमेरे टिपत असलेल्या दृश्यातून मार्ग-ठरवणे, इप्सित ठिकाणापर्यंत पोहोचलो का ते ठरवणे, अशा गोष्टी करतात. ड्रोन्स ने भूभाग आरेखन, पीकांचे स्वास्थ्य ठरवणे, शहरातील अतिक्रमणे ओळखणे हे तर शक्य आहेच पण त्याशिवाय वस्तूंची ने-आण पण करता येते. त्याच्या या बहुविध कामांमध्ये एआय मदत करते. यांचा वाढता वापर पाहता नवउद्योजकांनी या तंत्रज्ञानाकडे जरूर लक्ष द्यावे.

वाहतुकीतील गाड्यांच्या देखभालीमध्येही एआयचा वापर वाढला आहे. बऱ्याचशा गाड्यांमध्ये आता अनेक संवेदके(सेन्सर्स) बसवलेले असतात. त्यातून येणाऱ्या डेटा (माहिती, विदा ) चे परीक्षण करून कोठे बिघाड झाला आहे का किंवा होणार आहे का त्याचा अंदाज बांधता येतो. अचानक होणाऱ्या बिघाडाची अथवा अपघाताची शक्यता कमी होते.

वाहतूक सेवा वापरण्याच्या सुलभतेसाठी एआय आधारित संभाषण प्रणाल्या (चॅट बॉट ) चांगले मदतनीस ठरू शकतात. विविध मार्ग सुचवणे, तिकिटे काढणे, वेळापत्रक इत्यादी अनेक प्रश्नांचे निरसन करू शकतात, तेही २४ x ७ आणि कपाळावर आठी न आणता!! पेट्रोल, डिझेल सारख्या पारंपरिक इंधनांपासून विजेवर धावणाऱ्या गाड्यांकडे आपण वळत आहोत. त्यातील महत्त्वाचा घटक म्हणजे `बॅटरी'. चांगल्या कार्यक्षमतेत चालण्यासाठी चार्जिंग कधी आणि किती करावे याचे नियोजन एआय करतो. यामुळे बॅटरीचे आयुष्य वाढते. एआयचे एवढे फायदे असले तरी काही समस्या जरूर आहेत. आकाशात उडणारे ड्रोन्स जमिनीवरचे सर्वच टिपत असल्याने खाजगी, संवेदनशील किंवा संरक्षित क्षेत्रांवर उडवले गेले तर? यावर कहर म्हणजे ते ड्रोन शत्रू-राष्ट्राने बनवलेले असेल तर? यामुळे सुरक्षा धोक्यात येऊ शकते. नैतिकता आणि कायद्याचे उदाहरण म्हणजे स्वयंचलित गाडीने जर अपराध केला तर कोणाला जबाबदार धरायचे? असे एक ना अनेक प्रश्न उद्भवतात. तरीही, वाहतूक क्षेत्रात एआय अडथळ्यांच्या मार्गावरून पुढे जात आहे. त्याची दिशा आपले जीवन अधिक सुकर करण्याकडे आहे, हे निश्चित!