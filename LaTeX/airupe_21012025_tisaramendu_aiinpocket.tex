\chapter{'कृत्रिम बुद्धिमत्ता' तुमच्या खिशात}

``हे सिरी, उद्या पाऊस पडेल का?'' किंवा ``हे गुगल, मला जगजीत सिंग यांची गझल ऐकवशील का?'' अशा सूचना दिल्या की मोबाईलआज्ञाधारक नोकराप्रमाणे ती कामे त्वरित करतो. गाडी चालवताना रस्ता दाखवणे, पुस्तक खरेदीसाठी शिफारस करणे, किंवा वाढदिवसाची आठवण करून देणे, ही कामे तो सहज करतो. कॉल आणि संदेशापुरता मर्यादित असलेला मोबाईल आता एआय (कृत्रिम बुद्धिमत्ता, आर्टिफिशिअल इंटेलिजन्स) मुळे 'स्मार्ट' झाला आहे. तुमच्या नकळतच, एआय तुमच्या खिशात पोहोचला आहे. या तंत्रज्ञानाचा वापर कसा होतो, त्यामागील प्रणाली काय आहे, आणि त्याचे फायदे व संभाव्य धोके यावर नजर टाकूया.

आज बहुतेक लोकांकडे स्मार्टफोन आहेत. त्यात एआयचा मोठ्या प्रमाणावर वापर केला जातो. यामुळे शिक्षण, आरोग्य, खरेदी-विक्री आणि प्रवास या क्षेत्रांत लोकांना सहज प्रवेश मिळतो, ज्यामुळे लोकशाहीकरण वाढलेच आहे.

एआयद्वारे आपल्याला उत्तरे मिळतात, कामे ही होतात, पण अशा बऱ्याच ऍप्स मध्ये जाहिराती सुद्धा दाखवल्या जातात. तुम्हाला जरी सेवा फुकट मिळत असली तरी विविध उत्पादनाच्या जाहिराती दाखवून खरेदीस उद्युक्त केले जाते. आपले मन वळवण्याचा अथवा विशिष्ट पद्धतीने बनवण्याचा प्रयत्नही केला जातो. समाज-माध्यमातील (सोशल मीडिया) शिफारसींनी समाज-मन विशिष्ट अंगाचे बनवण्याचा प्रयत्नही केला जातो. त्यामुळे हे सर्व वापरताना बोलविता धनी कोण आहे, हे लक्षात ठेवणे महत्त्वाचे आहे.

मोबाईल मध्ये आपला डेटा इतक्या विविध प्रकारात आणि प्रमाणात असतो की त्यातील एआय तुम्हाला व्यवस्थित 'ओळखते'. गमतीत बोलायचं झालं तर, तुमची आई तुम्हाला जेवढे ओळखते त्यापेक्षा जास्त!! बरं , हे चांगले की वाईट? डेटा दिला नाही तर नवनवीन सेवा कशा मिळणार, आणि दिला तर चौर्यकर्माची भीती. त्यामुळे कुठल्या ऍप्स वर विश्वास ठेवायचा हे ठरवावे लागते.

एआयच्या प्रणाल्या (अल्गोरिदम्स) खरोखरच वरदान ठरतात. तुम्हाला जाणवले आहे का की आपल्याला दररोज खरंतर असंख्य संदेश, ईमेल्स येत असतात. त्यातील आगंतुक-अनावश्यक (स्पॅम) खड्याप्रमाणे बाजूला केले जातात. आरोग्याविषयी प्रणाल्यांमध्ये (ऍप्स ) हृदयगतीसारख्या घटकांचा अभ्यास करून तंदुरुस्तीच्या योजना दिल्या जातात. पुढच्या रस्त्यावर गर्दी दिसत असेल तर पर्यायी मार्ग सुचवला जातो. फोटो सुंदर येण्यास मदत केली जाते. आपल्याला आवडू शकतील अशीच गाणी आणि चित्रपट सुचवल्यामुळे शोधत बसावे लागत नाही, वेळ वाचतो. परदेशप्रवासात तर पावलापावलांवर मदत होते, दुभाषा म्हणून, हॉटेल शोधण्यास, प्रेक्षणीय स्थळांची माहिती घेण्यास, एक ना अनेक. परिणामी, मोबाईलवरील अवलंबित्व प्रचंड वाढले आहे. काहीही लक्षात ठेवायची गरज नाही आणि विचार करण्याचीही नाही असे होऊ लागले आहे. ऑक्सफर्ड शब्दकोशाचा 'वर्ड ऑफ द इयर' (२०२४ वर्षातील सर्वोत्तम शब्द) चा मान 'ब्रेन रॉट' (मेंदू-विचार-बुद्धीचा ह्रास) या शब्दास मिळाला आहे. मोबाईल, एआय, समाज माध्यमे यांच्या अति वापराने होणाऱ्या प्रज्ञा क्षयाचा तो द्योतक आहे. 'अति सर्वत्र वर्जयेत्' हा नियम पाळणे महत्त्वाचे आहे.

आपल्या खिशात बसणारा हा एआयरुपी 'अल्लाउद्दीनचा दिवा' भविष्यात काय काय घेऊन येऊ शकतो याची झलक पाहुयात. संवर्धित-वास्तव (एआर , ऑगमेंटेड रियालिटी) या तंत्रज्ञानाच्या द्वारे घराच्या कागदावरील नकाशावर त्याचे त्रिमितीय (३D) मॉडेल पाहता येईल, कॅमेरा पुढील वस्तूची इतंभूत माहिती तेथेच आपल्याला वाचता येईल, तसेच खेळांमध्ये (गेमिंग) तर क्रांती घडणार आहे. अनेक भाषा येणारा सर्वज्ञ दुभाषी मिळेल जेणेकरून कोठेही प्रवासाला गेल्यावर संभाषणाची काहीच अडचण येणार नाही. शिक्षण पुस्तकी ना राहता त्याला दृक-श्राव्य माध्यमांची जोड तर मिळेलच पण आभासी जगतात प्रयोग पण करिता येतील. ही झाली काही वानगी दाखल उदाहरणे, पण सर्वच विषयात मोबाईल मधील एआय मदतीला असेल. या सहभागाला अजूनही खूप वाव आहे आणि खरंतर अंत्योदय झाल्याशिवाय तंत्रज्ञानाचे फलित ते काय?

या सकारात्मक गोष्टींबरोबरच काही चिंतासुद्धा आहेत. मोबाईल मधील माहितीचा वापर कसा होतो आहे, ती विकली जात आहे का, त्याद्वारे आपल्याला लक्ष-बंधक (अटेन्शन-ट्रॅपड) तर बनवलं जात नाहीयेना याचा सजग विचार स्वतः आपण केला पाहिजे. समाज, कंपन्या आणि सरकार यांनी संभाव्य धोके कमी करण्याची सुसंगत धोरणे आखण्याची गरज आहे. या दिव्यातील राक्षसाला आपल्या नियंत्रणात ठेवायचे का आपल्याला गिळंकृत करू द्यायचे हे आपण ठरवायचे.