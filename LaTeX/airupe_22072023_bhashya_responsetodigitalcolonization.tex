\chapter{'डिजिटल वसाहतवादा'ला उत्तर }

{\textit{कृत्रिम बुद्धिमत्तेवर आधारित भारतीय प्रणाल्या बनविण्यासाठी प्रयत्न व्हायला हवेत. मोठी बाजारपेठ असल्याने परदेशी कंपन्या तर यात उतरणारच आहेत; पण त्यातही भारतीयांचा मोठ्या प्रमाणात पुढाकार हवा. भारताने ही संधी गमावून चालणार नाही.}}

\vspace{1.5em}

'चॅटजीपीटी' बनवणाऱ्या 'ओपनएआय' कंपनीचे मुख्य कार्यकारी अधिकारी सॅम ऑल्टमन काही दिवसांपूर्वी भारतात आले होते. त्यांनी मांडलेले एक मत खळबळजनक ठरले. 'भारतीय कंपन्यांनी चॅटजीपीटी सारखी गोष्ट बनवण्याच्या प्रयत्न करणे हे निरर्थक आहे' अशी दर्पोक्ती त्यांनी केली. या मतावर 'टेक महिंद्रा कंपनी'चे गुरनानी यांनी हे आव्हान स्वीकारल्याचे सांगितले.

काही भारतीयांना सॅम यांचे मत दुखावणारे वाटले; तर काहींना ते व्यावहारिक मर्यादांचे कथन वाटले. या वादात न पडता एक गोष्ट मात्र यातून नक्कीच पुढे आली, ती म्हणजे भारत हा आर्टिफिशिअल इंटेलिजन्स अर्थात कृत्रिम बुद्धिमत्तेचा फक्त वापरकर्ता राहणार आहे का निर्माणकर्ता पण बनणार आहे?

म्हणजेच फक्त 'भारतासाठी एआय' की 'भारताकडून एआय' (भारतासाठी एआय) यावर सीमित न राहता 'एआय-बाय-भारत' (भारताचे एआय) पण तो करणार आहे की नाही.

सध्याच्या तंत्रज्ञानयुगात एआयचे महत्त्व अनन्यसाधारण आहे. रोग-निदान, हवामान अंदाज यांपासून अगदी तुम्ही कोणते पुस्तक विकत घ्यावे, यापर्यंत एआय आपल्या जीवनात एकरूप झाला आहे. इतका की हे काही वेगळे आहे, याची कल्पनाच येत नाही.

एआय हा एक संगणकप्रणालीचा प्रकार असून तो आपल्यासमोर मोबाईलमधील ॲप स्वरूपात, शॉपिंग वेबसाईटमध्ये, उपकरणांमध्ये, चॅटबॉट अशा विविध रूपात येतो. ह्या एआय प्रणालींची निर्मिती जगभर होत असली तरी अमेरिकेचे यावर निर्विवाद वर्चस्व आहे.

वर उल्लेख केलेल्या अनेक प्रणालींवर अमेरिकी कंपन्यांचे स्वामित्व आहे. एका अर्थाने एआयच्या नाड्या बऱ्याच प्रमाणात अमेरिकेसारख्या देशांच्या हातात आहेत. ही गोष्ट आता चिंताजनक नसली तरी ती आपल्या सार्वभौमत्वाच्या दृष्टीने भूषणावह नक्कीच नाही. मग भारत काय करू शकतो याचा थोडक्यात विचार करू या. प्रचलित एआय प्रणालीला स्वयं प्रशिक्षणासाठी भरमसाठ आणि बहुविध माहिती (डेटा) लागतो.

भारताची लोकसंख्या आणि विविधता लक्षात घेता आपण डेटाच्या संभाव्य सोन-खाणीवरच बसलो आहोत. जसाजसा डिजिटल तंत्रज्ञानाचा वापर वाढत जाईल, तसातसा जास्तीत जास्त डेटा मिळत राहील. सध्या बऱ्याच परदेशी कंपन्या तुम्हाला काही सेवा, ऍप फुकट वापरायला का देतात याचा तुम्ही विचार केला आहे का?

तेथील जाहिरातींचे उत्पन्न हा एक भाग झाला तरी अनेक कंपन्या तुमचा डेटा गोळा करायला आलेल्या असतात आणि लालूच म्हणून सेवा फुकट देतात. आपल्यालाही फायदा होत असल्याने आपण कानाडोळा करतो. पण हे किती दिवस चालणार? परदेशी एआय सेवांवर किती दिवस अवलंबून राहणार? आपला डेटा, आपणच आपल्या 'एआय' सेवांसाठी वापरला तर फायदेशीर ठरेल, आर्थिकदृष्ट्या आणि सुरक्षेच्या दृष्टीनेसुद्धा.

जेथे जेथे शक्य असेल तेथे तेथे भारतीय (स्वदेशी) एआय निर्माण करता आला तर आत्मनिर्भर भारताच्या दिशेने ते मोठे पाऊल असेल. काही यशस्वी उदाहरणे आपल्या डोळ्यासमोर आहेत. परदेशी जीपीएस (ग्लोबल पोझिशनिंग सिस्टिम, उपग्रहाद्वारे जमिनीवरील ठिकाण शोधण्याची यंत्रणा) आपण मोबाईलमध्ये पाहतो, त्याच्या तोडीस तोडच नाही तर काकणभर सरस यंत्रणा 'नाविक' या नावाने इस्रोने बनवली आहे.

आणखी एक उदाहरण म्हणजे, डिजिटल आर्थिक व्यवहारांच्या तंत्रज्ञानात भारतीय युपीआय (युनिफाइड पेमेंट इंटरफेस) परकी प्रणालींच्या तुलनेत कोठेही कमी नाही. त्यामुळे न्यूनगंड न बाळगता आपल्या समस्यांवर आपणच 'एआय'मार्फत उत्तरे शोधण्याचा प्रयत्न केला पाहिजे. वानगीदाखल एक महत्त्वाची समस्या पाहू. भारतात इंग्रजी येणारे-समजणारे फारच कमी.

त्यात बहुतांशी मोबाईल ॲप, संगणकप्रणाल्या, डिजिटल उत्पादने-वस्तू या इंग्रजीच दाखवतात आणि समजतात. सध्या प्रसिद्ध झालेले चॅटजीपीटीपण इंग्रजी भाषेत जेवढे मस्त उत्तर देते, तेवढे मराठी-हिंदीत देते का? नाही. याचे कारण त्यामागे असलेल्या एल-एल-एम (लार्ज लॅन्व्हेज मॉडेल, बृहत भाषा प्रारूप) च्या प्रशिक्षणासाठी भारतीय भाषांच्या तुलनेत इंग्रजी डेटा खूप वापरला होता.

हे चित्र बदलायचे आले तर भारतीय भाषांची 'एल-एल-एम' प्रशिक्षित करावी लागतील. त्याचा वापर करून सर्व भारतीयांना वापरता येतील, अशा संभाषणप्रणाल्या बनावता येतील.

रेल्वेचे तिकीट काढायचे असेल, पुढील आठवड्याचा पावसाचा अंदाज विचारायचा असेल तर भारताच्या सर्वदूर कोपऱ्यातूनही कोणीही आपल्या बोलीभाषेत हे प्रश्न विचारून उत्तरे 'एआय'कडून घेऊ शकेल. सर्वसमावेशकतेच्या दृष्टीने ते गरजेचे आहे. अशा अनेक समस्यांना उत्तरे शोधण्याची गरज आहे आणि ती उत्तरे ते ही भारतीय डेटावर आधारित लागणार आहेत.

जसे रोगनिदानप्रणाली आपल्या भारतीयांच्या डेटावर प्रशिक्षित असेल तर निदान जास्त अचूक येणार. कायदेविषयक प्रश्नांची उत्तरे, भारतीय न्यायालयाच्या निकालांवर प्रशिक्षित असेल तर, उत्तरे अधिक बरोबर येणार नाहीत का? आता अशा भारतीय-स्वदेशी एआय प्रणाल्या कोण बनवणार? मोठी बाजारपेठ असल्याने परदेशी कंपन्या तर यात उतरणारच आहेत; पण त्यातही भारतीयांचासुद्धा मोठ्या प्रमाणात पुढाकार हवा.

कर्मधर्मसंयोगाने ती सोय नजीकच्या काळात होताना दिसते आहे. सध्या अनेक इंजिनिअरिंग कॉलेजेसमध्ये कम्प्युटर, एआय ह्या विद्याशाखांमध्ये मोठ्या प्रमाणात प्रवेश झाले आहेत. (बाकीच्या शाखांकडे जरा दुर्लक्षच झाले आहे, पण असो) ती सर्व मंडळी नजीकच्या भविष्यात एआय प्रणाली बनवण्यास सज्ज असतील. म्हणजे मागणीपण आहे, समस्यांच्या स्वरूपात आणि पुरवठा आहे; तंत्रज्ञांच्या स्वरूपात. हा दुग्ध-शर्करा योगच.

भारताने ही संधी गमावून चालणार नाही. महात्मा गांधींना मिठाचा सत्याग्रह करावा लागला, याचे कारण इंग्रजांनी मिठासारख्या अत्यावश्यक गोष्टीवर कर लावला होता म्हणून. सध्याच्या काळात अगदी मिठाएवढे नसले तरी, एआय हे अत्यावश्यक बनू लागले आहे. त्यावर परदेशी कंपन्यांचे वर्चस्व धोकादायक ठरण्याची शक्यता आहे.

पूर्वी भारतात उत्पन्न झालेला कापूस परदेशी नेला जाई, त्यावर प्रक्रिया करून कपडे बनवले जात आणि तो तयार माल भारतीय बाजारपेठेत विकावयास आणला जाई. या इतिहासाची वेगळ्या स्वरूपात पुनरावृत्ती तर होत नाहीये ना? याचा अर्थ इतर देशांशी सहकार्य, सामूहिक प्रकल्प, नवकल्पनांमध्ये सहभाग बंदच करायचा, असा मुळीच नाही.

डिजिटल वसाहतवादाला बळी पडत नाहीना, याची काळजी घेत स्वदेशी एआय तंत्रज्ञानामध्ये गुंतवणूक गरजेची आहे. तीच आत्मनिर्भर भारताच्या 'एआय'च्या मार्गक्रमणेची योग्य दिशा ठरेल.

नियमनाची शिफारस
एआय च्या प्रगतीची संभाव्य महा-घोडदौड लक्षात घेता ती कोठेही अनियंत्रित होऊ नये यासाठी दूरसंचार नियामक प्राधिकरणाने नुकतेच एक महत्त्वाचे पाऊल उचललेले आहे. त्यांनी ''एआयडीएआय'' (आर्टिफिशिअल इंटेलिजन्स अँड डेटा ऑथॉरिटी ऑफ इंडिया) च्या स्थापनेची शिफारस केली आहे.

त्याचे प्रमुख काम हे विविध क्षेत्रात 'एआय'मुळे संभावणारे धोके लक्षात घेणे, त्याचे नियमन करणारे धोरण ठरवणे, डेटा सुरक्षेच्या-गोपनीयतेच्या जागतिक दर्जाच्या उत्तम कार्यपद्धती सुचवणे, 'एआय' चा नैतिक वापर करण्यास प्रोत्साहन देणे, इत्यादी असणार आहे. या सर्व सूचनांचे योग्य पालन आणि अंमलबजावणी झाल्यास 'एआय'चे भारतातील भविष्य सुरक्षित आणि नियंत्रित होण्यास मदत होणार आहे.

