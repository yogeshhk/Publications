\chapter{शिक्षणक्षेत्रासाठी प्रभावी साधन}

पुण्यातील एका अभियांत्रिकी महाविद्यालयात `अतिथी अध्यापक' म्हणून शिकवताना एक गोष्ट सध्या लक्षात येत आहे की, नवनवीन तंत्रज्ञानामुळे विद्यार्थ्यांच्या `माहिती'च्या कक्षा रुंदावल्या असल्या, तरी `विचार क्षमता'कक्षा आकुंचन पावत आहे. वर्गात प्रश्न विचारल्यावर उत्तर शोधण्यासाठी विचार करायच्याऐवजी हात चॅटजिपीटीकडे वळतात. एवढेच नव्हे, तर एआय शिक्षणाच्या विविध अंगांमध्ये प्रवेश वाढत आहे. औद्योगिक आणि सामाजिक गरजांमुळे सरकारही शिक्षणात एआयच्या वापरास प्रोत्साहन देत आहे.  यंदाच्या केंद्रीय अर्थसंकल्पात अर्थमंत्री निर्मला सीतारामन यांनी एआयच्या आधुनिक दर्जाची केंद्रे स्थापन करण्यासाठी ५०० कोटी रुपयांची तरतूद केली आहे. तसेच, युनेस्कोने २०२५ चा आंतरराष्ट्रीय शिक्षण दिन `एआय आणि शिक्षण' याला समर्पित केला आहे. करोना काळात शाळा-विद्यालये बंद असल्याने दूरस्थ (रिमोट) शिक्षणाबरोबरच एआय आधारित ऍप्सचा खास करून स्वशिक्षणासाठी मोठ्या प्रमाणात वापर सुरू झाला होता तो आता इतर प्रक्रियांमध्ये पण सुरू झाला आहे. एकंदरीतच एआयचा शिक्षण क्षेत्रात कसा वापर होतो आहे, केला जाऊ शकतो ते संक्षेपात पाहुयात.

प्रत्येक वर्गात सर्वसाधारणपणे काही विद्यार्थी अतिशय हुशार असतात, काही सामान्य तर काहींना शिकणे कठीण जाते. शिक्षकांना सर्वसामान्य विद्यार्थ्यांसाठी शिकवावे लागते. त्यामुळे हुशार विद्यार्थ्यांना कंटाळा येतो, तर काहींना शिकवलेले डोक्यावरून जाते. यावर उपाय म्हणजे एआयच्या मदतीने वैयक्तिकृत (पर्सनलाईझ्ड) शिक्षण. आकलनाच्या गतीनुसार धडे-संकल्पनांची क्लिष्टता कमी-जास्त करता येते. एआय आधारित असे वैयक्तिक मार्गदर्शक उच्च-नीच, गरीब-श्रीमंत ह्या दऱ्या जागतिक स्तरावर कमी करत आहेत. शिक्षण सर्वांसाठी माफक दरात उपलब्ध होत आहे.  नवीन भाषा शिकण्यासाठी ड्यूओ लिंगोसारखे असंख्य मंच उपलब्ध झाले आहेत, घरबसल्या व फुकट वापरण्यासाठी. सुप्रसिद्ध `खान अकॅडमी' सुद्धा एआयचा चॅटबॉट स्वरूपात उपलब्ध आहे. त्याची विशेषता ही की विद्यार्थ्यांनी विचारलेल्या प्रश्नांना तडक उत्तर न देता (मग?) काही संकेत (हिंट) देऊन उत्तराकडे वळवले जाते. याने जास्त चांगले शिक्षण होते. हे एआय-रुपी शिक्षक तुमच्या प्रश्नांना उत्तर द्यायला कायम २४x७ उपलब्ध असतात, आणि चुका झाल्या तरी `छडी लगे छमछम' नाही!!  समजा, पाठ्यपुस्तकात एखादे इंजिन कसे चालते याची माहिती आहे, पण नुसत्या चित्रावरून कितीसे कळणार. तेथेच क्यू-आर कोड दिला असेल तर ऍपमध्ये त्या इंजिनचे चलन (ऍनिमेशन) दिसू शकते. एवढेच नाही तर काही शैक्षणिक प्रयोग चक्क आभासी प्रयोगशाळेत (वर्चुअल लॅब) मध्ये करता येतात. पुण्यातील शासकीय अभियांत्रिकी विद्यालयाने (सीओईपी) हा उपक्रम केला आहे. फायदा फक्त विद्यार्थ्यांनाच नाही, तर शिक्षकांनाही आहे. उत्तरपत्रिका तपासण्यासारखे जिकिरीचे काम नाही. एकेकाची अक्षरे, बहुरंगी भाषा पाहून सहनशीलता पणाला लागू शकते आणि एवढे करूनही माझा इथे १ मार्क का कापला असे विचारायला विद्यार्थी तयार. एआय-आधारित प्रणाली या तपासणीचे काम सोपे करू शकतात. या प्रणाली हस्ताक्षर ओळखतात, उत्तरे तपासतात आणि निबंधांनाही यथोचित गुण देतात. मार्कांचा अभ्यास करून विद्यार्थ्यांना कोठला विषय कळला नाहीये याचे अनुमान काढून तो पुन्हा शिकवता येऊ शकतो. फक्त उत्तरे तपासण्यातच नाही तर प्रश्न पत्रिका काढण्यातही एआयचा प्रभावी वापर होतो. काही धडे देऊन, पाहिजे असलेल्या क्लिष्टतेची प्रश्नपत्रिका तयार मिळू शकते. एआयची अजून एक मदत म्हणजे शैक्षणिक मजकूर, नोट्स बनवणे, स्लाईड्स तयार करणे इत्यादी.

संशोधन करणाऱ्या विद्यार्थ्यांसाठी `पेरप्लेक्सिटी' आणि `चॅटजीपीटी' आधारित `डीप रिसर्च' प्रणाली वरदान ठरत आहेत. यामुळे केवळ साहित्य सर्वेक्षण(लिटरेचर सर्वे) नव्हे, तर नवीन संशोधन पेपर लिहिण्यातही मोठी मदत होते. एवढे सगळे फायदे दिसत असताना काही तोटेसुद्धा समोर येत आहेत. शिक्षकांना भीती नोकरी जाण्याची. पण ते खूप खरे नाही. मानवी शिकवण्याला तोड नाही, तेही हाडाचा शिक्षक असेल तर. आठवा तुमचे आवडते शिक्षक. त्यांच्याकडून विद्यार्थी केवळ विषयच शिकत नसतो तर त्याला घडवण्याचे काम होत असते. ते मात्र अजून एआयला जमलेले नाही.