\chapter{सामर्थ्य आहे सूचक-प्रश्नांचे!!}

{\textit{चॅटजीपीटीचा सध्या बोलबाला आहे.  त्याची उपयुक्तता काय,  इथपासून ते त्यामुळे नोकऱ्या जातील का,  इथपर्यंत अनेक प्रश्न विचारले जात आहेत.  त्यावरील उत्तरेही वेगवेगळी आहेत.  पण सर्जनशीलता ही श्रेष्ठ असते हेच खरे .  }}

\vspace{1.5em}

चॅटजिपीटीने सध्या इंटरनेट वर नुसता धुमाकूळ घातला आहे. बहुतांश लोकं त्याच्या संभाषण क्षमतेने अवाक झाले आहेत. आणि का नाही होणार? जसा एखादा सुविद्य-ज्ञानी उत्तर देईल तशी, व्याकरण-योग्य उत्तरे दिलेली पहिली की त्याचं मागे असललेली 'कृत्रिम बुद्धिमत्ता' (एआय -आर्टिफिशिअल इंटेलिजन्स) हे क्षेत्र किती पुढे आले आहे याची कल्पना येते. आत्तापर्यंत तुम्ही चॅटजिपीटी विषयी इंटरनेट वर, लेखांमधून सर्वसाधारण माहिती वाचली असेलच. आपल्यापैकी काहींनी हे वापरूनही पहिले असेल. त्याविषयी जरा सखोल विवेचन पुढे पाहू.

चॅटजिपीटी ही एक संभाषण-प्रणाली (चॅटबॉट) आहे. आपण प्रश्न विचारायचे, त्याने त्याची समर्पक उत्तरे द्यायची, अगदी दोन व्यक्तीमध्ये संभाषण होते त्याप्रमाणे. आत्तापर्यंत हे बघितले नसेल तर चॅटजिपीटी https://chat.openai.com/chat या संकेतस्थळावर अगदी फुकट उपलब्ध आहे. नक्की वापरून बघा!!

चॅटजिपीटी कसा चालतो?
चॅटजिपीटीच्या मागे जे तंत्रज्ञान आहे त्याचे नाव भाषा-प्रारूप (एल एम, लँगवेज मॉडेल ) असे आहे. हे तंत्रज्ञान समजण्यासाठी त्याला साधर्म्य असणारे आणि रोजच्या व्यवहारातील एक उदाहरण पाहू. आपण मोबाईलवर जेंव्हा संदेश (एसेमेस) टाईप करतो, तेंव्हा आपल्याही पुढचा शब्द सुचवला जातो. आणि बऱ्याच वेळेला ते इतके चपखल असतात की आपल्याला वाटू शकते की मोबाइल आपल्या मनातले कसे ताडु शकतोय? या मागचे तंत्रज्ञान समजायला सोप्पे आहे. तुम्ही सध्या जे टाईप करत आहेत तसे तुम्ही कधी टाईप केले आहे का ते तो पाहतो आणि तसे पूर्वीचे संदेश पाहून त्याला समजू शकते की पुढे कुठला शब्द येणार आहे. समजा , तुम्ही टाईप करताय ``लेट्स गो फॉर अ ----''. अश्या प्रकारचे संदेश तुम्ही अनेक वेळेला पाठवले आहेत. त्या सर्वांमध्ये पुढचा शब्द, सर्वात जास्त वेळेला, 'कॉफी' असा असेल तर तो 'कॉफी' हा शब्द सुचवेल. अशा प्रकारे, 'आत्तापर्यंत टाईप केलेले शब्द पाहून पुढील शब्द सुचवणे' याला भाषा-प्रारूप म्हणतात. एसेमेसचे हे उदाहरण तुमच्या पूर्वीच्या १००-२०० संदेशांवर (डेटा) आधारित पुढील शब्द सुचवतो. जर डेटा जास्त असेल तर उत्तर अधिक अचूक येण्याची शक्यता जास्त. सध्या प्रचलित असलेली बृहत भाषा प्रारूपे (लार्ज लँगवेज मॉडेल) फारच मोठ्या डेटा स्रोतांवर आधारित असल्याने जास्त प्रभावशाली आणि अचूक असतात. लार्ज लँगवेज मॉडेलचा वापर करून, एका मागून एक असे शब्द घेऊन त्याचे बनते वाक्य, वाक्यांचे परिच्छेद आणि परिच्छेदांचे लेख. अश्या प्रकारची भाषा-निर्मिती (जनरेटिव्ह एआय) चा भाग आहे.

चॅटजिपीटीमधील 'चॅट' म्हणजे संभाषण-गप्पा. जिपीटी मधील 'जि' म्हणजे जनरेटिव्ह, जे शब्द जनरेट/निर्मिती करते, 'पी' म्हणजे 'प्रिट्रेंड', जे मोठ्या डेटा वर आधीच प्रशिक्षित आहे, आणि 'टी' म्हणजे ट्रान्सफॉर्मर, हे न्यूरल नेटवर्कच्या एका आराखड्याचे (आर्किटेक्चर) चे नाव आहे. चॅटजिपीटीहा 'जिपीटी ३.५' या लार्ज लँगवेज मॉडेलवर बेतलेला आहे. त्याला अनेक मानवी प्रश्नोत्तरे भारावून संभाषणासाठी खास प्रशिक्षित (फाईन ट्यून) केलेला आहे. 'जिपीटी ३.५' हे एक क्लिष्ट आणि अजस्त्र न्यूरल नेटवर्क आहे. त्यात १७५ अब्ज घटक (पॅरामीटर) आहेत. ते जगभरातील बऱ्याचशा भाषा स्रोतांवर (विकिपीडिया, पुस्तके, इ.) प्रशिक्षीत केला आहे. त्यामुळे तो जणू एक महान भाषा सर्वज्ञच(!) झाला आहे. या विस्तृत प्रशिक्षणामुळेच त्याला सर्वज्ञानातील (जनरल नॉलेज) कोणताही प्रश्न विचारला तरी त्याच्याकडे त्याची काही ना काही माहिती असते.

पण चॅटजिपीटी खरंच सर्वज्ञ आहे का? तर उत्तर आहे, नाही!! त्याचे प्रशिक्षण जरा जुन्या, म्हणजे २०२१ पर्यंतच्या डेटा वर झाले आहे. त्याला अगदी नजीकच्या घटनांविषयी काही विचारले तर तो सरळ 'सॉरी' म्हणतो. अजून एक मर्यादा म्हणजे त्याचे प्रशिक्षण सार्वजनिक डेटा वरच झाले असल्याने त्याला तुमच्या किंवा कोणाच्याही वैयक्तिक माहितीची कल्पना नाही. तो फक्त त्याला प्रशिक्षित केली माहिती योग्य प्रकारे तुमच्या पुढे ठेवतो. आता कधी कधी काही डेटा मध्ये चुकीची माहिती असू शकते, तर तो तीही माहिती छातीठोकपणे सांगतो. म्हणूनच चॅटजिपीटीवर पूर्णपणे अवलंबून राहता येत नाही. त्याने दिलेल्या उत्तरचे परीक्षण हे करणे कधीही श्रेयस्कर. असे काही नकारात्मक पैलू आणि मर्यादा असल्यातरी चॅटजिपीटी ही एक भन्नाट गोष्ट आहे, हे मान्यच करावे लागेल.

सूचक-प्रश्न म्हणजे काय?
चॅटजिपीटीची खरी क्षमता जोखायची असेल, त्याच्याकडून पाहिजे ते आणि पाहिजे तसे उत्तर काढायचे असेल तर त्याला प्रश्नही (प्रॉम्प्ट) फार अचूक आणि सूचक विचारावे लागतात. प्रॉम्प्ट अघळपघळ तर उत्तरही तसेच. पण जरा सटीक प्रश्न, ज्यात हे हवे हे नको असे सांगितले असेल, तर बरोब्बर उत्तर मिळण्याची शक्यता वाढते. सूचक-प्रश्न विचारणे हेच आता नवीन शास्त्र म्हणून उदयास येत आहे. त्या विद्या-शाखेचे नाव प्रॉम्प्ट अभियांत्रिकी. चॅटजिपीटी काय किंवा तत्सम इतर लार्ज लँगवेज मॉडेल वापरण्यासाठी प्रॉम्प्ट अभियांत्रिकीचे कौशल्य गरजेचे ठरणार आहे. अश्या प्रकारच्या नोकऱ्या आता उपलब्ध सुद्धा व्हायला लागल्या आहेत. भविष्यात त्याला अधिकच वाव मिळणार. चांगला प्रॉम्प्ट अभियंता होण्यासाठी त्याला स्वतःच्या नैपुण्याचे एक क्षेत्र तर असावेच लागणार पण त्याच बरोबर लार्ज लँगवेज मॉडेल आणि एआय चे ज्ञानही आवश्यक ठरणार असेल. एखाद्या यंत्र अभियंत्याला दुचाकीचे इंजिन विषयी काही माहिती पाहिजे असेल तर सूचक-प्रश्न विचारण्यासाठी त्याला त्यातील परवलीचे शब्द, खाचा-खोचा पण आधीच माहिती पाहिजेत. अश्या प्रकारचे संयुक्त ज्ञान सर्व शाखांना गरजेचे ठरणार आहे.

चॅटजिपीटीचे परिणाम काय?
चॅटजिपीटी एवढे प्रभावशाली असेल तर त्यामुळे आपल्या नोकऱ्यांवर गदा येऊ शकते का? तर त्याचे उत्तर 'होय' असेच आहे. अगदी सगळ्या नाही तरी काही, नक्की. जे तेच-तेच लेख लिहितात, त्याच विषयांवर निबंध लिहावे लागतात, तश्याच ई-मेल, तसेच प्रस्ताव, अश्या पाट्या टाकणाऱ्या नोकऱ्यांवर परिणाम होऊ शकतो. पण ज्यांचे काम सृजनशील, नवा-निर्मिती-पूर्ण आहे त्यांना चिंता करण्याची गरज नाही. लार्ज लँगवेज मॉडेलना हरवणे आत्ता तरी दुरापास्त वाटतंय.

चॅटजिपीटीचा बोलबाला चालू असतानाच त्याच्या जनकांनी पुढील लार्ज लँगवेज मॉडेलची म्हणजे 'जिपीटी ४' ची घोषणा केली आहे. गुगल सारख्या इतर बलाढ्य कंपन्याही या भाषा-युद्धात हिरीरीने उतरल्या आहेत. कोणाची सरशी होईल हे सांगणे कठीणच, पण आपल्या हातात एवढेच आहे की चॅटजिपीटीसारख्या शोधांचा आपल्या कामात पुरेपूर पण सजग वापर करणे आणि लक्ष असे ठेवणे की आपले काम असे असेल की ते चॅटजिपीटीला करणे अशक्य असेल !!