\chapter{वित्तीय सेवेसाही हजर !}

रात्री अचानक मोबाईलवर एक संदेश (नोटिफिकेशन) आला –‘तुमच्या खात्यातून आताच एक व्यवहार झाला आहे, तो संशयित वाटत आहे, तो तुम्हीच केला आहे का?’ पाहतो तर काय! व्यवहारातील समोरच्या पक्षाचे नाव अनोळखी, ठिकाण दूरवरचे. लगेच व्यवहार रद्द करण्याच्या सूचना दिल्या आणि मोठा सुस्कारा सोडला. अशा संशयित किंवा गैरव्यवहारांची ओळख पटवणे आता ‘एआय’मुळे शक्य झाले आहे. जगभरात सेकंदाला लाखो व्यवहार होत असतात, त्यातून फसव्या गोष्टी खड्यासारख्या बाजूला काढणे मानवाला खूप अवघड (मुश्किल ही नही, नामूमकिन है) आहे ते काम एआय लीलया करते. अशा प्रकारे वित्तक्षेत्रात एआय अनेक कामांसाठी वापरता येते ते पाहुयात.

वित्तीय सेवा म्हटले की सर्वप्रथम बँक आठवते. बँकेचे मुख्य काम म्हणजे ठेवी स्वीकारणे आणि त्याचा वापर करून इतरांना कर्ज देणे. हे कर्ज देताना फार काळजी घ्यावी लागते की ते बुडणार नाही याची. सगळेच कर्जदार काही बँकेतील लोकांच्या परिचयाचे-घरोब्याचे नसतात मग ठरवायचे कसे की कर्ज मागणारा अनोळखी मनुष्य ते वेळेत फेडेल की नाही? काही ठोकताळे अनुभवाने शिकता येतात पण ते फार प्रभावी नसतात. येथे एआय मदतीला येते. पूर्वीच्या असंख्य कर्ज व्यवहारांचा अभ्यास करून एआय कर्ज-पात्रतेचे प्रारूप (मॉडेल) तयार करतो. यामध्ये आर्थिक माहितीबरोबरच कौटुंबिक पार्श्वभूमी, खर्चाच्या सवयी यांचाही विचार होतो. हे मॉडेल कर्ज मंजुरीसाठी निर्णायक ठरते. त्यामुळे स्पष्टीकरणक्षमता (एक्स्प्लेनबिलिटी) वाढवण्याचा प्रयत्न केला जात आहे, म्हणजे कर्ज नाकारल्यास त्यामागचे कारणही स्पष्ट करता येईल.

यंत्रमानवी सल्लागार
पूर्वी गुंतवणूक म्हटली की बँकेत किंवा पोस्टात मुदत-ठेव ठेवणे हा मुख्य मार्ग असायचा. आता लोक शेअर बाजार आणि म्युच्युअल फंडांकडे वळले आहेत. एखाद्या शेअरचा भाव वाढेल की घटेल, हे ठरवण्यासाठी तज्ज्ञ कंपनीच्या स्थितीचा अभ्यास करतात आणि गणितीय तंत्रे वापरतात. आता एआय आधारित प्रणाल्यांमुळे (अल्गोरिदमिक ट्रेडिंग) हे अधिक प्रगत झाले आहे. ‘वॉल स्ट्रीट’मध्ये हे मोठ्या प्रमाणावर वापरले जाते. क्षणार्धात मोठे खरेदी-विक्री व्यवहार स्वयंचलित होतात. जेम्स सायमन्स यांच्या ‘रेनेसाँस’ कंपनीने यात विशेष प्रगती केली आहे. भारतातही मोठ्या कंपन्यांपासून ते स्टार्टअप्सपर्यंत अनेकजण शेअर बाजारामध्ये एआयचा वापर करत आहेत. व्यक्ती तितक्या प्रकृती असल्याने प्रत्येकाची गुंतवणूक (इन्व्हेस्टमेंट) सारखी नसते. प्रत्येक गुंतवणूकदाराची मानसिकता वेगळी असते. वय, आर्थिक स्थैर्य आणि जोखीम घेण्याची क्षमता (रिस्क ऍपेटाईट) यानुसार गुंतवणुकीचे निर्णय घेतले जातात. मानवी गुंतवणूक सल्लागारांप्रमाणेच एआयसुद्धा मदत करू शकतो. हजारो पर्यायांचा अभ्यास करून तो योग्य पर्याय सुचवतो. केवळ फिक्स्ड डिपॉझिट नव्हे, तर शेअर बाजार, क्रेडिट कार्ड्स, सोने-चांदी यांसारख्या गुंतवणुकीचे पर्यायही सुचवले जातात. त्यांना यंत्रमानवी सल्लागारच म्हणतात. ‘बेटरमेंट’, ‘वेल्थफ्रंट’ सारख्या अनेक कंपन्या यासाठी प्रसिद्ध आहेत.

‘‘भविष्य निर्वाह निधीतून मुदतपूर्व पैसे काढता येतात का? कोणत्या अटींवर?’’ असे प्रश्न पडल्यास गुगलवर शोधण्याऐवजी एआयवर आधारित चॅटबॉटला विचारणे सोयीचे ठरते. समजले नाही, तर पुन्हा विचारता येते, वेगवेगळे पैलू समजावून घेता येतात. येथे एआयचा वापर प्रभावी होतो. चॅटजिपीटी सारख्या बृहत-भाषा-प्रारूपावर आधारित अनेक चॅटबॉट आपल्याला दिसतात, विशेषत: बँकांच्या संकेत-स्थळांवर, तेही २४ x ७ तास प्रश्नांची उत्तरे देतात, कंटाळा ना करता, कोठलीही सुट्टी (लंच -ब्रेक) ना घेता!!

‘एआय’चे अनेक फायदे असले तरी काही धोकेही आहेत. आर्थिक माहिती गोपनीय असली पाहिजे, पण ती कधीकधी परवानगीशिवाय मॉडेल्स प्रशिक्षणासाठी वापरली जाऊ शकते. काही एआय अल्गोरिदम्समध्ये पूर्वग्रह आढळतो. विशिष्ट गटांतील लोकांना कर्ज मंजूर होते, तर काहींना नाकारले जाते, जरी ते आर्थिकदृष्ट्या सक्षम असले तरी. त्यामुळे पारदर्शकता आणि स्पष्टीकरणक्षमता वाढवण्यासाठी उपाययोजना सुरू आहेत. कायद्यानेही त्यावर बंधने आणली आहेत.

शेवटी, जो पर्यंत जग आहे तो पर्यंत पैसा आणि त्याचे महत्व कायम आहे आणि ते तसे आहे तोवर एआय आपल्यासाठी ‘लाख’मोलाचे ठरणार आहे.