\chapter{शेतीच्या समस्यांवरील अत्याधुनिक उतारा}

मायक्रोसॉफ्टचे प्रमुख सत्य नाडेला यांनी त्यांच्या भाषणात बारामतीतील एका अभिनव उपक्रमाचा विशेष उल्लेख केला होता. ऊस उत्पादनासाठी एआय कशा प्रकारे उपयोगात आणली गेली, हे त्यांनी नमूद केले. `फार्म ऑफ द फ्युचर' (भविष्यातील शेती) या उपक्रमांतर्गत सुमारे १६ लाख स्थानिक शेतकरी जोडले गेले आहेत. उदाहरणार्थ आता, कमी पाण्यात, कमी जागेत आणि कमी खर्चात अधिक गोडसर ऊस उत्पादित करण्याचे उद्दिष्ट एआयच्या मदतीने साध्य झाले आहे. इतर अनेक क्षेत्रांप्रमाणेच कृषी क्षेत्रातसुद्धा एआयने मुसंडी मारली आहे.

कल्पना करा, पाण्याचा प्रत्येक थेंब, खताचा प्रत्येक कण, बियाण्याचा प्रत्येक दाणा अत्यंत कार्यक्षमतेने वापरला गेला तर? एआयचे लक्ष्य हेच आहे. पारंपरिक शेतीत आधुनिक तंत्रज्ञानाचा समावेश करून उत्पादन वाढवणे व शेतकऱ्यांचे जीवनमान सुधारणे हा उद्देश आहे. शेती हा जोखमीचा व बेभरवशाचा व्यवसाय मानला जातो. हवामान, खत-बियाण्यांची उपलब्धता आणि बाजारभावातील चढ-उतार यांवर शेतीचा डोलारा सांभाळावा लागतो. पण एआयचा वापर करून विविध मार्गाने यातील  आव्हानांचा यशस्वी सामना करणे शक्य आहे, ते पाहुयात.

`स्मार्ट' शेतीत खुरपणी, पेरणी, पाणीपुरवठा, कापणी यांसारखी कामे स्वयंचलित पद्धतीने होऊ शकतात. ज्या देशांत मोठ्या प्रमाणावर जमीन आहे (उदा. अमेरिका), तेथे यंत्रांचा वापर जास्त किफायतशीर ठरतो. मानवी चालढकल, चुका टाळल्या जातात आणि उत्पादकता वाढते. उपग्रहांद्वारे हवामानाचा अचूक अंदाज घेता येतो. संवेदक (सेन्सर्स) जमिनीतील आर्द्रता आणि घटक मोजतात. ते वापरून एआयच्या मदतीने हवामानानुसार पाण्याचे अचूक प्रमाण ठरवता येते. यामुळे पाण्याची तब्बल ३० टक्के बचत करून दुप्पट पीक घेणे आता अशक्य नाही.

पीकांवर लागणारी कीड ही मोठी समस्या आहे. ड्रोन, उपग्रहाद्वारे मिळालेल्या चित्रांवरून कीड लवकर ओळखता येते. औषधफवारणी वेळेवर करता येते, यामुळे नुकसान कमी होते. आंध्र प्रदेशातील शेतकऱ्यांनी बुरशीचा प्रादुर्भाव रोखण्यासाठी याचा यशस्वी वापर केला आहे. भारत सरकारच्या `नॅशनल पेस्ट सर्व्हेलन्स सिस्टीम' (राष्ट्रीय कीड संनिरीक्षण प्रणाली) शेतकऱ्यांना कीड-प्रसाराबाबत सतर्क करण्याचे काम करते. बदलते हवामान ही अजून एक मोठी समस्या आहे. देशातील काही भागात पाणलोटाची ठीकठाक सोय असली तरी अर्ध्याअधिक शेतकऱ्यांना निसर्गावरच अवलंबून राहावे लागते. दुष्काळ प्रतिरोधक पिकांचे, विविध कामांचे वेळापत्रक एआयद्वारे तयार करता येते. राजस्थानातील शेतकऱ्यांनी भरडधान्य पिकासाठी अशा सल्ल्याचा उपयोग केला आहे.

भूतकाळातील उत्पादनांचा डेटा (माहिती, विदा), हवामान, मातीचे गुणधर्म यांचा अभ्यास करून एआय उत्पादनाचा अंदाज देऊ शकते. याचा आधार घेऊन शेतकरी पीक कधी बाजारात आणायचे हे ठरवू शकतो. शेतमालाला भाव कधी चांगला मिळेल, तोपर्यंत साठवणुकीची काय सोय करता येईल यांचे चांगले नियोजन करता येते. मालाचा अपव्यय कमीत कमी करता येते. मागणी व पुरवठा याचा बरोबर मेळ घालून दोन्ही पक्षांना (शेतकरी आणि व्यापारी-ग्राहक) फायदा होईल असे सौदे सुचवता येतात. गावातील, राज्यातील, देशातीलच नाही तर जगभरातील चालू असणाऱ्या व्यवहारांची माहिती मिळवणे शक्य असल्याने कोठे सर्वाधिक फायदा होईल याचा अंदाज बांधता येतो. शेती शास्त्रज्ञांना, सल्लागारांना व अधिकाऱ्यांना प्रत्येक शेतकऱ्याबरोबर चर्चा करून त्यांचे समाधान करणे शक्य होईलच असे नाही. शेतीत वैज्ञानिक मदत पुरवणाऱ्या संवाद प्रणाली (उदा. `किसान ई-मित्र' सारखे चॅटबॉट्स) उपलब्ध आहेत. या प्रणाली शेतीविषयक प्रश्नांची उत्तरे देतात, तेही विविध भारतीय भाषांमध्ये.

एआयच्या मदतीने शेतीत आमूलाग्र बदल घडत आहेत. शेतकरी सक्षम होत आहेत. अन्नदाता समृद्ध झाला तर देशही समृद्ध होतो. यासाठी सरकार व खाजगी कंपन्यांनी संशोधन आणि एआय तंत्रज्ञनाच्या उपाययोजनांवर भर द्यायला हवा. शेतकऱ्यांनीही डिजिटल साक्षरता आत्मसात करून तंत्रज्ञानाचा लाभ घ्यावा, जेणेकरून शेती हा ही एक निरंतर फायदेशीर व्यवसाय होईल.