\chapter{मी आणि माझा डेटा}

{\textit{मी आणि माझा डेटा हे द्वैत हळूहळू संपुष्टात येणार आहे .  जेवढा डिजिटल तंत्रज्ञानाचा वापर वाढेल तेवढी आपल्याबद्दलची माहिती डिजिटल - डेटा स्वरूपात होणार आहे. तुम्ही आणि तुमचा डेटा  यात खूप फारकत नसणार आहे. जणू डेटा-अद्वैताचा प्रवास सुरू झाला आहे .  त्या प्रवासात आपल्याला सजग राहावे लागणार आहे . }}

\vspace{1.5em}

नुकत्याच वाचनात आलेल्या बातमीत एक  `कोवीन' या भारताच्या लसीकरण्याच्या माहिती (डेटा) तथाकथित चौर्यकर्माच्या शक्यतेविषयी होती तर दुसऱ्या बातमीमध्ये जपानने सर्व सार्वजनिक डेटा एआय'च्या प्रशिक्षणासाठी मुक्त करण्याचा विचार करीत असल्याचे म्हटले होते. वरकरणी जरी दोन्ही बातम्या वेगळ्या वाटत असल्या तरी त्यांचे मूळ हे खाजगी-संरक्षित डेटा विषयी आहे. माझी माहिती, माझे लेख, सार्वजनिक आणि कोणीही वापरण्यास खुले झाले आहेत का? हे चुकीचे वाटत असले (आणि ते आहेच) तरी याविषयाला अनेक पैलू आहेत. मुख्य मुद्दा असा आहे कि माझा डेटा कोणी, कसा आणि माझ्या परवानगीने (किंवा शिवाय) वापरणे रास्त आहे का?

सध्याचे युग तंत्रज्ञानाचे आणि त्यातही एआयचे मानावे लागेल. दररोजच्या जीवनात एआयचा वापर वाढतच चालला आहे. प्रचलित एआय हे प्रामुख्याने डेटा वर अवलंबून असते. हा डेटा सार्वजनिक किंवा खाजगी असू शकतो. जसे, सध्या सुप्रसिद्ध झालेले चॅटजिपीटी हे संभाषणाचे ॲप जगभरातील असंख्य सार्वजनिक वेबसाइट्स, लेख, पुस्तके, यांच्या शब्दराशींवर प्रशिक्षित आहे. जेवढा जास्त डेटा तेवढे ॲप अचूक चालते. पण त्याला तुमच्या खासगी डेटा विषयी माहिती नसल्याने तो त्यासंदर्भात उत्तरे देऊ शकत नाही. परंतु काही कंपन्या जास्त अचूकतेसाठी, अधिक डेटा मिळवण्याच्या हव्यासापायी वाममार्गाने तुमचा खाजगी डेटा किंवा तुमची मालकी असलेला डेटा मिळवण्याचा प्रत्यत्न करतात. हा नक्कीच अपराध आहे. असे प्रकार रोखण्यासाठी काय करता येईल  ते पाहू.

आपला डेटा, त्याची गोपनीयता आणि त्याचे संरक्षण हा कळीचा मुद्दा ठरत आहे. हे केले नाही तर कोणते धोके उद्भवू शकतात त्याची माहिती सर्वांना असणे गरजेचे आहे. उदाहरणार्थ , समजा आपण एक `फिटनेस-ॲप' (जसे घड्याळातली आरोग्य प्रणाली) वापरात आहोत. त्यात तुमचे रोजचे व्यायाम, शारीरिक नोंदी, सायकलिंगचा वा पळण्याचा मार्ग, इ. साठवले जात आहेत. आता हा डेटा कोण, कसा वापरत असेल त्याची तुम्हाला कल्पना असते का? आपल्याला वाटते कि आपला डेटा संकलित करून आपल्याला छान सार-चित्र (`डॅशबोर्ड') दाखवणासाठी असेल, पण तसे ते करतीलच याची खात्री नाही. हा डेटा इतर लोकांना पाठवला (विकला) जाऊ शकतो. ॲप `इन्स्टॉल' करीत असताना आपण कळत नकळत आपल्या मोबाईलमधील काय काय पाहण्याची परवानगी दिलेली आहे त्यानुसार फिटनेस सोडून इतर डेटा पण ॲप निर्माते ओढू शकतात. यासाठी ॲपच्या गोपनीयतेच्या परवानग्या (`प्रायव्हसी सेटिंग') तपासावे. गरज नसलेल्या परवानग्या काढून टाकायला हव्यात. ॲपचे निर्माते विश्वासार्ह आहेत हे तपासूनच इन्स्टॉल करावे. नाहीतर आपल्या मोबाईलला किंवा कॉम्पुटरला व्हायरस हल्ल्याचा धोका निर्माण होऊ शकतो.

संभाव्य धोके बघून आपण ठरवले कि इंटरनेट वर काहीच माहिती लिहायची नाही, ई-मेल पाठवायच्याच नाहीत, सोशल मीडिया वापरायचच नाही तर हे ही बरोबर नाही, नाही का? नव-नवीन कल्पनांना, नवीन एआय च्या प्रणालींना डेटाची गरज असते. ते थांबले तर त्याची प्रगती पण रोडावेल. याला उपाय म्हणजे डेटा संरक्षण आणि सजग संमती. कोणता डेटा खरंच खाजगी आहे आणि काय इतरांनी किंवा एआय ने पहिले तर चालेल याचा नीर-क्षीर विवेक आपल्याला पाहिजे. आपला डेटा सुरक्षितपणे आणि सांगितलेल्या कामाकरिताच वापरला जातोय का याची खात्री करून घ्यावी लागेल. कोणत्याही ॲपने, वेबसाईटने, संगणक प्रणालीने आपला डेटा घेताना आपली सजग (मूक किंवा छुपी नाही) संमती विचाराने-घेणे आवश्यक ठरते. इन्स्टॉल करण्याच्या वेळेस दाखवले जाणारे संमतीपत्र (लायसन्स ऍग्रीमेंट) न पाहताच होकार (ऍग्री) म्हणणे टाळले पाहिजे. सर्वसाधारणपणे मोठ्या कंपन्या डेटाचा दुरुपयोग करणार नाहीत अशी आशा ठेवणे भाबडेपणाचे ठरते. तेव्हा डोळ्यात तेल घालून आपल्या डेटाचा वापर कोठे आणि कसा होतोय यावर नजर असली पाहिजे. क्लिष्ट आणि कटकटीचे वाटले तरी.

आपला डेटा कोठे कोठे असतो याची सर्वमान्यांना कल्पनाही नसते. ढोबळमानाने मोबाईल किंवा कॉम्पुटर वरून जर काही काम केले, काही माहिती पाठवली, मागविली, इंटरनेट वर टाकली, तर तेवढाच डेटा असतो असा समज आहे. तो तर असतोच पण, इंटरनेटला जोडल्या गेलेल्या सर्व गोष्टी डेटा संकलित करू शकतात. रस्त्यावरून जाताना वेबकॅम आपले चित्र काढू शकतो, खरेदीसाठी कार्ड वापरले तर तो पण डेटा असतो, रुग्णालयातील तपासणीच्या नोंदी, अगदी अंगठ्याचे ठसे इलेक्ट्रॉनिक पद्धतीने घेतल्यास तोही आपला डेटाच असतो. हा डेटा घेणाऱ्या व्यक्ती, दुकाने, ॲप, सरकारे इ .आपल्या डेटा चे काय करणार आहेत याचीपण माहिती आपल्याला असली पाहिजे.

डेटा'चा उपयोग खरेपणाने केल्यास तो चांगल्यासाठी ठरू शकतो. एखाद्या वेबसाइटवर तुम्ही पूर्वी केलेल्या वस्तूंची माहिती पाहून, एआय प्रणाली आपल्याला आवडू शकणाऱ्या वस्तू अचूक सुचवू शकते. एक्स रे किंवा इतर स्कॅन पाहून रोगनिदान करणे एआयला, त्यानी पाहिलेल्या असंख्यजणांच्या डेटा वरूनच शक्य होते. चॅटजिपीटी सारखी प्रणाली अगदी मानवासारखे उत्तर देऊ शकते त्याला कारणच की त्याला मिळालेला आपल्या सर्वांचा सार्वजनिक डेटा. त्यामुळे आपला डेटा द्यायचाच नाही किंवा सर्व डेटा उपलब्ध करून द्यायचा ह्या दोन्ही टोकाच्या भूमिका झाल्या. सतर्कपणे एक मध्यम मार्ग निवडायला हवा. भारताप्रमाणे अनेक देशांनी वेगवेगळ्या पद्धतीने डेटा संरक्षण आणि गोपनीयतेविषयी नियम आणि कायदे आणले आहेत. युरोपातील जीडीपीआर (जनरल डेटा प्रोटेक्शन रेग्युलेशन ) यासारखे कायदे वैयक्तिक डेटा संरक्षित करण्याचा प्रयत्न करीत आहेत त्यानुसार ॲप निर्मात्यांना, कंपन्यांना त्यानुसार आपल्या डेटा वापराविषयी बदल करून अधिक सुरक्षित करण्यास भाग पडले जात आहे. असे असले तरीही आपण गाफील न राहता आपल्या दैनंदिन जीवनात डेटाविषयी जागरूक राहणे गरजेचे आहे.

मी आणि माझा डेटा हे द्वैत हळूहळू संपुष्टात येणार आहे. जेवढा डिजिटल तंत्रज्ञानाचा वापर वाढेल तेवढी आपल्याबद्दची माहिती डिजिटल-डेटा स्वरूपात उपलब्ध होणार आहे. तुम्ही आणि तुमचा डेटा यात खूप फारकत नसणार आहे, जणू, डेटा-अद्वैताचा प्रवास सुरु झाला आहे. हे मार्गक्रमण सुकर आणि फलदायी होण्यासाठी सतर्कतेची आणि सजगतेची आवश्यकता आहे नाहीतर हा मार्ग खाच-खळग्यांचा व धोक्याचा बनू शकतो. तर, सावधान!!