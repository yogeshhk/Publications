\chapter{‘एआय’ला नैतिकतेचा लगाम}

बिझिनेस २०’च्या कार्यक्रमात पंतप्रधान श्री नरेंद्र मोदी यांनी सध्याच्या बहुचर्चित एआय (आर्टिफिशियल इंटेलिजन्स, कृत्रिम बुद्धिमत्ता) च्या भविष्यातील विकासासंदर्भात मार्गदर्शन केले. एआय-प्रगतीचा वारू चौखूर उधळत आहे आणि त्याचे फायदेही आपल्याला मिळत असले तरी त्याला नियमांचं लगाम घालण्याची वेळ आली आहे अश्या आशयाचे आवाहन त्यांनी केले. कोठल्याही तंत्रज्ञानाचा बरे-वाईटपणा प्रामुख्याने त्याच्या वापरकर्त्यावर, त्यांच्या मूल्यांवर आणि नैतिकतेवर अवलंबून असतो. म्हणूनच एआयचाही वापर चांगल्या, नैतिक आणि विधायक कामाकरिताच होईल याच्या खात्रीसाठी जागतिक स्तरावर नियमन आराखडा (रेग्युलेटरी फ्रेमवर्क) बनवण्याचे आवाहन त्यांनी केले, जेणेकरून एआयचा प्रवास, नैतिकतेच्या मार्गानेच होईल. म्हणूनच सध्या ‘एथिकल एआय’ (नैतिक कृत्रिम बुद्दीमत्ता) हा एक कळीचा मुद्दा बनला आहे, त्याच्या काही पैलूंविषयी जाणून घेऊयात.

एआय संगणक प्रणाली बनवण्यासाठी प्रचंड प्रमाणात माहिती (डेटा) ची आवश्यकता असते. जसा डेटा तशी एआय प्रणाली. म्हणजेच डेटा जर चांगला तपासलेला, वैविध्यपूर्ण आणि कामाला चपखल बसणारा असेल तर एआय प्रणाली पण उत्तम बनते. पण तसे नसेल तर, म्हणजे, डेटा मध्ये काही पूर्वग्रह असतील, तो एकांगी असेल तर ते वापरून बनलेली एआय प्रणालीसुद्धा तसेच वागेल. लहान मुलांना जसे शिकवू त्या प्रमाणे ते बनतात, अगदी तसेच. उदाहरणार्थ, एखाद्याला कर्ज द्यायचे आहे की नाही, हे ठरवण्यासाठी समजा एक एआय प्रणाली आहे. तिच्या प्रशिक्षणासाठी फक्त उच्च्भ्रू आणि पुरुष खातेदारांचाच डेटा वापरला गेला तर ती इतर वर्गातील खातेदारांना किंवा महिलांना कर्ज देण्याचा निर्णय कसा बरे सुचवेल? डेटामध्ये असा पूर्वग्रह (बायस) नसणे हे एआयच्या नियमनातील एक महत्वाची गरज आहे.

एआय प्रणाली बहुतांशी एक झाकली मूठ (ब्लॅक बॉक्स) असते. त्याच्या आत काय चालले आहे, ती कशावरून एखादा निर्णय घेते आहे हे सहसा समजत नाही. मग वादविवादाचे प्रसंग येतात. वरील उदाहरणानुसार समजा एखाद्याला कर्ज नामंजूर झाले, तर ते का झाले हे कळले पाहिजे. फक्त वैयक्तिक समाधानासाठीच नाही तर कायद्याने सुद्धा हे गरजेचे होत आहे. यालाच ‘विशद’ (एक्सप्लनेबल) एआय असे म्हणता येईल. कोठल्याही निर्णयाची कारणमीमांसा करता येणे हाही एआय-नियमनाचा महत्वाचा भाग आहे.

पूर्वग्रह हा फक्त डेटा मध्येच असतो असे काही नाही, तर काही वेळेस तो संगणक प्रणालीत (अल्गोरिदम) मध्ये मुद्दामून आणला जातो, तेही त्या बनवणाऱ्या कंपनीच्या फायद्यासाठी. सध्या इंटरनेट वरून खरेदी करणे खूप प्रचलित आहे. तेथे तुम्हाला वस्तू सुचवणारी एक एआय प्रणाली कार्यरत असते. ती समजा तुम्हाला ठराविक पद्धतीच्या, ठराविक निर्मात्याच्याच वस्तू सुचवत असेल तर तुम्ही एका अर्थाने फसवलेच जाताय, नाही का? ज्यात त्यांचा फायदा जास्त ती वस्तू पुढे-पुढे केली जाते. अश्याच प्रकारे, व्हिडीओ बघताना, तुम्हाला तासनतास खिळवून ठेवतील असेच व्हिडीओ समोर आणले जातात. अगदी व्यसन व्हावे इतका हा प्रकार भयानक होत चालला आहे. यात वेळ वाया जाणे हे तर आहेच पण ती संगणक प्रणाली कोणाला काय विडिओ सुचवत आहे याचा कधी विचार केला आहे का? लहान मुलांना, त्यांनी पाहू नये, असे सुचवले जाते आहे का? मग हा एक अनैतिकतेचाच भाग नाही का? हा विषय, म्हणजेच ‘संगणक-एआय प्रणालीने केलेले व्यसन’, हेही एआय नियमनात मोडते.

एआय प्रणालीने जबाबदारीने निर्णय दिले पाहिजेत जेणेकरून त्याने आपले नुकसान होणार नाही. यालाच ‘जबाबदार’ (रिस्पॉन्सिबल) एआय म्हणता येईल. एआयचा वापर सध्या फक्त कर्ज देणे, व्हिडीओ किंवा वस्तू सुचवणे एवढाच मर्यादित नाही तर तो जीवनाच्या सर्व अंगाना स्पर्श करतो आहे. अगदी वैद्यकीय सारखा जीवन-मरणाशी निगडित क्षेत्रामध्येही एआयचा वापर मोठ्या प्रमाणात होऊ लागला आहे. रोग निदान, नवीन औषधांची निर्मिती, इत्यादी शाखांमध्ये एआय वापरले जात आहे. येथे समजा एखादा निर्णय चुकीचा दिला गेला तर केवढ्यात पडेल? म्हणूच ‘जबाबदार एआय’ ही कायद्याच्या, नियमनाच्याच दृष्टीनेच नाही तर आपल्या जीवनासाठी सुद्धा एक गंभीर बाब आहे.

एआयमुळे नोकरी जाऊ शकते, शिक्षित-अशिक्षित यातील दरी रुंदावू शकते, या सारख्या समस्या तर आधीच एआयला भेडसावत आहेत त्यात एआय जर अनैतिक कृत्ये करू लागला तर दुष्काळात तेरावा महिना. त्यामुळेच एआय च्या पुढील प्रगतीची पावले फार सावधगिरीतून टाकली पाहिजेत. समर्पक नियमांच्या आधारे ते शक्य होईल.

दोन देशांमधील युद्धे अजूनही काही प्रमाणात दृश्य स्वरूपात, धरती-जल-आकाश येथे, लढली जात असली तरी मोठ्या प्रमाणात, छुप्या पद्धतीने ती इंटरनेट जगतात (सायबर वॉरफेअर) लढली जातात. उद्योगधंदे बंद पाडणे, आर्थिक व्यवहार ठप्प करणे, इत्यादी कामे करण्याची क्षमता या एआय आधारित सायबर आक्रमणात असू शकते. त्याचा बिमोड करण्यासाठीही जागतिक स्तरावर एआय नियमांची गरज आहे.

नजीकच्या काळात अतिप्रसिद्ध पावलेल्या चॅटजिपीटी सारख्या संभाषण-प्रणाल्या कोणत्याही प्रश्नांना अतिशय उत्तम आणि समर्पक उत्तरे देतात. त्यांचा प्रसार आता लेख-कथा लिहिणे, व्यवसायातील नियोजनाच्या सूचना देणे, एवढेच नाही तर, त्या संगणकाच्या प्रणाल्या लिहिण्यात (कोडिंग) मध्ये पण होऊ लागला आहे. अगदी शाळकरी मुले पण गृहपाठ करण्यासाठी वापरू लागले आहेत. तर ह्या मिळत असलेल्या उत्तरांची किती काळजी घेतली पाहिजे याचा आपण अंदाज करू शकतो. ‘बॉम्ब कसा बनवावा?’ या सारख्या प्रश्नांनाही जर उत्तरे मिळायला लागली तर? मोठे आव्हानच आहे या प्रणाली निर्मात्यांच्या समोर. यासाठी पण एआयचे नियमन अतिशय आवश्यक ठरते.

एआय नियमनाची गरज किती महत्वाची आहे हे पाहिले पण ते आणायचे कसे? कोणी? कशा स्वरूपात? हे महत्वाचे प्रश्न राहतात. कोणताही खेळ खेळायचा म्हणजे त्याचे नियम पाळलेच पाहिजेत. रहदारीचे नियम पाळले तरच कोंडी होणार नाही. त्याप्रमाणेच एआयच्या वापराचेपण नियम बनवावे लागतील. ‘नियमन’ म्हणजे फक्त हे करू नका, ते करू नका असे सांगणे नको. पण नक्की काय काय केले पाहिजे, मार्गदशक तत्वे (बेस्ट प्रॅक्टिसेस) काय, उत्तरांनी भरकटू नये व लक्ष्मणरेषा ओलांडू नये यासाठी संगणक-प्रणाली-आधारित-संरक्षक-झापडे (गार्डरेल्स) पण ठरवले पाहिजेत. हे सर्व करताना ते नियम खूप क्लिष्ट आणि शब्द जंजाळ ही नसावेत. कारण, नियम जर कठीण असतील तर ते टाळण्याकडे कल राहील आणि एआयची पुढील प्रगती पण खुंटेल. तर तसे न होऊ देता, सर्वसमावेशक, सर्व भागीदारांच्या सहकार्याने आणि सध्याच्या ‘वसुधैव कुटुम्बकम’ च्या युगात जगद्मान्य असे नियम असले पाहिजेत. त्या दृष्टीने प्रयत्नही सुरु झाले आहेत.

मागील महिन्यात गुगल, मेटा , ओपनएआय आणि मायक्रोसॉफ्ट सारख्या महाबलाढ्य कंपन्यांनी अमेरिकन सरकारसोबत, एआयच्या सुरक्षित विकासाबद्दल करार केला आहे. ‘भारतीय दूरसंचार नियामक प्राधिकरण (ट्राय)’ ने एआयचे नियमन करण्यासाठी “जोखीम-आधारित आराखडा” प्रस्तावित केला आहे. यास्वरुपाच्या अनेक प्रयत्नांच्या फलस्वरूप एआय नियमन लवकरच अजून विस्तृत स्वरूपात अस्तित्वात येईल अशी अशा करूयात. एआयचा वापर अधिक सुरक्षित, समर्पक, अचूक, सन्मार्गाने, नैतिकतेने आणि जबाबदारीने झाला तर सर्वांनाच त्याचा फायदा होणार आहे.