% Copyright page
\thispagestyle{empty}
% \null\vfill

\begin{center}
\includegraphics[width=0.2\linewidth,keepaspectratio]{YHK_Color_OutOfTheBox_tight} \\[1.5em]

\textbf{\Huge एआय-रूपे}\\ [0.5em]
{\small(`कृत्रिम बुद्धिमत्ते'ची 'विविध क्षेत्रातील रूपे)}\\[0.5em]

लेखक: \textbf{{\large डॉ. योगेश हरिभाऊ कुलकर्णी}}\\[1.5em]
\end{center}

\vspace{3.5em}

\begin{flushleft}

प्रकाशक: डॉ. योगेश हरिभाऊ कुलकर्णी (self-published)\\
पत्ता:  पाषाण ,  पुणे ८ \\
फोन:  +91 9890251406\\
ईमेल: yogeshkulkarni@yahoo.com\\[1.5em]

\vspace{3.5em}

प्रथम आवृत्ती: २०२५\\[0.5em]

% ISBN: [ISBN नंबर]\\[0.5em]

कॉपीराइट © २०२५ डॉ. योगेश हरिभाऊ कुलकर्णी\\[0.5em]

{\textit{सर्व हक्क राखीव. या पुस्तकाचा कोणताही भाग प्रकाशकाच्या लेखी परवानगीशिवाय कोणत्याही स्वरूपात पुनर्मुद्रित किंवा पुनर्प्रकाशित करता येणार नाही.  या पुस्तकात व्यक्त केलेली मते लेखकाची व्यक्तिगत आहेत.}}\\[1.5em]

{\large Legal Notice:}\\
{\textit{All rights reserved. No part of this publication may be reproduced, distributed, or transmitted in any form or by any means, including photocopying, recording, or other electronic or mechanical methods, without the prior written permission of the publisher, except in the case of brief quotations embodied in critical reviews and certain other noncommercial uses permitted by copyright law.}}
\end{flushleft}
\vfill\null
\clearpage

\begin{dedication}
मुक्त-स्रोत (`ओपन-सोर्स') चळवळीस  समर्पित  
\end{dedication}

\clearpage

\chapter*{पुनर्मुद्रणाविषयी}

या पुस्तकातील लेख विविध ठिकाणी अगोदरच प्रकाशित झालेले आहेत .  येथे थोड्याफार प्रमाणात संपादन करून केवळ संकलित करण्यात आलेले आहेत.  यातील बरेचसे लेख `मिडीयम'  या संकेत स्थळातील 'देसी स्टॅक' या प्रकाशनात पण उपलब्ध आहेत. 

\vspace{1.5em}
सविस्तर माहिती खालील प्रमाणे:
% \begin{itemize}
	% \item लेख क्रमांक १ : `कृत्रिम बुद्धिमत्ता म्हणजे काय?', महाराष्ट्र टाइम्स, १  मार्च २०१९
	% \item  लेख क्रमांक २ : `यंत्र बुद्धिमत्ता म्हणजे काय?', लिंक्डीन, ३ मे २०२०  
	% \item  लेख क्रमांक ३ - १४ : दै. सकाळ मधील `तिसरा मेंदू' सदर, जानेवारी ते मार्च २०२५
	% \item  लेख क्रमांक १५ - २२ : दै. सकाळ मधील `भाष्य', `विज्ञानवार्ता'  सदर, २०२३ -२०२४. 
	% \item  लेख क्रमांक २३:   नवीन लेख ,  या पुस्तकनिमित्त. 
% \end{itemize}

\begin{itemize}
	\item लेख १: `कृत्रिम बुद्धिमत्ता म्हणजे काय?', महाराष्ट्र टाइम्स, १  मार्च २०१९
	\item  लेख २:  `यंत्र बुद्धिमत्ता म्हणजे काय?',  लिंक्डीन, ३ मे २०२०  
	\item  लेख ३ - १४: दै. सकाळ,  सदर `तिसरा मेंदू', जाने-मार्च २०२५
	\item  लेख १५ - २२: दै. सकाळ, सदर `भाष्य' व `विज्ञानवार्ता', २०२३-२४. 
	\item  लेख २३:   नवीन लेख ,  या पुस्तकनिमित्त. 
\end{itemize}


\chapter*{शीर्षकाविषयी }
% एआय-रूपे यातील 'रूपे '  या शब्दाचे विविध अर्थ - प्रयोजने संभवतात. एक तर ते 'रूप ' या शब्दाचे बहुवचन दाखवते .  '--- रूपे प्रकट झाले '  अशा पद्धतीने एआय चे आगमन विविध क्षेत्रात झाले आहे असे दर्शवता येते. संस्कृत च्या दृष्टीने बघितल्यास टी  'रूप '  शब्दाची 'सप्तमी विभक्ती '  वाटून त्याचा अर्थ ठिकाण - जागेवर असा होतो .  या सर्वांचा परिपाक म्हणून हे शीर्षक . 

``एआय-रूपे'' या शीर्षकात `रूपे' शब्दाचे तीन अर्थ सूचित होतात. एक म्हणजे `रूप' चे बहुवचन. एआय'ची अनेक रूपे. दुसरा अर्थ त्याचे प्रकट होणे, जसे ``--- रूपे प्रकट झाले',  विविध क्षेत्रांत प्रकट होणे. तिसरा अर्थ`रूप '  शब्दाची सप्तमी विभक्ती, म्हणजे `रूपामध्ये'. एआय'च्या माध्यमातून जग पाहण्याचा दृष्टिकोन अधोरेखित होतो. हे शीर्षक एआय'च्या आगमनाचे विविध आयामांतील प्रतिबिंब दाखवते.