\chapter*{पुनर्मुद्रणाविषयी}

या पुस्तकातील लेख विविध ठिकाणी अगोदरच प्रकाशित झालेले आहेत .  येथे थोड्याफार प्रमाणात संपादन करून केवळ संकलित करण्यात आलेले आहेत.  यातील बरेचसे लेख 'मिडीयम '  या इंटरनेट वरील संकेत स्थळातील 'देसी स्टॅक' या प्रकाशनाखाली पण उपलब्ध आहेत. 

सविस्तर माहिती खालील प्रमाणे:
\begin{itemize}
	\item लेख क्रमांक १ : 'कृत्रिम बुद्धिमत्ता म्हणजे काय?', महाराष्ट्र टाइम्स, १  मार्च २०१९
	\item  लेख क्रमांक २ : 'यंत्र बुद्धिमत्ता म्हणजे काय?', लिंकडीन, ३ मे २०२०  
	\item  लेख क्रमांक ३ - १४ : दै. सकाळ मधील 'तिसरा मेंदू' सदर, जानेवारी ते मार्च २०२५
	\item  लेख क्रमांक १५ - २२ : दै. सकाळ मधील 'भाष्य' व 'विज्ञानवार्ता' सदर, २०२३ -२०२४  या काळात . 
\end{itemize}



\chapter*{शीर्षकाविषयी }
% एआय-रूपे यातील 'रूपे '  या शब्दाचे विविध अर्थ - प्रयोजने संभवतात. एक तर ते 'रूप ' या शब्दाचे बहुवचन दाखवते .  '--- रूपे प्रकट झाले '  अशा पद्धतीने एआय चे आगमन विविध क्षेत्रात झाले आहे असे दर्शवता येते. संस्कृत च्या दृष्टीने बघितल्यास टी  'रूप '  शब्दाची 'सप्तमी विभक्ती '  वाटून त्याचा अर्थ ठिकाण - जागेवर असा होतो .  या सर्वांचा परिपाक म्हणून हे शीर्षक . 

"एआय-रूपे" या शीर्षकात "रूपे" शब्दाचे दोन अर्थ सूचित होतात — एक म्हणजे एआयची अनेक रूपं: विविध क्षेत्रांत त्याचे प्रकट होणे, जसे "--- रूपे प्रकट झाले" अशा शैलीत दाखवता येते. दुसरा अर्थ संस्कृतमधील "रूपे", 'रूप '  शब्दाची 'सप्तमी विभक्ती , म्हणजे "रूपामध्ये", ज्यातून एआयच्या माध्यमातून जग पाहण्याचा दृष्टिकोन अधोरेखित होतो. हे शीर्षक एआयच्या आगमनाचे विविध आयामांतील प्रतिबिंब दाखवते.