\chapter{बुडीत-खात्याचा मोह}

माझ्या कॉलेज काळातील एक आठवण आहे. हिंदी चित्रपटसृष्टीत बलवान आणि पीळदार शरीरयष्टीच्या हिरोंचा बोलबाला सुरु झाला होता. साहजिकच तरुणांमध्ये त्याचे अनुकरण करण्याची ओढ-क्रेझ निर्माण झाली होती. माझ्या काडी-पैलवान मित्राला तर ‘सलमान’ व्हायचं वेडच लागल होतं. आई-बाबांच्या मागे लागून लागून त्याने सुमारे ५०००/- (त्या काळाचे) रुपयांचे जिमचे साहित्य आणून घरातील त्याच्या खोलीत एक वैयक्तिक व्यायामशाळाच थाटली होती. काही दिवस उत्साहात तासंतास व्यायाम केला. नंतर जसे शरीराने गाऱ्हाणे  मांडायला सुरवात केली तसे व्यायामाकडे दुर्लक्ष होऊ लागले. नंतर तर त्या सर्व सामुग्रीची अडचणच वाटू लागली. वर्ष उलटून गेलं आणि व्यायाम तर मात्र २ आठवड्यानंतरच बंद झालेला होता. ‘मी त्यात एवढे पैसे घातले आहेत’ या विचारांनी ते सर्व मिळेल त्या किमतीत (म्हणजे ‘अर्ध्या’) विकायची ही इच्छा होईना. आर्थिक नुकसान होणारच होतं. पण ती सगळी सामग्री तशीच ठेवणं हेच खरं नुकसान नव्हतं का? आपण अनेक वेळा काही गोष्टींमध्ये “आधीच एवढं केलंय” म्हणून अडकून पडतो आणि मग परत फिरता येत नाही. या मेंटल मॉडेल (मन:प्रारूप) ला म्हणजेच विचार-चित्राला ‘संक कॉस्ट फॅलसी’ म्हणजेच ‘बुडीत-खात्याचा मोह’ म्हणू शकतो. हा असा मानसिक सापळा आहे जो तुमच्या वैयक्तिक निर्णयांपासून ते सरकारच्या धोरणांपर्यंत सगळीकडे दिसतो.

या विचार-चित्रात, आपण एखाद्या गोष्टीत वेळ, पैसा किंवा श्रम घातल्यामुळे ती सोडणं आपल्याला कठीण जातं, जरी पुढे काहीही फायदा होणार नसलं, तरीही. आपण म्हणतो, "इतकं केलंय, तर आता सोडणं योग्य नाही." पण वास्तवात मागचं नुकसान लक्षात घेऊन पुढचं नुकसान वाढवणं अजूनच घातक ठरतं. उदाहरणार्थ, सरकार एखादा प्रकल्प हाती घेतं. काही कारणांमुळे किंवा नीट नियोजन न केल्यामुळे जनतेकडून अपेक्षित प्रतिसाद मिळत नाही. वापर कमी होतो पण देखभाल खर्च करावाच लागतो. तरीही सरकार तो प्रकल्प बंद करत नाही. कारण? "आधीच एवढा खर्च झालाय!" अशा प्रकारची अनेक उदाहरणं आपल्याला आजूबाजूला दिसतात, ती पाहूयात.

अनेकदा आपण अशा नात्यांमध्ये अडकून पडतो जे मानसिक त्रास, अपमान किंवा सतत दुःख देत असतात. तरीही आपण ती नाती टिकवतो, कारण आपण आधी खूप काही दिलेलं असतं. वेळ, भावना, कधी कधी आर्थिक मदतसुद्धा. "इतकी वर्षं एकत्र आहोत, आता तोडून कसं चालेल?" हा विचारच आपल्याला अडकवून ठेवतो. पण सत्य असं असतं की, चुकीचं नातं जितकं लवकर संपवता येईल, तितकं उरलेलं आयुष्य आनंदात घालवता येईल हा विचार करणे सुद्धा कधी कधी ‘गुन्ह्या’सारखे वाटत असले तरी त्याचा विचार करावा हेच हे विचार-चित्र सुचवतं. 

एखाद्या छोट्या शहरातल्या दुकानात, जिथे ग्राहक येत नाहीत, विक्री होत नाही, तरी दुकानदार म्हणतो की “मी या दुकानात इतका पैसा घातलाय, आता बंद कसं करू?” पण याचा विचार होत नाही की रोज दुकान उघडून ठेवणं, माल आणणं, वीजबिल भरणं, हे सर्व अजून तोटा करत आहेत. कधी कधी जुना व्यवसाय सोडून काही नवीन सुरू करणं हेच शहाणपणाचं असतं.

एखादा चित्रपट बघायला एखाद्या पॉश चित्रपटगृहात जावं आणि नेमका तो चित्रपट जाम बोअर निघावा. वेळ वाया जातोय, हे कळत असताना सुद्धा  “आधीच एक तास झाला आहे, दीडशे रुपयाचं तिकीट काढलाय,” असं म्हणत आपण तो आता शेवटपर्यंत पाहतो आणि उरलेला वेळही वाया घालवतो. जे आधीच वाया गेलं आहे, त्याची भर अजून वेळ वाया घालवून का करायची?

काही शेअर्स मध्ये गुंतवलेले पैसे दिवसोंदिवस कमी-कमी होत आहेत हे लक्षात येत असूनही लोक त्या शेअर्सना चिकटून राहतात. कारण, “आधीच पैसे घातलेत, आता विकलं तर नुकसान होईल.” पण हे लक्षात घेतलं जात नाही की ज्यामध्ये गुंतवणूक अकार्यक्षम ठरतेय तीचा प्रत्येक दिवस एक ‘नवीन नुकसान’ आणत असतो.

ही सगळी उदाहरणं आपल्याला एक गोष्ट शिकवतात. मागे जे झालं त्यासाठी आपण पुढे चुकीचे निर्णय घेऊ नयेत. जेव्हा आपण फक्त "आधीच इतकं केलंय" म्हणून काही करत राहतो, तेव्हा आपला वेळ, पैसा आणि मानसिक ऊर्जा वाया जाते. ह्या मानसिक सापळ्याची जाणीव ठेवणं गरजेचं आहे.
गीतकार साहिर लुधियानवी यांच्या ह्या ओळी  ‘संक कॉस्ट फॅलसी’ या मेंटल मॉडेलला अगदी चपखल बसतात - “वो अफ़्साना जिसे अंजाम तक लाना न हो मुमकिन, उसे इक ख़ूबसूरत मोड़ दे कर छोड़ना अच्छा।”
