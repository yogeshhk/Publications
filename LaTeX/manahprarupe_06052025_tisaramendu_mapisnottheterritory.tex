\chapter{नकाशा म्हणजेच भूभाग नव्हे}

पाश्चिमात्य देशांच्या नजरेत भारताची प्रतिमा बदलत आहे. ‘सापांचे खेळ करणाऱ्यांचा देश’ हे जुनं चित्र आता मागे पडलं असलं, तरी काही चित्रपट व बातम्यांमुळे भारतात सर्वत्र गरीबी व अस्वच्छता असल्याचं दाखवलं जातं. ‘जागतिक भूक/उपासमारी निर्देशांक’ (ग्लोबल हंगर इंडेक्स) मध्ये भारत आजही पाकिस्तान, बांगलादेश, नेपाळ, श्रीलंकेच्या मागे (दाखवलेला) आहे.

हे सर्व पाहून एखादा परदेशी व्यावसायिक भारतात येतो आणि पाहतो तर काय, चित्र तेवढे काही वाईट नाही. इतकं कमी प्रति-व्यक्ती उत्पन्न असलेला देश जगातील सर्वात मोठ्या स्मार्टफोन बाजारांपैकी एक कसा असू शकतो? अतिप्रगत देशांमध्ये सुद्धा कागदावर मतदान होत असल्याने मोजणीस महिनोंमहिने लागतात तर भारतात काही दिवसात निकाल जाहीर पण होतो. यूपीआय व्यवहार करणारा फेरीवाला, नेटफ्लिक्सवरील शो-वर चर्चा करणारा रिक्षाचालक किंवा मोबाईलवर हवामानाची माहिती पाहणारा शेतकरी, अशी एक ना अनेक उदाहरणे दिसतात. त्यामुळे जगापुढे आलेल्या-ठेवलेल्या प्रतिमेत काही अंशी सत्य असले तरी ते पूर्ण वास्तव नक्कीच  नाही. याच विसंगतीच्या मेंटल मॉडेलला (मन:प्रारूप) अथवा विचार चित्राला ‘मॅप इज नॉट द टेरिटरी’ म्हणजेच ‘नकाशा-भूभाग-फारकत’ किंवा सोप्या भाषेत ‘नकाशा म्हणजेच भूभाग नाही’ असे नामकरण करू शकतो. पोलिश-अमेरिकन वैज्ञानिक आल्फ्रेड कोर्जिब्स्की यांनी मांडलेली ही संकल्पना चार्ली मंगर आणि शेन पॅरिशसारख्या विचारवंतांनी जनमानसात पोचवली. हे विचारचित्र आपल्याला सतत आठवण करून देते की, नकाशा म्हणजे वास्तव नाही. आकडेवारी, चार्ट्स, सोशल मीडिया पोस्ट्स किंवा एखादी कथा (आख्यान, नॅरेटिव्ह) हे वास्तवाचं अपूर्ण चित्र असतं. ते उपयोगी असलं तरीही पूर्ण सत्य मानणं धोकादायक ठरू शकतं.

आपल्या मेंदूला सतत व खूप बारीक विचार करायला ऊर्जा खर्च करायची नसते. मग तो क्लुप्त्या (शॉर्टकट्स) वापरतो. समजण्यासाठी आपण गोष्टी साध्या-सोप्या करतो. उदाहरणार्थ, नकाशा रस्त्यांची व गावांची माहिती जरी ढोबळ मानाने देत असला तरी प्रवास ठरवायला उपयुक्त ठरतो. पण प्रत्यक्षात जमिनींवर चित्र थोडे वेगळे असू शकते, खोद-कामांमुळे मार्ग बदललेले असू शकतात, नकाशात दाखवलं हॉटेल तेथे नसू शकतं. पण त्यामुळे खूप काही अडत नाही. पण जेव्हा आपण नकाशालाच वास्तव समजतो, तेव्हा आपण अशा स्थितीत निर्णय घेतो जे प्रत्यक्षात अस्तित्वात असेल असे नाही, याचे भान मात्र ठेवावे लागते. अजून स्पष्टतेसाठी ‘नकाशा-भूभाग-फारकत’ या विचार-चित्राची अजून काही उदाहरणे पाहुयात. 

एका छोट्या शहरातून आलेल्या विद्यार्थ्याच्या रेझ्युमे (बायोडेटा) वर केवळ तो आय-आय-टी किंवा आय-आय-एम लिहिलेले नाही म्हणून नोकरीसाठी-संधीसाठी नाकारला जातो. येथे ‘नकाशा’ म्हणजेच तो ‘बायोडेटा’ व त्यात प्रतीत होणारी शिक्षणसंस्थेची जनमानसातील प्रतिमा, आणि ‘भूभाग’ म्हणजेच त्या विद्यार्थ्यांचे प्रत्यक्ष कौशल्य व क्षमता. आपणास माहिती आहे की भारतातील सर्व यशस्वी उद्योजकांनी फक्त याच नामांकित संस्थांमधून शिक्षण घेतलेलं नाहीये, आणि तरीही ते यशाच्या शिखरावर पोहोचलेले आहेत. 
एखाद्याचं इंस्टाग्राम पाहा, सुट्ट्यांचे मनमोहक फोटो, सतत प्रेरणादायक व सकारात्मक लेखन, प्रमाणबद्ध-फिट शरीरं. असं वाटतं की त्यांचं आयुष्य परिपूर्ण आहे. पण या पोस्ट्स किंवा बराचसा सोशल मिडिया हा एक ‘नकाशा’ आहे. तो नीटनेटका आणि सजवलेला असतो; पण प्रत्यक्ष 'भूभाग' गुंतागुंतीचा, आणि पूर्णतः अस्ताव्यस्त असू शकतो.

बरं वाटत नसताना सुद्धा  एखाद्याचे काही वैद्यकीय चाचण्यांचे रिपोर्ट्स सर्वसामान्य मर्यादेत असू शकतात. म्हणूनच डॉक्टरचा सल्ला घेणे आवश्यक असते. तो थकवा, निरुत्साह आणि अस्वस्थता वेगळ्या कारणांनी असू शकतो. चाचण्या म्हणजे ‘नकाशा’, उपयोगी, पण कधी कधी अपूर्ण. मानसिक आरोग्य, आहार, झोप आणि योग्य अनुभवसिद्ध वैद्यकीय निदान हा खरा ‘भूभाग’ जो  अनेकदा या "नॉर्मल" आकड्यांच्या पलीकडे असू शकतो.

गृहसंकुलाच्या जाहिरातीतील सोयीसुविधा, दाखवलेले रहिवासी कसे लक्षवेधक असतात. किती इमारती आहेत व किती मजले आहेत या सारख्या गोष्टी जरी प्रत्यक्षातही खऱ्या दिसत असल्यातरी इतर काही गोष्टींमधे मात्र मोठी तफावत असू शकते. म्हणजेच ‘जाहिरात’ हे काही प्रत्यक्ष ‘उत्पादन’ नव्हे. 

भारतासारख्या देशात, जिथे अनेक विरोधाभास एकत्र नांदतात, तिथे आकडेवारी अनेकदा वास्तवापेक्षा वेगळी असते. येथील कोट्यवधी लोक रिपोर्ट्सच्या ओळींमध्ये नव्हे, तर त्याच्या मधल्या जागेत जगत असतात, तिथे ‘नकाशा-भूभाग-फारकत’ हे विचारचित्र अधिक महत्त्वाचे ठरते.

जेव्हा आपण नकाशाला भूभाग समजणं थांबवतो, तेव्हा आपण अधिक चांगले प्रश्न विचारतो, लक्ष देऊन ऐकतो, आणि निर्णय घेण्याआधी जरा थांबून विचार करतो. आकड्यांवर सर्वस्वी अवलंबून ('डेटा-ड्रिव्हन') असणं म्हणजे कायमच ‘सत्य-ड्रिव्हन' असणं असे नव्हे. नकाशे उपयोगी असतात, पण ते ब्रह्मवाक्य-पवित्र नसतात. ते मार्गदर्शक असावेत, अंधविश्वासू करणारे नाही.

