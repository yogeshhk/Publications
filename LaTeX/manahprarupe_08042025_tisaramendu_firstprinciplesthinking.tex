\chapter{‘आद्य तत्वविचारा’चे महत्व}

इलेक्ट्रिक गाड्या जरी सध्या लोकप्रिय झाल्या असल्या, तरी त्यांचा उगम अलीकडचा नाही. अनेक दशकांपूर्वीच त्यांची निर्मिती झाली होती, पण त्या ग्राहकांपर्यंत पोहोचण्यात अयशस्वी ठरल्या. आधुनिक काळातही मोठ्या कंपन्या इलेक्ट्रिक गाड्यांकडे फारशा वळल्या नाहीत. मग एक अवलिया आला आणि कोठलीही पार्श्वभूमी अथवा अनुभव नसताना त्याने या क्षेत्रात उडी मारून पूर्ण चित्रच बदलवून टाकले. 
त्याने स्वतःला विचारले, "प्रदूषण कमी करणे हे प्रमुख उद्दिष्ट असूनही पेट्रोल-डिझेलऐवजी इलेक्ट्रिक गाड्या का नाहीत?" संशोधनाअंती समजले की बॅटरीच्या अवास्तव किमतीमुळे मोठ्या कंपन्या या गाड्या तयार करत नाहीत. मग पुढचा प्रश्न :  "बॅटरी इतकी महाग का आहे?" बॅटरी बनवण्यासाठी लागणारे साहित्य कोणते? उत्तर मिळाले : लिथियम, निकेल, कोबाल्ट, ग्रॅफाइट यांसारखे काही दुर्लभ धातू. पुढील प्रश्न : "या कच्च्या मालाची एकत्रित किंमत किती?" उत्तर मिळाले : तयार बॅटरीच्या किंमतीपेक्षा खूपच कमी! येथे आशेचा किरण सापडला. मग विचार आला, "निर्मिती प्रक्रिया अधिक स्वस्त करता येईल का?", "डिझाइन बदलता येईल का?", "मोठ्या प्रमाणावर उत्पादन केल्यास किंमत आणखी कमी होईल का?" अशा सततच्या मूलभूत प्रश्नांमधून उत्तर शोधत त्याने बॅटरी निर्मितीत क्रांती घडवली. जे भल्याभल्यांना जमले नाही ते एका नवख्या माणसाने केले आणि प्रस्थापितांना मागे टाकले. त्या माणसाचे नाव इलॉन मस्क आणि त्या गाडीचे नाव टेस्ला.
अशाप्रकारे  प्रश्न विचारत, मूलभूत तत्वांपर्यंत पोहोचून उत्तर शोधत ‘फर्स्ट प्रिन्सिपल्स थिंकिंग’, म्हणजेच ‘आद्य तत्व विचार’ या विचारपद्धतीचा जन्म होतो. हे एक प्रभावी मेंटल मॉडेल (मनःप्रारूप) अथवा विचार-चित्र आहे. 
‘आद्य तत्व विचार’ हे विचार-चित्र कसे कार्य करते?
कोणत्याही समस्येचे बारकाईने निरीक्षण करणे, प्रश्न विचारत तुकडे करत राहणे, जोपर्यंत आपण मूलभूत प्रश्नांपर्यंत पोहोचत नाही तोपर्यन्त. मग त्या मूळ प्रश्नाचे निराकरण करायचे आणि मग उलटे परत मूळ समस्येपर्यंत येता येता संपूर्ण उपाय निर्मिती करायची, ही ‘आद्य तत्व विचार’ या विचारचित्राची पद्धत-प्रक्रिया आहे. कोठलाही पूर्वग्रह (आणि अनुभव) नसेल आणि "हे असंच चालत आलंय" या संकल्पना दूर ठेवल्यास  ही प्रक्रिया जास्त प्रभावी ठरते. 

इलॉन मस्कने ही पद्धत स्पेसएक्स (अंतराळ संशोधन) आणि न्यूरालिंक (मेंदू-संगणक संवाद तंत्रज्ञान) यांसारख्या संकल्पनांमध्येही यशस्वीपणे वापरली. त्यामुळे त्याला आधुनिक 'आद्य तत्व विचार' चा प्रवर्तक म्हणता येईल. हे विचारचित्र फक्त तंत्रज्ञान क्षेत्रातच वापरता येते असे बिलकुल नाही. आपण काही इतर विविध क्षेत्रातील उदाहरणे बघुयात. 
गुंतवणूक आणि घर खरेदी
लोकांना वाटते की घर विकत घेणे ही चांगली गुंतवणूक आहे. पण खरंच आहे का? 'आद्य तत्व विचार' वापरून विचारले पाहिजे: "ही गुंतवणूक भावनिक आहे की परताव्यासाठी?" "भाड्याने दिल्यास आणि किंमत वाढल्यास किती परतावा मिळेल?" "घराची किंमत एवढी का आहे? त्यातील कच्च्या मालाचा, जमिनीचा, आणि स्थानमहात्म्याचा किती वाटा आहे?" या मूलभूत प्रश्नांमधून अनेक वेळा लक्षात येते की वेगवेगळ्या ठिकाणी किंमती फुगवल्या जातात. “त्याचे निराकरण कसे होऊ शकेल?” अशापद्धतीने केलेला विचार शहाणपणाने गुंतवणूक करण्यास मदत करू शकतो.

आरोग्य आणि वजन नियंत्रण
समजा तुम्हाला वजन कमी करायचे आहे. मग विचारायचे— "वजन वाढते कशाने?" उत्तर: खाल्लेल्या उष्मांकांची (कॅलरीज) मात्रा खर्च केलेल्या उष्मांकांपेक्षा जास्त असल्याने. मग पुढचे प्रश्न: "खाण्यात बदल करायचा की उष्मांक खर्च करण्याच्या पद्धती बदलायच्या?" खाण्यावर लक्ष द्यायचे ठरवल्यावर, “खाण्यात उष्मांक कोणापासून जास्त मिळतात?”, "कोणत्या गोष्टी वजन वाढवतात— तळलेले, गोड पदार्थ का इतर काही?" “मग दिवसातून चारवेळेला भरपेट खाण्याऐवजी दोनदाच मोजका पण समतोल आहार घ्यायचा?”, "पोषण आणि उष्मांक संतुलित कसे ठेवायचे?" ही विचारसरणी तज्ज्ञांच्या मार्गदर्शनासह वापरल्यास वजन संतुलित राखण्यासाठी एक शास्त्रशुद्ध आणि शाश्वत योजना तयार करता येते.
अंतिम विचार
'आद्य तत्व विचार' ही संकल्पना सर्वव्यापी आहे. तंत्रज्ञान, व्यवसाय, नोकरी, कला, संशोधन, खेळ, राजकारण अशा अनेक क्षेत्रांत याचा उपयोग करून समस्यांचे व्यावहारिक आणि नाविन्यपूर्ण उपाय शोधता येतात. ही विचारपद्धती आत्मसात केल्यास नवे दृष्टिकोन, नावीन्यपूर्ण शोध, आणि यशस्वी निर्णय घेणे सोपे होते. तुम्ही कधी वापरली आहे का ही पद्धत?

