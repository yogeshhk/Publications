\chapter{गत-संधीची किंमत}

भारतात दरवर्षी लाखो तरुण-तरुणी यूपीएससी, एमपीएससी सारख्या स्पर्धापरीक्षांची तयारी करताना दिसतात. काही जण यासाठी पाच-सहा वर्षांचं आयुष्य खर्च करतात. मात्र या परीक्षा अत्यंत स्पर्धात्मक असून, अंतिम यशस्वी होणाऱ्यांची संख्या अत्यल्प असते. देशसेवा, स्थैर्य आणि प्रतिष्ठा हे आकर्षण मान्य असले तरी, बहुतांश जणांच्या पदरी फक्त थकवा, निराशा आणि वाया गेलेली वर्षं येतात. मग याचा विचार का होत नाही? या गेलेल्या वर्षांची भरपाई कशी होणार? याच काळात दुसरे काही ठीकठाक (खूप लाभाचे नसले तरी) करता आले नसते का? खरेतर अशा ‘अभ्यासू’ तरुणांना विविध क्षेत्रात जाता येऊ शकते. विधी-कायदे, संगणक, वित्त इत्यादी क्षेत्रात खूपच गरज आहे. परदेशी भाषा शिकून कौशल्याधारित कामे भारताबाहेर मिळू शकतात.  सध्या जोमात असलेल्या स्टार्टअप्समध्ये करिअर घडवण्याची संधी आणि शिकवण्याची आवड असेल तर त्याही पेशात कितीतरी संधी आहेत.  एक ना अनेक. पण सर्व लक्ष एका परीक्षेकडे लागल्यामुळे उर्वरित सर्व दारं नकळत बंद होतात. मग या ‘गमावलेल्या’ संधींचा का विचार होत नाही? याच प्रश्नावर आधारित ‘मेंटल मॉडेल’ (मन: प्रारूप) अर्थात विचार-चित्राला ‘ऑपॉर्च्युनिटी कॉस्ट’ म्हणजेच ‘गत संधीची किंमत’ असे म्हणू शकतो. 

आपण आयुष्यात दररोज असंख्य निर्णय घेतो. काही मोठे, काही छोटे. पण प्रत्येक निर्णयामागे एक वा अनेक गत-संधींच्या किंमती दडलेल्या असतात, म्हणजेच आपण निवडलेली गोष्ट न करता मिळू शकणाऱ्या पर्यायाचं मूल्य. ही संकल्पना अर्थशास्त्रातून आली असली तरी तिचं महत्त्व केवळ आर्थिक व्यवहारांपुरतं मर्यादित नाही, ती आपल्या आयुष्याच्या प्रत्येक क्षेत्रात लागू होते. याची काही उदाहरणे पाहुयात.

कल्पना करा, तुम्ही रविवारची सुट्टी एखाद्या मोठ्या मॉलमध्ये खरेदी करत, खाण्या-पिण्यात आणि खरेदी (विंडो-शॉपिंग!!) करत घालवली. तुमचे पाचेक हजार खर्च आणि ६-७ तास सहज गेले. थोडी मजा झाली, पण जरा विचार करा, याच दिवशी जवळच्या उद्योजकतेवर आधारित कार्यशाळेला हजेरी लावली असती, तर? एखाद्या यशस्वी स्टार्टअप संस्थापकाकडून शिकायला मिळालं असतं. तिथे ओळखी वाढल्या असत्या. कदाचित एखादी कल्पना सुचली असती जी भविष्यात व्यवसायात रूपांतरित होऊ शकली असती. मॉलमध्ये गेलेल्या दिवसाचं बिल तुमच्या खिशातून गेलं: ₹५,000. पण कार्यशाळेला न गेल्याचं बिल कोणीच दाखवत नाही. हेच ते अदृश्य नुकसान म्हणजेच ‘गत संधीची किंमत’. म्हणजेच  एखादी गोष्ट निवडताना, आपण ज्या दुसऱ्या गोष्टी सोडून देतो, त्यांचे मूल्य.

एका चांगल्या पगाराच्या नोकरीसाठी होकार देताना, तुम्ही तुमचा स्टार्टअप सुरू करण्याची संधी सोडत आहात का? नजीकच्या भविष्यात नोकरी जास्त पैसे देईल, स्थिरता देईल पण गमावलेली स्टार्टअपची संधी काहीतरी जगावेगळं-असामान्य करण्याची शक्यता तर गमावत नाहीये ना?

रोज तासंतास सोशल मीडियावर घालवताना, कदाचित एक नवीन कौशल्य शिकण्याची संधी हातातून जात आहे का? त्यामुळे पुस्तकं वाचणे मागे पडत आहे का? कित्येक दिवसात सहकुटुंब पत्ते खेळले नाहीयेत का? 

घर खरेदी करणं हे अनेकांचं स्वप्न असतं, पण त्यासाठी घेतलेलं कर्ज आणि त्यात गुंतवलेलं भांडवल दुसऱ्या गुंतवणुकीत वापरलं असतं तर अधिक चांगला परतावा मिळाला असता का? भाड्याने राहणं काहींना खर्चिक वाटतं, पण यामुळे लवचिकता, आर्थिक तरलता आणि इतर संधी मिळू शकतात. दिसतं ते घर, पण दिसत नाही ती ‘गत -संधीची किंमत’.

ऑफिसमधल्या प्रत्येक मीटिंगला ‘हो’ म्हणणं म्हणजे तुमचं लक्ष, विचारशक्ती आणि मोकळा वेळ गमावणं, नाही का? हा वेळ सखोल-सर्जनशील काम किंवा महत्त्वाच्या निर्णयांसाठी वापरता आला असता. व्यत्ययांमुळे होणारं खरं नुकसान वेळेचं नसतं तर तुम्ही गमावलेल्या संधींचं असतं.

आपण सहसा जे मिळतं त्यावर लक्ष देतो, जे गमावतो त्यावर नाही. हे आपल्या विचारसरणीचं अपंगत्व आहे. ‘गत संधीची किंमत’ समजून घेणं म्हणजे केवळ शहाणपण नव्हे तर ही एक मानसिक शक्ती आहे. तीच आपल्याला सरळ वाटचाल करणाऱ्यांपासून वेगळं करते. विचारपूर्वक जगणाऱ्यांकडे नेते. तुमच्या प्रत्येक ‘होकारा’त अनेक ‘नकार’ दडलेले असतात याचं भान येणं महत्वाचं आहे. पुढच्या वेळी जेव्हा तुम्ही कुठलाही निर्णय घ्याल, तेव्हा फक्त "हे का?" इतकंच नाही, तर "यामुळे मी काय गमावत आहे?" हाही प्रश्न विचारायला विसरू नका.
