\chapter{दृष्टी पलीकडील सृष्टी}

ब्रिटिश राजवटीतील एक गोष्ट सांगतात. एका गावात सापाचं प्रमाण फार वाढलेलं होतं. या समस्येवर तोडगा म्हणून अधिकाऱ्यांनी एक उपाय-योजना केली. गावात दवंडी पिटवली की जे लोक साप पकडून कार्यालयात आणतील त्यांना इनाम देण्यात येईल. जेवढे जास्त साप, तेवढी अधिक इनामाची रक्कम. दिवसेंदिवस मोठ्या संख्येने मृत साप जमा होऊ लागले. काही दिवस उलटून गेल्यावरही सापांचा त्रास काही कमी झाला नाही. अधिकाऱ्यांना शंका आली. तपास केल्यावर असे सापडले की लोक साप पाळून त्यांची प्रजनन करून पैसे कमवायला लागले आहेत. हे लक्षात येताच त्यांनी इनाम योजना थांबवली. पैसे मिळवण्याचा मार्ग बंद झाल्यावर लोकांनी साप मोकळ्यात सोडले. पूर्वीची समस्या आता अधिक गंभीर झाली. ज्या उद्देशाने योजना सुरू केली होती, त्याचे फायदे केवळ थोड्या काळापुरते दिसले. मात्र लोकांनी त्याचा गैरफायदा घेतल्याने मूळ उद्देशालाच हरताळ फासला.
 म्हणून कोणताही विचार करताना, निर्णय घेताना नजीकच्या परिणामांचाच फक्त विचार न करता त्याचा संभाव्य व दूरगामी परिणामांचा विचार करणे या मेंटल मॉडेल (मन:प्रारूप) म्हणजेच ‘विचार-चित्रा’ला ‘सेकंड ऑर्डर थिंकिंग’  अथवा ‘दूरगामी परिणामांचा विचार’ म्हणू शकतो.  यापद्धतीत एका मागून एक प्रश्न विचारत राहायचे की ‘मग काय होईल?’, ‘मग काय होईल? असा एकामागून एक मागोवा घेत विचार केला जातो. सर्वसाधारणपणे, आपण एकाच प्रश्नाचे उत्तर शोधून कामाला लागतो. याला 'फर्स्ट ऑर्डर थिंकिंग' म्हणजे तात्कालिक विचार म्हणता येईल. परंतु 'सेकंड ऑर्डर थिंकिंग' पद्धतीने विचार करतांना, त्या उत्तरानंतर होणाऱ्या परिणामांची शृंखलाही तपासली जाते.
उदाहरणार्थ, बुद्धिबळात जसे आपण फक्त लगेचच्या चालीचाच फक्त विचार नाही तर त्याने पुढे काय काय होऊ शकते त्यावरही भविष्यातील चालींचा आराखडा बांधत असतो. 'परिणामांचे परिणाम' जोखणे, हे मूलतत्त्व आहे. चार्ली मंगर यांनी म्हटले होते की, "अशा पद्धतीने विचार करणे बिलकुल सोपे नाही, आणि ज्याला ते सोपे वाटते, तो मूर्ख समजावा." गमतीचा भाग सोडला तर त्यांच्या म्हणण्याचा उद्देश असा होता की उथळ आणि झापडे लावून केलेला विचार हा एक सापळा आहे आणि तो महागात पडू शकतो. म्हणूनच हे महत्वाचे तत्व काही उदाहरणांच्या आधारे पाहुयात. 
वजन कमी करण्यासाठी अनेकजण झटपट कॅलरीज कमी करण्याचा मार्ग अवलंबतात, कारण त्यांना त्वरित परिणाम हवा असतो. पण हा तात्कालिक विचार दीर्घकाळात उलट परिणाम करतो. लवकर वजन कमी झाल्याने शरीराची पचन-चयापचय क्रिया मंदावते, थकवा जाणवतो, आणि हार्मोनल असंतुलन होऊ शकते. परिणामी, काही काळाने वजन पुन्हा वाढू लागते. शॉर्टकट्सपेक्षा शाश्वत आणि संतुलित सवयींचा स्वीकार करणे अधिक फायदेशीर ठरते.
नफ्यामध्ये जोरदार आणि त्वरित वाढ करण्यासाठी अनेक कंपन्या कर्मचार्‍यांना तडकाफडकी कमी करण्याचा निर्णय घेतात. हे प्रथमदर्शनी फायदेशीर वाटत असले, तरी हा तात्कालिक विचार अनेक दीर्घकालीन समस्यांना आमंत्रण देतो. कर्मचार्‍यांची संख्या कमी केल्यावर मनोबल घटते, नवसंशोधन मंदावते, आणि चांगली कामगिरी करणारे कर्मचारी सुद्धा कालांतराने संधी शोधत निघून जातात. या सर्व गोष्टींचा एकत्रित परिणाम म्हणजे कंपनीची कार्यक्षमता कमी होऊ शकते, आणि दीर्घकालीन नफा वाढवण्याचे उद्दिष्ट साध्य होऊ शकत नाही.
दैनंदिन धावपळीत पालक अनेकदा मुलांना शांत ठेवण्यासाठी त्यांच्याकडे स्मार्टफोन देतात, जेणेकरून ते व्यस्त राहतील आणि त्रास देणार नाहीत. हा उपाय तात्काळ शांतता देतो, पण यामुळे मुलांमध्ये मोबाईल-व्यसनाची सुरुवात होते. लहान वयात स्क्रीनची  सवय लागल्यामुळे लक्ष केंद्रित करण्याची क्षमता कमी होऊ शकते. मुले एककल्ली आणि घरकोंबडी होऊ शकतात. आजची शांतता उद्याचे बालपण समस्याग्रस्त बनवू शकते.
वरील उदाहरणांवरून वाटेल की ‘दूरगामी परिणामांचा विचार’ म्हणजे फक्त नकारात्मक असतो,पण तसे नक्कीच नाही. अनेकवेळा निर्णयांचे सकारात्मक परिणामही पुढच्या टप्प्यावर उघड होतात. उदाहरणार्थ, मुलींना शिकवणे हे प्रथमदर्शनी पाहता त्यांना स्वयंपूर्ण बनवण्याचा एक मार्ग आहे असे वाटते आणि ते खरेच आहे. शिक्षणामुळे त्या आत्मनिर्भर बनतात, त्यांना आर्थिक स्थैर्य लाभते, आणि त्यांची आत्मसन्मानाची भावना वृद्धिंगत होते. पण यापलीकडे, या निर्णयाचे दुसऱ्या स्तरावरील परिणाम अधिक खोलवर आणि समाजहिताचे असतात. शिक्षित आणि स्वतः कमावणाऱ्या स्त्रिया सहसा लहान कुटुंब ठेवण्याची निवड करतात, त्यामुळे कुटुंब नियोजन सुलभ होते आणि सामाजिक समतोल राखला जातो. स्त्रियांना ‘उंबरठा’ ओलांडू न देणाऱ्या समूहांमध्ये कुटुंबे मोठी दिसतात. म्हणूनच या समस्येला उपाय म्हणून आलेल्या ‘बेटी पढाओ’ सारख्या योजनांमध्ये ‘दूरगामी परिणामांचा विचार’ दिसून येतो. केवळ वैयक्तिक नव्हे तर सामूहिक प्रगतीसाठीही हा विचार महत्त्वाचा ठरतो.
आजच्या धावपळीच्या जगात, ‘सेकंड ऑर्डर थिंकिंग’ अर्थात ‘दूरगामी परिणामांचा विचार’ आपल्याला थांबून चिंतन करण्याची आठवण करून देतो. लगेच सुचलेले उत्तर तेवढेच परिपूर्ण असू शकते असे नाही. त्यामुळे कोणताही निर्णय घेताना थोडा वेळ घ्या, स्वतःला विचारा: 'पुढे काय होईल?' आणि त्यानंतर काय होईल? ही सवय अनपेक्षित परिणामांपासून वाचवून शहाणपणाचे आणि प्रभावी निर्णय घेण्यास मदत करू शकते. शेवटी, ‘सेकंड ऑर्डर थिंकिंग’ ही एक पद्धत नसून, एक शांत पण प्रभावी दृष्टिकोन आहे,  गोंगाटातली एक निःशब्द ताकद.
