% Copyright page
\thispagestyle{empty}
% \null\vfill

\begin{center}
\includegraphics[width=0.2\linewidth,keepaspectratio]{YHK_Color_OutOfTheBox_tight} \\[1.5em]

\textbf{\Huge मन:प्रारूपे}\\ [0.5em]
{\small(विविध क्षेत्रातील `मेंटल मॉडेल्स'ची तोंडओळख )}\\[0.5em]

लेखक: \textbf{{\large डॉ. योगेश हरिभाऊ कुलकर्णी}}\\[1.5em]
\end{center}

\vspace{3.5em}

\begin{flushleft}

प्रकाशक: डॉ. योगेश हरिभाऊ कुलकर्णी (self-published)\\
पत्ता:  पाषाण ,  पुणे ८ \\
फोन:  +91 9890251406\\
ईमेल: yogeshkulkarni@yahoo.com\\[1.5em]

\vspace{3.5em}

प्रथम आवृत्ती: २०२५\\[0.5em]

% ISBN: [ISBN नंबर]\\[0.5em]

कॉपीराइट © २०२५ डॉ. योगेश हरिभाऊ कुलकर्णी\\[0.5em]

{\textit{सर्व हक्क राखीव. या पुस्तकाचा कोणताही भाग प्रकाशकाच्या लेखी परवानगीशिवाय कोणत्याही स्वरूपात पुनर्मुद्रित किंवा पुनर्प्रकाशित करता येणार नाही.  या पुस्तकात व्यक्त केलेली मते लेखकाची व्यक्तिगत आहेत.}}\\[1.5em]

{\large Legal Notice:}\\
{\textit{All rights reserved. No part of this publication may be reproduced, distributed, or transmitted in any form or by any means, including photocopying, recording, or other electronic or mechanical methods, without the prior written permission of the publisher, except in the case of brief quotations embodied in critical reviews and certain other noncommercial uses permitted by copyright law.}}
\end{flushleft}
\vfill\null
\clearpage

\begin{dedication}
`मेंटल मॉडेल्स 'यांना प्रसिद्धी देणाऱ्या चार्ली मंगर आणि शॉन पेरिश यांना समर्पित  
\end{dedication}

\clearpage

\chapter*{पुनर्मुद्रणाविषयी}

या पुस्तकातील लेख विविध ठिकाणी अगोदरच दै. सकाळ मधील `तिसरा मेंदू'  या सदरात प्रकाशित झालेले आहेत .  येथे थोड्याफार प्रमाणात संपादन करून केवळ संकलित करण्यात आलेले आहेत.  यातील बरेचसे लेख `मिडीयम'  या संकेत स्थळातील 'देसी स्टॅक' या प्रकाशनात पण उपलब्ध आहेत. 


\chapter*{शीर्षकाविषयी }
नाव: **"मन : प्रारूपे अथवा विचारचित्रे"**

या पुस्तकाच्या शीर्षकात "मन" म्हणजे मानवी विचारसरणीचे केंद्र असून "प्रारूपे" किंवा "विचारचित्रे" (`मेंटल मॉडेल्स') म्हणजे जग, समस्या व निर्णय यांना समजून घेण्यासाठी आपल्यामनात तयार होणाऱ्या अंतर्गत चौकटी. "प्रारूप" हा शब्द मॉडेल्स'चा, संरचनेचा, विश्लेषणाचा सूचक आहे, तर "विचारचित्र" हे अधिक दृश्य आणि सहज समजणारे रूपक आहे. हे शीर्षक अशा संकल्पनांचा अभ्यास करायला प्रेरणा देते ज्या आपले निर्णय, वर्तन आणि समज निश्चित करतात — म्हणजेच मनात तयार होणाऱ्या विचारांच्या नकाशांचा मागोवा.
