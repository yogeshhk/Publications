% Copyright page
\thispagestyle{empty}
% \null\vfill

\begin{center}
\includegraphics[width=0.2\linewidth,keepaspectratio]{YHK_Color_OutOfTheBox_tight} \\[1.5em]

\textbf{\Huge मन:प्रारूपे}\\ [0.5em]
{\small(विविध क्षेत्रातील `मेंटल मॉडेल्स'ची तोंडओळख )}\\[0.5em]

लेखक: \textbf{{\large डॉ. योगेश हरिभाऊ कुलकर्णी}}\\[1.5em]
\end{center}

\vspace{3.5em}

\begin{flushleft}

प्रकाशक: डॉ. योगेश हरिभाऊ कुलकर्णी (self-published)\\
पत्ता:  पाषाण ,  पुणे ८ \\
फोन:  +91 9890251406\\
ईमेल: yogeshkulkarni@yahoo.com\\[1.5em]

\vspace{3.5em}

प्रथम आवृत्ती: २०२५\\[0.5em]

% ISBN: [ISBN नंबर]\\[0.5em]

कॉपीराइट © २०२५ डॉ. योगेश हरिभाऊ कुलकर्णी\\[0.5em]

{\textit{सर्व हक्क राखीव. या पुस्तकाचा कोणताही भाग प्रकाशकाच्या लेखी परवानगीशिवाय कोणत्याही स्वरूपात पुनर्मुद्रित किंवा पुनर्प्रकाशित करता येणार नाही.  या पुस्तकात व्यक्त केलेली मते लेखकाची व्यक्तिगत आहेत.}}\\[1.5em]

{\large Legal Notice:}\\
{\textit{All rights reserved. No part of this publication may be reproduced, distributed, or transmitted in any form or by any means, including photocopying, recording, or other electronic or mechanical methods, without the prior written permission of the publisher, except in the case of brief quotations embodied in critical reviews and certain other noncommercial uses permitted by copyright law.}}
\end{flushleft}
\vfill\null
\clearpage

\begin{dedication}
`मेंटल मॉडेल्स 'यांना प्रसिद्धी देणाऱ्या चार्ली मंगर आणि शॉन पेरिश यांना समर्पित  
\end{dedication}

\clearpage

\chapter*{पुनर्मुद्रणाविषयी}
या पुस्तकातील लेख दै. सकाळ मधील `तिसरा मेंदू' या सदरात यापूर्वी प्रकाशित झालेले असून, येथे ते आवश्यक त्या हलक्या संपादनासह एकत्रित स्वरूपात मांडले आहेत. अनेक लेख `मिडीयम'  या संकेत स्थळावरील देसी स्टॅक या प्रकाशनातही उपलब्ध आहेत. वाचकांना एकसंध आणि सोपे संदर्भ मिळावेत, हा या पुनर्मुद्रणाचा मुख्य हेतू आहे.

\chapter*{शीर्षकाविषयी }
`मन : प्रारूपे' या पुस्तकाचे शीर्षक स्वतःतच एक लहानसं विधान आहे. विचारांच्या जगाचे दार उघडणारे, आणि मानवी बुद्धीच्या प्रवासाला दिशादर्शक ठरणारे. येथे ‘मन’ हा शब्द केवळ भावना किंवा विचारांचे स्थान म्हणून नाही, तर मानवी जाणिवा, समज, निर्णयक्षमता आणि आकलन यांचे केंद्र म्हणून वापरला आहे. मन म्हणजे आपण जग अनुभवतो त्या सर्व प्रक्रियांचा पाया, आणि त्यातूनच उगम पावतात आपले निर्णय व निष्कर्ष.

‘प्रारूपे’ हा शब्द या संदर्भात विशेष अर्थपूर्ण आहे. प्रारूप म्हणजे एखाद्या वस्तूचे, घटनेचे किंवा विचाराचे आंतरिक रूप, त्याची चौकट, त्याचे कार्य समजून घेण्याची रचना. इंग्रजीतील ‘मॉडेल्स’ किंवा ‘फ्रेमवर्क्स’ यांची मराठी प्रतिमा म्हणून ‘प्रारूपे’ हा शब्द; तो विचारांच्या मुळावर बोट ठेवतो. ही प्रारूपे म्हणजे आपल्या मनात नकळतपणे तयार होणारे नमुने, जग पाहण्याचे, समस्या विश्लेषित करण्याचे आणि निर्णय घेण्याचे मार्गदर्शक नकाशे.

या शीर्षकाने पुस्तकाचा आशय थेट सुचवला आहे. ते वाचकाला सांगते की पुढील पानांमध्ये आपण त्या मानसिक नकाशांचा अभ्यास करणार आहोत, जे आपल्या वर्तनाला दिशा देतात, पूर्वग्रहांना जन्म देतात, आणि समजुतींची मर्यादा किंवा व्यापकता ठरवतात.  
