\chapter{कमीतून जास्त}

बऱ्याच भारतीय शहरांप्रमाणे पुण्यातही वाहतुकीचा बहुतांश भार काही मोजक्या रस्त्यांवरच असतो. सुमारे २०\% रस्त्यांवर ८०\% वाहतूक केंद्रित असते. बाणेर-हिंजवडी, जे.एम. रोड, एफ.सी. रोड आणि स्वारगेट-टिळक रस्ता हे मुख्य मार्ग विशेषतः गर्दीच्या वेळात प्रचंड ताण सहन करतात. शहरात शेकडो किलोमीटर रस्ते असले तरी, गर्दी आणि विलंब नेहमी या काही मोजक्या मार्गांवरच केंद्रित असतात. सार्वजनिक वाहतुकीच्या बसेस सुद्धा आपला बहुतांश वेळ फक्त याच काही गर्दीच्या रस्त्यांवर अडकून घालवताना दिसतात. याला उपाय म्हणजे पुण्यातील सर्व रस्त्यांपेक्षा हे अतिरहदारीचे रस्ते निवडणे आणि त्यांच्या समस्या प्राधान्यक्रमाने सोडवणे. अशा मोजक्या-सीमित बाबींवर लक्ष केंद्रित करून मोठा बदल घडवून आणण्याच्या मेंटल मॉडेल (मन:प्रारूप) ला अथवा विचारचरित्राला ‘पॅरेटो तत्व’ किंवा ‘८०/२० नियम’ असे म्हणतात, याला सोप्या भाषेत ‘कमीतून जास्त’ असे म्हणू शकतो. हेच तत्त्व वापरून, पुणे महानगरपालिका स्मार्ट सिग्नलिंग, मेट्रो विस्तार आणि निवडक रस्ते रुंदीकरण यांसारख्या उपाययोजनांवर लक्ष केंद्रित करत आहे. उदाहरणार्थ, नळ स्टॉप चौकातील वाहतूक सुलभ केल्यामुळे संपूर्ण मार्गावरील प्रवासाचा वेळ लक्षणीय कमी झाला. पायाभूत सुविधांमध्ये सगळीकडे पैसे उधळण्यापेक्षा, हे काही मोजके "महत्त्वाचे रस्ते" सुधारण्याचे धोरण अधिक परिणामकारक ठरते.

इटालियन अर्थशास्त्रज्ञ ‘विल्फ्रेडो परेटो यांनी १९व्या शतकात हे तत्त्व मांडले. त्यांनी पाहिले की इटलीतील ८०\% जमीन केवळ २०\% लोकांच्या मालकीची आहे. असे चित्र जगभरातील अनेक नैसर्गिक व सामाजिक यंत्रणांमध्ये ही वारंवार दिसते. भारतातही हे तत्त्व अनेक ठिकाणी ठळकपणे दिसते. या तत्वातील ८०\% आणि २०\% या आकड्यांच्या मागे अगदी  फार तंतोतंतपणे पाहू नये तर त्याचा भावार्थ घ्यावा. थोड्याशा कारणांमुळेच बहुतांश परिणाम घडतात, हेच या तत्त्वाचं सार आहे. मुद्दा जास्त मेहनत करण्याचा नाही तर चतुरपणे, योग्य ठिकाणी करण्याचा आहे. याचा उपयोग केवळ वाहतूक-नियंत्रणापुरता मर्यादित नाही. व्यवसायापासून वैयक्तिक आयुष्यापर्यंत, सर्वत्र हे तत्व आपल्याला दिशा दाखवते. त्याची काही उदाहरणे पाहुयात. 

एका स्टार्टअपने पाहिलं की ८०\% तक्रारी २०\% ग्राहकांकडून येतात. त्यांनी त्या मोजक्या ग्राहकांच्या समस्या सोडवल्या आणि संपूर्ण सेवा सुधारली.

जगातील, तसेच भारतातीलही  फार थोडे लोकं इतर सर्व जनतेपेक्षा जास्त कमावतात. अतिशय कमी लोक आयकर भरत असल्याने ते इतर सर्वांच्या कल्याणासाठी भार उचलतात. 

आपण एखाद्या मध्यमवर्गीय कुटुंबाचा खर्च पहिला तर जवळजवळ बहुतांशी भाग एक-दोन मोठ्या गोष्टी, जसे घर भाडे (अथवा कर्जाचा हप्ता) किंवा मुलांच्या शिक्षणात जातो. तेथे नीट-सखोल विचार करून निर्णय घेतला तर योग्य बचत होऊ शकते. 

आरोग्याचा विचार केला तर बहुतेक घरांमध्ये फक्त २०\% अन्नपदार्थ जसे साखर, तेलकट आणि प्रोसेस्ड फूड हे ८०\% आरोग्य समस्यांना कारणीभूत ठरतात. हेच पदार्थ बदलले तर मोठा फरक पडू शकतो. 

तुमच्या मोबाईलमध्ये शेकडो ऍप्स असले तरी ८०\% वेळ तुम्ही त्यातील ठराविक २०\% ऍप्स वरच घालवता, नाही का? त्या मोजक्याच ऍप्स च्या उपयोगावर नियंत्रण ठेवले तर मोठ्याप्रमाणात वेळ वाचू शकतो. 

समाजजीवनात सुद्धा तुमचे आप्त-स्वकीय-मित्र हे मोजकेच असतात. वेळप्रसंगी अर्ध्या रात्री धावून येणारे २०\%, बाकी ८०\% असतात केवळ ओळखी, कामापुरत्या. त्यामुळे कुठली नाती जपून ठेवायची आणि कोणाचा खूप विचार करायचा नाही हे या तत्वामुळे जरूर ओळखता येते. 

‘परेटो तत्व’ हे कोठे लक्ष केंद्रित करायचे ते सांगते. सध्याच्या जगात भारंभार गोष्टी करण्यावर खूप भर असतो. पण त्यासर्वातून अपेक्षित परिणाम मिळतोय का हे तपासून पाहिलं जात नाही, मग होते नुसती दमणूक आणि हात राहतात रिकामेच. आपण जेव्हा कामाच्या भल्या मोठ्या यादीने गुदमरतो, तेव्हा हे लक्षात ठेवा की बहुतेक गोष्टी या आवश्यक-परिणामकारक नसतात. काही मोजक्याच कृती खरंतर परिणाम घडवतात. काय करावं यापेक्षा काय टाळावं हे ओळखणं जास्त मौल्यवान ठरतं. भारतासारख्या देशात  ८०/२० नियम हे एक अमूल्य मार्गदर्शक साधन आहे. आपली ऊर्जा सर्वत्र खर्च करण्याऐवजी, काही मोजक्या महत्त्वाच्या बाबींवर केंद्रित केली, तर परिणाम अधिक आणि टिकाऊ होतो. थोडक्यात सांगायचं तर कमी करून अधिक साध्य करणं ही युक्ती नाही, आळशीपणा नाही तर तीच खरी शक्ती आहे.





