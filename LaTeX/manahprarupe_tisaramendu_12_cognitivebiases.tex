\chapter{मनाच्या पूर्वग्रहाचे खेळ}
संगणकात अथवा मोबाईलमध्ये सॉफ्टवेअर-ऍप इन्स्टॉल करताना तुम्ही ‘नियम व अटी’ (टर्म्स अँड कंडिशन्स) आणि घेतल्या जाणाऱ्या ‘परवानग्या’ (पर्मिशन्स) वाचता का? का अशा अटींना होकार देऊन पुढे जाता? एका सर्वेक्षणानुसार केवळ ९% गुंतवणूकदारच संपूर्ण अटी वाचतात, उरलेले सोडतात. हा आकडा केवळ निष्काळजीपणा दाखवत नाही, तर आपल्या मेंदूची एक नैसर्गिक कमतरता देखील दर्शवतो. एवढ्या मोठ्या कंपन्या काही दगाबाजी करणार नाहीत हा पूर्वग्रह (बायस) आपल्या मनात ठाम असतो. अशाच आळसामुळे असो किंवा अंध-विश्वासामुळे, आपण सहज सायबर आक्रमणांना दार किलकिले करून ठेवतो. 

आपल्या मेंदूत अनेक पूर्वग्रह (‘बायसेस’) किंवा शॉर्टकट्स असतात, जे आपल्याला कष्ट घेण्यापासून वाचवतात पण चुकीच्या निर्णयांकडे नेतात. आजच्या धावपळीच्या जीवनात हे शॉर्टकट्स सतत वापरले जातात. पूर्वग्रह अधिक दृढ होतात आणि आपण त्या मनाच्या खेळात अडकतो. अशा नकळत होणाऱ्या फसवणुकीच्या मेंटल मॉडेल (मन:प्रारूपाला) अथवा विचार-चित्रांच्या समूहाला ‘कॉग्निटिव्ह बायसेस’ (मानसिक पूर्वग्रह) किंवा सध्या-सोप्या भाषेत ‘मनाच्या पूर्वग्रहाचे खेळ’ म्हणू शकतो. १९७० च्या दशकात मानसशास्त्रज्ञ डॅनियल काहनेमन आणि आमोस टव्हर्स्की यांनी हे स्पष्ट केलं की, आपले निर्णय बहुतेक वेळा भावनांवर आणि सहजतेवर आधारित असतात, वास्तवावर नव्हे. असे पूर्वग्रह अनेक प्रकारचे असतात. त्याची काही उदाहरणे पाहुयात. 

तुम्ही बाजारात गालिचा घ्यायला गेलात. दुकानदार किंमत सांगतो १२,०००. तुम्ही घासाघीस करून ६,००० वर घेता आणि स्वत:ला हुशार समजता. पण त्या गालिच्याची खरी किंमत कदाचित २,०००च असण्याची शक्यता खूप, नाही का?  हा प्रकार भारतातच नाही तर आशियातील अनेक देशांत, विशेषतः चीनमध्ये दिसतो. मी शांघायमध्ये वस्तू विकत घेताना सुरुवातीला नेहमी सांगितलेल्या किंमतीच्या फक्त तृतीयांश किंमतीत बोलणी सुरु करायचो. नंतर सौदा सुमारे अर्ध्या किमतीत होई. दोघेही खुश!  याला "अ‍ॅंकरिंग बायस" (आधार-पूर्वग्रह) म्हणतात, जिथे सुरुवातीला ऐकलेली किंमत आपल्या निर्णयावर प्रभाव टाकते. 

आपली काही ठाम मते-विचार असतात त्यांनाच आपण कवटाळून बसलेलो असतो. आपल्या अवतीभोवती त्याला पूरक अशाच गोष्टी आपल्याला दिसत राहतात आणि विरोधातली माहिती आपण नकळत दुर्लक्षित करतो. याला ‘कन्फर्मेशन बायस’ (दृढीकरण-पूर्वग्रह) म्हणतात. घरी ‘टाटा पंच’ ही गाडी विकत घेण्याची चर्चा सुरु असेल तर रस्त्यावरून जाताना दर १-२ मिनीटांनी टाटा-पंच दिसायला लागतात, बरोबर ना? समाज-माध्यमात, पत्रकारितेत एकाच विचारधारेला वाहिलेले गट ठळकपणे दिसतात. उदाहरणार्थ, जर एखाद्या नेत्यावर भ्रष्टाचाराचे आरोप झाले, तरी त्याचे समर्थक म्हणतील, “हे विरोधकांचे कारस्थान आहे,” कारण त्यांचा मेंदू आधीच ठरवून बसलेला असतो. त्यामुळे सत्य शोधण्याऐवजी, आपल्या विश्वासाला पाठिंबा देणारी माहितीच खरी वाटते.

कोणतीही घटना मनात ताजी असेल, तर आपण ती अधिक सामान्य-खरी समजतो. हाच ‘अ‍ॅव्हेलेबिलिटी बायस’ (स्मरणसुलभता-पूर्वग्रह). उदाहरणार्थ, जर नुकतेच एखाद्या हवाई अपघाताची बातमी ऐकली असेल, तर विमान प्रवास धोकादायक वाटू लागतो, जरी आकडेवारी वेगळीच असली तरी. आपल्या मेंदूला जे लगेच आठवते तेच खरं वाटते.

जुनी स्थिती कायम ठेवायची मनाची प्रवृत्ती म्हणजे ‘स्टेटस क्वो बायस’ (स्थितीस्थैर्य-पूर्वग्रह). अनेकदा नवीन आणि फायदेशीर पर्याय असतानाही आपण जुन्या सवयींच्या पलीकडे जाण्यापासून घाबरतो. एखाद्या कर्मचाऱ्याला चांगली नवीन नोकरीची ऑफर असली, तरी तो म्हणतो, “सध्या तशी स्थिती ठीक आहे, नवीन नोकरी धोकादायक वाटते.” आपल्या सवयी, सामाजिक अपेक्षा आणि अनिश्चिततेची भीती आपल्याला नवीन संधींचा वापर करायला टाळाटाळ करायला भाग पाडते.

‘लॉस अ‍ॅव्हर्जन’ म्हणजेच ‘नुकसान टाळण्याचा पूर्वग्रह’. आपण नफा मिळण्यापेक्षा नुकसान टाळण्याला अधिक महत्त्व देतो. उदाहरणार्थ, शेअर बाजारात तोटा होण्याची भीती बाळगून अनेक लोक पैसे गुंतवत नाहीत, जरी दीर्घकालीन फायदा होण्याची शक्यता असली तरी.  काही दु:खी कुटुंबे पाहून, संसाराचा गाडा ओढताना होणारी दमछाक बघून अनेकजण लग्न करण्याचे टाळतायत. आपल्याकडे जे आहे (त्यातील सर्वात महत्वाचे म्हणजे ‘स्वातंत्र्य’) ते गमावण्याची भीती आपल्याला पुढे जायची हिंमत देत नाही. 

आपल्या आयुष्यात अनेकदा महत्त्वाचे निर्णय घ्यावे लागतात जसे की शिक्षण, गुंतवणूक, मतदान, व्यवसाय, लग्न इत्यादी. जर आपले निर्णय पूर्वग्रहांनी प्रभावित असतील, तर त्यातून नुकसान होऊ शकते. ते पूर्वग्रह ओळखणे महत्त्वाचे आहे. त्यांचा अभ्यास करणे आणि त्यांचा अडथळा ओळखणे आपल्याला अधिक विवेकपूर्ण बनवू शकते.
दिलेल्या उदाहरणांवरून आपण हे शिकू शकतो की थोडा तटस्थपणा आणि साक्षेपी भाव ठेवण्याची सवय लागल्यास तुम्ही प्रभावी विचार करणारे होऊ शकता.

