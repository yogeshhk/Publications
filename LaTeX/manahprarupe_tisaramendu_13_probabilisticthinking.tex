\chapter{पाऊस पडेल काय?}

प्रसिद्ध हिंदी चित्रपट ‘शोले’तील ‘जय’ (अमिताभ बच्चन) ला कोठलाही निर्णय घेण्यासाठी छापा-का-काटा करण्याची सवय असते, आठवतंय? इथे भोळ्या ‘वीरू’ला वाटत असतं की, आपण जिंकण्याची शक्यता ५०% तरी आहे. तशाच पद्धतीने बुगू-बुगू भोलानाथला ‘पाऊस पडेल काय?’ असं विचारल्यावर त्याने मान डोलावली तर ‘हो’ नाहीतर ‘नाही’. येथेही खात्री नाही तर शक्यतेचा खेळ. क्रिकेट मॅच मध्ये कोण फलंदाजी आधी करणार हे पण नाणेफेकीवर ठरते. दोन्ही संघांना ‘टॉस’ जिंकण्याची समान संधी, म्हणजेच ५०% शक्यता. या सर्व उदाहरणांमध्ये खात्रीलायक काही न कळता शक्यता वर्तवल्या जात आहेत. आयुष्यातही अनेक वेळा निर्णयांमध्ये खात्रीशीर सांगता येत नाही. बहुतेक ठिकाणी असतो तो फक्त अंदाज. उदाहरणार्थ, उद्या पाऊस पडेल का हे अचूक सांगता येत नाही, म्हणून हवामान खातं सांगतं, “८०% शक्यता आहे” किंवा “२०% आहे.” पण आपण या शक्यतेचा वापर आपण कसा करायचा? छत्री न्यायची की नाही? सर्वसाधारणपणे विचार केला तर ५०%च्या वर शक्यता असेल तर ‘न्यायाची’ आणि कमी असेल तर ‘नाही’. पण प्रत्येक वेळी हे सूत्र लागू होत नाही. ते प्रसंगानुरूप बदलू शकते. समजा, तुम्हाला अतिमहत्वाच्या मीटिंगला जायचे आहे, भिजून चालणारच नाहीये तर २०% इतकी कमी शक्यता असूनसुद्धा तुम्ही छत्री न्याल आणि समजा मौजमजेसाठीच-ट्रेकिंगला बाहेर पडत असाल तर ८०% इतकी जास्त शक्यता असून सुद्धा छत्री नेणार नाहीत, बरोबर ना? अशा पद्धतीने संभाव्यतेचा-शक्यतेचा वापर करून निर्णय घेण्याचा मेंटल मॉडेल (मन:प्रारूप) ला म्हणजेच विचारचित्राला ‘प्रोबॅबिलिस्टिक थिंकिंग’ (संभाव्यतेचा विचार) म्हणतात. 

‘संभाव्यतेचा विचार’ वैयक्तिक निर्णयांपुरता मर्यादित नाही. तो सामूहिक निर्णयांमध्येही अत्यंत उपयुक्त ठरतो. उदाहरणार्थ, सरकार पावसाच्या अंदाजावरून पूर नियोजन करतं, विमा कंपन्या पीक वीमा दर ठरवतात, शहरांमध्ये वाहतूक नियोजन होतं, हे सगळं शक्यतांच्या आधारावर. अशाच प्रकारे ‘संभाव्यतेचा विचार’ याची इतर काही उदाहरणे पाहुयात. 

शेअर बाजारात किंवा म्युच्युअल फंडात गुंतवणूक करताना, कोणताही पर्याय खात्रीशीर परतावा देत नाही, हे लक्षात घेतलं पाहिजे. यशस्वी गुंतवणूकदार विविध शक्यतांचा विचार करतात जसे की आर्थिक परिस्थिती, कंपनीची कामगिरी, इत्यादी. यावरून भाव वर-खाली जाण्याच्या शक्यता जोखतात व आपल्या जोखीम-क्षमते (रिस्क ऍपेटाइट) नुसार निर्णय घेतात. 

पालक म्हणून तुम्ही तुमच्या सातवीतील मुलाला कोडिंग क्लासमध्ये घालावे का? यामुळे तो भविष्यात तंत्रज्ञान क्षेत्रातच जाईल याची कोणतीही खात्री नाही, पण जर अगदी कोडिंग नाही तर संगणकीय विचारांची लवकर ओळख झाल्यामुळे भविष्यातील योग्यतेची ‘शक्यता वाढत असेल’, तर तो एक योग्य निर्णय असू शकतो. समजा आधीच कळले की त्याला ते काही आवडत नाहीये किंवा जमत नाहीये तेंव्हा मार्ग बदलून दुसऱ्या मार्गाचा विचार करता येऊ शकतो. 

ऐच्छिक शस्त्रक्रिया किंवा प्रतिबंधात्मक चाचण्या अनेकदा “जोखमीचे घटक” पाहून निवडल्या जातात. जर हस्तक्षेप न केल्यास हृदयविकार १०% वाढण्याची शक्यता असेल तर आताच कृती करण्यासाठी ही जोखीम पुरेशी आहे का? वय जास्त असेल तर गुंतागुंत (कॉम्प्लिकेशन्स) वाढण्याची शक्यता किती? तज्ञ डॉक्टरच हे स्वीकार्य पातळी (थ्रेशोल्ड) च्या खाली का वर आहे ते ठरवून निर्णय घेऊ शकतात. 

नोकरी करावी की स्टार्टअप? यामध्येही संभाव्य यशाच्या शक्यतेचा विचार केला जातो. स्वतः:चा स्वभाव, बाजारपेठ, कौटुंबिक आधार, आणि स्वयंचलन (ऑटोमेशन) ची जोखीम हे सगळे घटक निर्णय ठरवतात.

उत्तर प्रदेशातील मतदान सर्वेक्षण असो वा तामिळनाडूमधील युतीबद्दलचे अंदाज, निवडणुका म्हणजे शक्यतांचाच खेळ असतो. पण मतदार अनेकदा ६०% विजयाच्या अंदाजाला निश्चित निष्कर्ष समजतात आणि  हे विसरून जातात की उरलेली ४०% शक्यता प्रत्यक्षात येऊ शकते आणि अनेकदा येतेही, विशेषतः भारताच्या वैशिष्ट्यपूर्ण (जनतेच्या भल्यासाठी काहीही!!) राजकीय परिस्थितीत.
आपण बरेचदा निश्चिततेच्या शोधात असतो. आपल्याला अचूक उत्तरं हवी असतात. पण जग हे शक्यतांवर आधारित असतं. संभाव्यतेचा विचार आपल्याला चांगले निर्णय घेण्यास मदत करतो, अचूक अंदाजाची हमी देत नाही. तो आपल्याला अवास्तव आत्मविश्वास टाळायला, अपेक्षा व्यवस्थापित करायला आणि जोखीम टाळायला शिकवतो.
जीवन हे हमखास निष्कर्षांची मालिका नसून, संधी आणि अंदाजांवर आधारित एक खेळ आहे. आपल्याला चांगल्या पैजा लावता आल्या पाहिजेत. आणि विशेषतः भारतात, जिथे मान्सून, बाजारपेठा, राजकीय समीकरणे, आणि माणसांचे वर्तन हे सर्व फारच अनिश्चित आहे, तिथे शक्यता समजून निर्णय घेणं ही फक्त शहाणपणाची नाही, तर आवश्यकतेची गोष्ट आहे.
डॉ. योगेश हरिभाऊ कुलकर्णी


