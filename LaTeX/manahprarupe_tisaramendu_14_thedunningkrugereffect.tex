\chapter{उथळ पाण्याला खळखळाट फार}

दरवर्षी फेब्रुवारी महिना येऊ लागला की, देशात अर्थतज्ज्ञांची लाट येते. कारण अर्थसंकल्प घोषित होणार असतो म्हणून.  केंद्रीय स्पर्धा परीक्षेची तयारी करणारा रमेशही त्याला कसा अपवाद असेल? तो कररचना, अनुदान आणि वित्तीय धोरणांवर आत्मविश्वासाने बोलायला लागतो, अगदी अर्थमंत्र्यांच्या अविर्भावात. एक-दोन महिन्यात, जेव्हा आयपीएलचा हंगाम असतो तेव्हा रमेश क्रिकेट-विश्लेषक होऊन जातो. टीव्हीवर खेळपट्टी बघून ती कोणाला साथ देईल याचा अंदाज तो छातीठोकपणे सांगतो. जगात कोठेही युद्धाचे वारे वाहायला लागले की रमेश सामरिक-नीती तज्ज्ञ म्हणून तयार. कोणी, कुठून आणि किती सैन्य तैनात केले पाहिजे याचा सखोल आराखडाच तो जाहीर करतो. एवढ्या विविध विषयात मुशाफिरी करणाऱ्या रमेशला त्याच्या वैयक्तिक आयुष्यात ‘प्रीलिम परीक्षा’ काही सुटत नाहीये, अनेक वर्षे झाली तरी. त्याला अभ्यासक्रमातलं विचारलं की ‘वित्तीय तूट आणि व्यापारी तूट यातील फरक काय?’ तर तो अजूनही गडबडतो. त्याची विद्वत्ता व्हॉट्सऍपचे व्हिडीओ फॉरवर्ड करून, चहा पीत त्यावर जोरजोरात चर्चा कारण्यापुरतीच आहे खरे तर, पण आव असा आणतो की विचारू नका. मला नक्की वाटते की आमचा रमेश हे काही अपवादात्मक उदाहरण नाहीये, तुमच्या आसपास असे अनेक ‘बहूआयामी तज्ज्ञ’ तुम्ही पहिले असतील. अशा अति-आत्मविश्वासी पण पोकळ विद्वत्तेच्या उदाहरणास ‘डनिंग-क्रुगर इफेक्ट’ मेंटल मॉडेल (मन: प्रारूप अथवा विचार-चित्र) म्हणतात. सोप्या भाषेत, मराठीतील म्हणीनुसार ‘उथळ पाण्याला खळखळाट फार’. हा एक ‘कॉग्निटिव्ह बायस’चा (पूर्वग्रहदूषित विचारसरणी) प्रकार आहे जेथे अत्यल्प ज्ञान किंवा क्षमता असणाऱ्यांना आपण ‘लयं भारी’ असा जोरदार आत्मविश्वास असतो.
१९९९ साली मानसशास्त्रज्ञ डेव्हिड डनिंग आणि जस्टिन क्रुगर यांनी हे मेंटल मॉडेल जगासमोर आणले. यामागची कहाणी गंमतीशीर आहे.  एका माणसाने चेहऱ्यावर लिंबाचा रस लावून दोन बँका लुटल्या, कारण त्याचा विश्वास होता की त्यामुळे तो सीसीटीव्ही कॅमेर्‍यांना दिसणार नाही! तो ना वेडा होता, ना नशेत होता, फक्त त्याला स्वतःच्या अज्ञानाची अजिबात जाणीव नव्हती. वेगळ्या भाषेत सांगायचं तर, "जितकं कमी माहिती असतं, तितकं आपल्याला माहित नसल्याचंही माहित नसतं." जे लोक नवखे असतात, त्यांच्याकडे स्वतःच्या कौशल्याचं अचूक मूल्यांकन करण्यासाठी पुरेशी माहिती नसते, त्यामुळे ते स्वतःला खूप हुशार समजतात. उलट जे खरे तज्ज्ञ असतात, ते अनेकदा स्वतःच्या क्षमतेबद्दल संकोच करतात, कारण त्यांना वाटतं की इतरांनाही तितकंच ज्ञान आहे.
भारतीय संदर्भात हा परिणाम अनेक ठिकाणी दिसतो. स्पर्धा परीक्षा देणाऱ्यांमध्ये, टीव्ही-सोशल मीडियावरील चर्चांमध्ये, आणि व्हॉट्सअ‍ॅप युनिव्हर्सिटीमधून पदवी घेतलेले लोक भौगोलिक राजकारण, वैद्यकीय शास्त्र आणि संविधान यावर निर्विवाद मतं मांडताना दिसतात. याची काही दैनंदिन आयुष्यात दिसणारी उदाहरणे पाहुयात. 
कोविड-१९ च्या काळात, कितीतरी लोक कोणती औषधं चालतात हे ठामपणे सांगत होते. घरगुती उपाय, ‘आजीबाईच्या बटव्यातील’ औषधे, वेगवेगळे काढे सर्रास सुचवले जात होते, त्यांच्याकडे वैद्यकीय पार्श्वभूमी नसतानाही. 
कंपन्यांमधील नुकतेच रुजू झालेले (फार नावाजलेल्या कॉलेजमधून शिकलेले) कर्मचारी अनेकदा बैठकींमध्ये अतिसोप्या कल्पना फार ठामपणे मांडतात, त्या मागच्या गुंतागुंतीची त्यांना जाणीव नसते. त्याचवेळी अनुभवी व्यवस्थापक मात्र विचारपूर्वक निर्णय घेतात, जोखमी समजून पुढे जातात. बरेचदा निर्णय हे केवळ तांत्रिक अथवा आर्थिक बाबींवर अवलंबून नसतात, मानवी संबंध, कंपनीतील राजकारण हे भारी पडत असते आणि तेथे अनुभवाशिवाय गत्यंतर नसते. 
पहिल्यांदाच व्यवसाय करणारे काही उद्योजक, आपल्या अजमावून न पाहिलेल्या कल्पनांनी गुगल-मायक्रोसॉफ्टसारखी कंपनी उभी करणार असल्याचा आत्मविश्वास बाळगतात. केवळ सादरीकरण आणि दुर्दम्य विश्वास घेऊन ते बाजारात उडी घेतात. ज्याचा अभ्यास केला नसतो काही वर्षे ‘घासली’ नसतात अशा क्षेत्रात केवळ ‘इंप्रेशन’ महत्वाचे ठरत नाही, तर काम दाखवावे लागते. 
डनिंग-क्रुगर इफेक्ट समजून घेण्यामागचा उद्देश इतरांची खिल्ली उडवणे नाही, तर स्वतःकडे डोळसपणे पाहणे हा आहे. आपल्यापैकी प्रत्येकालाच काही ना काही गोष्टी अज्ञात असतात. विशेषतः नवीन क्षेत्रांमध्ये. सगळेच काही ‘इलॉन मस्क’ नसतात की जेथे जाईल त्यात यश मिळवेल. सर्वसामान्यांना यशासाठी गरज असते ती अपार कष्टाची, बौद्धिक क्षमतेची आणि महत्वाचे  म्हणजे नम्रतेची; म्हणजेच "मला माहित नाही" हे स्वीकारण्याची आणि त्यात लाज न वाटण्याची क्षमता.
आपण या मेंटल मॉडेलने ग्रासलो आहे का हे शोधण्यासाठी स्वतःला काही प्रश्न विचारूया. ‘या विषयावर आत्मविश्वासाने बोलायला माझ्याकडे पुरेसा अनुभव आहे का?’, ‘मी अशा लोकांशी चर्चा केली आहे का, जे माझ्याशी असहमत आहेत किंवा ज्यांना अधिक माहिती आहे?’, ‘मी ठाम बोलतोय, पण हे खरंच ज्ञानावर आधारित आहे का’
भारतीय परंपरेतील एक सुंदर विचार आहे,  खरा विद्वान तो नाही की ज्याला सगळं माहित आहे, तर तो आहे ज्याला हे माहित आहे की "आपल्याला अजून खूप काही माहित नाही."
आजच्या सोशल मीडियाच्या युगात प्रत्येकाकडे एक माईक आहे, ऑनलाइन किंवा ऑफलाइन. सर्वांना ‘व्यक्त’ होण्याची मुभा आहे. अशा काळात, डनिंग-क्रुगर इफेक्ट ही केवळ सावधगिरीची गोष्ट नाही, तर दिशा दाखवणारा एक मानसिक होकायंत्र आहे. तो आपल्याला सांगतो की, बोलण्याआधी थांबून विचार कर, लिहिण्याआधी वाच, नेतृत्व करण्याआधी कार्यकर्ता हो आणि सल्ला देण्याआधी अनुभव घे. सरकार असो, कार्यालय, शाळा किंवा घर, चांगले निर्णय घ्यायचे असतील, तर आत्मविश्वास आणि कौशल्य यामधील अंतर ओळखणं आणि भरून काढणं आवश्यक आहे, नाही का?

