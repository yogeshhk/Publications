\chapter{निर्हेतुक चुका, ‘क्षमापात्र’}

आज सकाळची वेळ गडबडीची होती. ८:३० वाजले होते. अर्ध्या तासात एक महत्त्वाची मीटिंग गाठायची होती. मी रिक्षा केली आणि लवकरच लक्षात आले की, रिक्षावाला जवळचा रस्ता न घेता चकवा देतोय. मनात शंका आली की, हा मुद्दाम `फसवतोय` का? मी स्थानिक असूनही मला परगावचा समजून लांबचा रस्ता घेतोय? `मी सांगितलेला रस्ताही त्याला कळत नव्हता.` दोघांत चांगलीच बोलाचाली झाली. ‘चार शब्द’ ऐकवले गेले. पोहोचलो, थोडा उशीर झाला, पण ठीक होते. आमच्या ‘संभाषणात’ असे कळले की तोच परगावाचा होता, कुटुंबाला गावी सोडून आलेला. नवीन शहर आणि मागे राहिलेल्या घराची विवंचना याने त्याचे चित्त थाऱ्यावर नसेल म्हणून `त्याच्याकडून` रस्ता लांबचा घेतला गेला. `नंतर माझ्या लक्षात आले` की, त्याचा काही फसवण्याचा हेतू नव्हता. उगाच बोललो. `सोडून द्यायला हवे होते.` त्याची चूक निर्हेतुक असल्याने `क्षमापात्र` होती, नाही का? माझ्याच जीवाची उगाचच घालमेल झाली. अशा ताण आणणाऱ्या, गैरसमज होणाऱ्या छोट्या-छोट्या गोष्टी `टाळून` जीवन साधे `आणि` सोपे करायला हवे.

आपल्या रोजच्या आयुष्यात अशा घटना वारंवार होतात. मंदिरात कोणी रांगेत घुसतं, ऑफिसमधला सहकारी महत्त्वाच्या मेसेजला उत्तर देत नाही, कॉलनीतील क्रिकेट खेळताना एखादी ‘सिक्स’ तुमच्या खिडकीवर येते, आणि कोणी माफीही मागत नाही. तेव्हा आपले मन `पटकन` निष्कर्ष काढते की, “हे मुद्दामच केले गेले असेल.” पण खरंच असे असते का? बहुतेक वेळा कारण वेगळे असते: विसरभोळेपणा, अकार्यक्षमता, दुर्लक्ष किंवा साधा मूर्खपणा. 

इथे ‘हॅनलॉनचे रेझर’ हे मेंटल मॉडेल (मनःप्रारूप) अथवा विचारचित्र मदतीला येते. हे विचारचित्र रॉबर्ट जे. हॅनलॉन यांनी मांडले. याच आशयाची कल्पना इतिहासात अनेकांनी, अगदी इ.स. १७७४ पासूनच मांडलेली आहे. नेपोलियन बोनापार्टने एकदा म्हटले होते की, ‘`जे अकार्यक्षमतेमुळे घडल्यासारखे वाटते, त्यामागे वाईट हेतू आहे असे समजू नका’. ‘रेझर’ म्हणजे दाढीचे ब्लेड. इथल्या संदर्भात सांगायचे तर, हे असे तत्त्व आहे, जे अनावश्यक कारणे कापून टाकते आणि साधे उत्तर निवडायला सांगते. म्हणजेच, लोकांमध्ये वाईट हेतू शोधण्याऐवजी, आधी साधे कारण शोधा. हे तत्त्व जरी पश्चिमेत जन्माला आले असले, तरी भारतासारख्या १.४ अब्ज लोकांच्या देशात याचे महत्त्व फार मोठे आहे. इथे वाहतुकीत, सरकारी कार्यालयात, एकत्र कुटुंबात आणि समाज माध्यमांमध्ये आपला संयम रोज तपासला जातो. अशा वेळी हे तत्त्व फक्त एक वरवरचा, गमतीशीर विचार नसून, एक महत्त्वाचा, जगण्यासाठी उपयोगी विचार ठरतो. याची काही आणखी उदाहरणे पाहूयात.

तुम्ही बॉसला एक रिपोर्ट पाठवता आणि उत्तरच येत नाही. लगेच मनात येते की, “मी काही चुकलो का?” `पण खरे कारण असू शकते की, बॉस सतत मीटिंगमध्ये असतील, तुमचा मेल स्पॅममध्ये गेला असेल किंवा (आपले आवडते कारण) ते विसरभोळे असल्याने त्यांना पुन्हा आठवण करून द्यावी लागेल.

तुमच्या खास मित्राने तुमचा वाढदिवस लक्षात ठेवला नाही. तुम्हाला वाटेल, “त्याला काही फरकच पडत नाही!” पण शक्यता आहे की, त्याच्या कामाच्या व्यापात किंवा सोशल मीडियावरील शेकडो मेसेजच्या गर्दीत तुमच्या वाढदिवसाची आठवण राहिली नसेल. 

एखाद्या गावात रस्त्यांची आणि नाल्यांची दुर्दशा पाहून आपल्याला वाटते, की हे मुद्दाम करतायत का? एवढा निधी खर्च होऊनही उपयोग का होत नाही? त्यामागे असेही कारण असू शकते की, संबंधित अधिकारी स्थापत्य अभियंता (सिव्हिल इंजिनियर) नसतील. कदाचित कला, इतिहास किंवा भाषा यांसारख्या विषयातून स्पर्धा परीक्षा उत्तीर्ण होऊन त्यांची नियुक्ती या पदावर झालेली असेल. ते खूप प्रयत्न करत असतील, पण तो त्यांचा विषयच नसल्याने कामाचा दर्जा चांगला नसेल.

आपल्या देशात लोकांमध्ये आपुलकी आहे, पण संयम कमी आहे. त्यामुळे छोटेसे दुर्लक्षदेखील आपण मोठा अपमान समजतो. हॅनलॉनचे रेझर आपल्याला विचारायला शिकवते की, “यामागे दुसरे काही साधं कारण असेल का?” हे तत्त्व चुकीचे वागणे माफ करायला सांगत नाही, पण आपल्या मनाचा ताण नक्कीच कमी करायला मदत करते. शक्य तिथे चूक लक्षात आणून द्यावी आणि सरकारी अनास्थेबद्दल आवाज उठवावा, पण परिस्थिती समजून घेण्याचाही प्रयत्न करावा. इतरांच्या चुकीत खोडसाळपणा शोधण्याऐवजी, आपण जर थोडे समजून घेतले, तर आपल्यालाच मनःशांती मिळते. पुढच्या वेळी एखादा रिक्षावाला तुमच्या अपेक्षेपेक्षा वेगळा रस्ता घेत असेल, तेव्हा असा विचार करा, कदाचित त्या रस्त्यावर मेट्रोचे काम सुरू असेल किंवा तुम्हाला माहीत नसलेला एखादा मोठा खड्डा पडला असेल. तो फसवतोय असे नाही, तोही तुमच्यासारखाच, रोजच्या गरजा भागवण्यासाठी धडपडणारा एक सर्वसाधारण माणूस आहे.


