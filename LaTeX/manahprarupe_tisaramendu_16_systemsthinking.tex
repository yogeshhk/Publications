\chapter{समग्र विचारांची शक्ती }

पावसाळा सुरू झाला की नेहमीचं दृश्य दिसतं, रस्त्यांवर पाणी साठलेलं. मुंबई, पुणे आणि बंगलोरसारखी मोठी शहरेही याला अपवाद नाहीत. खरंतर या महत्वाच्या आणि जागतिक व्यवहार असलेल्या शहरांकडून अधिक चांगल्या नियोजनाची अपेक्षा असते. पण मग जनतेला 'वॉटर पार्क'चा चकटफू अनुभव कसा मिळणार? स्थानिक नेत्यांना जनतेवर असेलेली कणव दाखवण्यासाठी बिस्किटवाटपाची संधीही कशी मिळणार? खरंतर, नालेसफाई, ड्रेनेज नियोजन आणि अंमलबजावणीवर वेळेवर लक्ष दिलं गेलं असतं, तर अशा गोष्टी टाळता आल्या असत्या. सध्या जनतेचा आक्रोश आणि तंग वातावरण निवळण्यासाठी इकडे-तिकडे थोडी डागडुजी केली जाते, बैठका होतात, जबाबदारी निश्चित करण्याचे आश्वासन दिले जाते, बस्स, नंतर पुढच्या वर्षी, पहिले पाढे पंचावन्न.

या समस्या तितक्या क्लिष्ट नाहीत. नीती, निधी आणि तंत्रज्ञान असूनही त्या का टिकून राहतात? कारण आहे आपल्या तुटक, तात्कालिक विचारसरणीत. आपण समस्यांची लक्षणं वेगवेगळी समजून घेतो, पण त्या मूळ कारणांचा गुंता सोडवायला टाळतो. खोलात जाऊन, बहुअंगी कारणमीमांसा टाळतो कारण त्यासाठी वेळ, बुद्धी आणि समन्वय लागतो. तातडीनं काहीतरी ‘करून दाखवलं’ म्हणून जाहिरात केली जाते आणि समग्र विचार टाळला जातो. पण अनेक वेळा स्पष्ट झालं आहे की, ‘सिस्टिम्स थिंकिंग’ हे मेंटल मॉडेल (मन:प्रारूप) ) अथवा विचारचित्र येथे खूप गरजेचं ठरतं.

सिस्टिम्स थिंकिंग हा जगाकडे पाहण्याचा एक विशिष्ठ दृष्टिकोन आहे. कुठलीही गोष्ट अथवा घटना किंवा एक स्वतंत्र घटक म्हणून नव्हे, तर परस्परांवर परिणाम करणाऱ्या, एकमेकांशी जोडलेल्या घटकांच्या रूपात अभ्यास करणे हे या विचारचित्राचे तत्व आहे. म्हणजे जसं एखादं जंगल समजून घ्यायचं, केवळ एखादे झाडं किंवा एखादा प्राणी अभ्यासून चालणार नाही तर, इतर सर्व घटक, त्यांचे परस्परांवर असलेले परिणामही लक्षात घ्यावे लागतील. बॉस्टन येथील एमआयटीचे जे. फॉरेस्टर यांनी ‘सिस्टम डायनॅमिक्स/थिंकिंग’ या शास्त्रशाखेची मांडणी केली तर चार्ली मंगर यांनी याला मेंटल मॉडेल्सच्या चौकटीत मान्यता दिली. याची आपल्या सभोवतालची काही उदाहरणे पाहुयात. 

भारतामध्ये सिस्टिम्स थिंकिंगचा अभाव दिसतो तो आपल्या नद्यांच्या स्वच्छता मोहिमांमध्ये. उदाहरणार्थ, १९८६ मध्ये सुरू झालेली ‘गंगा अ‍ॅक्शन प्लॅन’ ही योजना काही दशकांनंतर आणि हजारो कोटी रुपयांनंतरही पूर्णत्वास गेलेली दिसत नाही. गंगा अजूनही प्रदूषितच आहे. का? कारण आपण फक्त प्रामुख्याने सांडपाण्याच्या प्रकल्पांवर लक्ष केंद्रित केलं, पण वरच्या पातळीवरचं कचराप्रबंधन, स्थानिक प्रशासन आणि लोकांची सवय बदलणं हे सर्व घटक बऱ्यापैकी दुर्लक्षित केले. ते सगळं एकाच ‘सिस्टम’चा भाग आहे त्या सर्वांचा विचार करून उपाययोजना ठरवल्या पाहिजेत.

ग्रामीण भागात कॉम्पुटर-टॅबलेट वाटप केलं जातं. उद्दिष्ट चांगलं आहे. पण नियमित वीज नाही, शिक्षक प्रशिक्षित नाहीत, आणि अभ्यासक्रम कालबाह्य असल्यामुळे उपयोग होत नाही. बदल हवा असेल, तर पोषण, शिक्षक प्रेरणा, पायाभूत सुविधा आणि पालक साक्षरता यांसह पूर्ण पार्श्वभूमी विचारात घ्यावी लागेल.

वाहतूक कोंडी रोखण्यासाठी उड्डाणपूल बांधले जातात. काहीसा दिलासा मिळतो. पण लोकसंख्येचा वाढता लोंढा आणि वाहनांची वाढ यामुळे ते उपाय अल्पकालीन ठरतात. फक्त रस्ते नव्हे, तर शहर नियोजन, सार्वजनिक वाहतूक, आणि कडक नियमांची अंमलबजावणी गरजेची आहे.

भारतातील शेती आजही मोठ्याप्रमाणात नैसर्गिक पाण्यावर, सरकारी अनुदानावर, सावकारी पतपुरवठ्यावर आणि बाजारभावाच्या अनिश्चिततेवर तोल सांभाळत आहे. थोडे काही चुकले तर गरीब शेतकऱ्यापुढे आत्महत्येशिवाय पर्याय उरत नाही. कर्जमाफी ही तात्पुरती मलमपट्टी आहे. सिस्टिम्स थिंकिंगने विचार केल्यास ‘शेती ही केवळ पेरणी आणि कापणी नाही’, तर एक आर्थिक, सामाजिक, (कधीकधी राजकीय) आणि नैसर्गिक जाळं आहे. त्यासाठी सिंचन, बाजारपेठा, जमीनधारण, आणि पीकविमा अशा सर्व पातळ्यांवर विचार करणं गरजेचं आहे.

आजचा वेगवान काळ आपल्याला त्वरेने निर्णय घ्यायला भाग पाडतो. पण सिस्टिम्स थिंकिंग सांगते की, गडबडीत चुकण्याची शक्यता वाढते. योजना आखताना ‘फीडबॅक लूप्स’ (प्रतिक्रया-पडसाद तपासणी-बदल), ‘डिलेज ‘ (विलंब, कालांतराने होणारे परिणाम), आणि ‘इमर्जन्स‘(अनपेक्षित परिणाम) यांची जाण असणं गरजेचं आहे.

भारतीय जुगाड ही कल्पकता असली, तरी तो समग्र विचारांशिवाय कधी कधी उलट परिणाम देतो. खरी प्रगती हवी असेल, तर तात्पुरते उपाय करणारे नव्हे, तर संपूर्ण चित्र पाहणारे विचारवंत होणं गरजेचं आहे. पुढच्या वेळी एखादी समस्या दिसली, तेव्हा "काय बिघडलंय?" एवढंच न विचारता, "या मागची संपूर्ण प्रणाली काय आहे?" हा प्रश्न स्वतःला विचारणार का? 


