\chapter{ताजे स्मरण...निर्णयास कारण }

नुकताच एक मोठा आणि अतिशय दुर्दैवी विमान अपघात झाला. विमानातील एक सोडून सर्वच मृत्युमुखी पडले. विमान जिथे कोसळले, तिथेही काहींचा मृत्यू झाला, अनेक जखमी झाले. हे सर्व पाहून, ऐकून मन विषण्ण होते. जीवन किती क्षणभंगुर आहे, याची जाणीव होते. याचा आणखी एक परिणाम म्हणजे आपल्याला विमान प्रवासाची तीव्र भीती वाटू शकते. तथापि, आकडेवारी पाहता, कार अपघातात मृत्यूची शक्यता विमान अपघातांपेक्षा खूपच जास्त आहे. तरीही, विमान अपघाताची कल्पना अधिक भयावह वाटते. का?

या प्रश्नाचं उत्तर आहे‘अ‍ॅव्हेलेबिलिटी ह्युरिस्टिक’ नावाच्या मेंटल मॉडेल (मन:प्रारूप) अथवा विचारचित्रामध्ये. मनातील विचारांमध्ये 'हाजीर तो वजीर' या तत्त्वानुसार, नुकतेच आलेले विचार अधिक प्रभावी ठरतात. म्हणजेच या मॉडेलचा महत्त्वाचा पैलू आहे,'रिसेन्सी इफेक्ट'. याचा अर्थ, जे अलीकडे घडलंय, त्याचा आपल्या निर्णयांवर अधिक प्रभाव पडतो. आपला मेंदू निर्णय घेताना नेहमी खोलवर विचार करत नाही. अनेकदा लगेच आठवणीत येणाऱ्या गोष्टींवर आधारित निर्णय घेतो. ‘हाजीर तो वजीर’ हे विचारचित्र याच विषयी आहे. जे लक्षात ताजं असतं, ते प्रभाव टाकतं, पण ते आकडेवारीनुसार खरं असतंच असं नाही.

खरंतर हे मेंटल मॉडेल ही एक मानसिक सवय आहे. एखादी घटना लक्षात राहिली की आपण समजतो ती फार महत्त्वाची आहे, नेहमीच घडत असणार. ती डोक्यात खोलवर रुजते आणि पुढचे निर्णय त्यावरच आधारित होतात. आपले मेंदू शक्यता (प्रोबॅबिलिटी) मोजत नाहीत, तर कधी काय ऐकले गेले, हे लक्षात ठेवतात. भभारतासारख्या देशात, जिथे सोशल मीडियावर, कौटुंबिक चर्चांमध्ये आणि व्हॉट्सअ‍ॅप मेसेजेसमधून सतत काही ना काही ऐकायला मिळतं, तिथे अलीकडे ऐकलेल्या गोष्टींचा प्रभाव अधिक असतो. हा प्रभाव वैयक्तिक निवडींपुरताच मर्यादित नसून, धोरणांवरही पडतो. त्याची काही उदाहरणे पाहूया. 

निवडणुका काळाच्या थोडेच आधी सर्व पक्षांकडून लोकप्रिय घोषणांची खैरातच सूरु होते. ती प्रलोभने समाज मनात ताजी असतानाच मत दिलं जात असल्याने फायदा होतो ही यामागची अपेक्षा असते आणि अशातऱ्हेने आधीच्या चारवर्षांच्या अपयशावर सहजपणे पांघरूण टाकता येते.

कोविड काळात शेअर बाजार कोसळला होता. तेव्हा अनेकांनी घाबरून शेअर्स, म्युच्युअल फंड विकले होते. कुणाला हे लक्षात आलं नाही की (अगदी खात्री देता येत नसली तरी) शेअर बाजार तसेच सेन्सेक्स हा सर्वसाधारणपणे दीर्घकाळात सातत्याने वाढत असतो. नजीकच्या काळातील मथळे वाचून निर्णय होतात, शुद्ध आकड्यांच्यावर, पुराव्याच्या माहितीवर (डेटा) नव्हे.

जाहिराती या सतत आपल्या डोळ्यासमोर दिसत अथवा कानावर पडत असल्याने, एखादी गोष्ट विकत घेताना त्या जाहिरातीचीच आठवण होऊन तेच उत्पादन विकत घेतले जाते ना की त्याच्या गुणधर्माचा अभ्यास करुन. आपल्याला कधी कधी त्रास होतो, चीड येते, पण तरीही त्या दाखवल्या जातात कारण त्या नाकारात्मक भावनेतही त्या लक्षात नक्की राहतात आणि याचा उत्पादन विक्रीत फायदा होतो. 

आयआयटी किंवा बिट्समधून शिकणारा एखादा विद्यार्थी कॉलेज सोडून स्टार्टअप करतो आणि यशस्वी होतो. ही कहाणी ऐकून अनेकजण नोकरी-शिक्षण सोडतात. पण यशस्वी उदाहरणं मोजकीच असतात. बहुसंख्य अपयशी प्रयत्न दिसत नाहीत.

नामांकित विद्यापीठांतील विद्यार्थ्यांना मिळणाऱ्या कोट्यवधीच्या पगाराचे आकडे पाहून, पालक आपल्या आठवीतल्या मुलालाही जेईई क्लासला घालतात.मात्र त्या मुलाचा कल, बौद्धिक क्षमता आणि पुढचं वास्तव लक्षात घेतलं जात नाही.

या सगळ्या उदाहरणांमधून एक स्पष्ट धोक्याची घंटा ऐकू येते, ‘नजीकता’ आणि ‘महत्त्व’ यात गोंधळ होण्याची शक्यता. ‘अ‍ॅव्हेलेबिलिटी ह्युरिस्टिक’ हे मॉडेल हा धोका टाळायला शिकवतं. आपण नजीकच्या (पण क्वचित घडणाऱ्या) घटनांवर विश्वास ठेवतो आणि दररोज जीवघेण्या ठरणाऱ्या गोष्टी, जसं की प्रदूषण, मधुमेह, बेदरकार ड्रायव्हिंग यांकडे साफ दुर्लक्ष करतो.

मग यावर उपाय काय? थांबा आणि स्वतःला विचारा की ही गोष्ट खरंच सामान्य आहे का, की फक्त अलीकडे घडल्यामुळे लक्षात राहिली आहे? यावरील माहिती (डेटा) शोधा. भावनांपेक्षा आकड्यांवर आणि पुराव्यावर विश्वास ठेवा. काही गृहितकं (ऍझम्पशन्स) केली आहेत का ते तपासा. कारण जर असं नाही केलं, तर आपण आकाशातून कोसळणाऱ्या विमानाला घाबरू, पण आपल्याच पायाखालच्या रस्त्यातल्या खड्ड्याकडे दुर्लक्ष करू.



