\chapter{जावे त्याच्या वंशा तेव्हा कळे}

भारतात लोकांचा सरकारी यंत्रणांवरील विश्वास केवळ अकार्यक्षमतेमुळेच नाही, तर जबाबदारीच्या अभावामुळे देखील खिळखिळा झाला आहे. जरा बारकाईने पाहा. आरोग्यमंत्री स्वतःच्या कुटुंबीयांवर उपचार नामांकित खासगी रुग्णालयांमध्ये करून घेतात, तर करोडो सामान्य लोक सरकारी रुग्णालयांतील गर्दीत वाट पाहत राहतात. शिक्षणमंत्री आपली मुले सरकारी शाळांमध्ये घालण्याचे सोडून, ‘पॉश’ आंतरराष्ट्रीय शाळांमध्ये घालतात. मातृभाषेचा आग्रह धरणे मान्यच आहे पण त्याच्या अंमलबजावणीस हिंसक स्वरूप देणाऱ्यांची मुले कुठल्या माध्यमांच्या शाळेत शिकतात? हे सर्व बघून प्रसिद्ध विचारवंत आणि लेखक नसीम निकोलस तालेब यांनी विचारलेला एक धारधार प्रश्न आठवतो, "जो आर्किटेक्ट-बिल्डर स्वतः निर्मिलेल्या इमारतीत राहायला तयार होत नाही, त्याच्यावर तुम्ही विश्वास ठेवाल का?" उत्तर अगदी स्पष्ट आहे, पण हा साधा नियमही अनेकदा दुर्लक्षित केला जातो. तालेब यांचे तत्त्व आहे की, जो सल्ला देतो किंवा निर्णय घेतो, त्याच्यावर त्या निर्णयांचे प्रत्यक्ष परिणामही यायला हवेत, तरच तो त्याचे काम गांभीर्याने करेल. हीच संकल्पना आहे ‘स्किन इन द गेम’ या मेंटल मॉडेल (मन:प्रारूप) अथवा विचारचित्राची. जर एखाद्याला त्याच्या निर्णयाच्या यशाचं श्रेय मिळत असेल, तर संभाव्य अपयशाची जबाबदारीही त्यानेच घ्यायला हवी. रस्ते विभागाच्या अधिकाऱ्यांना, त्यांनी ‘पास’ केलेल्या रस्त्यांवरून दुचाकीवरून नेहमी प्रवास करायला भाग पाडले तरच रस्त्यांची गुणवत्ता सुधारेल, नाही का? या मेंटल मॉडेलची इतर काही उदाहरणे पाहुयात. 
जेव्हा एखादा संस्थापक स्वतःचे पैसे आपल्या स्टार्टअपमध्ये गुंतवतो, तेव्हा गुंतवणूकदार त्याच्यावर अधिक विश्वास ठेवतात. कारण तेथे ‘स्किन इन द गेम’ हे तत्व लागू पडते. याच्या उलट, काही कंपन्यांचे प्रमुख (सीईओ), धंद्यात कितीही नुकसान होत असलं तरी भरघोस पगार घेत राहतात आणि वेळप्रसंगी कंपनी डुबवून निघूनही जातात कारण त्यांची वैयक्तिक काहीच गुंतवणूक नसते. 
गावोगावी शाळांमध्ये ‘पोषक आहार योजना’ चालवल्या जातात. त्यात अन्न  बनवण्यासाठी कोणी बाहेरचा आचारी नेमला जात नाही. विद्यार्थ्यांच्या आयाच ते काम आळीपाळीने करतात. आपण बनवलेले अन्न  आपलेच मूल खाणार असल्याने चांगला दर्जा राखला जातो. 
काही वित्त-विश्लेषक-सल्लागार फक्त इतरांना सल्ले देतात, पण त्यांनी सुचवलेल्या ठिकाणी स्वतः  पैसे गुंतवत नाहीत. अशा सल्ल्यांचा काय उपयोग? त्यांचं ‘स्किन’ तिथे नसतं. पण जे विश्लेषक स्वतः त्या शेअर्समध्ये गुंतवणूक करतात, त्यांच्या शिफारसी अधिक जबाबदार, विचारपूर्वक असतात. त्यांच्या निर्णयांचा परिणाम त्यांच्या स्वतःच्या खिशावर होतो  त्यामुळे सल्ला देखील प्रामाणिक असतो.
एखादा गैरसरकारी संस्थेचा (एनजीओ) कार्यकर्ता जर शहरातून येऊन ग्रामीण भागात दोन दिवस कार्यक्रम करून निघून जात असेल तर त्याला काय समजणार? पण जो कार्यकर्ता त्या गावात राहतो, तिथले रस्ते, दवाखाने, वीज याच्या चांगल्या-वाईट गोष्टींचा भाग असतो, त्याचे निर्णय त्याच्या योजना अधिक समर्पक असतात.
आपल्या आदरस्थानी असलेले राजे, छत्रपती शिवाजी महाराज, राणा प्रताप, थोरले बाजीराव हे महान योद्धे स्वत: लढाईत अग्रेसर असायचे. त्यांनी आखलेल्या समरनीतीचा, योजनांचा थेट परिणाम त्यांच्या आयुष्यावरही होणार असल्याने निर्णयांची संपूर्ण जबाबदारी त्यांच्यावर असायची. ही असामान्य धडाडी पाहून त्यांचे सैन्य देखील मग जीवावर उदार होऊन साथ द्यायचे. 
आधुनिक भारताचं भविष्य केवळ नवकल्पनांवर आणि वाढीवर अवलंबून नाही, तर विश्वासावर देखील आहे. संस्थांवर, नेतृत्वावर, आणि एकमेकांवर. विश्वास निर्माण होतो तेव्हा, जेव्हा निर्णय घेणारे लोक त्या निर्णयांचे परिणाम स्वतः भोगतात. ‘स्किन इन द गेम’ हे मेंटल मॉडेल आपल्याला एक पुरातन सत्य पुनर्रबिंबित करते ते म्हणजे जेव्हा फायदा तुमचा असेल, तेव्हा नुकसानही तुमचंच असायला हवं. 
‘स्किन इन द गेम’ ही केवळ सरकारी, सामाजिक, व्यावसायिक किंवा आर्थिक उत्तरदायित्व घेण्याची गोष्ट नाही तर ती एक नैतिक जबाबदारीची चौकट आहे. तुम्ही ज्या निर्णयांचा इतरांवर परिणाम करत आहात, त्याचे काही परिणाम तुमच्यावरही हवेत, तेव्हा तुमचं निर्णय घेण्याचं वर्तन पारदर्शक, संवेदनशील आणि विश्वासार्ह होतं. जेव्हा दुसऱ्याला किंमत मोजायला लावून निर्णय घेणे शक्य असतं, तेव्हा खऱ्या नेतृत्वाची, जबाबदार नागरिकत्त्वाची, आणि नैतिक व्यावसायिकतेची खरी कसोटी सुरु होते. या मेंटल मॉडेलची कसोटी एकदम सोपी आहे,  “जर मी चुकलो, तर त्याची शिक्षा मलाच मिळणार आहे का आणि ती भोगण्याची माझी तयारी आहे का?” जोपर्यंत सत्तेच्या स्थानावर असलेले लोक “हो” असं प्रामाणिक उत्तर देत नाहीत, तोपर्यंत कुठेतरी नुकताच बांधलेला पूल पुन्हा कोसळू शकतो!!

