\chapter{श्रीखंडाची चौथी वाटी}

लग्नसमारंभात हल्ली प्रचलित असलेल्या ‘बुफे’ पद्धतीपेक्षा मला पंगतीची व्यवस्थाच जास्त आवडते. ‘जड’ प्लेट हातात घेऊन, उभे राहून, ताटकळत खाण्यापेक्षा, पंगतीत आरामात बसून पक्वान्नांचा विशेष आनंद घेता येतो. अशाच एका पंगतीत गोड पदार्थ होता, माझे आवडते श्रीखंड. मग काय विचारता सोय नाही. पहिली वाटी संपवली. फारच भारी. इच्छा संपेना. दुसरी घेतली. छान वाटलं. तिसरी खरंतर नको होती पण घ्यावी लागली, वधू-वरांच्या आग्रहाखातर. आनंद कमी झाला आणि कशीबशी संपवली. आतामात्र बास म्हणायचं असं  ठरवताच समोरच्या ‘पार्टी’तील ओळखीच्यांनी पैज लावली. केवळ ‘इगो’ पायी चौथी वाटी घेतली पण अगदी गळ्याशी येऊन कधीही पुढचा सोपस्कार होईल असे झाले. पहिल्या वाटीला हवेहवेसे वाटणारे श्रीखंड आता बघवतपण नव्हते. हा प्रसंग ‘द लॉ ऑफ डिमीनीशींग रिटर्न्स’ हे मेंटल मॉडेल (मन:प्रारूप) म्हणजेच विचारचित्र दर्शवतो. ह्याला आपण ‘लाभ ह्रासाचा नियम’ म्हणू शकतो. या नियमाचा उगम जरी अर्थशास्त्रातील असला तरी तो आपले काम, नातेसंबंध, आरोग्य आणि जवळजवळ प्रत्येक निर्णयावर लागू होतो. काही गोष्टी सुरुवातीला मोठा फायदा देतात, पण जसजसे आपण त्या अधिक करत जातो, तसतसे त्याचा परिणाम कमी होतो आणि कधी कधी तोटा होऊ लागतो. आज ज्या युगात "जास्त मेहनत = जास्त यश" असं समजलं जातं, तिथे हा नियम समजून घेणं फार गरजेचं आहे. कारण प्रत्येक गोष्टीत अधिक घालून उपयोग होईलच, असं नाही. याची काही उदाहरणे पाहुयात. 

स्वातंत्र्यानंतर भारतातील अन्न-धान्य उत्पादनाची स्थिती बिकट होती. मोठ्याप्रमाणात तुटवडा असल्याने परदेशातून ते मिळवण्यासाठी हात पसरावे लागायचे. मालही पाठवला जायचा तो निकृष्ट दर्जाचा. पण तो खाण्यासाठी वापरण्यावाचून गत्यंतर नव्हते. यावर आमूलाग्र उपाय म्हणून १९६०-७० च्या दशकात भारतात हरित क्रांती झाली. त्यावेळी सिंचन, नवीन बियाणं आणि खतांमुळे शेती उत्पादनात मोठी वाढ झाली. भारत स्वयंपूर्ण झाला. पण पुढील दशकांमध्ये याच पद्धतीने शेती केली गेली, तेव्हा उत्पादन फारसे वाढले नाही आणि खतांचा अति वापर केल्याने जमिनीची गुणवत्ताही खराब झाली. जे उपाय आधी फायदेशीर वाटत होते, तेच नंतर अकारण खर्चिक आणि अपायकारक ठरले.

रोज ३०-४० मिनिटे व्यायाम केल्याने ऊर्जा वाढते, मनःस्थिती सुधारते आणि आरोग्य उत्तम राहते. हा सुरुवातीचा मोठा फायदा आहे. पण तोच व्यायाम दिवसातून तीन-चार तास, कोणत्याही मार्गदर्शनाशिवाय केल्यास शरीराला दुखापत, प्रचंड थकवा आणि मानसिक ताण वाढू शकतो. इथे व्यायामाचा ‘अतिरेक’ फायद्याऐवजी तोट्याचा सौदा ठरतो.

परीक्षेच्या आदल्या रात्री जागून अभ्यास करणे कदाचित फायद्याचे ठरू शकते. पण जर हीच सवय बनली आणि रोजच रात्री जागरण केले, तर हळूहळू स्मरणशक्ती आणि एकाग्रता कमी होऊ लागते. सुरुवातीला फायदेशीर वाटणारी कृती, सवयीची झाल्यावर तिची परिणामकारकता घटते.

नवीन पिढीला तंत्रज्ञान-स्नेही बनवण्यासाठी शाळेत टॅबलेटवर शिकवले जाते. शैक्षणिक ॲप्समुळे विषय खेळकर पद्धतीने समजतात आणि लक्षात राहतात. पण जसजसा टॅबलेटच्या वापराचा वेळ वाढू लागतो, तसतशी त्यावर इतर ॲप्स आणि खेळांचे आक्रमण होते. मग अभ्यासापेक्षा करमणूकच जास्त होऊ लागते आणि मूळ हेतू बाजूला राहतो.

सुरुवातीला एखादे ॲप किंवा सॉफ्टवेअर वापरण्यास अतिशय सोपे असते. पण ते प्रसिद्ध झाल्यावर त्यात नवनवीन 'फीचर्स' टाकण्याची स्पर्धा सुरू होते. हळूहळू ते ॲप इतके किचकट आणि संथ होते की मूळ वापरकर्तेच त्याला कंटाळून नवीन, सोप्या पर्यायाकडे वळतात. इथेही 'जास्त' फीचर्स देणे फायद्याऐवजी तोट्याचे ठरते.

पैसा कोणाला नको आहे. वर्षाला ₹३ लाख कमावणाऱ्या व्यक्तीसाठी ₹१० लाखांचे उत्पन्न आयुष्य बदलून टाकणारे ठरू शकते. कालांतराने १० लाखांचे २० आणि २० लाखांचे ५० लाख झाल्यावर अनेक स्वप्ने साकार होतात, पण त्याचबरोबर खर्च आणि जबाबदाऱ्याही वाढतात. एका टप्प्यानंतर जेव्हा उत्पन्न ₹१ कोटींवरून ₹२ कोटी होते, तेव्हा मिळणाऱ्या आनंदात फारसा फरक पडत नाही; उलट कामाचा ताण आणि जबाबदाऱ्या वाढल्याने सुख कमी होऊ शकते.

एखाद्या मित्राशी आठवड्यातून एखादवेळेस बोलणं चांगलचं, ते संबंध दृढ करतं. गरजेच्या काळात दरदिवशीही कॉल केला तरी ठीक वाटतं. पण दिवसातून पाच वेळा फोन केला, तर कंटाळा येतो, चिडचिड होते, नाही का?

थोडक्यात, ‘लाभ-ह्रासाचा नियम’ हे मॉडेल निराशावादी नसून, ते एक व्यावहारिक सत्य आहे. ते आपल्याला शिकवते की, ‘नेहमीच ‘जास्त’ म्हणजे ‘चांगले’ असे नाही.’ कधीतरी योग्य वेळी थांबणे, हेच अधिक फायदेशीर ठरते. हा नियम आपल्याला "थांबण्याची योग्य मर्यादा कोणती?" हा प्रश्न विचारायला प्रवृत्त करतो.

आजच्या स्पर्धेच्या आणि सतत धावपळीच्या जगात, हा नियम एका ‘ब्रेक’सारखे काम करतो, जो आपल्याला अधिक प्रभावी आणि सुज्ञ बनवतो. प्रत्येक प्रयत्नाला एक मर्यादा असते आणि ती वेळेवर ओळखली, तरच फायदा होतो. कधीकधी, थोडे 'कमी' करणे हाच सर्वात मोठा शहाणपणा असतो, नाहीतर श्रीखंडाप्रमाणेच कोणतीही चांगली गोष्ट 'गळ्याशी' यायला वेळ लागत नाही.

