\chapter{मनी वसे ते चहू दिसे }

२०२३ मध्ये सर्वोच्च न्यायालयाने कलम ३७० रद्द करण्याचा सरकारचा निर्णय कायम ठेवला. साहजिकच, सत्ताधारी पक्षाने त्याचे जोरदार स्वागत केले, पण जम्मू आणि काश्मीरमधील नेते आणि अनेक विरोधकांनी त्याला तीव्र विरोध केला. पुढे २०२४ मध्ये, त्याच न्यायालयाने सरकारने सुरू केलेली ‘इलेक्टोरल बाँड्स’ योजना असंवैधानिक ठरवली. यावेळी विरोधक आणि अनेक सामाजिक संघटनांनी “हा लोकशाहीचा विजय आहे,” म्हणत या निर्णयाचे स्वागत केले. पण सत्ताधारी पक्षाच्या समर्थकांनी याला “न्यायालयाचे सीमोल्लंघन (ओव्हररीच)” म्हटले. खरा मुद्दा असा आहे की, सर्वोच्च न्यायालयाचे काम संविधानाचा अर्थ लावणे, पुरावे तपासणे आणि निष्पक्ष निर्णय देणे आहे, जे ते सातत्याने करत असते. मात्र, जनतेच्या आणि राजकीय पक्षांच्या प्रतिक्रिया त्यांच्या निष्ठा व पूर्वग्रहांमधूनच उमटतात. या नैसर्गिक मानसिक प्रवृत्तीला ‘कन्फर्मेशन बायस’ किंवा ‘पुष्टीकरण पूर्वग्रह’ हे मेंटल मॉडेल (मन:प्रारूप) अथवा विचारचित्र म्हटले जाते. हा एका प्रकारचा ‘कॉग्निटिव्ह बायस’ अर्थात मानसिक-वैचारिक पूर्वग्रह आहे.

पुष्टीकरण पूर्वग्रह ही आपल्या मेंदूची एक नैसर्गिक प्रवृत्ती आहे. आपण आपल्या विश्वासांना बळकटी देणारी माहिती शोधतो, ऐकतो, लक्षात ठेवतो आणि विरोधी माहितीकडे दुर्लक्ष करतो. याचाच अर्थ, दोन भिन्न मतांचे लोक एकाच घटनेकडे पाहूनही आपापल्या मताचीच अधिक खात्री बाळगतात. हे विशेषतः राजकीय, धार्मिक किंवा भावनिक यांसारख्या संवेदनशील विषयांमध्ये अधिक दिसून येते. हा एक मानसिक शॉर्टकट आहे, जो काही वेळा उपयुक्त ठरतो. आपल्या जगाविषयीच्या धारणेला किंवा अहंकाराला धक्का लागू नये म्हणून मेंदू ही सोपी वाट निवडतो. गुंतागुंतीच्या जगात प्रत्येक वेळी नव्याने विचार करण्याची ऊर्जा वाचवण्यासाठी ही एक नैसर्गिक संरक्षणप्रणाली आहे. पण अनेकदा, विशेषतः गुंतागुंतीच्या समस्यांमध्ये, तो चुकीच्या निर्णयांकडे नेऊ शकतो. याचे परिणाम म्हणजे खोटी बातमी खरी मानणे, अयोग्य व्यक्तींवर विश्वास ठेवणे किंवा संकुचित विचारसरणीत अडकून राहणे. याची काही उदाहरणे पाहुयात.

शेअर बाजारात अनेक गुंतवणूकदार एखाद्या ‘नावाजलेल्या’ कंपनीवर किंवा उद्योजकावर इतका दृढ विश्वास ठेवतात, की त्या उद्योजकाची पुढची पिढी आली तरी तो कायम राहतो. ते त्या कंपनीबद्दलची केवळ सकारात्मक माहितीच लक्षात घेतात. ते केवळ त्याच आर्थिक सल्लागारांना किंवा वृत्तवाहिन्यांना फॉलो करतात, जे त्यांच्या गुंतवणुकीच्या निर्णयाला दुजोरा देतात. कायदेशीर अडचणी, आर्थिक घसरण आणि कौटुंबिक कलह यांसारख्या नकारात्मक बाबींचा कंपनीच्या भविष्यावर होणारा परिणाम माहीत असूनही त्याकडे दुर्लक्ष केले जाते. शेअरची किंमत घसरली तरी, ते याला "ही तात्पुरती बाब आहे," असेच समजतात आणि अधिक गुंतवणूक करून तोटा वाढवतात.

विवाहाच्या बाबतीतही मुला-मुलीच्या अनुरूपतेपेक्षा ‘घराण्याला’ अधिक महत्त्व दिले जाते. आपला विश्वास असतो की इतक्या प्रसिद्ध, तालेवार घराण्यात सर्व काही चांगलेच असणार. काही ज्ञात अप्रिय घटनांकडे सोयीस्करपणे दुर्लक्ष केले जाते. किरकोळ उणिवांकडे कानाडोळा करण्याच्या नावाखाली मोठ्या समस्याही झाकल्या जातात, ज्याचे पर्यवसान नंतर दुर्दैवी घटनांमध्ये होते. ‘आपला अंदाज चुकला’ हे मान्य करण्याच्या त्रासापेक्षा ‘सगळं ठीक होईल’ या भ्रमात राहणे मेंदूला सोपे वाटते.

बहुतांश पालकांना वाटते की मुला-मुलींनी डॉक्टर किंवा इंजिनीअरच व्हावे, आणि त्यातही कॉम्प्युटर शाखाच निवडावी. काही यशस्वी उदाहरणे पाहून त्यांना खात्री पटते की हाच एकमेव सुरक्षित मार्ग आहे. नामांकित महाविद्यालयांतूनही सर्वांना नोकऱ्या मिळत नाहीत किंवा इतर क्षेत्रांतही उत्तम संधी आहेत, ही वस्तुस्थिती समोर असूनही, या पुष्टीकरण पूर्वग्रहामुळे पाल्याच्या कमी गुणांकडे आणि आवडीनिवडीकडे दुर्लक्ष करून, त्याला कॉम्प्युटर शाखेतच प्रवेश मिळवून दिला जातो.

सोशल मीडिया किंवा टीव्हीवरील चर्चा पाहिल्या की दोन स्पष्ट तट पडलेले दिसतात. तिथे जणू युद्धभूमीचे स्वरूप आलेले असते. पूर्वग्रह इतके तीव्र असतात की एखादी चांगली किंवा वाईट घटना घडली तरी दोन्ही बाजू समोरासमोर उभ्या ठाकलेल्या दिसतात. सोशल मीडिया ॲप्सदेखील संगणक प्रणाल्यांच्या (अल्गोरिदम्स) आधारे असे फीड तयार करतात, की तिथे फक्त आपल्या मतांशी जुळणाऱ्या पोस्ट्स दिसतात. यालाच ‘इको-चेंबर’ म्हणतात, जिथे विरोधी विचारांना प्रवेशच नसतो. यामुळे वैचारिक कट्टरता वाढते आणि समाजात संवादऐवजी संघर्षच अधिक दिसू लागतो. आपणही विरोधी मते दिसल्यास त्यांना ब्लॉक किंवा ट्रोल करतो, नाही का? खरंतर, आपल्या वैचारिक विकासासाठी परस्परविरोधी मते जाणून घेणे आवश्यक असते, ज्याला भारतीय परंपरेत ‘शत्रूबोध’ असे म्हटले आहे. समोरच्याची बाजू नीट अभ्यासली तरच त्यावर सुसंस्कृतपणे मात करता येते, अन्यथा केवळ आरडाओरड आणि शाब्दिक संघर्षच दिसतो.

मग या पुष्टीकरण पूर्वग्रहावर मात कशी करायची? सर्वात आधी, आपल्याला जागरूक व्हायला हवे. स्वतःला विचारा: 'मी एखाद्या विषयाच्या सर्व बाजू तपासत आहे, की केवळ मला पटेल तीच माहिती स्वीकारत आहे? माझी मते पुराव्यांवर आधारित आहेत की केवळ विश्वासांवर?' विरोधी मते सक्रियपणे शोधा आणि वाचा. व्यक्तीऐवजी विचारांवर चर्चा करा. समोरच्या व्यक्तीचा युक्तिवाद शक्य तितक्या चांगल्या प्रकारे समजून घेऊन मगच त्यावर प्रतिक्रिया द्या. यामुळे चर्चेची पातळी उंचावते. “मला यावर पुरेशी माहिती नाही,” असे म्हणण्याची सवय लावा. नेहमी बरोबर असण्याचा हट्ट सोडा. चुका मान्य करायला शिका, कारण त्यातूनच आपण शिकतो. तेव्हाच खऱ्या अर्थाने शहाणपणाचा प्रवास सुरू होतो.


