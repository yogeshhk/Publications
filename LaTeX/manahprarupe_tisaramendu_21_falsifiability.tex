\chapter{बाप दाखव, नाहीतर श्राद्ध कर}

मी कदाचित पहिली-दुसरीत असेन. उन्हाळ्याच्या सुट्टीत सोलापूरजवळ माझ्या मामाच्या गावाला गेलो होतो. मोठ्या, सपाट मैदानी भागात फुटबॉल खेळताना मी माझ्या मोठ्या मामेभावाला विचारलं, “आपलं हे गाव तिथे पुढे दिसतंय तिथपर्यंतच का?” तो हसून म्हणाला, “ती आडवी रेषा दिसतेय, ते क्षितिज आहे. त्यापलीकडेही खूप गावं आहेत. पृथ्वी ही फुटबॉलसारखीच गोल आहे.” मला विश्वासच बसला नाही. जमीन तर मला सपाट दिसतेय. पुढच्याच वर्षीच्या सुट्टीत आत्याकडे कोकणात गेलो होतो. सायंकाळी समुद्रकिनाऱ्यावर असताना, एक जहाज जणू पाण्याबाहेर हळूहळू वर येताना दिसलं. सुरुवातीला केवळ त्याचं टोक, मग हळूहळू ते संपूर्ण जहाज. तेव्हाच भावाचं म्हणणं पटलं. पृथ्वी गोल आहे. हा सिद्धांत एका छोट्या मुलालाही तपासता आला. मीच माझं पूर्वीचं मत खोडून काढलं.
ही घटना जरी अगदीच ‘बालिश’ असली तरी एक महत्त्वाचा मुद्दा अधोरेखित करते. आपण एखाद्या कल्पनेचं मूल्यांकन करताना क्वचितच विचारतो, “ही गोष्ट खोटी ठरवता येईल का?” इथेच ‘फॉल्सीफायेबिलीटी’ मेंटल मॉडेल (मन:प्रारूप) अर्थात, ‘असत्यसिद्धिक्षमता’ म्हणजेच "खोटेपणा सिद्ध करता येण्याची शक्यता" हे विचारचित्र कामास येते. ही कल्पना वैज्ञानिक तत्त्वज्ञ कार्ल पॉपर यांनी मांडली होती. यानुसार, कोणताही दावा वैज्ञानिक किंवा तार्किक मानायचा असेल, तर तो तपासता येणारा आणि गरज पडल्यास खोटा ठरवता येणारा असावा. जर एखादा सिद्धांत कधीच चुकीचा ठरवता येणार नसेल, तर ते विज्ञान नाही, ती अंधश्रद्धा आहे. याची काही उदाहरणे पाहूयात. 
भारतात एका योगी-बाबांनी त्यांचे एक उत्पादन मधुमेह (डायबेटीस) बरा करतं, असा प्रचार व जाहिरात केली होती. हजारो लोकांनी विश्वास ठेवला, मोठा खप झाला, पण काहींना शंका आल्याने त्यांनी पुराव्याची व शास्त्रीय तपासणीची मागणी केली. प्रकरण न्यायालयात गेल्यावर त्या बाबांनी आपण असा दावा केलाच नव्हता, असे सांगितले. नवीन संशोधन किंवा उत्पादन करायला कोणाचीच बंदी नाही, पण इथे काय करायला हवं होतं? सांख्यिकी दृष्ट्या स्वैर (रँडम) चाचण्या घ्यायला हव्या होत्या. उत्पादनाच्या वापराआधी आणि वापरानंतर रक्तातील साखरेचे प्रमाण तपासून मधुमेह कमी झाल्याचा दावा सिद्ध करता आला असता, नाही का?
राजकारणात तर कोण काय दावे करेल आणि आश्वासनं देईल, हे सांगताच येत नाही. ‘कोकणाचा कॅलिफोर्निया’ आणि ‘मुंबईचं शांघाय-सिंगापूर’ कधी होणार, हे देवालाही ठाऊक नसेल. याचप्रमाणे ‘गरिबी हटवू’ किंवा ‘भ्रष्टाचार मिटवू’ यांसारख्या शिळ्या घोषणांकडे आता कोणी लक्षही देत नाही. यावर उपाय म्हणून निवडणुकांपूर्वी प्रसिद्ध केलेला जाहीरनामा-’वचन’नामा कायदेशीर दस्तऐवज बनवायला हवा. त्यातील आश्वासनांना पक्षाचे नेते जबाबदार असायला हवेत आणि त्यांनी त्यावर स्वाक्षरी करायला हवी. या जाहीरनाम्यात मोजता येणाऱ्या आकड्यांचा समावेश असावा. कोणती योजना, किती लोकांना, किती पैशांत देणार, याचं स्पष्ट विवरण हवं. यासाठी पैसा कुठून उभारणार, याचा संपूर्ण ताळेबंद (बॅलन्स शीट) त्यात असावा. या घोषणांची अंमलबजावणी न झाल्यास दंडात्मक कारवाई, अगदी पक्षाची नोंदणी रद्द करण्यापर्यंतची तरतूद, या दस्तऐवजात स्पष्टपणे नमूद असावी. जनतेला त्यांचे दावे तपासता येतील आणि वेळप्रसंगी खोटे ठरवता येतील. मगच त्यांच्या बोलण्यात सत्यता येईल.
आपल्या मतांची सत्यताही तपासता यायला हवी. उदाहरणार्थ, बर्‍याच पालकांचं मत असतं, की मुलांना मोबाईल दिला तर ते जास्त ‘ढ’ होतात. तर काहींना वाटतं, की विविध गोष्टी शिकल्याने ते ‘स्मार्ट’ होतात. यातलं खरं काय? अशा वेळी वर्तणूकशास्त्रातील अभ्यासक वेगवेगळ्या गटांवर प्रयोग करून हे तपासतात. यातून जे निष्कर्ष निघतील, तेच खरे मानायला हवेत.
एखादा स्टार्टअप संस्थापक म्हणतो, “हे अ‍ॅप पुढचं झोमॅटो होईल”. पण हा दावा तपासता येईल का? त्याला विचारायला हवं, “म्हणजे नक्की काय?” जर तो म्हणाला, “मी ३ महिन्यांत १०,००० रोजचे ग्राहक मिळवेन”, तर हे मात्र तपासता येईल. अशी तपासता येणारी उद्दिष्टं ठरवल्याशिवाय ‘फॉल्सीफायेबिलीटी’ हे मेंटल मॉडेल प्रभावीपणे वापरता येत नाही.
भारत एक आध्यात्मिक-धार्मिक देश आहे. हे खूप सुंदर आहे, पण कधीकधी श्रद्धेचा गैरवापर होतो. “ही पूजा केली की श्रीमंत व्हाल,” असा दावा तपासता येतो का? तर, नाही, पण जर ‘श्रीमंत’ म्हणजे किती पैसे, हे आधीच ठरवलं तर मात्र तपासता येतो! ‘फॉल्सीफायेबिलीटी’ श्रद्धेवर हल्ला करत नाही, तर ती श्रद्धाळूंना फसवणुकीपासून वाचवते.
‘जागतिक भूक निर्देशांक’ (ग्लोबल हंगर इंडेक्स) मध्ये भारताचा क्रमांक बांगलादेशपेक्षाही खालचा आहे. हे ऐकून आश्चर्य वाटतं. मग इथे विचारायला हवं की, यामागे मोजमाप काय आहे? ही आकडेवारी कशावर आधारित आहे? हे वास्तव आहे की कोणत्यातरी जागतिक संस्थेचा खोडसाळपणा, हे सिद्ध करता येतं. 
प्रत्येक वेळी कोणी कोणताही दावा केला, की आपण विचारायला हवं: “कशामुळे हे खोटं ठरू शकेल?” जर याचं उत्तरच मिळत नसेल, तर त्या गोष्टीवर विश्वास ठेवण्यात अर्थ नाही. भारताला फक्त स्वप्नं पाहणारे कोणत्याही आश्वासनांवर विश्वास, नेत्यांवर अंध-श्रद्धा ठेवणारे नागरिक नकोत; त्याला विचार करणारे, चाचणी करणारे आणि प्रश्न विचारणारे नागरिक हवेत. ‘फॉल्सीफायेबिलीटी’ हे केवळ वैज्ञानिकांचं हत्यार न राहता, प्रत्येक भारतीयाच्या विचारसरणीचं साधन बनलं पाहिजे. जी श्रद्धा तपासणी मानत नाही, ती श्रद्धा काय कामाची? 
