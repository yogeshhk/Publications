\chapter{सर सलामत तो पगडी पचास}

दुसऱ्या महायुद्धाच्या काळात अमेरिकेच्या लष्करासमोर एक मोठे संकट उभे राहिले. त्यांची अनेक बॉम्बर विमाने शत्रूच्या हल्ल्यात नष्ट होत होती. जी काही विमाने परत येत, त्यांची तपासणी केली असता असे दिसून आले की, बहुतेक गोळ्या पंखांवर आणि शेपटीवर लागल्या होत्या, तर इंजिन सुस्थितीत होते. त्यामुळे लष्करी तज्ज्ञांच्या मते, जिथे गोळ्या लागल्या आहेत, ते भाग अधिक मजबूत करायला हवेत.
पण, अब्राहम वॉल्ड नावाच्या एका हुशार गणितज्ञाने वेगळाच मुद्दा मांडला. त्याने स्पष्ट केले की, आपण फक्त परत आलेली विमाने पाहत आहोत. ही विमाने गोळ्या लागूनही परत आली आहेत, म्हणजेच त्या जागांवर गोळ्या लागूनही ती वाचली आहेत. याचाच अर्थ, त्या जागा फारशा धोकादायक नाहीत. खरी धोकादायक जागा तर ती होती, जिथे गोळी लागल्यामुळे विमान परत येऊच शकले नाही. आणि म्हणूनच आपल्याला त्या भागांवर कोणतीही इजा दिसत नाही. वॉल्डने सल्ला दिला की, ज्या भागांवर काहीच झाले नाही असे वाटते, म्हणजेच इंजिन, त्यालाच अधिक संरक्षण द्यायला हवे. त्याच्या या विचारानुसार योग्य बदल करण्यात आले आणि अनेक वैमानिकांचे प्राण वाचले.
ही गोष्ट केवळ युद्धापुरती मर्यादित नाही, तर आपल्या दैनंदिन विचार करण्याच्या पद्धतीबद्दलही आहे. आपण बहुतेक वेळा फक्त यशस्वी गोष्टी पाहतो आणि त्यावरूनच निष्कर्ष काढतो. याउलट, जे अपयशी ठरले, त्यांचा अभ्यास करणे अधिक महत्त्वाचे ठरते.
या विचारसरणीच्या मेंटल मॉडेलला (मन:प्रारूप) अथवा विचारचित्राला ‘सर्व्हायवरशीप बायस’ म्हणतात. यालाच मराठीत ‘उत्तरजिवीत्व पूर्वग्रह’ असे म्हणू शकतो. ‘सर सलामत तो पगडी पचास’ ही हिंदी म्हण याला चपखल बसते. याचा अर्थ, पगडी घालण्याची हौस खुप असली तरी डोकं असलं तरच, म्हणजे जगलो तरच ते शक्य आहे.  आपण केवळ यशस्वी आणि टिकून राहिलेल्या गोष्टी पाहतो. आपण नेहमीच यशाच्या कथा ऐकतो, त्या सांगितल्या जातात, छापल्या जातात, पण अपयशाच्या कथा सहसा पडद्याआड राहतात. याच कारणामुळे आपण अनेकदा चुकीचे निष्कर्ष काढतो. याची काही उदाहरणे पाहूया.
तरुण पिढीचे आदर्श असलेले बिल गेट्स, मार्क झकरबर्ग आणि स्टीव्ह जॉब्स यांनी शिक्षण पूर्ण न करता उद्योग सुरू केला. नंतर ते आर्थिक आणि जागतिक प्रभावाच्या दृष्टीने प्रचंड यशस्वी झाले. मग काही जण विचार करतात की, “त्यांनी केले, मग मी का नाही करू शकत?” पण याच वाटेवर जाऊन ज्यांनी सर्वस्व गमावले, त्यांच्याकडे आपले लक्ष जात नाही. त्यांनीही तितकाच प्रयत्न केला होता, पण त्यांची गोष्ट कोणीच सांगत नाही.
हेच शेअर मार्केटच्या बाबतीतही घडते. ‘एखाद्या स्टॉकने दहा वर्षांत शंभरपट परतावा दिला, त्यात गुंतवणूक केली असती तर तुम्ही आज करोडपती असता,’ अशा जाहिराती आपण पाहतो. पण कोणी आपल्याला हे सांगत नाही की, हजारो स्टॉक्सपैकी फार थोडेच टिकून राहतात. या काळात इतर अनेक कंपन्या बंद पडल्या आणि गुंतवणूकदारांचे पैसे बुडाले. पण त्यांची चर्चा कुठेच होत नाही.
बॉलिवूडमध्येही हेच चित्र आहे. “शाहरुख खान काहीही नसताना मुंबईत आला आणि सुपरस्टार झाला,” ही कथा सर्वांना माहीत आहे. हे उदाहरण डोळ्यासमोर ठेवून असंख्य तरुण-तरुणी आपले भविष्य आजमावायला मायानगरीत येतात. त्या सर्वांचे काय होते? त्यांचे स्वप्नही शाहरुख सारखेच होते, पण त्यांच्या वाट्याला वेगळे वास्तव आले.
आज अनेक तरुण यूट्यूब चॅनेल सुरू करतात किंवा इंस्टाग्रामवर व्हिडीओ-रील्स बनवतात. अनेकदा सुमार दर्जाचा ‘कन्टेन्ट’ असूनही, काहींचे व्हिडीओ व्हायरल होतात आणि त्यांना प्रचंड प्रसिद्धी व पैसा मिळतो. अशा गोष्टींचा बोलबाला होतो. हे पाहून अनेक तरुण या ‘इझी मनी’च्या मागे धावत आहेत. पण सगळे थोडेच कमाई करू शकतात. त्यांच्याकडे कोणी बघतही नाही. आपण फक्त यशस्वी झालेल्यांची उदाहरणे पाहतो, बाकीच्यांचे अपयश कुठेच समोर येत नाही.
मग यावर उपाय काय? प्रत्येक यशस्वी गोष्टीकडे बघताना स्वतःला एक प्रश्न जरूर विचारावा, "ह्याच गोष्टी करताना जे अपयशी झाले, त्यांचे काय?" कुठल्याही यशामागे अपयशाचे प्रमाण किती आहे, हे शोधण्याची सवय लागली पाहिजे. यशस्वी व्यक्तींना देव मानण्याऐवजी, त्यांच्या केवळ अंतिम यशाकडे न पाहता त्यांच्या प्रयत्नांचा आदर करा. आणि हेही लक्षात ठेवा की, कधीकधी नशिबाचीही साथ मिळालेली असते. हे यश अनेकदा अपवादात्मक असते हे लक्षात घ्या. आपण ‘सरासरी’ निकालांवर लक्ष केंद्रित करून आपली योजना आखली पाहिजे.
आपण भारतीय समाज म्हणून यशाचे मोठे कौतुक करतो आणि यशस्वी व्यक्तींना देव मानतो. पण ही उदाहरणे केवळ त्या स्पर्धेत टिकून राहिलेल्यांची आहेत. ‘सर्व्हायवरशीप बायस’ म्हणजे फक्त वाचलेल्या लोकांच्या कहाण्या ऐकून आपला दृष्टिकोन ठरवणे. म्हणूनच, यशाची प्रत्येक कहाणी ऐकताना त्यामागे लपलेल्या अपयशाचाही विचार करायला हवा. कायम लक्षात ठेवायला हवं की, जी विमाने परत आलीच नाहीत, त्यांचे काय झाले?
