\chapter{सोडी सोन्याचा पिंजरा}

सुरेश चांगल्या गुणांनी उत्तीर्ण होऊन ‘डॉक्टर’ झाला होता. एका मोठ्या खाजगी रुग्णालयात त्याला कनिष्ठ पदावर नोकरी मिळाली होती. त्याने दिवस-रात्र काम केले. वरिष्ठ डॉक्टरांच्या मार्गदर्शनाखाली मन लावून काम केल्याने त्याचे निदान, औषधयोजना आणि प्रत्यक्ष शस्त्रक्रिया अधिक अचूक आणि विना गुंतागुंतीच्या होऊ लागल्या. वरिष्ठ पदे मिळत गेली. त्याचे नाव केवळ डॉक्टरांमध्ये नव्हे तर रुग्णांमध्येही आदराने घेतलं जायचं. त्यांच्या हातातली स्थिरता, शस्त्रक्रियेतलं कौशल्य, आणि माणुसकीनं ओतप्रोत भरलेली रुग्णसेवा यामुळे तो रुग्णालयाचा आधारस्तंभ झाला. साहजिकच, जेव्हा वैद्यकीय संचालकपद रिक्त झालं, तेव्हा कोणतीही शंका न घेता त्याची निवड झाली. पण काही महिन्यांतच परिस्थिती बदलू लागली. विभागीय मिटिंग्स, अनेक प्रकारचे रिपोर्ट्स आणि व्यवस्थापनाने दिलेली ‘टार्गेट्स’ पूर्ण करण्याच्या नादात त्याचे रुग्णांकडे वेळ देणं कमी व्हायला लागलं. आजारावर सखोल चिंतन करण्याऐवजी तो तपासण्या 'उरकू' लागला. मग अनेक विभागांमध्ये कुरबुरी सुरू झाल्या. जो सुरेश तणावातही शांतपणे अचूक निर्णय घ्यायचा, तोच आता गोंधळलेला आणि हतबल दिसू लागला. काही परिचारिका कुजबुजत म्हणू लागल्या, “डॉक्टर म्हणून ते भारी होते, पण ही नवीन जबाबदारी त्यांच्यासाठी नाहीये”. डॉ. सुरेश यांच्या बाबतीत जे घडले, ते दुर्मिळ नाही. हे 'पीटर्स प्रिन्सिपल' (पीटरचे तत्त्व) या मेंटल मॉडेलचे (मन:प्रारूप) अथवा विचारचित्राचे  उत्तम उदाहरण आहे. सोप्या भाषेत याला ‘पदोन्नतीमुळे आलेली अकार्यक्षमता’ म्हणता येईल.

ही संकल्पना शिक्षणतज्ज्ञ लॉरेन्स जे. पीटर यांनी १९६९ मध्ये मांडली. त्यानुसार, "प्रत्येक संस्थेत कर्मचारी अखेरीस अशा पदावर पोहोचतो, ज्यासाठी लागणारी कौशल्ये त्याच्याकडे नसतात." याचा अर्थ, पदोन्नती देताना व्यक्तीच्या आजवरच्या कामगिरीचा विचार होतो, पण नव्या जबाबदारीसाठी तो सक्षम आहे की नाही, हे पाहिले जात नाही. त्यामुळे कर्मचारी अशा पदावर पोहोचतो, जिथे त्याची कौशल्ये अपुरी पडतात आणि तो तिथेच अडकतो. परिणामी, संस्थेची गती मंदावते. याची काही इतर उदाहरणे पाहुयात. 

असेच बँकांमध्ये घडते. खात्यांचे बिनचूक काम आणि मोठ्या व्यवहारांची जबाबदारी यशस्वीपणे सांभाळल्यामुळे एखाद्या कर्मचाऱ्याला शाखा व्यवस्थापक बनवले जाते. पण तिथे कर्जवाटप, कर्मचारी व्यवस्थापन आणि ग्राहक संबंध सांभाळण्यासाठी पूर्णपणे वेगळी कौशल्ये लागतात. याचा परिणाम म्हणून अनेक शाखा केवळ नेतृत्वाची कमतरता असल्यामुळे अडचणीत येतात.

उत्तम शिक्षक जेव्हा शाळेचा मुख्याध्यापक बनतो, तेव्हा त्याचे लक्ष शिकवण्याऐवजी कागदपत्रे, बजेट आणि प्रशासकीय कामांमध्ये अडकते. परिणामी, शाळेची गुणवत्ता घसरते आणि मुलांना नवनवीन पद्धतीने शिकवण्याचे त्याचे मूळ आवडते काम बाजूलाच राहते.

क्रिकेटमध्येही अनेक उत्तम-स्टार खेळाडूंना कर्णधार बनवले जाते. पण संघाचे व्यवस्थापन, खेळाडूंची मनस्थिती समजून घेण्याची क्षमता आणि माध्यमांना सांभाळणे यात ते कमी पडतात. याचा परिणाम संघाच्या तसेच त्यांच्या वैयक्तिक कामगिरीवरही होतो.

राजकारणातही असेच होते. कोणाच्याही हाकेला अर्ध्या रात्रीत धावून जाणारा कार्यकर्ता-नेता जेंव्हा राज्याचा मंत्री होतो तेंव्हा त्याची भूमिका बदलणे अपेक्षित असते. प्रशासकीय कौशल्याअभावी त्याला धोरणे ठरवणे किंवा विभाग चालवणे जमत नाही. काम करवून घेण्याच्या पद्धतीत बदल झालेले असतात. ते न समजल्याने यामुळे नाराजी वाढते आणि संपूर्ण विभाग अकार्यक्षम होतो.

टेक स्टार्टअपमध्ये एखादा प्रतिभावंत कोडर जेव्हा टीम लीड बनतो, तेव्हा त्याला माणसे हाताळावी लागतात. त्याला वरिष्ठांची आणि प्रसंगी विक्री विभागाचीही मर्जी सांभाळावी लागते. प्रमोशन मिळते, पण ते पद 'सोन्याचा पिंजरा' होऊन जाते.

मग यातून मार्ग काय? आपल्याला पदोन्नतीचा अर्थ नव्याने समजून घ्यावा लागेल. वरिष्ठ पद देणे हेच एकमेव बक्षीस नसावे; त्याऐवजी स्वायत्तता, उद्दिष्ट आणि कौशल्य वाढवण्याची संधी देणे अधिक परिणामकारक ठरू शकते. कर्मचाऱ्याला पुढच्या पदासाठी तयार करा, तो आपोआप तयार होईल असे गृहीत धरू नका. वैयक्तिक पातळीवरही विचार करा, "मी या नवीन भूमिकेत स्थिरावतोय की फक्त निभावतोय?". पुढच्या पदासाठी प्रयत्न करण्याआधी, ती जबाबदारी तुमच्या कौशल्यांना साजेशी आहे का, हे तपासा.

नेतृत्व करणाऱ्यांनी अशी संस्कृती निर्माण करायला हवी, जिथे पदोन्नती नाकारणे कमीपणाचे नव्हे, तर शहाणपणाचे लक्षण मानले जाईल. केवळ उच्च पद म्हणजे यश, हे समीकरण चुकीचे आहे. आपल्या समाजात पद, पगार आणि मोठे केबिन हेच यशाचे मापदंड आहेत. पण ‘पीटरचे तत्त्व’ आपल्याला शिकवते की, क्षमतेपलीकडची वाढ ही प्रगती नसून अधोगती असते. सोन्याचा असला तरी तो पिंजरा असू शकतो.
