\chapter{माझं झालं की झालं}

जगभरात हवामान बदलाचा प्रश्न दिवसेंदिवस गंभीर होत आहे. तापमानवाढ, वाढती समुद्रपातळी, जंगलतोड आणि पाणीटंचाई ही संकटे आता कोणत्या एका देशापुरती मर्यादित राहिलेली नाहीत. या ह्रासाला प्रामुख्याने विकसित देश जबाबदार आहेत, ज्यांनी औद्योगिक क्रांतीपासून निसर्गाचा बेसुमार वापर करून समृद्धी साधली. त्यांनी पृथ्वीच्या संसाधनांचा स्वार्थासाठी वापर केला, प्रदूषण केले आणि आता तेच विकसनशील राष्ट्रांना पर्यावरणाचे ज्ञानामृत पाजत आहेत.
वातावरण, हवामान आणि समुद्र ही कोणाच्याही मालकीची नसून ‘जागतिक सामायिक (सार्वजनिक) संपत्ती’ आहे, हे ते सोयीस्करपणे विसरले. त्यामुळे तिच्या वापराची जबाबदारीही जागतिक असायला हवी. पण स्वार्थाने घेतलेले निर्णय आणि “माझं झालं की झालं” या वृत्तीने प्रदूषणाचा भार इतरांवर ढकलला. यामुळेच आज आपण ‘ट्रॅजेडी ऑफ कॉमन्स’ म्हणजेच ‘सार्वजनिक संसाधनांची शोकांतिका’ या मेंटल मॉडेलच्या (मन:प्रारूप, विचारचित्र) परिणामांचा सामना करत आहोत. या मॉडेलनुसार, जेव्हा व्यक्ती किंवा देश सार्वजनिक संसाधनाचा वापर करताना फक्त स्वतःच्या फायद्याचा विचार करतात, तेव्हा ते संसाधन हळूहळू संपते आणि अंतिमतः सर्वांचेच नुकसान होते.
या संकल्पनेचा पाया १८३३ मध्ये अर्थतज्ज्ञ विल्यम फॉर्स्टर लॉयड यांनी घातला. पुढे १९६८ मध्ये उत्क्रांतीशास्त्रज्ञ गॅरेट हार्डिन यांनी ती अधिक सविस्तरपणे मांडली. या सिद्धांतानुसार, जेव्हा प्रत्येक जण सार्वजनिक हिताचा विचार न करता सामायिक संसाधनातून जास्तीत जास्त फायदा घेण्याचा प्रयत्न करतो, तेव्हा समस्या निर्माण होतात. हार्डिन यांनी एका सार्वजनिक कुरणाचे (ज्याला इंग्रजीत “कॉमन्स” म्हटलं जातं) उदाहरण दिले. जर प्रत्येक गुराख्याने मर्यादित गुरे चारली, तर कुरण टिकून राहील. पण प्रत्येकाने स्वतःच्या फायद्यासाठी जास्त गुरे आणल्यास, कुरण काही काळातच नष्ट होईल. या शोकांतिकेचे मूळ म्हणजे, आपण दीर्घकालीन नुकसानाकडे दुर्लक्ष करून तात्काळ फायद्याचा विचार करतो. याची काही उदाहरणे पाहुयात. 
भारतात याची अनेक उदाहरणे दिसतात. घराबाहेर कचरा फेकणे, सार्वजनिक जागेवर अतिक्रमण करणे किंवा विजेची चोरी करणे. सार्वजनिक शौचालये असूनही ती वापरण्यायोग्य नसतात. कारण “मी एकट्याने स्वच्छ ठेवून काय होणार?” हा विचार प्रत्येक जण करतो आणि अंतिमतः कोणीच जबाबदारी घेत नाही. माझं झालं की झालं.
एखाद्या व्यक्तीने रस्त्यावर नको त्या ठिकाणी गाडी पार्क केली की, त्याचे अनुकरण इतरही करतात. लवकरच, चालण्यासाठी जागाच उरत नाही आणि हाच ‘नवा नियम’ (न्यू नॉर्मल) बनतो.
शेतीतील पाण्याचा अतिवापर हे आणखी एक उदाहरण. जास्त उत्पादनासाठी शेतकरी भूगर्भातील पाण्याचा प्रचंड उपसा करतात. यामुळे पाण्याची पातळी खोल जाते, बोरवेल कोरड्या पडतात. याचे पर्यवसान दुष्काळ, स्थलांतर आणि कृषी संकटात होते.
अनेक ठिकाणी उपजीविकेच्या नावाखाली बेकायदेशीर जंगलतोड चालते. जेव्हा शेकडो लोक तोच विचार करतात, तेव्हा जंगल नाहीसे होते, पर्जन्यचक्र बिघडते आणि मानवी जीवन उद्ध्वस्त होते.
डिजिटल जगातही हेच घडते. एकाच व्हिडिओ-सेवेचे अकाऊंट अनेक जण वापरतात, ऑनलाइन कोर्सेसचे लॉगिन शेअर केले जातात. यामुळे कंपनीचे उत्पन्न घटते आणि सेवेचा दर्जा खालावतो.
यावर उपाय आहे का? होय, नक्कीच. जिथे मालकी स्पष्ट असते, तिथे संसाधनांची जपणूक होते. खासगी मालमत्ता सार्वजनिक मालमत्तेपेक्षा जास्त जपली जाते, कारण तिथे जबाबदारी निश्चित केलेली असते. राजस्थानमधील काही गावांमध्ये पंचायतींनी सामाजिक नियम घालून विहिरींच्या वापरावर नियंत्रण ठेवले. ‘आपलं गाव, आपली जबाबदारी’ यासारखी तत्त्वे संसाधनांचा शाश्वत वापर सुनिश्चित करतात.
दुसरा उपाय म्हणजे प्रोत्साहन किंवा दंड. जपानमध्ये कचऱ्याच्या वर्गीकरणासाठी कठोर नियम आहेत. नियम मोडल्यास दंड आणि शिस्तबद्ध पालनासाठी बक्षीस दिले जाते. भारतातही अशी प्रभावी वर्तनप्रणाली आणायला हवी.
या सगळ्यातून आपण काय शिकायचे? जे काही ‘सार्वजनिक’ आहे, गावाचे, देशाचे आणि जगाचे, ते जपणे ही आपली सर्वांची जबाबदारी आहे. “कोणीच जबाबदार नाही” अशी भूमिका घेतल्यास ती साधनसंपत्ती नष्ट होते. भारतासारख्या प्रचंड लोकसंख्या आणि मर्यादित संसाधने असलेल्या देशात ही शोकांतिका टाळण्याचा एकच मार्ग आहे: लोकांनी एकत्र येऊन, नियम पाळून आणि स्वार्थ बाजूला ठेवून सार्वजनिक संपत्तीची काळजी घेणे. जेव्हा आपण वाहतूककोंडी पाहतो तेव्हा ठरवूया अशी कोंडी मीच गाडी दामटल्याने कधीही होणार नाही. सगळ्यांनीच नियम  पळाले, पोलिसांना, वॉर्डन्सना सहकार्य केले तर आपणच सर्व लवकर या कोंडीतून सुटू, नाही का?   


