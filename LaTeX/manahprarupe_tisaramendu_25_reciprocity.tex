\chapter{देणाऱ्याचे हात घ्यावे}


२०२१ मध्ये अफगाणिस्तानमध्ये तालिबानने सत्ता घेतली आणि काबूलमध्ये गोंधळ माजला. तेव्हा भारताने "ऑपरेशन देवी शक्ती" नावाची एक मोठी मोहीम राबवली. यात ८०० हून अधिक भारतीय नागरिक आणि अफगाण अल्पसंख्यांकांना सुरक्षितपणे बाहेर काढण्यात आले. पण हे सहज शक्य नव्हते. थेट विमान पाठवणे धोकादायक होते. म्हणून भारताने उझबेकिस्तान आणि ताजिकिस्तान यांसारख्या जवळच्या देशांकडून मदत मागितली, त्यांच्या आकाशमार्गाचा वापर, लँडिंगची परवानगी, इंधन व इतर तांत्रिक सुविधा, यांसाठी. या देशांनी भारताला मोठ्या मनाने मदत केली. का? अगदी नेमके बोट ठेवता येत नसले तरी भारताचा जगन्मान्य उदारपणा त्याला कारणीभूत असावा. काही वर्षांपूर्वी भारताने, कोविड काळात, "व्हॅक्सीन मैत्री" उपक्रमांतर्गत बऱ्याच छोट्या, विकसनशील देशांना लस, संरक्षक किट्स आणि औषधे भेट दिली होती. जेव्हा ते अडचणीत होते, तेव्हा भारत त्यांच्यामागे उभा होता. म्हणून जेव्हा भारत अडचणीत होता, तेव्हा या देशांनीही आपल्यामागे उभे राहणे केवळ नैतिकच नव्हे तर आपुलकीचे मानले. हेच आहे ‘रेसिप्रोसिटी’ म्हणजेच परस्परतेचे मेंटल मॉडेल (मन:प्रारूप, विचारचित्र). जेव्हा तुम्ही मदत करता, तेव्हा तीच भावना तुम्हाला संकटात सावरायला येते. कधी कधी उलटेही होऊ शकते. जर तुम्ही कोणाला त्रास दिला तर त्याची व्याजासकट परतफेड होण्याची शक्यता जास्त असते. ‘पेरले तसे उगवते’ हेच खरे. याची काही उदाहरणे पाहुयात.
महाराष्ट्रातील हिवरे बाजार या छोट्याशा गावाने १९९०च्या दशकात एक वेगळाच इतिहास घडवला. इथे ना कोणी मोठे भांडवल आणले, ना सरकारने फार मोठ्या योजना दिल्या. पण गावकऱ्यांनी एकमेकांना साथ दिली. कोणी बांध घातले, कोणी विहिरी खोदल्या, ज्याच्याकडे ट्रॅक्टर होता, त्याने तो वापरायला दिला आणि जे अन्न शिजवू शकले, त्यांनी श्रमिकांसाठी जेवण तयार केले. इथे कुठलाही करार नव्हता, ना कुणी शासकीय आदेश दिला होता. फक्त एक अदृश्य नियम होता. "आज मी तुझी मदत करतो, उद्या तू माझी कर." हीच परस्परता! ही आपली संस्कृतीच आहे, आणि म्हणून ती आपल्याला फार सहजपणे समजते.
आपण साड्यांच्या दुकानात गेल्यावर दुकानदार आपल्याला चहा (उन्हाळ्यात कोल्ड्रिंक) देतो. हा केवळ पाहुणचार म्हणून नाही, तर ते एक प्रभावी परस्परतेचे तंत्र आहे. आपण चहा घेतला, म्हणजे काहीतरी खरेदी करण्यास आपोआप अजून प्रवृत्त होतो. रिकाम्या हाताने बाहेर पडणे मनाला खटकते. तसेच नात्यांमध्ये. जर एखादा समोरच्याचे नीट ऐकून घेतो आणि समजून घेतो तर समोरचाही (सर्वसाधारण परिस्थितीत) तशाच पद्धतीने वागतो, नाही का? ऑफिसमध्ये, जेव्हा मॅनेजर आपल्या टीमचे कौतुक करतो, जास्त वेळ थांबून तोही काम करतो, तेव्हा ती टीम अटीतटीची (डेडलाइनची) वेळ आल्यावर त्याच्यासाठी दिवस-रात्र मेहनत घेते. जे शिक्षक विद्यार्थ्यांना अभ्यासाबरोबरच आयुष्यबद्दल, करियरबद्दल आपुलकीने मार्गदर्शन करतात, त्यांना मुलांकडून फक्त शैक्षणिक यशच नाही, तर आजन्म आदरही मिळतो.
नाही म्हणायला, बऱ्याच वेळेला चांगले वागण्यात फायदाही दडलेला असतो. इंटरनेटच्या जगात, अनेक सोशल मीडिया इन्फ्लुएन्सर मोफत सल्ले देतात, उदाहरणार्थ फिटनेस टिप्स, आर्थिक योजना. जर मनापासून, कळकळीने आणि अनुभवसिद्ध सल्ले सांगितले तर लोकांमध्ये विश्वास निर्माण होतो. नंतर तेच लोक त्यांचे कोर्स खरेदी करतात.
परस्परतेची सावलीही असते. प्रत्येक गोष्टीची एक दुसरी बाजूही असते. जसा चांगुलपणा परत येतो तसा वाईटपणाही. तीही ‘परस्परताच’ म्हणावी. जर आपण कुणाकडे दुर्लक्ष केले, तर तेही आपल्याला दुर्लक्षित करतात. जर आपण कुणाला त्रास दिला, तर तेही कधी ना कधी तसाच परत देतात. हेच तत्त्व राष्ट्रांमध्येही लागू पडते. एक देश निर्बंध लावतो, तर दुसरा देशही त्याचे आपल्या पद्धतीने (अगदी लगेच नाही तरी योग्य वेळ आल्यावर) बरोबर उत्तर देतो. समर्थ रामदास स्वामींनी दासबोधात म्हटले आहे की “धटासी आणावा धट । उधटासी पाहिजे उधट । खटनटासी खटनट । अगत्य करी ॥” म्हणजेच, दांडग्यासाठी दुसरा दांडगा योजावा. उर्मट माणसास दुसरा उर्मटच योजला पाहिजे. लबाड असेल त्यास दुसऱ्या लबाड माणसासमोर उभे करावे. याप्रमाणे जशास तसे अगत्य योजावे.
एवढे सर्व समजल्यानंतर आपण कशाला कोणाशी मुद्दामून वाईट वागू, नाही का? चांगुलपणाचे मार्गच रास्त. आज तुम्ही जर कुणासाठी चांगले करत असाल, तर उद्या ते चांगले कुठल्या ना कुठल्या रूपात परत तुमच्याकडे येणार आहे. हीच खरी परस्परतेची शक्ती आहे. तिचा उपयोग करा, ती तुमचे जग बदलू शकते. प्रसिद्ध कवी विंदांनी म्हटल्याप्रमाणे “ देणार्‍याने देत जावे, घेणार्‍याने घेत जावे, घेता घेता एक दिवस, देणार्‍याचे हात घ्यावे”.

