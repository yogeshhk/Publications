\chapter{मी करोडपती झालो तर … }

सध्या यूएई (संयुक्त अरब अमिराती) संदर्भातील एक बातमी केवळ भारतातल्या धनिक वर्गाचेच नव्हे, तर उच्च मध्यमवर्गाचेही लक्ष वेधून घेत आहे. कारण यूएईने आपला प्रतिष्ठेचा ‘गोल्डन व्हिसा’ अवघ्या ₹२३ लाखांत देण्याची घोषणा केली आहे. त्याद्वारे तुम्ही तेथे तहहयात वास्तव्य करू शकता आणि नागरी सोयींचा लाभ घेऊ शकता. यापूर्वी या व्हिसासाठी काही कोटी रुपयांची गुंतवणूक आवश्यक होती. त्यामुळे भारतातील अनेक उच्च उत्पन्न गटातील व्यक्ती याकडे आकर्षित होत आहेत. या आकर्षणाचे सर्वात महत्त्वाचे कारण म्हणजे यूएईमध्ये व्यक्तिगत आयकर (डायरेक्ट पर्सनल इन्कम टॅक्स) नाही. पण बदल्यात मिळतात जागतिक दर्जाच्या सुविधा. कोण विचार करणार नाही?

याउलट भारतात काय परिस्थिती आहे? जे नागरिक प्रामाणिकपणे आयकर भरतात, त्यांना काय मिळते? अनेकदा मोडकळीस आलेली सार्वजनिक यंत्रणा, रस्त्यांवरील खड्डे, आणि आरोग्याच्या प्राथमिक सुविधाही नाहीत. असं वाटण्याचं कारण म्हणजे, भारतात अप्रत्यक्ष कर जरी सर्वांना लागू होत असला, तरी प्रत्यक्ष वैयक्तिक आयकर केवळ २-३% लोकच भरतात. उरलेले बरेच लोक खरेच गरीब असल्याने करमुक्त मर्यादेत येतात, तर काही जरी उत्पन्न अधिक असले, तरी त्यांच्या व्यवसायामुळे त्यांना ‘सरसकट’पणे  थेट आयकरभरावा लागत नाही. मग प्रश्न उभा राहतो की, या मोजक्या करदात्यांनीच बाकीच्यांचा भार वाहायचा का?

या विचारांतून एक धाडसी कल्पना निर्माण होते की, भारतानेही जर यूएईप्रमाणे थेट वैयक्तिक आयकर पूर्णपणे बंद केला, तर काय होईल? अर्थव्यवस्था कोसळेल का? की लोक त्यांचे वाचलेले पैसे व्यवसायांत गुंतवतील, जास्त खर्च करतील आणि त्या वाढलेल्या व्यवहारांमधून मिळणाऱ्या अप्रत्यक्ष करांमधून सरकारला जास्त महसूल मिळेल? याचे उत्तर आपण ही योजना प्रत्यक्षात आणण्याआधी ‘विचारप्रयोगांच्या’ (थॉट एक्सपेरिमेंट्स) मेंटल मॉडेलमधून म्हणजेच मन:प्रारूप अथवा विचारचित्रातून  शोधू शकतो.

विचारप्रयोग म्हणजे एखाद्या कल्पनेला वास्तवात अंमलात न आणता, मनातच त्या कल्पनेचे परिणाम, शक्यता आणि धोके तपासणे. हे जणू कल्पनाशक्तीच्या साहाय्याने घेतलेलं वैचारिक प्रतिरूप (सिम्युलेशन) आहे. यात "जर असं झालं, तर काय?" असे प्रश्न विचारून निर्णय घेण्याची स्पष्टता वाढवली जाते. हे विचारचित्र म्हणजे फक्त तात्त्विक चर्चा नाही, तर ते एक प्रभावी विचारसाधन आहे. यासाठी प्रयोगशाळा, निधी, किंवा सरकारची परवानगी लागत नाही तर लागतो तो केवळ कल्पकतेचा उपयोग, थोडं तर्कशास्त्र आणि विचारांचं धाडस.

विचारप्रयोगांचा वापर प्राचीन काळापासून आहे. बृहदारण्यक उपनिषदात ऋषी याज्ञवल्क्य आणि राजा जनक यांच्यात आत्म्याच्या स्वरूपाबाबत संवाद आहे, तोही प्रत्यक्ष अनुभव न घेता, केवळ वैचारिक कल्पनांमधून.

आजही, जेव्हा न्यायालय एखाद्या कायद्याच्या संभाव्य दुरुपयोगाबाबत चिंता व्यक्त करतं, तेव्हा ते ‘या कलमाचा गैरवापर झाला, तर काय?’ असा विचारप्रयोग करते. वास्तवात काही घडलं नसतानाही, फक्त कल्पनेतून सर्व शक्यता तपासून पाहणं केलं जातं. 

आपल्याला वाटेल की विचारप्रयोग हे केवळ विचारवंत किंवा वैज्ञानिक करतात, पण तसे नाही. ते आपल्या रोजच्या आयुष्यातही लपलेले असतात. त्याची काही उदाहरणे पाहुयात:

एक पिता आपल्या मुलीला परदेशात शिक्षणासाठी पाठवायचं की नाही, हे ठरवताना विचार करतो की, “ती यशस्वी झाली तर? अपयशी झाली तर? मी तिला थांबवलं आणि ती दुःखी झाली तर?” या प्रक्रियेतून तो निर्णय घेण्यास तयार होतो.

धोनी, एखाद्या वर्ल्ड कप सामन्यात फलंदाजीच्या क्रमवारीत स्वतःला वर पाठवायचं का, हे ठरवताना, मनातल्या मनात पिचची गती, सामन्याचं दडपण, विरोधकांची मनःस्थिती हे सगळं काही क्षणात तपासतो, फायदे-तोटे मोजतो आणि मग त्वरित निर्णय घेतो. हाच विचारप्रयोग.

एका पुण्यातील उद्योजकाने दुसऱ्या शहरात विस्तार करताना विचार केला की, “जर वितरण यंत्रणा फसली, तर काय? पण जर यशस्वी झाली, तर?” दोन्ही बाजूंचा तौलनिक अभ्यास करून मगच तो पुढचं पाऊल टाकतो.

काही अर्थतज्ज्ञ यूबीआय (युनिव्हर्सल बेसिक इनकम) लागू करण्याची शिफारस करतात. यात प्रत्येक नागरिकाला, त्याच्या उत्पन्नाची पातळी काहीही असो, सरकारकडून एक ठराविक रक्कम नियमितपणे दिली जाते, जीवन जगण्यासाठी आवश्यक किमान आर्थिक आधार म्हणून. इथे विचारप्रयोग असा असतो की, “जर प्रत्येक भारतीयाला ₹१०,००० महिना दिले, तर लोक त्यातील किती साठवतील, किती खर्च करतील? त्या खर्चामुळे महागाई वाढेल का?” या प्रश्नांची उत्तरं शोधण्यासाठी सांख्यिकीसारख्या शास्त्रांचा वापर करून विचारप्रयोग सुरू होतात. सखोल विश्लेषणानंतरच अशा योजना पुढे आणल्या जातात.

विचारप्रयोग महत्त्वाचे का? आपण अनेकदा काहीतरी बिघडल्यानंतरच दुरुस्ती करतो. विचारप्रयोग उलट सांगतात की, आधीच कल्पनांमध्ये सर्व ‘क्रॅश टेस्ट’ करा, जेणेकरून वास्तवात धक्का बसणार नाही. महत्वाचे म्हणजे, केवळ नुकसान टाळण्यासाठीच विचारप्रयोग करायचे असं नाही बरं  का, तर सुखद शक्यतांचाही विचार करता येतो, जसं की, “मी करोडपती झालो तर …”
