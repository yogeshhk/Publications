\chapter{आपला तो बाब्या …}

एका लहानशा खेड्यात, उन्हाने कोरड्या पडलेल्या जमिनीवर पहिल्या पावसाच्या सरी कोसळत आहेत. शेतकरी आपल्या शेतात उभा आहे, दोन्ही हात पसरलेले, डोळे मिटलेले, त्याचं चेहरा आनंदाने फुललेला. हा पाऊस म्हणजे त्याच्यासाठी जीवन आहे. आता त्याच्या कपाशीच्या बियाण्यांत जीव येईल,पीक चांगलं येईल, छान भावाला विकलं गेलं तर मागाच्यावर्षीसकटचं कर्ज फिटेल, त्याच्या कुटुंबाचं वर्ष थोडं सोपं जाईल. त्याच्या समोरचं चित्र, ओलसर, उज्वल आणि आशेचं. पण काहीशा अंतरावर, एक शिक्षक, त्याच पावसाकडे बघत असतो, चिंता घेऊन. गेल्या दोन पावसाळ्यांपासून त्याच्या एकल शाळेच्या खोलीचं छप्पर गळतंय. आता गावापासून येणारा रस्ता चिखलमय होणार, आणि मुलं काही दिवस आजारपणाने अथवा गणवेश सुकला नाही म्हणून येणारच नाहीत. मुलांप्रमाणेच त्याचा भविष्याची चिंता. त्याच्या समोरचं चित्र, ओलसर, काळं आणि उदास. तेच गाव. तोच पाऊस. पण दोन अगदी वेगळी वास्तव. यालाच ‘रेलेटिव्हिटी’ मेंटल मॉडेल म्हणजेच सापेक्षता (मन:प्रारूप) म्हणतात. ती सापेक्षता जी दररोज आपल्या आयुष्यात प्रकट होते. जरी सर्वसाधारणपणे सगळीच लहान मुलं द्वाड-खोडकर असली तरी आपल्याला वाटतं की नाही, “आपला तो बाब्या आणि दुसर्‍याचं ते कार्ट”, तसंच काहीसं. 
आइन्स्टाईनच्या सापेक्षतावादाने आपल्याला शिकवलं की अवकाश आणि काळ-वेळ स्थिर नसतात. ते पाहणाऱ्याच्या गतीवर आणि स्थानावर अवलंबून असतात. एक जण जर रेल्वे प्लॅटफॉर्मवर उभा असेल आणि दुसरा त्या गाडीमध्ये असेल, तर दोघंही तीच गाडी पाहतात, पण अनुभव वेगळा असतो. एकाला गाडी गाडी पुढे जाताना दिसेल तर दुसऱ्याला प्लॅटफॉर्म मागे जाताना आणि दोघंही बरोबर असतात! हे फक्त विज्ञान नाही तर हे जीवनाचं तत्त्वज्ञान आहे. जीवनात ही सापेक्षता अनेक ठिकाणी दिसते. त्याची काही उदाहरणे पाहुयात. 
खेडेगावातील एखाद्या कुटुंबासाठी ₹५०० म्हणजे आठवड्याचं राशन. पण मुंबईच्या समुद्रकिनाऱ्याच्या रेस्टॉरंटमध्ये हे ₹५०० एका स्टार्टरसाठीही पुरणार नाहीत. रक्कम एकच, पण जगण्याची वास्तव वेगवेगळी.
आदिवासी भागातील विद्यार्थ्याने जर ६०% गुण मिळवले, तर अधिक महत्वाचे. तो अनेक अडचणींवर मात करून आलेला असतो. भाषेची अडचण, वीज नाही, शिकवण्या नाहीत. त्याच वेळी एका शहरातील मुलाला ८५% गुण असूनही अपेक्षेपेक्षा कमीच मानायला हवं, नाही का? इथे केवळ आकडे पूर्ण गोष्ट सांगत नाहीत. परीक्षा एकच असली तरी गुणांवरून चिकाटी, जिद्द मोजता येत नाही. 
आपल्याकडे कोठल्याही कौटुंबिक समारंभात "संध्याकाळी ७” ही वेळ असेल तर त्यावेळी यजमानही त्या ठिकाणी हजर असण्याची शक्यता कमीच. पण कॉर्पोरेट मीटिंगमध्ये ५ मिनिट उशीरही चालत नाही. बॉस त्याची नोंद ठेवतो आणि मग त्याचा वापर कोठे होतो ते सांगायलाच नको. वेळेचं महत्व सुद्धा सापेक्षच.
नाटकात नाट्यमय (‘लाऊड’) अभिनय गरजेचा असतो आणि तो प्रेक्षकांना थेट भिडतो, पण तसाच अभिनय एखाद्या चित्रपटात अथवा वेबसिरीजमध्ये केल्यास ‘ओव्हर अ‍ॅक्टिंग’ वाटू शकते. 
एका कलात्मक चित्रपटाला आंतरराष्ट्रीय महोत्सवात पारितोषिक मिळतं, पण स्थानिक चित्रपटगृहामध्ये प्रेक्षक तो अर्धवट सोडून निघून जातात. त्याचवेळी एखाद्या मसाला चित्रपटाला टाळ्यांचा कडकडाट होतो, पण समीक्षकांकडून त्याला ‘दर्जाहीन’ म्हटलं जातं. चांगल्या सिनेमाचं मोजमापही आपापल्या दृष्टिकोनावर ठरतं.
एखाद्या राष्ट्रासाठी रशियासोबत व्यापार करणे ‘आर्थिक गरज’ असते, तर दुसऱ्या राष्ट्रासाठी तेच ‘धोका’ मानले जाते. एकाच धोरणाचं मूल्यांकन वेगवेगळ्या देशांकडून त्यांच्या स्वतःच्या सुरक्षेच्या आणि हितसंबंधांच्या चष्म्यातून केलं जातं.
कधी एखाद्या क्रिकेटपटूची ३० धावांची खेळी सामना जिंकवणारी ठरते, कारण ती योग्य वेळेस आली असते. तर दुसऱ्याच सामन्यात त्याच्या ७० धावाही कमी लेखल्या जातात, कारण त्या कूर्मगतीने काढलेल्या असतात. आकड्यांपेक्षा वेळ आणि परिस्थिती खेळींचे महत्व ठरवतात.
सापेक्षता मेंटल मॉडेल का समजून घ्यायचे? ते आपल्याला थांबून विचार करायला शिकवते. सत्य हे एकाच कोनातून दिसणारं नसतं. अनेक वाद, विशेषतः समाज-माध्यमात, राजकारणात आणि अगदी कुटुंबातसुद्धा, यामुळे होतात की दोघंही स्वतःच्या जागेवरून योग्य असले तरी त्यांचे विचार विरुद्ध असतात.
सापेक्षतेचा स्वीकार म्हणजे सत्याला नाकारणं नव्हे तर त्याची अनेक बाजुंनी ओळख करून घेणं. थोडंसं स्थान बदललं की नवीन दृष्टिकोन समोर येतो. म्हणून पुढच्यावेळी कोणी जग वेगळ्या प्रकारे पाहत असेल, तर त्याला लगेच चुकीचं ठरवू नका. स्वतःला विचारा: "तो असं काय पाहतोय, की जे मला दिसत नाहीये?" मग सत्य कोणाचं? ते तुम्ही कुठे उभे आहात, यावर ठरतं. प्लॅटफॉर्मवर की गाडीच्या आत. 
