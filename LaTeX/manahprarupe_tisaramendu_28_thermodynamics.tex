\chapter{वाढता वाढता वाढे …}


एखाद्या मोठ्या शहरात सकाळच्या वेळेस तुम्ही सिग्नलवर थांबला असाल, तर एक गोष्ट नक्की जाणवते की, सिग्नल सुटताच मोठा गोंधळ उडतो. एकाच वेळी सगळ्यांना पुढे जायचे असते; रिक्षा उजवीकडून शिरते, दुचाकी मधून वाट काढतात आणि एखादी गाडी आडवी घुसते. याच वेळेला पादचाऱ्यांनाही रस्ता ओलांडायचा असतो. कुणीच हा गोंधळ मुद्दाम करत नाही, पण तरीही तो घडतोच. कारण प्रत्येक जण स्वतंत्रपणे निर्णय घेत असतो आणि पोलीस असले तरी त्यांना हा गोंधळ आवरता येईल व इतके सर्व लोक त्यांचे ऐकतील, याची शक्यता नसते. ही विस्कळीतता म्हणजेच एंट्रॉपी. थर्मोडायनॅमिक्सचा (उष्मागतिकी शास्त्र) हा दुसरा नियम आहे. तो सांगतो की, कोणत्याही व्यवस्थेत बाह्य ऊर्जेचा वापर करून शिस्त न लावल्यास, ती आपोआपच गोंधळाकडे झुकते आणि हा गोंधळ दिवसेंदिवस वाढतच जातो, ‘वाढता वाढता वाढे’ या पद्धतीने. ‘आमच्या वेळेस असं नव्हतं’ हे म्हणूनच खरं ठरतं. शिस्त लावण्यासाठी आणि टिकवण्यासाठी ऊर्जा लागते. ही ऊर्जा खर्च करण्याची तयारी नसल्याने सिग्नलसारखे अराजक हमखास घडते.
थर्मोडायनॅमिक्स हा मेकॅनिकल इंजिनियरिंग (यंत्र-अभियांत्रिकी) किंवा भौतिकशास्त्रातील (फिसिक्स) केवळ एक विषय नसून, ते एक प्रभावी ‘मेंटल मॉडेल’ (मन:प्रारूप) देखील आहे. आयुष्यातील निर्णय, नातेसंबंध, दिनचर्या, कामाची शिस्त आणि सामाजिक संबंध यांसारख्या अनेक गोष्टी थर्मोडायनॅमिक्सच्या नियमांतून समजून घ्यायला मदत होते.
थर्मोडायनॅमिक्सचा पहिला नियम सांगतो की ऊर्जा निर्माण करता येत नाही आणि ती नाहीशीही होत नाही; ती केवळ एका प्रकारातून दुसऱ्या प्रकारात बदलते.
आपण खाल्लेलं अन्न, दिवसभर काम करण्यासाठी वापरलं जातं. शाररिक तसेच मानसिक प्रक्रियांसाठी ही ऊर्जा वापरली जाते. काही खाल्लेच नाही तर ऊर्जा येणारच नाही. ऊर्जा खर्च केली नाही तर मेद-स्वरूपात ती साठून राहते. 

‘वेळ’ हीसुद्धा अशीच एक गोष्ट आहे जी निर्माण करता येत नाही. प्रत्येक व्यक्तीकडे दररोज समान, म्हणजेच २४ तास असतात, मग ती सामान्य व्यक्ती असो की नोबेल पुरस्कार विजेता. पण या वेळेचे काय करायचे, ती कुठे आणि कशात गुंतवायची, हे मात्र पूर्णपणे आपल्या निवडीवर आणि शहाणपणावर अवलंबून असते.
त्याचप्रमाणे शेअर बाजारात ‘जोखीम’ कमी केली म्हणजे ती नाहीशी होत नाही; ती केवळ एका स्वरूपातून दुसऱ्या स्वरूपात रूपांतरित होते. मोठ्या गुंतवणूकदारांनी एका ठिकाणी जोखीम कमी केली, की ती संपूर्ण व्यवस्थेत दुसरीकडे कुठे तरी निर्माण होतेच.
थर्मोडायनॅमिक्सच्या दुसऱ्या नियमानुसार, विस्कळीतता (एंट्रॉपी) कायम वाढत असते, अगदी सिग्नलवर होणाऱ्या गोंधळाप्रमाणे. आपण घर झाडून स्वच्छ करतो, पण नियमित साफसफाई न केल्यास काही दिवसांतच धूळ साचते. ऑफिसमधील कामाच्या जबाबदाऱ्या वेळोवेळी पार पाडल्या नाहीत, तर कामाचा ढीग (आता संगणकात) साचतो. म्हणजेच, आपण काहीच केले नाही, तर गोष्टी बिघडतच जातात. त्यामुळे ऊर्जा हे केवळ वापराचे साधन नसून, ती आपल्या अस्तित्वाचा आणि प्रगतीचा आधार आहे.
थर्मोडायनॅमिक्समधील उष्णतेच्या प्रवाहाची कल्पना आपल्या सामाजिक वागणुकीतही दिसून येते. उष्णता जशी गरम वस्तूपासून थंड वस्तूकडे वाहते, त्याचप्रमाणे एखाद्या टीममधील एका सकारात्मक व्यक्तीचा प्रभाव इतरांवर पडतो. याउलट, सतत तक्रार करणारी व्यक्ती सर्वांची मनःस्थिती खराब करू शकते. त्यामुळे आपण कोणाच्या संगतीत वेळ घालवतो, याचा थेट परिणाम आपल्या ऊर्जेवर होतो.
एखाद्या देशाचा किंवा राज्याचा प्रमुख नेता जर स्वच्छ प्रतिमेचा, नीतिमान, पारदर्शक आणि लोकहितासाठी काम करणारा असेल, तर त्याची ही ऊर्जा, म्हणजेच त्याची मूल्यव्यवस्था, वर्तनशैली आणि सार्वजनिक आचारधर्म, हळूहळू संपूर्ण व्यवस्थेत झिरपतो. त्याच्या मंत्रिमंडळातील सदस्य, खालच्या पातळीवरील प्रशासक, अधिकारी आणि स्थानिक राजकारणी हे सर्व त्या वरच्या पातळीवरून वाहणाऱ्या नैतिकतेच्या दबावाखाली आपले वर्तन सुधारतात. त्यांना ठाऊक असते की चुकीच्या कृत्यांना वरून संरक्षण मिळणार नाही, उलट जबाबदारीची अपेक्षा आहे. त्यामुळे एक संस्थात्मक शिस्त आणि आंतरिक नियंत्रण निर्माण होते. ‘यथा राजा तथा प्रजा’ (काही वेळेस उलटेही सत्य होते!!) हा नियम दिसून येतो.
थर्मोडायनॅमिक्सचे नियम आपल्याला हे शिकवतात की जग एका विशिष्ट पद्धतीने चालते. ऊर्जा मर्यादित आहे, गोंधळ (एंट्रॉपी) नैसर्गिकरीत्या वाढतो, आणि व्यवस्था टिकवणे आव्हानात्मक असते. ही केवळ शास्त्राची नव्हे, तर जीवनशैलीची शिकवण आहे. “ऊर्जा जपा, विस्कळीततेला ओळखा आणि संतुलन राखा” ही त्रिसूत्री आपल्या वैयक्तिक आणि सामाजिक आयुष्यासाठीही तितकीच महत्त्वाची आहे.

