\chapter{कडू कारलं, साखरेत घोळलं}

१९८०च्या दशकाच्या उत्तरार्धात आणि १९९०च्या सुरुवातीला भारत एका मोठ्या बदलाच्या उंबरठ्यावर उभा होता. इतर प्रगत देशांप्रमाणेच भारतानेही डिजिटल युगात प्रवेश करणे गरजेचे होते, पण हे ओळखणारे फारच कमी होते. त्याकाळचे पंतप्रधान राजीव गांधी यांनी जेव्हा सरकारी कार्यालयांमध्ये आणि सार्वजनिक संस्थांमध्ये संगणक व माहिती तंत्रज्ञान यांचा वापर सुरू करण्याचा प्रयत्न केला, तेव्हा या प्रयत्नांना जोरदार विरोध झाला. राजकारणाच्या रंगमंचावर समाजवादी पक्ष आणि डाव्या विचारसरणीच्या अनेक गटांनी या निर्णयावर टीका करत, संगणक हे “नोकऱ्या हिरावून घेणारे यंत्र” असल्याचा प्रचार सुरू केला. त्या काळात “संगणक हटाओ, रोजगार बचाओ” हा नारा रस्तोरस्ती ऐकू येत होता. सरकारी कार्यालयांतील टायपिस्ट, लिपिक आणि इतर कर्मचाऱ्यांच्या नोकऱ्या धोक्यात येतील अशी भीती अनेकांच्या मनात होती. संगणक म्हणजे माणसांना विस्थापित करणारी थंड, बेरकी यंत्रं असा समज झाला होता. पण ज्या गोष्टीला त्या वेळी धोका मानले गेले, तीच गोष्ट काही वर्षांनी भारताच्या आर्थिक व तांत्रिक विकासाची नांदी ठरली.

१९९०च्या दशकाच्या अखेरीस भारताने माहिती तंत्रज्ञानाच्या क्षेत्रात जागतिक स्तरावर आपली छाप पाडली. बंगळूर, पुणे आणि हैदराबादसारख्या शहरांनी सॉफ्टवेअर, आउटसोर्सिंग आणि सेवा उद्योगांमध्ये झपाट्याने प्रगती केली. लाखो युवकांना चांगल्या पगाराच्या नोकऱ्या मिळाल्या, त्या थंड, बेरकी संगणकांमुळेच. ज्या तंत्रज्ञानाला आधी विरोध करण्यात आला होता, त्याचे रूपांतर एका संधीमध्ये झाले.

येथे ‘‘इनर्शिया’ (जडत्व) हे मेंटल मॉडेल (मन:प्रारूप) आपल्याला अशा विरोधांमागचं मूळ कारण उलगडून दाखवतं. भौतिकशास्त्रानुसार, स्थिर वस्तू स्थिरच राहते आणि गतिमान वस्तू गतिमानच राहते, जोपर्यंत एखादी बाह्य असंतुलित शक्ती तिच्यावर कार्य करत नाही. हीच संकल्पना आपल्या मनोवृत्ती, वर्तणूक आणि सामाजिक व्यवस्थांनाही लागू पडते. याची काही उदाहरणे पाहूया.

आजही आपल्या देशातील अनेक शाळांमध्ये जुन्या अभ्यासक्रमावर आणि पाठांतरावर आधारित परीक्षा पद्धतीवर भर दिला जातो. "हे तर नेहमीच असंच चालत आलंय," अशी एक मानसिकता तयार होते. शिक्षण पद्धतीत आमूलाग्र बदल करायचा झाल्यास प्रस्थापित पद्धतींना आव्हान द्यावे लागते, जे अनेकांना रुचत नाही.

राजकीय व्यवस्थांमध्येही अशीच एक ठरावीक जडत्वता दिसून येते. मतदार अनेकदा एखाद्या पक्षाला पाठिंबा देतात, जरी त्या पक्षाचा कारभार फारसा समाधानकारक नसला तरी, त्या पक्षाचे नवीन पिढीतील नेते अतिशय सुखवस्तू, विचारहीन व भरकटलेले असले तरी. केवळ परंपरेने, किंवा “नेहमीच यांनाच मत दिलंय,” या विचाराने निर्णय घेतला जातो. अशा वेळेस विचार न करता सवयीने निर्णय होतो आणि समाजाचा विकास खुंटतो.

सरकारी संस्था, विशेषतः सार्वजनिक क्षेत्रातील बँका, वर्षानुवर्षे जुन्याच पद्धती वापरत आहेत. गोष्टी ‘ऑनलाईन’ झाल्या असल्यातरी तक्रार निवारण खुद्द संस्थेत गेल्याशिवाय, ‘वाटाघाटी’ केल्याशिवाय होत नाही. बदलाची गरज असल्याचे सगळ्यांनाच ठाऊक आहे. तरीही त्या जड प्रक्रियांमध्ये हालचाल होत नाही, कारण “आजवर असंच चाललंय” ही मानसिकता आणि स्थिर-स्थावर झालेल्या प्रक्रियेतून मिळणाऱ्या लाभांवर पाणी सोडता येत नाही.  बरं , काही चुकले तर जबाबदारी कोण घेणार या भीतीपायीसुद्धा बदल टाळले जातात. म्हणूनच फायदे स्पष्ट असले तरी त्या बदलाची सुरुवात करणे कठीण जाते.

जडत्व ही मानसिकता केवळ सरकारपुरती मर्यादित नाही. ती आपल्या प्रत्येकाच्या रोजच्या जगण्यात दिसून येते. उदाहरणार्थ, एखादी जुनी सवय मोडणे, जसे की साखरयुक्त चहा पिणे. त्याचा त्याग किती कठीण वाटतो, नाही का? आरोग्यासाठी ते हानिकारक ठरत असले, तरी त्या सवयीचा मोह आपण टाळू शकत नाही. हेच ते जडत्व.

मात्र जडत्व हा शत्रू नसून एक तटस्थ शक्ती आहे. तिचा परिणाम नकारात्मक तेव्हाच होतो, जेव्हा आपण विचार न करता तिला आपल्या निर्णयांवर स्वार होऊ देतो. सवय, परिचय आणि स्थैर्य या गोष्टी चांगल्या असल्या तरी, बदलाची गरज असताना त्या अडथळा ठरू शकतात.

उपाय आहे, “सुरुवात करणे”, अगदी छोट्या पावलांनी. जसे एखादी नवीन सवय फक्त दोन मिनिटांसाठी सुरू करणे, किंवा एक छोटा निर्णय घेणे. हीच कृती हळूहळू गती निर्माण करते. एकदा का सुरुवात झाली, की पुढचा प्रवास आपोआप सुलभ होतो.

भारतात प्रतिभा आणि नवनवीन कल्पनांची कमतरता नाही. अडथळा आहे तो बदलाला सामोरे जाण्याचा. प्रस्थापित व्यवस्थेला केवळ प्रश्न विचारून फायदा नसतो तर एक समांतर बदल यशस्वी करून दाखवावा लागतो.  बदल न होण्यामागे लोकांची असमर्थता हे कारण नसून, बदलाला होणारा मानसिक विरोध आणि पर्याय देण्यात लागणारी धडाडी नसणे हे कारणीभूत ठरते.

म्हणून पुढच्या वेळी जर एखादी गोष्ट आपण “नेहमीच असं केलंय” म्हणून करत असाल, तर थांबा, आणि स्वतःला विचारा की, “हे खरंच प्रभावी आहे का?” जर उत्तर ‘नाही’ असेल, तर पहिले पाऊल टाका, त्या पहिल्या पावलाच्या प्रतिकाराला सामोरे जा. कारण एकदा का धक्का मिळाला, गाडी हलली, ‘मोसम’ (मोशन) पकडला, की तिचे पुढे जाणे अटळ असते.

