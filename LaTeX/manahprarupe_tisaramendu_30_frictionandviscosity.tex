\chapter{थांबा, पुढे गतिरोधक आहे}

अनेक मोठ्या खाजगी कंपन्यांमध्ये भरतीची प्रक्रिया वरवर पाहता पारदर्शक आणि योग्यता (मेरिट) आधारित वाटते. ऑनलाइन अर्ज करा, मुलाखत द्या आणि निवड झाली तर ऑफर मिळवा. पण या सगळ्या प्रक्रियेच्या मागे काही छुपे अडथळे कार्यरत असतात. अनेक वेळा अर्ज केलेले फॉर्म कुठे गेले हे कळत नाही. आपली पुढच्या फेरीसाठी निवड झाली आहे का, हे विचारण्यासाठी वेबसाईटवर दिलेल्या ईमेलला प्रतिसाद मिळत नाही. कोणी फोन उचलत नाही. कालांतराने कळते की जागा भरली गेली आहे. समजा, पुढच्या फेरीसाठी बोलावणे आले, मुलाखती झाल्या. तुम्हाला वाटते की छान उत्तरे दिली, पण पुन्हा निर्णयासाठी प्रतीक्षा. एचआर (ह्युमन रिसोर्स, मानव संसाधन विभाग) चा नंबर मिळवून विचारता नकार कळतो. कारण? ते दिले जात नाही. आपल्याला कशामुळे नाकारले, याचे स्पष्टीकरण मिळत नाही. हा सर्व केवळ देखावा तर नाही ना, अशी शंका येते. मुद्दाम उशीर करणे, कृत्रिम अडथळे आणणे, जेणेकरून उमेदवार कंटाळून अर्ज करण्याचा विचारच सोडून देतील. मग ती पदे शांतपणे अंतर्गत ओळखींतून भरली जातात. सर्वसामान्य उमेदवाराला जाणवणारा हा संघर्ष कधी मुद्दामहून तर कधी बेपर्वाई, अंतर्गत राजकारण, विस्कळीत प्रणाली आणि एकमेकांपासून वेगळे ठेवलेले अंतर्गत विभाग यांमुळे निर्माण झालेला असतो. भरती पारदर्शक नसल्याने आणि वेळेत प्रतिसाद मिळत नसल्याने उमेदवारांना 'फ्रिक्शन' (घर्षण, संघर्ष) व 'व्हिस्कॉसिटी' (अतरलता) या मेंटल मॉडेल (मन:प्रारूप) याचा अनुभव येतो. 
भौतिकशास्त्रात, 'फ्रिक्शन' म्हणजे दोन घन पृष्ठभाग एकमेकांवर घासले गेल्याने निर्माण होणारा प्रतिकार. यामुळे गती कमी होते, उष्णता निर्माण होते आणि पुढे जाण्यासाठी जास्त ऊर्जा लागते. तर ‘व्हिस्कॉसिटी‘ म्हणजे द्रवपदार्थाच्या वाहण्याला होणारा अडथळा. जसे मध पाण्याच्या तुलनेत हळू वाहतो, कारण त्याची 'व्हिस्कॉसिटी' (अतरलता) जास्त असते. या दोन्ही संकल्पना गतिरोधकता दर्शवणारे मेंटल मॉडेल म्हणून वापरता येतात. सामाजिक, संस्थात्मक किंवा वैयक्तिक पातळीवर कोणतीही 'प्रतिकारक शक्ती' असेल, तर ती गतिरोधकतेच्या स्वरूपात दिसते. उदाहरणार्थ, सामन (रावस) माशाचा प्रवास हा प्रवाहाविरुद्ध असतॊ, कष्टप्रद पण गंतव्याकडे पोहोचण्याची आस कधी ढळू न देता तो परिस्थितीला तोंड देत प्रवास चालू ठेवतो. अशी गतिरोधकता नेहमीच्या जीवनात केवळ अडथळ्यांच्या स्वरूपातच नाही तर संघर्षांच्या, कधीकधी लढ्याच्या रूपात प्रकट होते, अगदी, तुकोबारायांनी म्हणल्याप्रमाणे ‘रात्रंदिन आम्हा युद्धाचा प्रसंग’. अशाप्रकारे या मन:प्रारूपाची दैनंदिन जीवनात दिसणारी काही उदाहरणे पाहूया. 
विमान प्रवास घ्या. तिकीट बुक करणे सोपे वाटते, पण त्यात बदल करायचा झाल्यास ‘फ्रिक्शन’ सुरू होते. दडवलेले बदल-शुल्क, गोंधळात टाकणाऱ्या अटी आणि ग्राहक सेवेतील प्रतीक्षा. एअरलाईनच्या अंतर्गत प्रणाली, आरक्षण, ग्राहक सेवा आणि वेळापत्रक विभाग हे एकमेकांशी समन्वय ठेवत नाहीत आणि त्यामुळे आपल्या वाट्याला वाईट अनुभव येतो. ज्या कंपन्या ह्या गोष्टी सुधारतात, त्या प्रवाशांचा विश्वास जिंकतात.
व्हिसा प्रक्रिया हे आणखी एक उदाहरण. अर्ज करताना लागणारी कागदपत्रे, महिनोंमहिने अपॉइंटमेंट न मिळणे आणि कॉन्स्यूलेटच्या संथ कामकाजामुळे, पूर्ण प्रक्रिया संथ व अतरल दिसते. ज्या देशांनी हे डिजिटल केले, प्रक्रिया स्पष्ट केली आणि ट्रॅकिंग प्रणाली उपलब्ध केली, तिथे पर्यटक, कौशल्य आणि गुंतवणूक वाढली.
वैयक्तिक जीवनातही हे लागू होते. अस्ताव्यस्त टेबल, मंद झालेला संगणक किंवा एकाच वेळी अनेक टॅब उघडणे हे गतिमान कामातील अडथळेच आहेत. एका वेळेस एक काम नीट न करता सतत वेगवेगळ्या कामांमध्ये उडी मारणे, हे कार्यक्षमतेत अडथळा आणते. यावर उपाय म्हणजे कार्यक्षेत्र साफ ठेवणे, प्रत्येक कामाला ठराविक वेळ राखून ठेवणे आणि चांगली साधने वापरणे. ही गतिरोधकता संपली तर कार्यक्षमता वेगाने वाढू शकते.
या सर्व उदाहरणांवरून आपल्याला वाटेल की गतिरोधकता (‘फ्रिक्शन’ आणि ‘व्हिस्कॉसिटी’) ही नकारात्मकच गोष्ट असून ती टाळणे, कमी करणे गरजेचे आहे. तर, तसे नाही. या दोन्ही गोष्टी कधीकधी आवश्यक असतात. टायरला रस्ता धरून ठेवण्यासाठी 'फ्रिक्शन' लागतेच. इंजिन ऑईल जर पाण्यासारखे वाहायला लागले, तर त्याचा उपयोग नाही. अशाचप्रमाणे, द्वि-चरण तपासणी (टू-स्टेप व्हेरिफिकेशन) सारखा पर्याय 'फ्रिक्शन' वाटत असला तरी सुरक्षा वाढवतो. निर्णय प्रक्रियेमध्ये 'व्हिस्कॉसिटी' जर विचारपूर्वक असेल, तर ती फायद्याची असते. शरीरातील रक्त खूप पातळ किंवा घट्ट झालेले चालत नाही; त्यात समतोल असणे महत्त्वाचे आहे. 
म्हणून, पुढच्या वेळी जेव्हा तुम्हाला एखाद्या प्रक्रियेत अडथळा जाणवेल, तेव्हा तो लगेच कमी करण्याचा प्रयत्न करू नका. भौतिकशास्त्रज्ञासारखा विचार करा. 'फ्रिक्शन' कुठे आहे, 'व्हिस्कॉसिटी' कुठे आहे, ती कशामुळे आहे हे बघा आणि मग ती जर हानिकारक असेल तरच कमी करण्यासाठी कृती करा.
