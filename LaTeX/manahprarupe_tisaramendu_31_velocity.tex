\chapter{वेगे वेगे धावू }

रमेश नुकताच पदवीधर झाला होता. त्याने व्यवसायात उडी घ्यायचे ठरवले आणि गुंतवणूक व विमा एजन्सी सुरू केली. आवश्यक परीक्षा उत्तीर्ण होऊन त्याने परवानग्यादेखील मिळवल्या. आता ग्राहक मिळवण्यासाठी त्याची धावपळ सुरू झाली. दिवसभर व्यस्त वेळापत्रक; कधी बैठका, कधी फोन कॉल्स, कधी ईमेल्स, तर कधी स्वतःच ठरवलेली ‘टार्गेट्स’ पूर्ण करण्याची घाई. वरवर पाहता हे सर्व प्रगतीसारखे दिसत होते. पण थोडे खोलात पाहिल्यास लक्षात येत होते की, या धावपळीला कुठल्याही ठोस उद्दिष्टाची दिशा नव्हती. व्यवसाय विशेष वाढत नव्हता. असलेले ग्राहक टिकवण्यातच वेळ खर्च होत होता. नवीन गुंतवणूक संधींचा अभ्यास करून त्या ग्राहकांना समजावून सांगितल्यास अधिक ग्राहक जोडले जातील, पण त्यासाठी त्याला वेळच मिळत नव्हता. ना कोणती योजना, ना स्पष्ट दिशा; केवळ पळणे सुरू होते. हे जणू ट्रेडमिलवर वेगाने धावताना बाहेरील दृश्य स्थिर राहण्यासारखे होते.
इथे ‘व्हेलॉसिटी’ (वेग) हे मेंटल मॉडेल (मन:प्रारूप) आपल्याला योग्य दृष्टी देते. हे मॉडेल भौतिकशास्त्रातून घेतलेलं असलं, तरी याचा उपयोग आपल्या वैयक्तिक आणि व्यावसायिक आयुष्यातही तितकाच आहे. ‘व्हेलॉसिटी’ म्हणजे केवळ ‘स्पीड’ (गती) नव्हे, तर दिशेसह असलेली गती. केवळ किती गतीने चाललो आहोत हे महत्त्वाचे नाही, तर आपण योग्य दिशेने चाललो आहोत की नाही, हे समजून घेणे पण महत्त्वाचे आहे. आपण निर्णय घेताना, करिअर निवडताना किंवा व्यवसाय करताना, ही दिशा निर्णायक ठरते. आपण करत असलेल्या कष्टांमुळे आपण खरोखरच आपल्या ध्येयाकडे जात आहोत की केवळ थकत आहोत? हे समजून घेण्यासाठी हे मॉडेल उपयुक्त आहे. याची काही उदाहरणे पाहूया.
आज अनेकजण नवनवीन कोर्स करत राहतात, सतत नोकऱ्या बदलतात किंवा ओव्हरटाईम करतात. पण हे सर्व करताना त्यांचे अंतिम ध्येय काय आहे, याचा स्पष्ट विचार नसतो. त्यांना नेता बनायचे आहे की उद्योजक, की एखाद्या विशिष्ट कौशल्यात पारंगत व्हायचे आहे? हा विचारच स्पष्ट नसतो. म्हणूनच, ‘व्हेलॉसिटी’ हा प्रश्न विचारतो: तुमचा प्रवास तुम्हाला खरोखर तुमच्या ध्येयाजवळ नेत आहे का? की ‘एक ना धड, भाराभर चिंध्या’ असे प्रयत्न सुरू आहेत?
काही कंपन्या केवळ ट्रेंड्सच्या मागे धावतात. आज एआय (कृत्रिम बुद्धिमत्ता), उद्या ई-कॉमर्स, परवा पर्यावरणपूरक योजना. परंतु, जर यामागे त्यांच्या मूळ उद्दिष्टांचा विचार नसेल, तर अशा हालचालींमुळे केवळ लक्ष विचलित होते आणि ऊर्जा वाया जाते. याउलट, ‘व्हेलॉसिटी’ असलेले व्यवसाय त्यांच्या ध्येयाशी आणि मिशनशी सुसंगत असलेल्या गोष्टींवरच लक्ष केंद्रित करतात.
सरकारे अनेकदा केवळ कृती दाखवण्यासाठी एकामागून एक योजना सुरू करतात. एकाच रस्त्याचे उदाहरण घ्या: आधी डांबरीकरण, मग पाण्याच्या पाईपलाईनसाठी खोदकाम, त्यानंतर काँक्रिटीकरण आणि मग केबलसाठी पुन्हा खोदकाम. या सगळ्यानंतर मेट्रोचे काम सुरू होतेच. सरतेशेवटी नागरिकांच्या पदरात काय पडते, हे आपण पाहतोच. नुसताच पैशाचा व्यय. मग ‘सारे काही (नवीन) टेंडरसाठी’ तर नाही ना असे वाटायला लागते. खरी प्रगती तेव्हाच होते, जेव्हा या योजना नीट अभ्यास करून, योग्य क्रम ठरवून आणि दीर्घकालीन उद्दिष्टांशी जोडून राबवल्या जातात. 
आपल्या समाजात ‘हसल’ (धडपड) म्हणजेच प्रगती, असा एक गैरसमज रूढ झाला आहे. लवकर उठा, काम करा, जास्त काम करा. पण जर आपण दहा वेगवेगळ्या गोष्टींचा पाठलाग करत असू आणि त्यांच्यामागे का धावत आहोत हेच माहिती नसेल, तर आपण काहीच साध्य करू शकणार नाही. ‘व्हेलॉसिटी’ आपल्याला विचार करायला भाग पाडते की, मी कुठे चाललो आहे? माझ्या कृती त्या दिशेला पूरक आहेत का? माझ्या मार्गातील कोणते अडथळे मी दूर करू शकेन?
या मॉडेलचा एक महत्त्वाचा भाग म्हणजे नियमितपणे आपल्या मार्गाचे मूल्यमापन करून तो सुधारणे (‘कोर्स करेक्शन’). जसे एखादे क्षेपणास्त्र प्रक्षेपणानंतर वेळोवेळी आपला मार्ग तपासते आणि सुधारते, तसेच आपणही बदलत्या परिस्थितीनुसार, नवीन माहितीनुसार आणि आपल्या उद्दिष्टांनुसार आपल्या दिशेमध्ये आवश्यक बदल करायला हवेत. ‘व्हेलॉसिटी’ आपल्याला आठवण करून देते की आयुष्य ही शर्यत नसून एक अर्थपूर्ण यात्रा आहे. केवळ वेग असून चालत नाही, तर योग्य दिशाही हवी. अन्यथा, हाती फक्त थकवा, निसटलेल्या संधी आणि पश्चात्तापच उरतो. याउलट, जर आपल्या कृती ध्येयाशी सुसंगत असतील, तर लहान पावलांमधूनही मोठे परिवर्तन घडू शकते. म्हणून, पुढच्या वेळी आपल्या जीवनाच्या बोटीचे इंजिन किती शक्तिशाली आहे याकडेच लक्ष न देता, आपले होकायंत्र (कम्पास) योग्य दिशा दाखवत आहे की नाही, हेदेखील तपासा. कारण खरे यश केवळ गतीत नाही, तर योग्य दिशेने साधलेल्या वेगात आहे.
