\chapter{अवकाशात एक जागा नि तरफ द्या…}

रमेश आणि सुरेश दोघेही जिवलग मित्र. दोघेही एका नामांकित अभियांत्रिकी महाविद्यालयातून पदवीधर झालेले. दोघेही समान हुशार आणि मेहनती होते. सुरेशला यंत्रांमध्ये रस होता, म्हणून तो एका वाहननिर्मिती कंपनीत उत्पादन विभागात कामाला लागला. रमेशला संगणक आवडत असल्याने त्याने एका सॉफ्टवेअर कंपनीत काम सुरू केले. दोघेही दिवसाचे १०-१२ तास कष्ट करायचे, शिस्तबद्ध आणि निष्ठावान होते. पण दहा वर्षांनंतर त्यांचे आयुष्य वेगवेगळ्या मार्गांनी पुढे गेले.
वाहन उद्योगातील सुरेश एक स्थिर पगार मिळवत राहिला, त्याने सहकाऱ्यांचा आदर मिळवला आणि तो हळूहळू पदोन्नती घेत गेला. रमेशने काही काळानंतर, चांगला अनुभव घेतल्यानंतर धडाडी करून स्वतःची कंपनी सुरु केली. त्याला माहिती झालेली समस्या सोडवण्यासाठी हुशारीने व मेहनतीने एक सॉफ्टवेअर (संगणक प्रणाली) तयार केले. त्याने ते जागोजागी प्रदर्शित करून बऱ्याच लोकांपर्यत पोहोचवले. लोकांना त्याचा फायदा दिसू लागला. त्याचे ते उत्पादन आंतरराष्ट्रीय स्तरावर पोहोचले. डॉलरमध्ये उत्पन्न मिळू लागले. एकामागोमाग एक कंत्राटे मिळत रमेशच्या उत्पन्नातही प्रचंड वाढ झाली. इथे दोघा मित्रांच्या उत्पन्नातील फरक फार मोठा झाला. तो फरक त्यांच्या मेहनतीत नव्हता, तर त्यांनी ती मेहनत कशावर घेतली यात होता. कमी किंवा त्याच मेहनतीत जास्त फायदा मिळवणे, यालाच म्हणतात लेव्हरेज (प्रवर्धन) मेंटल मॉडेल (मन:प्रारूप). भौतिकशास्त्रात ‘लेव्हरेज’ म्हणजे कमी ताकद वापरून मोठे ओझे उचलण्याचा उपाय. नेहमीच्या जीवनात लेव्हरेज म्हणजे कमीत कमी साधने, तंत्रज्ञान, नातेसंबंध किंवा भांडवल वापरून आपल्या तुलनेने कमी कामाचे परिणाम अनेक पटींनी वाढवणे. काही लोक हे कौशल्य शिकून आपल्या कष्टांना गुणतात, तर काही जण तेवढीच मेहनत करूनही फारसे पुढे जाऊ शकत नाहीत. फरक केवळ लेव्हरेजचा असतो.
आधुनिक तत्त्वचिंतक नवल रविकांत यांनी लेव्हरेजचे तीन प्रमुख प्रकार सांगितले आहेत. पहिला प्रकार आहे 'श्रम' (लेबर), म्हणजेच इतरांकडून काम करून घेणे किंवा एक टीम तयार करणे. दुसरा प्रकार आहे 'भांडवल' (कॅपिटल), म्हणजेच पैशातून अधिक पैसा कमावणे, अर्थात गुंतवणूक. आणि तिसरा, सर्वात प्रभावी प्रकार म्हणजे 'परवानगी-मुक्त प्रवर्धन' (पर्मिशनलेस लेव्हरेज). यात अशा गोष्टी येतात, ज्या एकदा तयार केल्या की त्यांची प्रतिकृती बनवायला नगण्य खर्च येतो. उदाहरणार्थ, सॉफ्टवेअर, पुस्तके किंवा ऑनलाइन व्हिडीओ. हे एकदा तयार झाले की विशेष खर्च न करता लाखो लोकांपर्यंत पोहोचू शकतात आणि प्रचंड नफा मिळवून देऊ शकतात. याची काही इतर उदाहरणे पाहूया.
एका शिक्षिकेचा विचार करूया. त्या दररोज ५० विद्यार्थ्यांना शिकवतात. त्यांचे शिकवणे प्रभावी असले, तरी त्याची पोहोच मर्यादित आहे. पण त्याच शिक्षिकेने आपले व्याख्यान रेकॉर्ड करून ऑनलाइन उपलब्ध केल्यास, हजारो विद्यार्थी देशभरातून ते पाहू शकतात. काम तेच, पण पोहोच प्रचंड वाढली. नंतर काही व्याख्याने सशुल्क केल्याने पैसे ही मिळाले. हेच आहे तंत्रज्ञानाचे लेव्हरेज.
अर्थ-जगतात हे अधिक स्पष्टपणे दिसून येते. एक गुंतवणूकदार एखाद्या स्टार्टअपमध्ये ₹१ लाख गुंतवतो. कंपनी यशस्वी झाली, तर ही रक्कम शंभरपट वाढू शकते. त्याचे स्वत:चे कोणतेही अतिरिक्त श्रम नाहीत, फक्त भांडवलाचे योग्य नियोजन. हेच भांडवलाचे लेव्हरेज.
सरकारी क्षेत्रातही हे दिसून येते. एखाद्या अधिकाऱ्याने नागरिकांच्या अर्ज-सेवांसाठी एक ऑनलाइन पोर्टल तयार केल्यास, हजारो सेवा रोज अत्यल्प मनुष्यबळ वापरून हाताळल्या जाऊ शकतात. एकदाच केलेले काम, सतत परिणाम देत राहते.
पण लेव्हरेज ही दुधारी तलवारीसारखी आहे. चुकीच्या पद्धतीने वापरल्यास ती नुकसान करू शकते. उदा., वित्तीय लेव्हरेजचा अतिरेक केल्यास कर्जाच्या गर्तेत अडकण्याची शक्यता असते. तसेच, आपण आपल्या पदाचा किंवा प्रभावाचा गैरवापर केला, तर नातेसंबंध, विश्वास आणि प्रतिष्ठा या सगळ्यांचे नुकसान होऊ शकते. म्हणूनच लेव्हरेज वापरताना विवेकबुद्धी आणि दूरदृष्टी आवश्यक असते.
भारतीय तरुणांसाठी तंत्रज्ञानाचे लेव्हरेज ही संधी मोठी आहे. पूर्वी केवळ प्रस्थापितांच्या कवेत असणारे तंत्रज्ञान आता सर्वसामान्यांना उपलब्ध आहे. ज्ञान आणि कल्पकता असलेल्यांकडे आता मोठे लेव्हरेज आहे, केवळ श्रीमंत अथवा सत्ताधीशांकडे नव्हे.
दोन हजार वर्षांपूर्वी अर्किमिडीजने म्हटले होते, "मला उभे राहण्यासाठी अवकाशात एक जागा आणि एक बाजू लांब असलेली एक तरफ द्या, मी संपूर्ण पृथ्वी उचलून दाखवेन." आजच्या जगात ती तरफ म्हणजेच वेगवेगळ्या प्रकारचे लेव्हरेजेस. योग्य जागा निवडा, योग्य कौशल्ये वापरा आणि अशा गोष्टी तयार करा, ज्याच्या लेव्हरजने तुमचा प्रभाव आणि उत्पन्न वाढवत राहतील. कारण यश केवळ ताकदीत नाही, तर योग्य ठिकाणी लेव्हरेज वापरण्यात आहे.

