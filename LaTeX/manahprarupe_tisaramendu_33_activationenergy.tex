\chapter{ल्युडोतील सहाचं दान}

एकोणिसाव्या शतकाच्या उत्तरार्धात जेव्हा ब्रिटिश साम्राज्य भारतावर राज्य करत होते, तेव्हा अनेक ब्रिटिश अधिकारी भारतीय लोकांकडे आणि त्यांच्या उद्यमशीलतेकडे संशयाने व तुच्छतेने पाहत असत. त्यांना वाटत असे की भारतातील लोक छोटे-छोटे व्यवसाय करतील, पण मोठा उद्योग चालवण्यास सक्षम नाहीत. विशेषतः पोलादनिर्मितीसारखा आधुनिक आणि महत्त्वाकांक्षी उद्योग चालवणे हे त्यांच्या मते भारतासाठी अशक्य होते. पण जमशेदजी टाटा यांना हा अपमान जिव्हारी लागला. त्यांनी ब्रिटिशांच्या या संकुचित विचारसरणीला केवळ शब्दांनी नव्हे, तर कृतीने उत्तर देण्याचा आणि त्यातूनच एक स्वतंत्र, आत्मनिर्भर भारत घडवण्याचा संकल्प केला. ब्रिटिशांच्या शंका, वंशभेद आणि राजकीय अडथळे झुगारून, त्यांनी १९०७ मध्ये टाटा आयर्न अँड स्टील कंपनी (आजची टाटा स्टील) सुरू केली. जमशेदजींनंतरसुद्धा त्यांचा हा संकल्प त्यांच्या वारसांनी पुढे नेला आणि भारताच्या औद्योगिक परिवर्तनाचा पाया रचला. जमशेदजींची  कहाणी केवळ उद्योगाची नसून, ती माणसाच्या जिद्दीची, ताकदीची आहे. त्यांच्या जीवनातून आपण एक मेंटल मॉडेल (मन:प्रारूप) शिकू शकतो, ते म्हणजे ऍक्टिव्हेशन एनर्जी, अर्थात ‘स्फुल्लिंग ऊर्जा’. ही संज्ञा विज्ञानातून आली असली तरी, ती आयुष्यातही तितकीच लागू होते.
रसायनशास्त्रात, एखादी रासायनिक प्रक्रिया सुरू होण्यासाठी ज्या किमान ऊर्जेची गरज असते, तिला ऍक्टिव्हेशन एनर्जी म्हणतात. ही ऊर्जा मिळाली नाही, तर काहीच घडत नाही. जसे ल्युडो (सारिपाटाचे एक रुपडे) खेळात जोपर्यंत सहा या आकड्याचे दान पडत नाही, तो पर्यंत तुम्हाला खेळाला सुरुवातच करता येत नाही. आपल्या दैनंदिन जीवनातही नेमके हेच घडते. आपल्याकडे वेळ असतो, साधने असतात, ज्ञान असते, पण तरीही सुरुवात करायला एक “ठिणगी” लागते. हाच तो क्षण असतो, जेव्हा आपण स्थैर्यातून क्रियाशीलतेकडे वळतो. एखादे काम सुरू करणे कठीण वाटते, पण एकदा का सुरुवात झाली, की पुढे जाणे तुलनेने सोपे होते. हीच गोष्ट आपली टाळाटाळ, आळस आणि नवीन सवयी लावण्यातील अपयश यामागील कारण स्पष्ट करते. याची काही उदाहरणे पाहूया.
रमेश परीक्षा जवळ आलेली असूनही अभ्यास सुरू करत नव्हता. पण एक दिवस त्याचा मित्र (आणि प्रतिस्पर्धी) सुरेशशी बोलताना त्याला कळले की, सुरेशचा आठपैकी पाच धड्यांचा अभ्यास पूर्ण झाला आहे. हे ऐकताच रमेश खडबडून जागा होतो, आपले टेबल आवरतो, मोबाईल दुसऱ्या खोलीत ठेवतो, पुस्तक उघडतो आणि तासन्‌तास अभ्यासाला लागतो. सुरुवात करणे हेच सर्वात कठीण होते. अशा वेळी, एक छोटीशी ठिणगीही पुरेशी ठरते.
एका छोट्या व्यवसायालाही हे तत्त्व लागू होते. एका स्टार्टअपने एक उत्तम उत्पादन बनवले होते, पण विक्री मात्र होत नव्हती. एके दिवशी त्याच्या गुंतवणूकदाराचा फोन आला आणि त्याने थेट धमकी दिली की, कामगिरी सुधारली नाही, तर पुढचा निधी मिळणार नाही. यामुळे मोठी धावपळ उडाली. तातडीने सोशल मीडियावर मार्केटिंग सुरू झाले, एका प्रसिद्ध इन्फ्लुएन्सरला उत्पादनाबद्दल पोस्ट करण्यास सांगितले आणि ती पोस्ट व्हायरल झाली. बघता बघता उत्पादनाची विक्री वाढू लागली आणि गोष्टी पुन्हा रुळावर आल्या. एका फोनकॉलमधील धमकीने सर्व चित्र पालटले.
सुरुवात करणे कठीण असते. म्हणूनच आपण गुंतवणूक पुढे ढकलतो, कठीण संभाषण टाळतो किंवा एखादे ध्येय सतत लांबणीवर टाकतो. आणि आपण जेवढा वेळ सुरुवात न करता थांबतो, तेवढी ती आवश्यक ऊर्जा अधिकच जास्त वाटू लागते. पण याउलट, सुरुवातीचा अडथळा कमी केला की कामाला वेग येतो. 
ऍक्टिव्हेशन एनर्जी कमी करणे हे प्रगतीसाठी उपयोगी असले तरी, व्यवसायात काही वेळा प्रतिस्पर्ध्यांना रोखण्यासाठी ती मुद्दाम वाढवावी लागते. यालाच "बॅरियर टू एंट्री" म्हणजेच ‘प्रवेशातील अडथळा’ म्हणतात. जेव्हा एखादी कंपनी आपल्या व्यवसायात मोठी गुंतवणूक, विशेष तांत्रिक कौशल्य किंवा कठोर परवाना नियमांची गरज निर्माण करते, तेव्हा नवीन स्पर्धकांना प्रवेश करणे अत्यंत कठीण होते. उदाहरणार्थ, एक औषधनिर्माण कंपनी नवीन औषध बाजारात आणण्यासाठी अनेक वर्षे संशोधन, परवाने आणि मान्यता प्रक्रियेत घालवते. एकदा का त्या औषधाला पेटंट मिळाले की, इतर कंपन्यांना ते बनवता येत नाही आणि त्या उद्योगातील स्पर्धा आपोआप कमी होते.
आपण सभोवताली पाहतो की सर्वच यशस्वी लोक किंवा संस्था अत्यंत बुद्धिमान नसतात, पण ते ‘सुरुवात’ करतात. ‘ऍक्टिव्हेशन एनर्जी’ हे मन:प्रारूप आपल्याला सांगते की यशासाठी सर्वात महत्त्वाची गोष्ट बुद्धिमत्ता किंवा वेळ नसून, ती म्हणजे ‘सुरुवात’ करणे. छोट्याने सुरुवात करा. पोषक वातावरण तयार करा. स्वतःला कृतीसाठी उद्युक्त करा. कारण सवय असो, उद्योग असो किंवा एखादी सामाजिक चळवळ असो, प्रत्येकाची सुरुवात एका ठिणगीनेच होते. एखादा अपमान जसा मोठा उद्योगसमूह निर्माण करतो, त्याचप्रमाणे जुलमी शासनाविरुद्ध उचललेले मूठभर मीठसुद्धा स्वातंत्र्याची एक विराट चळवळ उभी करू शकते.

