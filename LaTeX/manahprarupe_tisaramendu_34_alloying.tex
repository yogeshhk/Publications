\chapter{दुग्ध-शर्करा योग }


दरवर्षी भारतात हजारो तरुण पदवीधर होतात. पण या पदव्या असूनही अनेकांना चांगली नोकरी मिळवण्यात अपयश येते. याच वेळी कंपन्यांच्या तक्रारी वेगळ्याच असतात. उपलब्ध पदवीधरांमध्ये त्यांना हवे तसे कौशल्य मिळतच नाही. हे केवळ ज्ञानाच्या अभावाचे नव्हे, तर कौशल्यांबरोबरच्या योग्य मिश्रणाच्या अभावाचे चित्र आहे. कोड (संगणक प्रणाली) लिहिता येतो म्हणून केवळ चालत नाही, संवादकौशल्यही तितकेच गरजेचे असते. कालांतराने नेतृत्व करणाऱ्याला सहवेदना (एम्पथी) नसेल, तर काही खरं  नाही. खरी कमतरता आहे ती योग्य मिश्रणाची. म्हणजेच ज्ञानाच्या, योग्य कौशल्यांच्या आणि दृष्टिकोनांच्या (अ‍ॅटिट्यूड) मिश्रणातून तयार होणाऱ्या संमिश्रणाची, 'अ‍ॅलॉय'ची.
धातुविज्ञानात (मेटॅलर्जी) 'अ‍ॅलॉयिंग' म्हणजे दोन किंवा अधिक धातू एकत्र करून त्यांचे गुणधर्म अजून वाढवणे. लोखंड आणि कार्बन एकत्र करून तयार होणारे स्टील, हे त्याचे प्रमुख उदाहरण. पण जेव्हा आपण या संकल्पनेला 'मेंटल मॉडेल' (मन:प्रारूप) म्हणून पाहतो, तेव्हा 'अ‍ॅलॉयिंग' म्हणजे फक्त धातूंचे मिश्रण नव्हे, तर कल्पना, कौशल्ये आणि दृष्टीकोन यांचे सर्जनशील एकत्रीकरण होय.
'अ‍ॅलॉयिंग' या संकल्पनेचा गाभा 'संश्लेषण' (सिंथेसिस) आहे. केवळ एकाच विषयात पारंगत होणे आज पुरेसे नाही; विविध क्षेत्रांचा समन्वय अधिक फलदायी ठरतो. जसे तांबे आणि कथील (टिन) यांचे मिश्रण 'कांस्य' (ब्रॉन्झ) तयार करते, तसेच दोन वेगळ्या शास्त्रांची युती नवी दिशा निर्माण करते. उदाहरणार्थ, अर्थशास्त्र आणि मानसशास्त्र एकत्र केल्यामुळे ‘वर्तणूक अर्थशास्त्र' (बिहेवियरल इकॉनॉमिक्स) जन्माला आले. हे म्हणजे समन्वय (सिनर्जी) जिथे एकत्र मिळालेला परिणाम वेगवेगळ्या भागांच्या बेरजेपेक्षा अधिक असतो, म्हणजेच दुग्ध-शर्करा योग म्हणायचा, अथवा  “दो और दो, पाँच”. याची काही उदाहरणे पाहूयात.
शिक्षण आणि तंत्रज्ञान या दोन वेगळ्या क्षेत्रांतील समन्वयातून 'एड-टेक' म्हणजेच शैक्षणिक तंत्रज्ञान क्षेत्र जन्माला आले. इथे शिक्षकाचे ज्ञान आणि एआय (आर्टिफिशिअल इंटेलिजन्स, कृत्रिम बुद्धिमत्ता) यांची शक्ती एकत्र येते, आणि त्यातून वैयक्तिक पातळीवर झेपेल तसे शिकवण्याची क्षमता (पर्सनलाइज्ड एज्युकेशन) निर्माण होते. केवळ शिक्षक किंवा केवळ संगणक यापैकी कोणतेही एकटे हे साध्य करू शकत नाही.
भारतीय शास्त्रीय संगीत आणि पाश्चात्य संगीत यांचे फ्युजन हे असेच उदाहरण. एक पारंपरिक तबलावादक आणि एक जॅझ वादक एकत्र काम करतात. प्रारंभी त्यांचे संगीत परस्परविरोधी वाटते, पण ज्या क्षणी त्या दोघांमध्ये सर्जनशीलतेबरोबर परस्पर आदर आणि समजून घेण्याची तयारी येते, त्याक्षणी एक नवा संगीतप्रकार तयार होतो. जुगलबंदी अफलातून होते. प्राचीन परंपरा आणि आधुनिक प्रयोगशीलता यांचा हा सुरेल मिलाफ.
लोकशाही धोरणे बनवताना जर केवळ आकडेवारीवर आधारित निर्णय घेतले, तर ते कधी कधी वास्तवात चालत नाहीत. पण जेव्हा अभ्यासक शास्त्रीय माहितीबरोबरच स्थानिक समाजाचा अनुभव, ज्ञान यांनाही समजून घेतात, तेव्हा धोरणे अधिक वास्तववादी आणि टिकाऊ ठरतात. हासुद्धा एक प्रकारचा अ‍ॅलॉयच.
उद्योजकतेमध्येही हेच लागू होते. केवळ नफा-तोट्याची भाषा समजणारा उद्योजक जर गावातील गरजांपासून दूर असेल, तर त्याचे उत्पादन अपुरे-तोकडे राहील. पण जर तो एखाद्या अशा सहकाऱ्याशी भागीदारी करतो, ज्याला समाजाची प्रत्यक्ष ओळख आहे, तर तयार होणारे उत्पादन केवळ विकले जात नाही, ते जीवनही बदलते. इथे नफा आणि उद्दिष्ट दोन्ही हातात हात घालून चालतात.
क्रिकेटसारख्या खेळातही आज 'फक्त अनुभव' पुरेसा नाही. एक यशस्वी प्रशिक्षक आता केवळ सराव आणि भावना नव्हे, तर आकडेवारी, डेटा, आणि विश्लेषण यांचा वापर करून निर्णय घेतो. खेळाडूंच्या फिटनेस डेटावर आधारित रणनीती आखली जाते, पूर्वीच्या सामन्यांचा अभ्यास करून फलंदाजी क्रम ठरवला जातो. हा आहे पारंपरिक जाणिवा आणि आधुनिक तंत्रज्ञानाचा अ‍ॅलॉय.
पण एक महत्त्वाची गोष्ट लक्षात ठेवायला हवी की, कोणतेही धातू मनात येईल तसे एकत्र करता येत नाही. संशोधनात चाचणी केली जाते, काही मिश्रणं उपयोगी ठरतात, तर काही अपयशी. विचारांचंही तसंच आहे. योग्य वेळ, योग्य प्रमाण आणि योग्य समज हवी. चुकीच्या लोकांचा एकत्रित प्रयत्न, किंवा तांत्रिक कौशल्याच्या अतिरेकात मानवी संवेदना हरवली, तर नुकसान पदरी पडते.
आपल्या समाजात अनेकदा “शुद्धते”चे महत्त्व अधोरेखित केले जाते. जात, परंपरा, विचारपद्धती या सर्वांमध्ये. पण अ‍ॅलॉयिंग आपल्याला सांगते की खरी ताकद एकसंधतेत नव्हे, तर विविधतेत आहे. एखादे उत्पादन, धोरण, नातेसंबंध किंवा संघटना हे सगळं फक्त “शुद्ध” रूपात टिकत नाही, तर योग्य मिश्रणात भरभराटीला येते. म्हणूनच, ‘एकाला चालो रे’  होण्याऐवजी आपण एखाद्या गटाचा भाग व्हावं, जिथे एकमेकांपासून शिकत नव्या शक्यतांचा शोध घेतला जातो. विचार करा आपणास कोण व्हायचं आहे, २४ कॅरेट शुद्ध पण जरा ठिसूळ सोनं, की थोडे तांबे मिश्रित पण मजबूत-तरीही-लवचिक २२ कॅरेट (‘अ‍ॅलॉय’) सोनं.

