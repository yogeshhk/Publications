\chapter{सुसंगती सदा घडो … }

एका लहानशा गावात किरण नावाच्या उद्यमशील तरुणाने सौरऊर्जेवर चालणाऱ्या, स्वस्त आणि प्रभावी दिव्यांची रचना आणि निर्मिती केली. आता तो त्यांच्या विक्रीचा प्रयत्न करत होता. उत्पादन चांगले असूनही आणि किंमत योग्य असूनही, विक्री काही केल्या वाढत नव्हती. मग एक दिवस, राष्ट्रीय दूरदर्शनवर हवामान बदलावर आधारित एक माहितीपट दाखवण्यात आला. त्यात त्या दिव्यांचा थेट उल्लेख नव्हता, पण लोकांना केरोसीनवर चालणाऱ्या दिव्यांचे दुष्परिणाम कळले आणि मनोमन पटले. गावातील शाळेतही मुलांना ही चित्रफीत दाखवल्याने, घरोघरी अक्षय ऊर्जेचे (रिन्यूएबल एनर्जी) महत्त्व पोहोचले. मग काय विचारता? दुसऱ्या दिवसापासून किरणच्या सौर-दिव्यांची मागणी अचानक वाढली. गावकरी पारंपरिक केरोसीनऐवजी पर्यायी ऊर्जास्रोतांकडे वळू लागले आणि विक्री झपाट्याने वाढली. हे त्या उद्योजकाने अधिक कष्ट घेतल्यामुळे घडले नाही, तर त्या माहितीपटाने उत्प्रेरकाचे (कॅटालिस्ट) काम केल्यामुळे घडले. अद्नान सामीच्या लोकप्रिय गीतातील ओळीप्रमाणे किरणच्या व्यवसायाला थोडी ‘लिफ्ट’ मिळाली आणि त्याची जोरात प्रगती सुरु झाली. 
हेच आहे 'कॅटालिस्ट' किंवा उत्प्रेरकाचे मेंटल मॉडेल (मन:प्रारूप). यात, एक असा बाह्य घटक असतो जो थेट तुमच्या प्रयत्नांत प्रत्यक्ष सहभागी नसतो, पण तरीही तो तुमच्या परिणामांमध्ये केवळ सानिध्याने वेगाने बदल घडवून आणतो. रसायनशास्त्रात उत्प्रेरक रासायनिक क्रिया अधिक वेगाने घडवतो आणि स्वतःमध्ये काहीही बदल न होऊ देता ती क्रिया पूर्ण करतो, तसेच जीवनातील काही प्रसंग, व्यक्ती, कल्पना किंवा साधने मोठा बदल घडवून आणू शकतात, अगदी 'जादू' केल्यासारखे.
आपण अनेकदा असे गृहीत धरतो की यश मिळवण्यासाठी खूप मेहनत, वेळ आणि धडपड करावी लागते. पण खरी प्रगती अनेकदा योग्य वेळ, योग्य संधी आणि योग्य ‘संगतीवर’ अवलंबून असते. मोरोपंतांच्या ‘सुसंगती सदा घडो ..’ या केकावली प्रमाणे. हेच तत्व 'कॅटालिस्ट' या मेंटल मॉडेलमधून समजते, जिथे केवळ सान्निध्य मोठा परिणाम घडवते. काही वेळा एक सल्ला, एक पुस्तक, एक संवाद, किंवा योग्य वेळी मिळालेली प्रेरणा आपल्या जीवनात नवे वळण आणू शकते. हे घटक तुमच्यासाठी थेट काही करत नाहीत, पण तुमचे काम अधिक प्रभावी बनवतात. याची काही उदाहरणे पाहूया.
एखादा विषय मुलांना कंटाळवाणा वाटत असतो, पण अचानक एक असे शिक्षक येतात, जे अत्यंत उत्साही आणि कल्पक असतात. तोच विषय, तीच पुस्तके, पण शिकवण्याची पद्धत आणि त्यातील ऊर्जा मुलांमध्ये कुतूहल निर्माण करते. त्या शिक्षकामुळे एखाद्या विद्यार्थ्याचा त्या विषयाकडे पाहण्याचा दृष्टिकोनच बदलतो आणि पुढे तोच विद्यार्थी विज्ञानात करिअर करतो. येथे ‘कॅटालिस्ट’ ठरतो, तो शिक्षक.
कोविड-१९ च्या काळात विविध माध्यमांतून मास्क वापरण्याची गरज सांगितली जात होती, पण खेड्यापाड्यांत तो संदेश प्रभावीपणे पोहोचत नव्हता. नंतर एका तरुणाने स्थानिक बोलीभाषेत, साध्या शब्दांत आणि विनोदी शैलीत मास्क वापरण्यावरचा एक व्हिडीओ तयार केला. तो सोशल मीडियावर व्हायरल झाला, लाखो लोकांनी पाहिला आणि गावोगावी मास्क वापरणे सुरू झाले. तोच संदेश आधीही दिला गेला होता, पण योग्य भाषा, माध्यम आणि सादरीकरणामुळे तो अधिक प्रभावी ठरला.
अनेकदा आपण एखाद्या नोकरीत कंटाळलेलो असतो, पण ती सोडण्याचे धाडस होत नाही. मग एक दिवस, मित्रासोबत सहज बोलताना तो म्हणतो, "तू सतत त्रासात दिसतोस, तू स्वतःसाठी काहीतरी चांगले शोधायला हवे." त्या एका वाक्यामुळे विचारांना चालना मिळते. आपण निर्णय घेतो, नोकरी बदलतो किंवा नवीन काहीतरी सुरू करतो. इथेही, कॅटालिस्ट ठरतो, तो संवाद.
प्रत्येक संगतीचा चांगलाच किंवा फलदायीच परिणाम होईल असे नाही. जसे मराठीतील प्रसिद्ध वाक्प्रचार आहे, “ढवळ्या शेजारी बांधला पवळा, वाण नाही पण गुण लागला’ हा संगतीने चुकीचेही घडू शकते, दुर्गुणही लागू शकतात याविषयी आहे, काहीसं तसंच. म्हणून ‘कॅटालिस्ट’ हा मदतनीस सिद्ध होत आहे ना याची खात्री करून घ्यावी लागते. 
या सर्व उदाहरणांमध्ये एक गोष्ट स्पष्ट होते की, बदल घडवण्यासाठी प्रत्येक वेळी मोठा धक्का देण्याची गरज नसते; कधीकधी एखादी संगती अथवा योग्य हस्तक्षेप पुरेसा असतो. आपण प्रत्येक परिणामावर नियंत्रण ठेवू शकत नाही, पण एखादी प्रक्रिया कोठे मागे पडत आहे का, हे ओळखून आपण योग्य ठिकाणी योग्य हस्तक्षेप करू शकतो. प्रसंगी स्वत:ला माझ्या एका छोट्याशा कृतीने मोठा बदल घडेल का? कारण आयुष्यात, अगदी रसायनशास्त्राप्रमाणेच, कधीकधी सर्वात मोठे परिणाम अगदी केवळ सानिध्याने घडतात; फक्त ती सुसंगती कोणाची हे कळणे महत्त्वाचे असते.

