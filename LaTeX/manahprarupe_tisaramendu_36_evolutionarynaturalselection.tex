\chapter{... तेथे लव्हाळे वाचती}

हजारो वर्षांपूर्वी जिराफांचे पूर्वज आजच्या जिराफांपेक्षा बरेच वेगळे होते. त्या काळात, आफ्रिकेच्या सॅव्हाना जंगलात झाडांची पाने हा अन्नाचा एक महत्त्वाचा स्रोत होता. जमिनीवरील गवत व झुडपे इतर प्राणीही खात असल्यामुळे स्पर्धा वाढली. अशा वेळी ज्या जिराफांची मान थोडी लांब होती, त्यांना उंच झाडांच्या फांद्यांवरील पाने सहज मिळू लागली. त्यामुळे त्यांना अधिक अन्न मिळाल्याने ते अधिक निरोगी राहिले आणि त्यांची संततीही अधिक संख्येने टिकून राहिली. हळूहळू, पिढ्यानपिढ्या, लांब मान असलेले जिराफच अधिक प्रमाणात टिकले आणि त्यांची संख्या वाढली. यालाच ‘नॅचरल सिलेक्शन’ म्हणजेच नैसर्गिक निवड म्हणतात. जेव्हा एखादे वैशिष्ट्य एखाद्या जीवाला त्याच्या पर्यावरणात टिकून राहण्यासाठी फायदेशीर ठरते, तेव्हा ते वैशिष्ट्य काळानुरूप अधिक प्रबळ होत जाते. यामुळेच आज दिसणारा जिराफ त्याच्या लांब मानेसह उत्क्रांत झाला आहे. ही संकल्पना "सर्व्हायव्हल ऑफ द फिटेस्ट" म्हणजेच 'जो जुळवून घेतो, तोच टिकतो' (‘उत्क्रांतीची नैसर्गिक निवड’) याचे अत्यंत प्रभावी उदाहरण आहे.
‘उत्क्रांतीची नैसर्गिक निवड’ ही मूलतः जीवशास्त्रातील संकल्पना असली तरी, ती आपल्या रोजच्या जगण्यालाही तंतोतंत लागू होते. म्हणूनच, याकडे एक मेंटल मॉडेल (मन:प्रारूप) म्हणून पाहता येते. जे बदलाशी जुळवून घेतात, ते टिकतात; जे घेत नाहीत, ते कालांतराने नाहीसे होतात. तुकोबारायांच्या ओळी "महापुरे झाडे जाती, तेथे लव्हाळे वाचती" हेच दर्शवतात. ‘उत्क्रांतीची  नैसर्गिक निवड’ हे मन:प्रारूप तीन महत्त्वाच्या आधारांवर उभे आहे. एक म्हणजे ‘वैविध्य’ (डायव्हर्सिटी), म्हणजेच वेगवेगळ्या प्रकारच्या कल्पना, पद्धती किंवा कौशल्यांचे अस्तित्व. दुसरे म्हणजे ‘निवड दबाव’, म्हणजे असे सामाजिक, आर्थिक किंवा पर्यावरणीय घटक जे काही विशिष्ट गोष्टींना इतरांपेक्षा जास्त फायदेशीर ठरवतात. आणि तिसरे म्हणजे ‘प्रसार’ म्हणजेच यशस्वी कल्पना किंवा वर्तनाचा इतरांमध्ये होणारा प्रसार किंवा मोठ्या प्रमाणात वापर. याची काही उदाहरणे पाहूया.
भारतामध्ये डिजिटल पेमेंट्सचा वेगाने झालेला प्रसार हे याचे ठळक उदाहरण आहे. जे पारंपरिक व्यवसाय केवळ रोख व्यवहारांवर अवलंबून राहिले किंवा बदलास तयार नव्हते, ते मागे पडले. पण ज्या दुकानदारांनी क्यू-आर कोड, ई-कॉमर्स आणि इन्व्हेंटरी मॅनेजमेंट सॉफ्टवेअर वापरायला सुरुवात केली, ते मात्र बदलाशी जुळवून टिकले.
टायपिंग किंवा शॉर्टहँडसारख्या एकेकाळच्या लोकप्रिय कौशल्यांना आज फारशी मागणी नाही. आता तर आवाजाचे थेट लेखनात रूपांतर (डिक्टेशन) करणारे तंत्रज्ञान आले आहे. ज्या लोकांनी स्वतःला अशा डिजिटल साधनांमध्ये प्रशिक्षित केले, ते आजही केवळ टिकूनच नाहीत, तर त्यांना मोठी मागणी आहे. कारण तंत्रज्ञानाच्या दुनियेत "स्थिर" असे काहीच नसते.
फोटो काढणे एके काळी केवळ फिल्मवर शक्य होते. पण जेव्हा डिजिटल तंत्रज्ञान आले, तेव्हा ज्या कंपन्यांनी ते स्वीकारले, त्या यशस्वी झाल्या. ‘कोडॅक’सारखी कंपनी, जी फिल्मवरच अवलंबून राहिली, तिने आपला बाजार गमावला. अगदी याचप्रमाणे मोबाईल उद्योगात नोकियाचा एक काळ होता. पण स्मार्टफोन आणि टचस्क्रीन तंत्रज्ञान आले, तेव्हा नोकिया वेळेवर बदल करू शकली नाही. त्याचवेळी ॲपल आणि अँड्रॉइडने बाजार काबीज केला.
घोकंपट्टीवर आधारलेली पारंपरिक शिक्षणपद्धती आता टीकेचा विषय ठरली आहे. त्या तुलनेत ज्या शिक्षणसंस्था मुलांमध्ये समस्या सोडवण्याची क्षमता, सर्जनशीलता आणि आंतरविद्याशाखीय  (इंटर डिसिप्लिनरी)  दृष्टिकोन रुजवतात, त्या अधिक यशस्वी ठरत आहेत. या बदलत्या जगात कोणत्याही शिक्षणपद्धतीला टिकून राहायचे असेल, तर तिला बदल स्वीकारावेच लागतील.
एकेकाळी युद्धातील सर्वात प्रभावशाली शक्ती म्हणजे घोडेस्वार सैन्य. ते वेगवान, शक्तिशाली आणि थेट हल्ल्यांसाठी वापरले जात असे. पण विसाव्या शतकात टँक, मशीनगन आणि हवाई दलाचा उदय झाला. दुसऱ्या महायुद्धात पोलंडचे पारंपरिक घोडेस्वार सैन्य जर्मनीच्या टँक-आधारित 'ब्लिट्झक्रिग' रणनीतीपुढे टिकू शकले नाही. आता तर चित्र पूर्णपणे बदलले आहे; केवळ शेकडो ड्रोन्स आणि क्षेपणास्त्रे पाठवून शत्रू राष्ट्राला नामोहरम करता येते. जे देश जुन्या पद्धतींमध्ये अडकले, त्यांना युद्धात मोठे नुकसान सहन करावे लागले.
नैसर्गिक निवडीचे वैशिष्ट्य हेच आहे की, ती कोणाचे यश, शक्ती, परंपरा किंवा निष्ठा पाहून निर्णय घेत नाही. ती फक्त हे पाहते की, कोण ‘जुळवून घेऊ शकते’. या मेंटल मॉडेलचा खरा अर्थ असा आहे की, यश कधीच कायमस्वरूपी नसते आणि टिकून राहण्याची कोणतीही हमी नसते.
जर आपल्याला टिकायचे असेल, तर ‘नामशेष होणे’ या संकल्पनेकडे अपयश म्हणून नव्हे, तर 'आता काहीतरी बदलायला हवे' अशी सूचना म्हणून पाहावे लागेल. हे जितके लवकर उमगेल, तितक्या लवकर आपण स्वतःमध्ये आणि आपल्या सभोवतालच्या जगात आवश्यक बदल घडवू शकू. कुणी सांगावे, ज्या गरजेमुळे जिराफाची मान लांब झाली, तशीच भविष्यात अधिक विचार करण्याच्या गरजेमुळे मानवाचे डोके मोठे होईल?


