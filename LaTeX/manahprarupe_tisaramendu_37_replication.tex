\chapter{बे एके बे, बे दुणे चार }

१९८० आणि ९०च्या दशकात, चीनने आपल्या आर्थिक प्रगतीची सुरुवात पाश्चिमात्य औद्योगिक मॉडेल्सची हुबेहुब नक्कल करून आणि तंत्रज्ञानाची उलट-अभियांत्रिकी (रिव्हर्स इंजिनिअरिंग) करून केली. इलेक्ट्रॉनिक्सपासून ते अवजड यंत्रसामग्रीपर्यंत अनेक उत्पादनांची चीनने बारकाईने प्रतिकृती केली. परदेशी कंपन्यांशी संयुक्त भागीदारी करत चीनने तंत्रज्ञान, उत्पादन प्रक्रिया आणि पुरवठा साखळीचे कौशल्य आत्मसात केले. चीनला "जगाचा कारखाना" असं म्हटलं जाऊ लागलं, कारण त्याचा भर स्वस्त उत्पादनावर आणि नक्कल करण्यावर होता.

पण हळूहळू चित्र बदलू लागलं. चीनने संशोधन आणि विकास (आर-अँड-डी), शिक्षण आणि स्थानिक प्रतिभेमध्ये मोठ्या प्रमाणात गुंतवणूक केली. २०१०च्या दशकात ‘हुवावे’, ‘शाओमी’ आणि ‘बीवायडी’ सारख्या कंपन्या केवळ नक्कल करणाऱ्या राहिल्या नाहीत, तर अनेक तंत्रज्ञानात आघाडीच्या ठरल्या. ही परिवर्तनकथा ‘रेप्लिकेशन’ या मेंटल मॉडेलचे (मन:प्रारूप) उत्कृष्ट उदाहरण आहे. याचा अर्थ असा की, केवळ जशीच्या-तशी नक्कल करणं आणि समजून घेतलेली पुनरावृत्ती यात मोठा फरक आहे. विचारपूर्वक केलेली पुनरावृत्ती ही परिणामकारक प्रगतीची रणनीती ठरू शकते, पण आंधळेपणाने केलेली नक्कल नेहमी चेष्टेचा विषय बनते आणि अपयशाकडे नेते.

जीवशास्त्रात, ‘रेप्लिकेशन’ म्हणजे गुणसूत्रांची (डीएनए) स्वतःची प्रत तयार करण्याची प्रक्रिया. जीवसृष्टीच्या सातत्यासाठी अत्यावश्यक असलेली ही प्रक्रिया केवळ पेशींपुरती मर्यादित नाही. ही संकल्पना एक मेंटल मॉडेल म्हणून शिकवते की यशस्वी उत्पादन, प्रक्रिया किंवा प्रणाली प्रभावीपणे पुन्हा वापरता येते. या संकल्पनेचं तत्त्व सोपं आहे, जे काही यशस्वी ठरतं, ते प्रथम समजून घ्या, त्यामागचं "का" जाणून घ्या, आणि मग ते शहाणपणाने पुन्हा राबवा. आपण प्रत्येक वेळी नव्याने चाकाचा शोध घेण्याची गरज नाही (“व्हाय रिइन्व्हेन्ट द व्हील?”). ‘बे एके बे’ झाले की नीट समजून घ्या, बरोबर झाले असेल तर ‘बे दुणे चार” आणि पुढे. याची काही उदाहरणे पाहुयात.

शिक्षणाचंही तसंच आहे. गुरुकुलाची पद्धत काळानुसार बदलत गेली. मॅकॉलेने इंग्रजांना हवी तशी शिक्षणपद्धती आणली तशी ती स्वातंत्र्योत्तर भारतातातील शाळांमध्ये बऱ्यापैकी तशीच उतरली. पण  ‘देश, काल, पात्र’ यांचा विचार न करता केल्याने, तर ती फारशी यशस्वी झाली नाही. भारताने यशस्वी कारकून, कामगार बनवले पण नवोन्मेषी शोधांमध्ये आपण मागे पडलो. त्यामुळे स्थानिक गरजांनुसार योग्य बदल करतच पुनरावृत्ती करावी लागते.

तसेच भारतीय स्टार्टअप्सपैकी अनेकांनी परदेशी यशस्वी मॉडेल्सची नक्कल केली. काहींनी त्यात यश मिळवलं कारण त्यांनी स्थानिक गरजांशी जुळवून घेतलं (उदाहरण: यु-पी-आय मुळे डिजिटल पेमेंट्समध्ये झालेली क्रांती). पण अनेक कंपन्या अपयशी ठरल्या कारण त्या मॉडेल्सना स्थानिक सामाजिक आणि आर्थिक वास्तवाशी मेळ घालता आला नाही.

आपल्या वैयक्तिक आयुष्यातही हे लागू होतं. इंटरनेटवर प्रसिद्ध झालेले "मॉर्निंग मिरॅकल" किंवा उत्पादकतेचे हॅक्स जेव्हा आपण आपल्या गरजा, सवयी आणि ध्येय न समजता अनुकरण करतो, तेव्हा ते फार काळ टिकत नाहीत. थेट नक्कल न करता, आपल्या गरजेनुसार बदल करूनच त्याचा उपयोग होतो.

‘रेप्लिकेशन’ आपल्याला एक अमूल्य पाठ शिकवतं, प्रगती अनेकदा नक्कलांमधूनच सुरू होते. लहान मूल मोठ्यांची नक्कल करून बोलायला शिकतं. शिकाऊ चित्रकार जुनी-प्रसिद्ध पाहून चित्र काढायला शिकतात. सुप्रसिद्ध गायक-गायिका त्यांच्या सुरुवातीच्या काळात जुन्या गायकांसारखेच गाताना वाटतात. पण कालांतराने जे स्वतःची शैली आणि आवाज निर्माण करतात, तेच टिकतात. मात्र, यामध्ये सावधगिरी आवश्यक आहे. संपूर्ण संदर्भ न समजता केवळ नक्कल केल्यास परिणाम शून्य होतात. उदाहरणार्थ, एक यशस्वी प्रयोग दुसऱ्या ठिकाणी करायचा असेल, तर स्थानिक परिस्थिती, संस्कृती आणि संसाधनांचा विचार करावा लागतो. अन्यथा अशी नक्कल उथळ ठरते आणि पुनरावृत्तीचा अयोग्य वापर हानिकारकही ठरू शकतो.

आजच्या जगात "एका तासात यशस्वी व्हा" अशा टेम्प्लेट्स, फॅड्स आणि प्रसिद्ध व्यावसायिक कहाण्या सर्वदूर पसरलेल्या आहेत. त्यामुळे काही यशस्वी गोष्ट पाहिल्यावर आपल्याला ती तशीच नक्कल करावीशी वाटते. पण विचार न करता नक्कल करणं म्हणजे मूळ मजकूर न वाचता फोटोकॉपी काढणं, जिथे चुका वाढतात, अर्थ हरवतो आणि मूळ उद्देशही गमावतो. म्हणून, काहीही नक्कल करण्याआधी स्वतःला विचारा: हे का यशस्वी झालं? आणि गरज पडल्यास मी ते बदलून घेण्यास तयार आहे का? ‘रेप्लिकेशन’ ही एक दुधारी तलवार आहे. ती एकाच वेळी उत्कृष्टतेचा प्रचार करू शकते किंवा सुमारतेचा प्रसार. निवड आपली आहे, यशस्वी चित्रपटाचे दाक्षिणात्य ‘रिमेक’ आपण किती दिवस काढत राहणार?
