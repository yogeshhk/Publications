\chapter{परस्पर संबंधांचे महाजाल }

अलीकडेच एक मोठा व्यापारी संघर्ष उफाळून आला, जेव्हा अमेरिकेने चीनवर मोठ्या प्रमाणात आयात-कर (टॅरिफ) लादले. त्याला प्रत्युत्तर देताना, चीनने दुर्मीळ खनिजांच्या निर्यातीवर निर्बंध घातले. ही खनिजे अशी आहेत की त्यांचा उपयोग स्मार्टफोनपासून सौरऊर्जा पॅनेलपर्यंत अनेक आधुनिक तंत्रज्ञानांत होतो. या संघर्षात भरडून निघाला तो भारताचा नुकताच उगवता इलेक्ट्रिक वाहन उद्योग. गरजेच्या खनिजांचा तुटवडा निर्माण झाला, उत्पादन रखडले आणि रोजगाराला खीळ बसली. अमेरिकेच्या एका निर्णयाचे परिणाम हजारो मैल दूर भारतात जाणवतात. हेच आहे ‘इकोसिस्टम’ नावाचे मेंटल मॉडेल (मन:प्रारूप). ते आपल्याला शिकवते की, जग वेगवेगळ्या, अलिप्त गोष्टींनी बनलेले नाही, तर या सर्व गोष्टी एकमेकांशी जोडलेल्या आहेत. परस्पर संबंधांचे महाजाल. एक घटक हलला की त्याचे पडसाद दुसरीकडे उमटतात. हे अगदी ‘फुलपाखराच्या फडफडीसारखे’ (बटरफ्लाय इफेक्ट) आहे, ज्यात एका कोपऱ्यातील फडफडणारे पंख दुसऱ्या कोपऱ्यात वादळ (केऑस) निर्माण करू शकतात.
आपण अनेकदा घटनांकडे एक स्वतंत्र घडामोड म्हणून पाहतो. पण आजचे जग तसे चालत नाही. तैवानमधील एका तांत्रिक धक्क्याचे परिणाम उत्तर प्रदेशातील मोबाईल दुकानदारांना, पुण्यातील वाहन उद्योगाला, आसाममधील विद्यार्थ्यांना आणि तामिळनाडूमधील लॉजिस्टिक कंपन्यांना जाणवतात. हे केवळ जागतिकीकरणाचे परिणाम नाहीत, तर हे एका परस्परसंबंधित जगाचे, म्हणजेच ‘इकोसिस्टम’चे चित्र आहे. ती स्थिर नसते; ती सतत बदलत असते, स्वतःला सुधारते किंवा कधीकधी कोसळतेही. त्यात सकारात्मक आणि नकारात्मक प्रतिक्रियांचा प्रतिसाद (फीडबॅक लूप्स) असतो, जो परिणामांना कधी वाढवतो किंवा स्थिर करतो. या व्यवस्थेत स्पर्धा आणि सहकार्य एकत्र नांदतात. ‘इकोसिस्टम’ ही कल्पना मुळात निसर्गशास्त्रातून आली आहे, जिथे झाडे, प्राणी, हवामान आणि जमीन यांचा परस्परसंबंध अभ्यासला जातो. पण आज हीच चौकट आपण व्यवसाय, आरोग्य, तंत्रज्ञान आणि शिक्षण या क्षेत्रांतही वापरू शकतो. याची काही उदाहरणे पाहूया.
कोरोना विषाणूचा उगम चीनमधील एका बाजारातून झाल्याची शक्यता वर्तवली गेली. या एका स्थानिक घटनेने संपूर्ण जगातील आरोग्य, शिक्षण, अर्थव्यवस्था आणि दैनंदिन जीवनाला धक्का दिला. भारतातही लॉकडाउन, बेरोजगारी, ऑनलाइन शिक्षण याचे दूरगामी परिणाम झाले. 
त्याचप्रमाणे, रशिया-युक्रेन युद्धामुळे भारतातील सामान्य ग्राहकांपर्यंत थेट परिणाम पोहोचला, कारण दोन्ही देश भारताला सूर्यफूल तेलाचा पुरवठा करणारे मोठे देश होते. तेलाच्या किमती वाढल्या आणि भारतीय स्वयंपाकघरांपर्यंत महागाईची झळ पोहोचली.
एखाद्या कृषी शास्त्रज्ञाने प्रयोगशाळेत विकसित केलेले गव्हाचे एक नवीन, उच्च-उत्पादन देणारे वाण (एक लहान बदल) भारतात 'हरित क्रांती' घडवते. यामुळे केवळ देशाची अन्नसुरक्षाच साधली जात नाही, तर सिंचन पद्धती, खतांचा वापर आणि ग्रामीण अर्थकारण यांवरही दूरगामी परिणाम होतात. एका बियाणात झालेला बदल संपूर्ण सामाजिक-आर्थिक व्यवस्थेला नवी दिशा देतो.
शहराच्या विकास आराखड्यात, एखाद्या निवासी भागाजवळ एक नवीन 'मेट्रो स्टेशन' मंजूर करण्याचा निर्णय (एक स्थानिक बदल) वरवर पाहता छोटा वाटू शकतो. पण त्यामुळे त्या परिसरातील घरांच्या किमती गगनाला भिडतात, लहान-मोठे व्यवसाय वाढीस लागतात, वाहतुकीची पद्धत बदलते आणि लोकांचे जीवनमान उंचावते. काहीवेळा यामुळे जुन्या रहिवाशांना विस्थापितही व्हावे लागते. एका स्टेशनचा निर्णय संपूर्ण परिसराचे चित्र बदलून टाकतो.
एखाद्या सोशल मीडिया ॲपवर कोणीतरी खोडसाळपणे पसरवलेली एक छोटीशी अफवा (एक डिजिटल फडफड) काही तासांतच एखाद्या कंपनीचे लाखो रुपयांचे नुकसान करते किंवा समाजात मोठा तणाव निर्माण करते. इथे तंत्रज्ञानाच्या परिसंस्थेतील एक नगण्य वाटणारी घटना वास्तवात मोठे वादळ निर्माण करते आणि तिचे परिणाम आर्थिक किंवा सामाजिक क्षेत्रात भोगावे लागतात.
‘इकोसिस्टम’ हे मानसिक प्रारूप आपल्याला सांगते की, सर्व काही आपल्या नियंत्रणात नसते आणि जे आपल्याला वरवर दिसते, ते संपूर्ण सत्य नसते. आपण उचललेल्या एका पावलाचे दूरगामी परिणाम होऊ शकतात, कधी फायदेशीर तर कधी हानिकारक. हे प्रारूप आपल्याला घाईगडबडीत निर्णय घेण्यापासून रोखते आणि विचारपूर्वक कृती करायला शिकवते. आपण ज्या जगात राहतो, ते एका विशाल, अदृश्य जाळ्यासारखे आहे; आपण एक धागा ओढतो आणि कंपने भलत्याच ठिकाणी जाणवतात.
ही विचारसरणी आपल्याला अधिक सजग नागरिक आणि माणूस बनवते. म्हणूनच, आपण सिग्नल मोडायचा नाही, हा एक छोटासा निश्चयसुद्धा सामाजिक शिस्तीच्या मोठ्या चळवळीची सुरुवात ठरू शकतो. सुरुवात तर करा. 
