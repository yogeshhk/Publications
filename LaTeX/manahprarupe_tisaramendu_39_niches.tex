\chapter{एकमेवाद्वितीय }

अनेक शहरांमध्ये सकाळी-सकाळी एक नेहमीचे चित्र दिसते, ते म्हणजे भाजीबाजाराचे. अनेक विक्रेते एकाच वेळी ओरडून आपापल्या भाजीपाल्याकडे ग्राहकांचे लक्ष वेधण्याचा प्रयत्न करत असतात. सगळे जण थोड्याफार प्रमाणात तेच विकत असतात, तेही त्याच लोकांना, त्याच पद्धतीने. पण या गोंधळात, एका शांत कोपऱ्यात एक छोटासा, साधा ठेला-स्टॉल असतो, जो काहीतरी वेगळे करत असतो. इथे फक्त एकच गोष्ट विकली जाते: विदेशी मशरूम्स. ना ओरडणे, ना घासाघीस. त्याचा ग्राहकवर्ग ठरलेला आहे: शेफ्स, आरोग्याबाबत जागरूक लोक आणि काही जिज्ञासू खरेदीदार, जे मुद्दामहून इथेच येतात. इतर विक्रेते जिथे कमी नफ्यावर झगडत असतात, तिथे हा विक्रेता मात्र भरभराटीस आलेला दिसतो. कारण तो स्वस्त विकत नाही, गोंगाट करत नाही, तर तो इतरांपेक्षा वेगळा आहे.
हे काही योगायोगाने घडलेले नाही. हे आहे ‘नीश’ (अतिविशेषता) या मेंटल मॉडेलचे (मनःप्रारूप) एक जिवंत उदाहरण. ही कल्पना मूळची पर्यावरणशास्त्रातील असली, तरी आजच्या जगात आपले काम, व्यवसाय आणि आयुष्य समजून घेण्यासाठी ती अत्यंत उपयुक्त ठरते.
निसर्गात ‘नीश’ म्हणजे एखाद्या जीवाचे पर्यावरणातील अतिविशिष्ट (स्पेशलाइज्ड) स्थान. तो काय खातो, कुठे राहतो, आणि इतर जीवांशी कसा संवाद साधतो, हे सर्व त्यात येते. दोन प्रजाती एकाच नीशमध्ये फार काळ टिकत नाहीत; स्पर्धेमुळे एकाला स्वतःत बदल करावाच लागतो. आपल्या मानवी जगातही असेच नीश असतात. हे नीश म्हणजे काय? तर, एक असे विशिष्ट स्थान किंवा अढळपद, जे एखादी व्यक्ती आपल्या कौशल्यांवर आणि अनुभवावर आधारित मिळवते. बाजारातील एक अनोखे उत्पादन, करिअरमधील दुर्मिळ स्पेशलायझेशन, ही सगळी ‘नीश’ मेंटल मॉडेलचीच उदाहरणे आहेत. आपल्याकडेही याचे अनेक दाखले आहेत त्यातील एक आपल्या जवळचे आहे ते म्हणजे, स्वर्गीय लता मंगेशकर. पुलंनी एकदा त्यांच्याविषयी म्हटले होते, ‘आकाशात सूर्य आहे, चंद्र आहे… आणि लताचा स्वर आहे!’ केवढे यथार्थ वर्णन, नाही का? सुप्रसिद्ध वचन, "झाले बहू, होतील बहू, परी या सम हा", म्हणजे ‘नीश ‘च. एकमेवाद्वितीय. या संकल्पनेची काही अजून उदाहरणे पाहूयात. 
अनेक वस्त्रोद्योगातील कंपन्या सर्वसामान्य ग्राहकांच्या मागे धावत असतात. पण ज्या कंपन्या एका विशिष्ट गरजेला लक्ष्य करतात, त्या अनेकदा अधिक यशस्वी ठरतात. उदाहरणार्थ, एक भारतीय स्टार्टअप जो फक्त बाळांसाठी रसायनमुक्त लंगोट (डायपर्स) बनवते. ही कंपनी विशेषतः अशा पालकांसाठी आहे, जे आपल्या मुलांच्या त्वचेसाठी अतिशय जागरूक आहेत. मुख्य प्रवाहातील ‘सरासरी’ फॅशनच्या गर्दीत सामील न होता, ही कंपनी बाजारात स्वतःचे अढळ स्थान निर्माण करते. साहजिकच, स्पर्धा कमी असल्याने किमतीवर पकड राहते.
सर्वसाधारण वकील घरगुती किंवा नेहमीच्या अपराधिक तंट्यांच्या केसेस घेतात. पण याउलट, एखादी वकील जर फक्त सायबर क्राईम आणि डिजिटल प्रायव्हसीसारख्या अत्याधुनिक आणि तुलनेने अपरिचित विषयांवर काम करत असेल, तर ती नक्कीच जास्त फी आकारू शकते. भारतात ही क्षेत्रे अजूनही विकसित होत आहेत. हजारो सामान्य वकिलांच्या स्पर्धेत न पडता, ती स्वतःची जागा मजबूत करते. हेच ‘नीश’.
शेतीतही हेच तत्त्व लागू होते. एखादा शेतकरी पारंपरिक ऊस, गहू न पिकवता, काळ्या तांदळाची जुनी जात किंवा इतर दुर्मीळ धान्ये पिकवतो. त्याची पैदावार कमी असली, तरी आरोग्याबद्दल जागरूक असलेले ग्राहक ते पीक प्रीमियम दराने खरेदी करतात. असे करून तो अस्थिर कृषी-व्यवस्थेत स्वतःसाठी स्थैर्य आणि संपत्ती निर्माण करतो.
डिजिटल युगात, ‘सर्वांसाठी सर्वकाही’ या जागतिक स्पर्धेत, केवळ 'इतरांपेक्षा चांगले' (बीइंग बेटर) असून भागत नाही, तर 'इतरांपेक्षा वेगळे' (बीइंग डिफरंट) असणे पण महत्त्वाचे ठरते.
पण, ‘नीश’ निवडणे म्हणजे कायमस्वरूपी यशाची हमी नव्हे. बदलत्या काळात एखादा ‘नीश’ कालबाह्य किंवा नष्टही होऊ शकतो. ग्राहकांच्या सवयी बदलतात, तंत्रज्ञान मागे पडते. म्हणूनच आपल्या नीशमध्ये तरबेज असलेल्यांनीसुद्धा बदलांसाठी तयार रहावे लागते. मशरूम विक्रेत्याला ग्राहकांच्या बदलत्या चवीनुसार. सायबर वकिलांना तंत्रज्ञानाशी संबंधित नवनवीन कायदे समजून घ्यावे लागतील. थोडक्यात, ‘विशेषतेसोबतच लवचिकता’ हाच ‘नीश’ या मेंटल मॉडेलचा गाभा आहे.
हे मेंटल मॉडेल नवउद्योजक आणि विचारवंत नवल रविकांत यांच्या ‘विशिष्ट ज्ञान’ (स्पेसिफिक  नॉलेज) या कल्पनेशी मिळतेजुळते आहे. तुमचे ज्ञान-कौशल्य असे असावे, की जे अतिविशिष्ट, दुर्मिळ आणि सहजासहजी शिकता येण्यासारखे नसेल. त्यामुळे जागतिक पटलावर तुम्ही एकमेवाद्वितीय ठरू शकता.
अशा प्रकारे, ‘नीश’ हे केवळ स्पर्धा टाळण्याचे साधन नाही, तर ते स्वतःचे एक वेगळे अस्तित्व निर्माण करण्याचा मार्ग आहे. हे जगापासून दूर जाणे नव्हे, तर आपण निवडलेल्या क्षेत्रात स्वतःला केंद्रस्थानी प्रस्थापित करणे आहे. “परी या सम हा" अशी स्तुती आपल्या वाट्याला आली, तर आयुष्याचे सार्थक झाले असे म्हणता येईल, नाही का?
