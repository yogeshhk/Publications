\chapter{पिंजऱ्याचे दार उघडावे}

या लेखमालेचे शीर्षक ‘तिसरा मेंदू’ हे बाह्य किंवा पूरक बुद्धिमत्तेचे प्रतीक आहे. म्हणूनच कृत्रिम बुद्धिमत्ता असो वा मानवी बुद्धिमत्तेला पूरक ठरणारी मेंटल मॉडेल्स (मनःप्रारूपे) असोत, योग्य प्रसंगी योग्य पद्धतीने विचार करणे किंवा करवून घेणे, हेच खरे परिपूर्ण बुद्धिमत्तेचे लक्षण ठरते. या प्रक्रियेमुळे आपली विचारक्षमता व्यापक होते आणि कोणत्याही प्रश्नाचा सर्वांगीण विचार करता येतो. लौकिकदृष्ट्या ‘कृत्रिम बुद्धिमत्ता आणि मेंटल मॉडेल्स’ या विषयांवरील लेखमालारूपी हा विचारप्रवाह जरी येथे थांबत असला तरी, ‘राम-राम’ करण्याआधी मात्र, थोडे ‘सिंहावलोकन’ आणि थोडे ‘पुढे काय?’ यासाठी हा लेखप्रपंच. त्यातही प्रामुख्याने ‘मेंटल मॉडेल्स’ विषयी. 
‘मेंटल मॉडेल्स’ ही केवळ तात्त्विक संकल्पना किंवा बौद्धिक कसरत नाही, तर जगाकडे पाहण्याची एक विशिष्ट पद्धत आहे. ही एक अशी विचाराची चौकट (फ्रेमवर्क) आहे, जणू एखादा चष्मा, ज्यातून आपण घटना, संधी आणि धोके अधिक स्पष्टतेने पाहू शकतो. सुप्रसिद्ध गुंतवणूकतज्ञ आणि तत्वज्ञ चार्ली मंगर यांनी म्हटल्याप्रमाणे, वेगवेगळ्या क्षेत्रांतील मॉडेल्सचे लॅटिसवर्क (म्हणजेच विचारांचे जाळे) तयार करणे हेच खरे कौशल्य आहे. कारण अशा पद्धतीनेच आपण वास्तवाकडे विविध दृष्टिकोनांतून पाहू शकतो, आणि एकांगी विचारसरणीपासून स्वतःचे संरक्षण करू शकतो.
योग्य मेंटल मॉडेल्स कशी निवडावीत?
आपल्याला उपलब्ध असलेल्या मेंटल मॉडेल्सचे विश्व अक्षरशः अमर्याद आहे. त्यांतील काही मॉडेल्स परस्परविरोधी भासतात किंवा इतरांपेक्षा उलट सल्ले देतात. येथे सारासार विवेक अत्यंत महत्त्वाचा ठरतो. कोणत्या परिस्थितीत काय वापरायचे, हे अनुभवानेच जमेल. सुरुवात अशा मॉडेल्सने करा, जी अनेक परिस्थितींमध्ये उपयोगी पडतात. उदाहरणार्थ: ‘इन्वर्जन’ म्हणजे उलट पद्धतीने विचार करणे, ‘ऑपॉर्च्युनिटी कॉस्ट’ म्हणजे एक गोष्ट निवडल्याने दुसरी कोणती संधी गमावली हे लक्षात घेणे, किंवा ‘सेकंड-ऑर्डर थिंकिंग’ म्हणजे निर्णयांच्या दूरगामी परिणामांचा विचार करणे.
आपल्या विचारशैलीला आव्हान देणारी आणि आपले अंधबिंदू (ब्लाइंड स्पॉट्स) उघड करणारी मॉडेल्स प्राधान्याने समजून घ्या. जसजशी अनुभवाची भर पडेल, तसतसे भौतिकशास्त्र, जीवशास्त्र, अर्थशास्त्र, मानसशास्त्र अशा विविध क्षेत्रांतील मॉडेल्स आपल्या विचारसंपदेत सामील करा. हे लक्षात ठेवा की, सर्व विषयांत तज्ज्ञ होणे आवश्यक नाही, परंतु गुंतागुंतीच्या समस्यांकडे पाहण्यासाठी भक्कम आणि विश्वसनीय चौकटी (फ्रेमवर्क) निर्माण करणे महत्त्वाचे आहे.
काय करावे आणि काय टाळावे?
ही मॉडेल्स घाईघाईने एकामागून एक गोळा करण्यापेक्षा, त्यांची निवडक आणि सखोल समज आवश्यक आहे. गुणवत्ता ही संख्येपेक्षा अधिक महत्त्वाची आहे. मात्र, कोणत्याही एका मॉडेलला अंतिम सत्य मानू नका. प्रत्येक मॉडेल हे वास्तवाचे केवळ सरलीकरण असते, ते संपूर्ण सत्य नव्हे. एकाच मॉडेलचा अतिरेक टाळा; अतिरेकी आसक्ती विचारांना मर्यादित करते, कारण ‘ज्याच्या हातात हातोडा असतो, त्याला प्रत्येक गोष्ट खिळ्यासारखी दिसू लागते.’
आपल्या ज्ञानाची मर्यादा ओळखा. कुठे आपली समज मजबूत आहे आणि कुठे कमकुवत, हे प्रामाणिकपणे स्वीकारा. गरज असेल तिथे मार्गदर्शन घेण्याची किंवा अधिक शिकण्याची तयारी ठेवा. सर्वात महत्त्वाचे म्हणजे, स्वतःला नेहमी प्रश्न विचारत राहा. काळ, परिस्थिती आणि समाज सतत बदलत असतो. त्यामुळे आपली मेंटल मॉडेल्सदेखील वेळोवेळी तपासून, सुधारून किंवा बाजूला ठेवण्याची तयारी ठेवावी लागते.
पुढे काय?
मेंटल मॉडेल्सचे शिक्षण ही आयुष्यभर चालणारी प्रक्रिया आहे. केवळ आपल्या आवडीच्या क्षेत्रातच नव्हे, तर वेगवेगळ्या विषयांत वाचन करा. त्यातून नवे दृष्टिकोन मिळतील. इतरांशी चर्चा करा, कारण शिकवण्याने आणि वादविवादाने आपली समज अधिक खोल होते.
स्वतःच्या विचारांची चिकित्सा करा. जेव्हा तुम्ही एखादा निर्णय घ्याल, तेव्हा त्याचे परिणाम कसे झाले, हे लिहून ठेवा. यामुळे कोणती मॉडेल्स आपल्याला उपयुक्त ठरतात, हे समजण्यास मदत होईल. वास्तवातील प्रश्नांवर ही मॉडेल्स तपासून पाहा, ते तुमच्यासाठी जिवंत प्रयोगशाळा ठरतील. 
शेवटचे आणि महत्त्वाचे
ही लेखमाला तुमच्यासाठी एक सुरुवात ठरावी. अथ मनःप्रारूपानुशासनम्!! आता मनःप्रारूप वापरण्याची शिस्त अंगीकारावी. खरी वाटचाल तेव्हा सुरू होते, जेव्हा तुम्ही या मॉडेल्सना तुमच्या आयुष्यात उतरवता, त्यांना स्वतःच्या अनुभवांनुसार घडवता आणि सतत जगाकडे नव्या नजरेने पाहण्याचा प्रयत्न करता. कारण शहाणपण म्हणजे सर्व उत्तरे माहीत असणे नव्हे, तर योग्य प्रश्न विचारण्याची आणि अधिक स्पष्टतेने पाहण्याची क्षमता मिळवणे होय.
आपण सर्वजण नकळतपणे कोणत्यातरी विचारपद्धतीच्या, सवयींच्या आणि पूर्वग्रहांच्या पिंजऱ्यात राहात असतो. तो पिंजरा सोन्याचा, म्हणजेच सुरक्षिततेचा, ओळखीचा आणि आरामाचा असतो. पण तो कितीही आकर्षक असला तरी, आपली विचारशक्ती, जिज्ञासा आणि जाणिवांचा संकोच करू शकतो. त्या सोयीच्या चौकटीबाहेर पाऊल ठेवावे, आपले पूर्वग्रह तपासावेत आणि विचारांच्या विशाल आकाशात झेप घ्यावी, जिथे नवीन ज्ञान, नवे दृष्टिकोन आणि खरी मोकळीक मिळते. म्हणूनच, आता नक्कीच “पिंजऱ्याचे दार उघडावे”... 
