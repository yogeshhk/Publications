\chapter{‘मन:प्रारूपे’, एक तोंडओळख}

जगातील सर्वात प्रसिद्ध गुंतवणूकदार कोण? असे विचारल्यावर बहुतेकांचे उत्तर असेल - वॉरन बफे. पण त्यांचे तुलनेने कमी प्रसिद्ध, तरीही असामान्य प्रतिभेचे सहकारी म्हणजे चार्ली मंगर. कोणत्याही समस्येवर किंवा प्रसंगावर चार्ली विविध दृष्टिकोनातून विचार करायचे. या दृष्टिकोनांना ते मेंटल मॉडेल्स (मन:प्रारूपे) म्हणत. सोप्या भाषेत याला आपण 'विचार-चित्रे' म्हणू शकतो, कारण एखाद्या परिस्थितीला दिला जाणारा प्रतिसाद हा आपल्या मनात कोरलेल्या विचार-चित्रावर अवलंबून असतो. मानवी मेंदू सुरवातीपासून  विचार करण्याची ऊर्जा वाचवण्यासाठी अशा विचार-चित्रांचा आधार घेतो आणि सहज निर्णय घेतो.

आपण स्वतःला तर्कशुद्ध आणि वस्तुनिष्ठ विचार करणारा प्राणी समजतो, पण प्रत्यक्षात असे सतत घडत नाही. अनेकदा भावनांच्या प्रभावाखाली, सवयीने किंवा पूर्वग्रहदूषित पद्धतीने निर्णय घेतले जातात. प्रदीर्घ अनुभव आणि वारंवारतेमुळे अशा पद्धती विकसित होतात. विविध प्रसंगांना तोंड देण्यासाठी वेगवेगळी विचार-चित्रे वापरण्यात येतात. आपण सर्वच जण यांचा वापर करीत असलो तरी काही विशिष्ठ विचारचित्रे अतिशय प्रभावीपणे उपयोगात आणता येतात. पुढील लेखांमध्ये अशीच काही विचार-चित्रे उदाहरणांसह पाहुयात.

या पुस्तकातील लेख दै. सकाळ मधील `तिसरा मेंदू' या सदरात यापूर्वी प्रकाशित झालेले असून, येथे ते आवश्यक त्या हलक्या संपादनासह एकत्रित स्वरूपात मांडले आहेत. 'तिसरा मेंदू'  हे बाह्य किंवा अतिरिक्त बुद्धिमत्तेचे प्रतीक आहे. जर आपल्या भात्यात अनेक मेंटल मॉडेल्स असतील, तर योग्य प्रसंगात योग्य मॉडेल वापरणे हीच अतिरिक्त बुद्धिमत्ता ठरते. त्यामुळे आपली विचारक्षमता वाढते आणि प्रश्नांचा सर्वांगीण विचार करता येतो.

यशाच्या पायऱ्या चढताना जबाबदाऱ्या वाढतात. गतिमान आणि गुंतागुंतीच्या जगात घेतलेल्या निर्णयांचे परिणाम दूरगामी असतात. म्हणूनच यशस्वी लोक कळत-नकळत मेंटल मॉडेल्सचा उपयोग करतात. निर्णय घेण्यासाठी, संधी ओळखण्यासाठी आणि अचूक ज्ञान मिळवण्यासाठी ही पद्धती उपयुक्त ठरते. विचार करण्याची क्षमता सरावाने वाढवता येते. भविष्यातील संभाव्य परिस्थितींसाठी विविध विचार-चित्रे वापरून आढावा घेता येतो. नक्की काय महत्त्वाचे आहे? त्वरित आणि दूरगामी परिणाम काय असतील? यासाठी विचार-चित्रे मदत करतात. नियमित वापर केल्यास ही मॉडेल्स भात्यातील बाणांप्रमाणे सहज वापरता येतात. नेहमीच्या वापरातील, तुलनेने सोपे आणि प्रसिद्ध असे एक विचारचित्र आणि त्याची एक-दोन उदाहरणे पाहुयात. 

“स्वॉट विश्लेषण” 
या पद्धती मध्ये एखाद्या कल्पनेचे, प्रकल्पाचे, संस्थेचे किंवा चक्क माणसाचे सुद्धा मूल्यमापन करता येते. “स्वॉट” याच्या चार अद्याक्षरांच्या आधारे ते होते, म्हणजेच ‘एस’ (स्ट्रेंग्थ, सामर्थ्य), ‘डब्ल्यू’ (विकनेस, कमकुवत बाजू), ‘ओ’ (ओप्पोर्च्युनिटी, संधी) आणि ‘टी’ (थ्रेटस, धोके). 

उदाहरणार्थ एखाद्या विद्यार्थ्याचे “स्वॉट विश्लेषण” असे असू शकते की त्याचे ‘एस’ म्हणजे शास्त्र व गणित या विषयात चांगली गती असणे, त्याचे ‘डब्ल्यू’ म्हणजे संभाषण कौशल्याचा अभाव, ‘ओ’ म्हणजे विविध अभियांत्रिकी व तंत्रज्ञानाच्या संशोधन क्षेत्रात असलेल्या अमाप संधी, त्याचप्रमाणे ‘टी’ म्हणजे, त्याच संधींसाठी असलेली प्रचंड स्पर्धा. असे विश्लेषण वैयक्तिक प्रमाणेच, संस्थात्मक किंवा व्यावसायिक स्तरावरही करता येते. अजून एक उदाहरण पाहुयात. नवीन मोबाईल फोन बाजारात आणण्याआधी त्याचे केलेले स्वॉट विश्लेषण असे असू शकते  ‘एस’ (स्ट्रेंग्थ, सामर्थ्य), म्हणजे वाढवलेली संगणकीय क्षमता आणि कृत्रिम बुद्धिमत्तेचा (एआय) अंतर्भाव.  ‘डब्ल्यू’ (विकनेस, कमकुवत बाजू) म्हणजे वाढलेली किंमत आणि बॅटरीचा खप. ‘ओ’ (ओप्पोर्च्युनिटी, संधी)  म्हणजे नवीन तंत्रज्ञान आणि सुविधांसाठी वाढती मागणी, त्याचप्रमाणे  ‘टी’ (थ्रेटस, धोके) म्हणजे  वेगवान तांत्रिक बदल, सायबर सुरक्षेचे धोके आणि मोठ्या ब्रँड्सकडून तीव्र स्पर्धा. या पद्धतीने विचार केला की सर्व महत्त्वाच्या बाबींचा विचार झाला आहे याची खात्री होते.

मेंटल मॉडेल्सचा प्रभावी वापर
अशाप्रकारची अनेक विचार चित्रे आपण उपयोगात आणू शकतो. ती अतिशय प्रभावी असली तरी प्रत्येक समस्येला रामबाण उपाय म्हणून सरसकट वापरता येत नाही. कोठे कोणते ‘मेंटल-मॉडेल’ (मन:प्रारूप) वापरायचे याचे तारतम्य बाळगावे लागते. अनुभवाने तेही जमायला लागते. कालांतरानी या तयार मेंटल मॉडेल्सबरोबरच आपण स्वनिर्मित मॉडेल्स ची पण भर घालू शकतो. चार्ली मंगर यांच्या म्हणण्यानुसार त्यांच्याकडे १०० हून अधिक मेंटल मॉडेल्स होती, आणि ती विविध क्षेत्रांतून घेतलेली होती. इतक्या मोठ्या संख्येने मन:प्रारूपे शिकणे अवघड असले, तरी सुरुवात करणे कठीण नाही! तर मग, तुम्ही मन:प्रारूपे समजावून घेण्यास व वापरण्यास सुरुवात करताय का? बघा, काही फायदा होतोय का!


\chapter{‘आद्य तत्वविचारा’चे महत्व}

इलेक्ट्रिक गाड्या जरी सध्या लोकप्रिय झाल्या असल्या, तरी त्यांचा उगम अलीकडचा नाही. अनेक दशकांपूर्वीच त्यांची निर्मिती झाली होती, पण त्या ग्राहकांपर्यंत पोहोचण्यात अयशस्वी ठरल्या. आधुनिक काळातही मोठ्या कंपन्या इलेक्ट्रिक गाड्यांकडे फारशा वळल्या नाहीत. मग एक अवलिया आला आणि कोठलीही पार्श्वभूमी अथवा अनुभव नसताना त्याने या क्षेत्रात उडी मारून पूर्ण चित्रच बदलवून टाकले. 
त्याने स्वतःला विचारले, "प्रदूषण कमी करणे हे प्रमुख उद्दिष्ट असूनही पेट्रोल-डिझेलऐवजी इलेक्ट्रिक गाड्या का नाहीत?" संशोधनाअंती समजले की बॅटरीच्या अवास्तव किमतीमुळे मोठ्या कंपन्या या गाड्या तयार करत नाहीत. मग पुढचा प्रश्न :  "बॅटरी इतकी महाग का आहे?" बॅटरी बनवण्यासाठी लागणारे साहित्य कोणते? उत्तर मिळाले : लिथियम, निकेल, कोबाल्ट, ग्रॅफाइट यांसारखे काही दुर्लभ धातू. पुढील प्रश्न : "या कच्च्या मालाची एकत्रित किंमत किती?" उत्तर मिळाले : तयार बॅटरीच्या किंमतीपेक्षा खूपच कमी! येथे आशेचा किरण सापडला. मग विचार आला, "निर्मिती प्रक्रिया अधिक स्वस्त करता येईल का?", "डिझाइन बदलता येईल का?", "मोठ्या प्रमाणावर उत्पादन केल्यास किंमत आणखी कमी होईल का?" अशा सततच्या मूलभूत प्रश्नांमधून उत्तर शोधत त्याने बॅटरी निर्मितीत क्रांती घडवली. जे भल्याभल्यांना जमले नाही ते एका नवख्या माणसाने केले आणि प्रस्थापितांना मागे टाकले. त्या माणसाचे नाव इलॉन मस्क आणि त्या गाडीचे नाव टेस्ला.
अशाप्रकारे  प्रश्न विचारत, मूलभूत तत्वांपर्यंत पोहोचून उत्तर शोधत ‘फर्स्ट प्रिन्सिपल्स थिंकिंग’, म्हणजेच ‘आद्य तत्व विचार’ या विचारपद्धतीचा जन्म होतो. हे एक प्रभावी मेंटल मॉडेल (मनःप्रारूप)आहे. 
‘आद्य तत्व विचार’ हे मेंटल मॉडेल कसे कार्य करते?
कोणत्याही समस्येचे बारकाईने निरीक्षण करणे, प्रश्न विचारत तुकडे करत राहणे, जोपर्यंत आपण मूलभूत प्रश्नांपर्यंत पोहोचत नाही तोपर्यन्त. मग त्या मूळ प्रश्नाचे निराकरण करायचे आणि मग उलटे परत मूळ समस्येपर्यंत येता येता संपूर्ण उपाय निर्मिती करायची, ही ‘आद्य तत्व विचार’ या विचारचित्राची पद्धत-प्रक्रिया आहे. कोठलाही पूर्वग्रह (आणि अनुभव) नसेल आणि "हे असंच चालत आलंय" या संकल्पना दूर ठेवल्यास  ही प्रक्रिया जास्त प्रभावी ठरते. 

इलॉन मस्कने ही पद्धत स्पेसएक्स (अंतराळ संशोधन) आणि न्यूरालिंक (मेंदू-संगणक संवाद तंत्रज्ञान) यांसारख्या संकल्पनांमध्येही यशस्वीपणे वापरली. त्यामुळे त्याला आधुनिक 'आद्य तत्व विचार' चा प्रवर्तक म्हणता येईल. हे विचारचित्र फक्त तंत्रज्ञान क्षेत्रातच वापरता येते असे बिलकुल नाही. आपण काही इतर विविध क्षेत्रातील उदाहरणे बघुयात. 
गुंतवणूक आणि घर खरेदी
लोकांना वाटते की घर विकत घेणे ही चांगली गुंतवणूक आहे. पण खरंच आहे का? 'आद्य तत्व विचार' वापरून विचारले पाहिजे: "ही गुंतवणूक भावनिक आहे की परताव्यासाठी?" "भाड्याने दिल्यास आणि किंमत वाढल्यास किती परतावा मिळेल?" "घराची किंमत एवढी का आहे? त्यातील कच्च्या मालाचा, जमिनीचा, आणि स्थानमहात्म्याचा किती वाटा आहे?" या मूलभूत प्रश्नांमधून अनेक वेळा लक्षात येते की वेगवेगळ्या ठिकाणी किंमती फुगवल्या जातात. “त्याचे निराकरण कसे होऊ शकेल?” अशापद्धतीने केलेला विचार शहाणपणाने गुंतवणूक करण्यास मदत करू शकतो.

आरोग्य आणि वजन नियंत्रण
समजा तुम्हाला वजन कमी करायचे आहे. मग विचारायचे— "वजन वाढते कशाने?" उत्तर: खाल्लेल्या उष्मांकांची (कॅलरीज) मात्रा खर्च केलेल्या उष्मांकांपेक्षा जास्त असल्याने. मग पुढचे प्रश्न: "खाण्यात बदल करायचा की उष्मांक खर्च करण्याच्या पद्धती बदलायच्या?" खाण्यावर लक्ष द्यायचे ठरवल्यावर, “खाण्यात उष्मांक कोणापासून जास्त मिळतात?”, "कोणत्या गोष्टी वजन वाढवतात— तळलेले, गोड पदार्थ का इतर काही?" “मग दिवसातून चारवेळेला भरपेट खाण्याऐवजी दोनदाच मोजका पण समतोल आहार घ्यायचा?”, "पोषण आणि उष्मांक संतुलित कसे ठेवायचे?" ही विचारसरणी तज्ज्ञांच्या मार्गदर्शनासह वापरल्यास वजन संतुलित राखण्यासाठी एक शास्त्रशुद्ध आणि शाश्वत योजना तयार करता येते.
अंतिम विचार
'आद्य तत्व विचार' ही संकल्पना सर्वव्यापी आहे. तंत्रज्ञान, व्यवसाय, नोकरी, कला, संशोधन, खेळ, राजकारण अशा अनेक क्षेत्रांत याचा उपयोग करून समस्यांचे व्यावहारिक आणि नाविन्यपूर्ण उपाय शोधता येतात. ही विचारपद्धती आत्मसात केल्यास नवे दृष्टिकोन, नावीन्यपूर्ण शोध, आणि यशस्वी निर्णय घेणे सोपे होते. तुम्ही कधी वापरली आहे का ही पद्धत?



\chapter{उलटा विचार, यशाचा आधार}

एखादा माणूस सतत नकारात्मकतेने बघत असेल, तर ते चांगले मानले जात नाही. पण जर हा दृष्टिकोन समजून आणि योग्य प्रकारे वापरता आला, तर तो खूप फायद्याचा ठरू शकतो. हाच विचार एका प्रसिद्ध मेंटल मॉडेलचा (मन:प्रारूप) गाभा आहे. त्याला ‘इन्व्हर्जन’ किंवा सोप्या भाषेत 'उलटा विचार' म्हणता येईल.  समजा, तुम्हाला तब्येत सुधारायची आहे. तुम्ही सकस आहार घ्याल, व्यायाम कराल, हे स्वाभाविक आहे. पण जर उलटा विचार केला, तर तुम्ही पाहाल की तब्येत बिघडवणाऱ्या सवयी कोणत्या आहेत आणि त्या टाळता येतात का? जंक फूड कमी करता येईल का? व्यसने सोडता येतील का? दिवसभर कॉम्प्युटर किंवा मोबाईलसमोर बसण्याची सवय मोडता येईल का? केवळ या नकारात्मक गोष्टी बंद केल्या, तरी तुमच्या आरोग्यावर सकारात्मक परिणाम होईल. हाच ‘उलट्या विचारा’चा गाभा आहे. समस्येकडे वेगळ्या (उलट्या) दृष्टिकोनातून पाहण्याची संधी देते. चुकांकडे लक्ष देऊन त्या टाळल्या, तर यशस्वी होण्याची शक्यता वाढते.  भारतीय तत्वज्ञानातही याची बीजे सापडतात. सत्याचा शोध घेताना नेति-नेति (हेही नाही, तेही नाही) ह्या पद्धतीने काय टाळायचे ते आधी ठरवले, की अंतिम सत्याचा-वास्तवतेचा मार्ग स्पष्ट होतो.  चार्ली मंगर एकदा म्हणाले होते, "मला फक्त हेच माहित करायचं आहे की मी कुठे मरणार आहे, म्हणजे मी तिथे जाणार नाही." हा विनोदी वाटणारा; पण खोल अर्थ असलेला विचार उलट्या विचारसरणीचे प्रमुख तत्व स्पष्ट करते. अपयश टाळणे, अनेकदा यश मिळवण्याइतकेच महत्त्वाचे असते.  
गुंतवणूक: हुशारीपेक्षा मूर्खपणा टाळा
गुंतवणुकीत यशस्वी होण्यासाठी गहन विश्लेषण आणि गुंतागुंतीच्या तंत्रांचा आधार न घेता, प्रथम मूर्खपणा टाळा. "कुठले शेअर्स मला श्रीमंत करतील?" असा विचार करण्याऐवजी, "कुठल्या चुका माझी संपूर्ण संपत्ती नष्ट करू शकतात?" असा प्रश्न विचारला पाहिजे. अनावश्यक कर्ज घेऊन गुंतवणूक करू नका. सर्वसाधारण बँका देतात त्यापेक्षा एखादी योजना दुप्पट-तिप्पट परतावा देण्याचे वचन देत असेल तर ते कसे खरे मानायचे? अतिशय ‘चमको’ योजनांवर आंधळा विश्वास ठेवू नका. कळपाच्या (मेंढरांच्या) मानसिकतेने व्यवहार करणे, भावनांवर आधारित निर्णय घेणे, आणि अत्याधिक जोखीम पत्करणे यांसारख्या चुका टाळल्या, तरी दीर्घकालीन संपत्ती निर्माण होऊ शकते.  मंगर म्हणतात, "खूप हुशार होण्याचा प्रयत्न करण्यापेक्षा, फक्त सातत्याने मूर्खपणा टाळल्यानेच आम्हाला किती मोठा दीर्घकालीन फायदा झाला आहे हे पाहून आश्चर्य वाटते."
व्यवसाय: चुका टाळा  
एखादी कंपनी यशस्वी होण्यासाठी काय करावे यावर लक्ष केंद्रित करण्याऐवजी, भूतकाळातील अपयशांचे विश्लेषण करून काय करू नये हे समजून घेतले पाहिजे. कोडॅकने, नोकियाने स्मार्टफोनच्या क्रांतीचा अचूक अंदाज लावला नाही आणि आपले वर्चस्व गमावले. "काय आपल्याला बुडवू शकते?" असा प्रश्न विचारल्याने कंपन्या संभाव्य आपत्ती आधीच ओळखून टाळू शकतात. भरमसाठ उत्पादनांच्या मागे न लागता, त्यातील न चालणारी, अद्वितीय नसणारी उत्पादने बंद केलेलीच बारी, नाही का?
निवडीतील गोंधळ कमी करणे  
आजच्या जगात पर्यायांची भरमार आहे. शिक्षण-करियरमध्ये, नोकरी-व्यवसायात, ते अगदी जोडीदार निवडीमध्येही अनेक पर्याय असतात. गोंधळायला होऊ शकते. फारच अचूक-तंतोतंत शोधत बसलो तर काही मिळत नाही आणि काहीही ठरवले नाही तर अतिसामान्य गोष्ट मिळण्याची शक्यता वाढते. याला उपाय म्हणून एक मध्यम-मार्ग म्हणजे, मला ‘काय नको’ हे आधी पक्के ठरवणे, मग उरलेल्या गोष्टींमध्ये निवड करणे. या पद्धतीत निवडीस चांगला वावही राहतो व निराशाही पदरी पडत नाही. ‘उलट विचारांचा’ हा ही एक प्रकार. 
यशासाठी वजाबाकीचा दृष्टिकोन  
आजच्या जगात सतत काहीतरी मिळवण्याची चढाओढ आहे. नवीन कौशल्ये, नव्या रणनीती, अधिक युक्त्या! पण उलट विचारसरणी आपल्याला वजाबाकीचे महत्त्व शिकवते. अनावश्यक गोष्टी ओळखून त्या दूर केल्या, की यशाचा मार्ग आपोआप मोकळा होतो.मग ज्या गोष्टी खरोखरच महत्वाच्या आहेत त्यांच्यावर लक्ष केंद्रित करणे शक्य होते. आजच प्रयत्न करून पहा: "मला अधिक चांगले होण्यासाठी काय करायला हवे?" असा विचार करण्याऐवजी, "मी काय करणे थांबवायला हवे?" असा प्रश्न विचारून पहा. उत्तरं कदाचित आश्चर्यकारक वाटतील, पण ती नक्कीच तुम्हाला योग्य मार्गावर आणतील.



\chapter{गत-संधीची किंमत}

भारतात दरवर्षी लाखो तरुण-तरुणी यूपीएससी, एमपीएससी सारख्या स्पर्धापरीक्षांची तयारी करताना दिसतात. काही जण यासाठी पाच-सहा वर्षांचं आयुष्य खर्च करतात. मात्र या परीक्षा अत्यंत स्पर्धात्मक असून, अंतिम यशस्वी होणाऱ्यांची संख्या अत्यल्प असते. देशसेवा, स्थैर्य आणि प्रतिष्ठा हे आकर्षण मान्य असले तरी, बहुतांश जणांच्या पदरी फक्त थकवा, निराशा आणि वाया गेलेली वर्षं येतात. मग याचा विचार का होत नाही? या गेलेल्या वर्षांची भरपाई कशी होणार? याच काळात दुसरे काही ठीकठाक (खूप लाभाचे नसले तरी) करता आले नसते का? खरेतर अशा ‘अभ्यासू’ तरुणांना विविध क्षेत्रात जाता येऊ शकते. विधी-कायदे, संगणक, वित्त इत्यादी क्षेत्रात खूपच गरज आहे. परदेशी भाषा शिकून कौशल्याधारित कामे भारताबाहेर मिळू शकतात.  सध्या जोमात असलेल्या स्टार्टअप्समध्ये करिअर घडवण्याची संधी आणि शिकवण्याची आवड असेल तर त्याही पेशात कितीतरी संधी आहेत.  एक ना अनेक. पण सर्व लक्ष एका परीक्षेकडे लागल्यामुळे उर्वरित सर्व दारं नकळत बंद होतात. मग या ‘गमावलेल्या’ संधींचा का विचार होत नाही? याच प्रश्नावर आधारित ‘मेंटल मॉडेल’ला (मन: प्रारूप)‘ऑपॉर्च्युनिटी कॉस्ट’ म्हणजेच ‘गत संधीची किंमत’ असे म्हणू शकतो. 

आपण आयुष्यात दररोज असंख्य निर्णय घेतो. काही मोठे, काही छोटे. पण प्रत्येक निर्णयामागे एक वा अनेक गत-संधींच्या किंमती दडलेल्या असतात, म्हणजेच आपण निवडलेली गोष्ट न करता मिळू शकणाऱ्या पर्यायाचं मूल्य. ही संकल्पना अर्थशास्त्रातून आली असली तरी तिचं महत्त्व केवळ आर्थिक व्यवहारांपुरतं मर्यादित नाही, ती आपल्या आयुष्याच्या प्रत्येक क्षेत्रात लागू होते. याची काही उदाहरणे पाहुयात.

कल्पना करा, तुम्ही रविवारची सुट्टी एखाद्या मोठ्या मॉलमध्ये खरेदी करत, खाण्या-पिण्यात आणि खरेदी (विंडो-शॉपिंग!!) करत घालवली. तुमचे पाचेक हजार खर्च आणि ६-७ तास सहज गेले. थोडी मजा झाली, पण जरा विचार करा, याच दिवशी जवळच्या उद्योजकतेवर आधारित कार्यशाळेला हजेरी लावली असती, तर? एखाद्या यशस्वी स्टार्टअप संस्थापकाकडून शिकायला मिळालं असतं. तिथे ओळखी वाढल्या असत्या. कदाचित एखादी कल्पना सुचली असती जी भविष्यात व्यवसायात रूपांतरित होऊ शकली असती. मॉलमध्ये गेलेल्या दिवसाचं बिल तुमच्या खिशातून गेलं: ₹५,000. पण कार्यशाळेला न गेल्याचं बिल कोणीच दाखवत नाही. हेच ते अदृश्य नुकसान म्हणजेच ‘गत संधीची किंमत’. म्हणजेच  एखादी गोष्ट निवडताना, आपण ज्या दुसऱ्या गोष्टी सोडून देतो, त्यांचे मूल्य.

एका चांगल्या पगाराच्या नोकरीसाठी होकार देताना, तुम्ही तुमचा स्टार्टअप सुरू करण्याची संधी सोडत आहात का? नजीकच्या भविष्यात नोकरी जास्त पैसे देईल, स्थिरता देईल पण गमावलेली स्टार्टअपची संधी काहीतरी जगावेगळं-असामान्य करण्याची शक्यता तर गमावत नाहीये ना?

रोज तासंतास सोशल मीडियावर घालवताना, कदाचित एक नवीन कौशल्य शिकण्याची संधी हातातून जात आहे का? त्यामुळे पुस्तकं वाचणे मागे पडत आहे का? कित्येक दिवसात सहकुटुंब पत्ते खेळले नाहीयेत का? 

घर खरेदी करणं हे अनेकांचं स्वप्न असतं, पण त्यासाठी घेतलेलं कर्ज आणि त्यात गुंतवलेलं भांडवल दुसऱ्या गुंतवणुकीत वापरलं असतं तर अधिक चांगला परतावा मिळाला असता का? भाड्याने राहणं काहींना खर्चिक वाटतं, पण यामुळे लवचिकता, आर्थिक तरलता आणि इतर संधी मिळू शकतात. दिसतं ते घर, पण दिसत नाही ती ‘गत -संधीची किंमत’.

ऑफिसमधल्या प्रत्येक मीटिंगला ‘हो’ म्हणणं म्हणजे तुमचं लक्ष, विचारशक्ती आणि मोकळा वेळ गमावणं, नाही का? हा वेळ सखोल-सर्जनशील काम किंवा महत्त्वाच्या निर्णयांसाठी वापरता आला असता. व्यत्ययांमुळे होणारं खरं नुकसान वेळेचं नसतं तर तुम्ही गमावलेल्या संधींचं असतं.

आपण सहसा जे मिळतं त्यावर लक्ष देतो, जे गमावतो त्यावर नाही. हे आपल्या विचारसरणीचं अपंगत्व आहे. ‘गत संधीची किंमत’ समजून घेणं म्हणजे केवळ शहाणपण नव्हे तर ही एक मानसिक शक्ती आहे. तीच आपल्याला सरळ वाटचाल करणाऱ्यांपासून वेगळं करते. विचारपूर्वक जगणाऱ्यांकडे नेते. तुमच्या प्रत्येक ‘होकारा’त अनेक ‘नकार’ दडलेले असतात याचं भान येणं महत्वाचं आहे. पुढच्या वेळी जेव्हा तुम्ही कुठलाही निर्णय घ्याल, तेव्हा फक्त "हे का?" इतकंच नाही, तर "यामुळे मी काय गमावत आहे?" हाही प्रश्न विचारायला विसरू नका.


\chapter{दृष्टी पलीकडील सृष्टी}

ब्रिटिश राजवटीतील एक गोष्ट सांगतात. एका गावात सापाचं प्रमाण फार वाढलेलं होतं. या समस्येवर तोडगा म्हणून अधिकाऱ्यांनी एक उपाय-योजना केली. गावात दवंडी पिटवली की जे लोक साप पकडून कार्यालयात आणतील त्यांना इनाम देण्यात येईल. जेवढे जास्त साप, तेवढी अधिक इनामाची रक्कम. दिवसेंदिवस मोठ्या संख्येने मृत साप जमा होऊ लागले. काही दिवस उलटून गेल्यावरही सापांचा त्रास काही कमी झाला नाही. अधिकाऱ्यांना शंका आली. तपास केल्यावर असे सापडले की लोक साप पाळून त्यांची प्रजनन करून पैसे कमवायला लागले आहेत. हे लक्षात येताच त्यांनी इनाम योजना थांबवली. पैसे मिळवण्याचा मार्ग बंद झाल्यावर लोकांनी साप मोकळ्यात सोडले. पूर्वीची समस्या आता अधिक गंभीर झाली. ज्या उद्देशाने योजना सुरू केली होती, त्याचे फायदे केवळ थोड्या काळापुरते दिसले. मात्र लोकांनी त्याचा गैरफायदा घेतल्याने मूळ उद्देशालाच हरताळ फासला.
 म्हणून कोणताही विचार करताना, निर्णय घेताना नजीकच्या परिणामांचाच फक्त विचार न करता त्याचा संभाव्य व दूरगामी परिणामांचा विचार करणे या मेंटल मॉडेलला(मन:प्रारूप)‘सेकंड ऑर्डर थिंकिंग’  अथवा ‘दूरगामी परिणामांचा विचार’ म्हणू शकतो.  यापद्धतीत एका मागून एक प्रश्न विचारत राहायचे की ‘मग काय होईल?’, ‘मग काय होईल? असा एकामागून एक मागोवा घेत विचार केला जातो. सर्वसाधारणपणे, आपण एकाच प्रश्नाचे उत्तर शोधून कामाला लागतो. याला 'फर्स्ट ऑर्डर थिंकिंग' म्हणजे तात्कालिक विचार म्हणता येईल. परंतु 'सेकंड ऑर्डर थिंकिंग' पद्धतीने विचार करतांना, त्या उत्तरानंतर होणाऱ्या परिणामांची शृंखलाही तपासली जाते.
उदाहरणार्थ, बुद्धिबळात जसे आपण फक्त लगेचच्या चालीचाच फक्त विचार नाही तर त्याने पुढे काय काय होऊ शकते त्यावरही भविष्यातील चालींचा आराखडा बांधत असतो. 'परिणामांचे परिणाम' जोखणे, हे मूलतत्त्व आहे. चार्ली मंगर यांनी म्हटले होते की, "अशा पद्धतीने विचार करणे बिलकुल सोपे नाही, आणि ज्याला ते सोपे वाटते, तो मूर्ख समजावा." गमतीचा भाग सोडला तर त्यांच्या म्हणण्याचा उद्देश असा होता की उथळ आणि झापडे लावून केलेला विचार हा एक सापळा आहे आणि तो महागात पडू शकतो. म्हणूनच हे महत्वाचे तत्व काही उदाहरणांच्या आधारे पाहुयात. 
वजन कमी करण्यासाठी अनेकजण झटपट कॅलरीज कमी करण्याचा मार्ग अवलंबतात, कारण त्यांना त्वरित परिणाम हवा असतो. पण हा तात्कालिक विचार दीर्घकाळात उलट परिणाम करतो. लवकर वजन कमी झाल्याने शरीराची पचन-चयापचय क्रिया मंदावते, थकवा जाणवतो, आणि हार्मोनल असंतुलन होऊ शकते. परिणामी, काही काळाने वजन पुन्हा वाढू लागते. शॉर्टकट्सपेक्षा शाश्वत आणि संतुलित सवयींचा स्वीकार करणे अधिक फायदेशीर ठरते.
नफ्यामध्ये जोरदार आणि त्वरित वाढ करण्यासाठी अनेक कंपन्या कर्मचार्‍यांना तडकाफडकी कमी करण्याचा निर्णय घेतात. हे प्रथमदर्शनी फायदेशीर वाटत असले, तरी हा तात्कालिक विचार अनेक दीर्घकालीन समस्यांना आमंत्रण देतो. कर्मचार्‍यांची संख्या कमी केल्यावर मनोबल घटते, नवसंशोधन मंदावते, आणि चांगली कामगिरी करणारे कर्मचारी सुद्धा कालांतराने संधी शोधत निघून जातात. या सर्व गोष्टींचा एकत्रित परिणाम म्हणजे कंपनीची कार्यक्षमता कमी होऊ शकते, आणि दीर्घकालीन नफा वाढवण्याचे उद्दिष्ट साध्य होऊ शकत नाही.
दैनंदिन धावपळीत पालक अनेकदा मुलांना शांत ठेवण्यासाठी त्यांच्याकडे स्मार्टफोन देतात, जेणेकरून ते व्यस्त राहतील आणि त्रास देणार नाहीत. हा उपाय तात्काळ शांतता देतो, पण यामुळे मुलांमध्ये मोबाईल-व्यसनाची सुरुवात होते. लहान वयात स्क्रीनची  सवय लागल्यामुळे लक्ष केंद्रित करण्याची क्षमता कमी होऊ शकते. मुले एककल्ली आणि घरकोंबडी होऊ शकतात. आजची शांतता उद्याचे बालपण समस्याग्रस्त बनवू शकते.
वरील उदाहरणांवरून वाटेल की ‘दूरगामी परिणामांचा विचार’ म्हणजे फक्त नकारात्मक असतो,पण तसे नक्कीच नाही. अनेकवेळा निर्णयांचे सकारात्मक परिणामही पुढच्या टप्प्यावर उघड होतात. उदाहरणार्थ, मुलींना शिकवणे हे प्रथमदर्शनी पाहता त्यांना स्वयंपूर्ण बनवण्याचा एक मार्ग आहे असे वाटते आणि ते खरेच आहे. शिक्षणामुळे त्या आत्मनिर्भर बनतात, त्यांना आर्थिक स्थैर्य लाभते, आणि त्यांची आत्मसन्मानाची भावना वृद्धिंगत होते. पण यापलीकडे, या निर्णयाचे दुसऱ्या स्तरावरील परिणाम अधिक खोलवर आणि समाजहिताचे असतात. शिक्षित आणि स्वतः कमावणाऱ्या स्त्रिया सहसा लहान कुटुंब ठेवण्याची निवड करतात, त्यामुळे कुटुंब नियोजन सुलभ होते आणि सामाजिक समतोल राखला जातो. स्त्रियांना ‘उंबरठा’ ओलांडू न देणाऱ्या समूहांमध्ये कुटुंबे मोठी दिसतात. म्हणूनच या समस्येला उपाय म्हणून आलेल्या ‘बेटी पढाओ’ सारख्या योजनांमध्ये ‘दूरगामी परिणामांचा विचार’ दिसून येतो. केवळ वैयक्तिक नव्हे तर सामूहिक प्रगतीसाठीही हा विचार महत्त्वाचा ठरतो.
आजच्या धावपळीच्या जगात, ‘सेकंड ऑर्डर थिंकिंग’ अर्थात ‘दूरगामी परिणामांचा विचार’ आपल्याला थांबून चिंतन करण्याची आठवण करून देतो. लगेच सुचलेले उत्तर तेवढेच परिपूर्ण असू शकते असे नाही. त्यामुळे कोणताही निर्णय घेताना थोडा वेळ घ्या, स्वतःला विचारा: 'पुढे काय होईल?' आणि त्यानंतर काय होईल? ही सवय अनपेक्षित परिणामांपासून वाचवून शहाणपणाचे आणि प्रभावी निर्णय घेण्यास मदत करू शकते. शेवटी, ‘सेकंड ऑर्डर थिंकिंग’ ही एक पद्धत नसून, एक शांत पण प्रभावी दृष्टिकोन आहे,  गोंगाटातली एक निःशब्द ताकद.


\chapter{नकाशा म्हणजेच भूभाग नव्हे}

पाश्चिमात्य देशांच्या नजरेत भारताची प्रतिमा बदलत आहे. ‘सापांचे खेळ करणाऱ्यांचा देश’ हे जुनं चित्र आता मागे पडलं असलं, तरी काही चित्रपट व बातम्यांमुळे भारतात सर्वत्र गरीबी व अस्वच्छता असल्याचं दाखवलं जातं. ‘जागतिक भूक/उपासमारी निर्देशांक’ (ग्लोबल हंगर इंडेक्स) मध्ये भारत आजही पाकिस्तान, बांगलादेश, नेपाळ, श्रीलंकेच्या मागे (दाखवलेला) आहे.

हे सर्व पाहून एखादा परदेशी व्यावसायिक भारतात येतो आणि पाहतो तर काय, चित्र तेवढे काही वाईट नाही. इतकं कमी प्रति-व्यक्ती उत्पन्न असलेला देश जगातील सर्वात मोठ्या स्मार्टफोन बाजारांपैकी एक कसा असू शकतो? अतिप्रगत देशांमध्ये सुद्धा कागदावर मतदान होत असल्याने मोजणीस महिनोंमहिने लागतात तर भारतात काही दिवसात निकाल जाहीर पण होतो. यूपीआय व्यवहार करणारा फेरीवाला, नेटफ्लिक्सवरील शो-वर चर्चा करणारा रिक्षाचालक किंवा मोबाईलवर हवामानाची माहिती पाहणारा शेतकरी, अशी एक ना अनेक उदाहरणे दिसतात. त्यामुळे जगापुढे आलेल्या-ठेवलेल्या प्रतिमेत काही अंशी सत्य असले तरी ते पूर्ण वास्तव नक्कीच  नाही. याच विसंगतीच्या मेंटल मॉडेलला (मन:प्रारूप)‘मॅप इज नॉट द टेरिटरी’ म्हणजेच ‘नकाशा-भूभाग-फारकत’ किंवा सोप्या भाषेत ‘नकाशा म्हणजेच भूभाग नाही’ असे नामकरण करू शकतो. पोलिश-अमेरिकन वैज्ञानिक आल्फ्रेड कोर्जिब्स्की यांनी मांडलेली ही संकल्पना चार्ली मंगर आणि शेन पॅरिशसारख्या विचारवंतांनी जनमानसात पोचवली. हे विचारचित्र आपल्याला सतत आठवण करून देते की, नकाशा म्हणजे वास्तव नाही. आकडेवारी, चार्ट्स, सोशल मीडिया पोस्ट्स किंवा एखादी कथा (आख्यान, नॅरेटिव्ह) हे वास्तवाचं अपूर्ण चित्र असतं. ते उपयोगी असलं तरीही पूर्ण सत्य मानणं धोकादायक ठरू शकतं.

आपल्या मेंदूला सतत व खूप बारीक विचार करायला ऊर्जा खर्च करायची नसते. मग तो क्लुप्त्या (शॉर्टकट्स) वापरतो. समजण्यासाठी आपण गोष्टी साध्या-सोप्या करतो. उदाहरणार्थ, नकाशा रस्त्यांची व गावांची माहिती जरी ढोबळ मानाने देत असला तरी प्रवास ठरवायला उपयुक्त ठरतो. पण प्रत्यक्षात जमिनींवर चित्र थोडे वेगळे असू शकते, खोद-कामांमुळे मार्ग बदललेले असू शकतात, नकाशात दाखवलं हॉटेल तेथे नसू शकतं. पण त्यामुळे खूप काही अडत नाही. पण जेव्हा आपण नकाशालाच वास्तव समजतो, तेव्हा आपण अशा स्थितीत निर्णय घेतो जे प्रत्यक्षात अस्तित्वात असेल असे नाही, याचे भान मात्र ठेवावे लागते. अजून स्पष्टतेसाठी ‘नकाशा-भूभाग-फारकत’ या विचार-चित्राची अजून काही उदाहरणे पाहुयात. 

एका छोट्या शहरातून आलेल्या विद्यार्थ्याच्या रेझ्युमे (बायोडेटा) वर केवळ तो आय-आय-टी किंवा आय-आय-एम लिहिलेले नाही म्हणून नोकरीसाठी-संधीसाठी नाकारला जातो. येथे ‘नकाशा’ म्हणजेच तो ‘बायोडेटा’ व त्यात प्रतीत होणारी शिक्षणसंस्थेची जनमानसातील प्रतिमा, आणि ‘भूभाग’ म्हणजेच त्या विद्यार्थ्यांचे प्रत्यक्ष कौशल्य व क्षमता. आपणास माहिती आहे की भारतातील सर्व यशस्वी उद्योजकांनी फक्त याच नामांकित संस्थांमधून शिक्षण घेतलेलं नाहीये, आणि तरीही ते यशाच्या शिखरावर पोहोचलेले आहेत. 
एखाद्याचं इंस्टाग्राम पाहा, सुट्ट्यांचे मनमोहक फोटो, सतत प्रेरणादायक व सकारात्मक लेखन, प्रमाणबद्ध-फिट शरीरं. असं वाटतं की त्यांचं आयुष्य परिपूर्ण आहे. पण या पोस्ट्स किंवा बराचसा सोशल मिडिया हा एक ‘नकाशा’ आहे. तो नीटनेटका आणि सजवलेला असतो; पण प्रत्यक्ष 'भूभाग' गुंतागुंतीचा, आणि पूर्णतः अस्ताव्यस्त असू शकतो.

बरं वाटत नसताना सुद्धा  एखाद्याचे काही वैद्यकीय चाचण्यांचे रिपोर्ट्स सर्वसामान्य मर्यादेत असू शकतात. म्हणूनच डॉक्टरचा सल्ला घेणे आवश्यक असते. तो थकवा, निरुत्साह आणि अस्वस्थता वेगळ्या कारणांनी असू शकतो. चाचण्या म्हणजे ‘नकाशा’, उपयोगी, पण कधी कधी अपूर्ण. मानसिक आरोग्य, आहार, झोप आणि योग्य अनुभवसिद्ध वैद्यकीय निदान हा खरा ‘भूभाग’ जो  अनेकदा या "नॉर्मल" आकड्यांच्या पलीकडे असू शकतो.

गृहसंकुलाच्या जाहिरातीतील सोयीसुविधा, दाखवलेले रहिवासी कसे लक्षवेधक असतात. किती इमारती आहेत व किती मजले आहेत या सारख्या गोष्टी जरी प्रत्यक्षातही खऱ्या दिसत असल्यातरी इतर काही गोष्टींमधे मात्र मोठी तफावत असू शकते. म्हणजेच ‘जाहिरात’ हे काही प्रत्यक्ष ‘उत्पादन’ नव्हे. 

भारतासारख्या देशात, जिथे अनेक विरोधाभास एकत्र नांदतात, तिथे आकडेवारी अनेकदा वास्तवापेक्षा वेगळी असते. येथील कोट्यवधी लोक रिपोर्ट्सच्या ओळींमध्ये नव्हे, तर त्याच्या मधल्या जागेत जगत असतात, तिथे ‘नकाशा-भूभाग-फारकत’ हे विचारचित्र अधिक महत्त्वाचे ठरते.

जेव्हा आपण नकाशाला भूभाग समजणं थांबवतो, तेव्हा आपण अधिक चांगले प्रश्न विचारतो, लक्ष देऊन ऐकतो, आणि निर्णय घेण्याआधी जरा थांबून विचार करतो. आकड्यांवर सर्वस्वी अवलंबून ('डेटा-ड्रिव्हन') असणं म्हणजे कायमच ‘सत्य-ड्रिव्हन' असणं असे नव्हे. नकाशे उपयोगी असतात, पण ते ब्रह्मवाक्य-पवित्र नसतात. ते मार्गदर्शक असावेत, अंधविश्वासू करणारे नाही.



\chapter{गड्या, आपुला गाव बरा}

हिंदी चित्रपटसृष्टीचे महानायक अमिताभ बच्चन यांनी सत्तर-ऐंशीच्या दशकात एकामागून एक यशस्वी चित्रपट दिले. ते प्रसिद्धीच्या शिखरावर होते. १९८३ च्या ‘कुली ‘ चित्रपटातील दुर्दवी अपघातातून सावरून आल्यानंतर तर ते जनमानसाच्या गळ्यातील ताईत बनले होते. कदाचित याच लोकप्रियतेचा फायदा व्हावा म्हणून (किंवा खरोखरच्या जनसेवेच्या ओढीने) त्यांनी राजकारणात पाऊल टाकले. लोकसभेच्या निवडणुकीत अलाहाबाद (आताचे प्रयागराज) येथून त्यांनी माजी मुख्यमंत्री बहुगुणांचा मोठा पराभव केला. पण तीन-एक वर्षातच, झालेल्या आरोपांमुळे, त्यांनी वैतागून, नाराज होऊन, राजकारणाला ‘दलदलीसारखं’ म्हणत या क्षेत्राचा निरोप घेतला. जे व्यक्तिमत्त्व एका चित्रपटाला खांद्यावर उचलू शकत होतं, ज्याच्या संवादांनी लाखो लोक भारावून जायचे, आणि ज्याने एक पूर्ण युग घडवलं, तेच व्यक्तिमत्त्व राजकारणातील खेळी, सत्तेचे समीकरण आणि प्रशासन यांमध्ये अपयशी ठरलं. कारण तो त्यांचा ‘रंगमंच’ नव्हता. खरं तर हे अपयश नव्हे, तर एक शिकवण आहे. एका क्षेत्रातील टाळ्यांची ग्वाही दुसऱ्या क्षेत्रात यश मिळवून देईलच, असं नाही. शेवटी, ‘गड्या, आपुला गाव बरा’ असे म्हणावे लागते. 

आज बहु-गुणी असण्याचा गौरव होतो. अगदी ‘जॅक ऑफ ऑल ट्रेड्स’ म्हणजेच ‘सगळ्याच गोष्टी थोड्याफार प्रमाणात येतात’ ही गोष्ट काहीशी प्रतिष्ठेची मानली जाते.  येथे ‘सर्कल ऑफ कॉम्पिटन्स’ (क्षमता-वर्तुळ) या मेंटल मॉडेल (मन:प्रारूप) म्हणजेच विचारचित्राची गरज भासते. ते आपल्याला क्षमतेच्या मर्यादांची जाणीव करून देते. ते आपल्याला सांगते की तुम्हाला काय येतं ते ओळखा आणि त्याहून महत्त्वाचं म्हणजे, काय येत नाही तेही! जे आपल्याला नीट कळले आहे, येते आहे, त्यात राहावे व प्रगती करावी नाहीतर ‘एक ना धड आणि भाराभर चिंध्या’ होण्याची शक्यता असते. गुंतवणूकदार चार्ली मंगर यांनी या संकल्पनेला प्रसिद्धी मिळवून दिली. उदाहरणार्थ, प्रसिद्ध गुंतवणूकदार वॉरेन बफे (चार्ली मंगर यांचे सहकारी) यांनी अनेक दशकांपर्यंत तंत्रज्ञान कंपन्यांमध्ये गुंतवणूक केली नाही. त्यांचा तंत्रज्ञानाला विरोध नव्हता, पण त्यांना त्या कंपन्यांचे दीर्घकालीन मूल्य समजत नाही, हे त्यांनी प्रामाणिकपणे मान्य केलं. ही विनयशीलता काही नकारात्मकता नव्हे तर तीच त्यांना डॉट-कॉम क्रॅशपासून वाचवू शकली आणि बर्कशायर हाथवे सारखी यशस्वी कंपनी घडवता आली. अशा या महत्वाच्या विचारचित्राची अजून काही उदाहरणे पाहुयात. 

गुंतवणूक क्षेत्रात काही काळापूर्वी क्रिप्टो करंसीचे (आभासी-सांकेतिक चलन) वारे वाहत होते. बऱ्याच लोकांना त्याच्या मागची संकल्पना, त्याचे मूल्य जोखण्याची पद्धत असे काहीही माहिती नव्हते तरी काहीजण आपल्या मित्रांकडून ‘टिप्स’ घेऊन आपली बचत टोकनमध्ये गुंतवायचे, अगदी ‘कळपातील मेंढ्यांप्रमाणे’. अशा प्रसंगी, आपल्याला जरा माहिती असलेल्या म्युच्युअल फंड किंवा इंडेक्स फंडमध्ये राहणं कधीही जास्त शहाणपणाचे नाही का?

नोकरीत सुरवातीपासून कष्ट करून, ज्ञान सम्पादन करून, आपल्या तांत्रिक कौशल्यावर अनेक जण बढत्या मिळवतात. थोड्या वर्षांनी ‘सिनियर’ (वरिष्ठ)  झाल्यावर आपोआप व्यवस्थापकाची (‘मॅनेजर’) ची जबादारी येते (खरंतर, दिली जाते). त्याची काहीही आवड नसताना, क्षमता नसताना फक्त सामाजिक दबावामुळे आणि वेतन वाढीमुळे लोकं ते स्वीकारतात. पण सत्य थोड्याच दिवसात कळते, तेथील गोंधळ, वरच्यांकडून आणि खालच्यांकडूनही ऐकावी लागणारी बोलणी पाहून काहींना नक्की वाटते की ‘गड्या, आपुला गाव बरा’, परत आवडत्या आणि चांगले जमणाऱ्या तांत्रिक कामातच जावे आणि त्यातच उत्तरोत्तर प्रगती करावी. 

एखाद्याने ‘मधुमेहावर घरगुती उपाय’ असा व्हॅट्सऍप वर मेसेज पाठवला तर विचार करायला पाहिजे, की पाठवणाऱ्याचे ते अधिकार क्षेत्र आहे का? त्यात त्याचा अभ्यास किती आहे म्हणजेच हा विषय त्याचा ‘क्षमतेच्या वर्तुळातील’ आहे का? हे पाहूनच ते सल्ले  मानावे की नाही हे ठरवावे. खऱ्या डॉक्टरऐवजी अशा सल्ल्यावर विश्वास ठेवणं धोकादायक ठरू शकतं.

हातात कधी बॅट न घेतलेले, सचिनला कव्हर ड्राइव्ह कसा मारायचा ते सांगत असतात? पुलंच्या भाषेत सांगायचे झाले तर “आपण स्वतः पुणे महानगरपालिकेत, उंदीर मारायच्या विभागात आहोत नोकरीला, हे विसरून ‘अमेरिकेची आर्थिक घडी नीट बसवण्याचा खरा मार्ग कोणता?’ यावर  मत ठणकावतात”. या उदाहरणांत आपण ‘क्षमता-वर्तुळाच्या’ बाहेर जात आहोत हे कळले पाहिचे. याला उपाय म्हणजे जागरूकता, अभ्यास आणि विनयशीलता. 

आजच्या चढाओढीच्या आणि गलबल्याच्या जगात, ’क्षमता-वर्तुळ’ हे विचारचित्र आपल्याला स्वतःची एक शांत, पण खंबीर ओळख देते. एखाद्या गोष्टीचा निर्णय घेण्याआधी तिला नीट समजून घ्या आणि आपल्या क्षमतेत नसेल तर दुसऱ्या तज्ज्ञांची मदत घ्या, असे सुचवते. मग विचार येतो की आपण असेच क्षमता-वर्तुळात अडकून, संकुचित (कूपमंडूक), राहायचे का? तर, नाही.  “गड्या, आपुला गाव बरा” म्हणताना  हे विचारचित्र आपल्याला नवीन शिकण्यापासून थांबवत नाही. तुम्ही हे वर्तुळ मोठे करू शकता, ‘गावाचा’ ‘देश’ करू शकता तर ‘देशा’चे ‘जग’. म्हणजेच तुमच्या क्षमतेचं क्षेत्र तेच ठेवून त्यात उत्तरोत्तर प्रगती करणे, हा खरा उद्देश आहे. 



\chapter{सोप्या मार्गाचे शहाणपण}

काल दुपारची गोष्ट. माझ्या मोबाईलचं वाय-फाय काही चालत नव्हतं. खूप प्रयत्न केले, सेटिंग्ज तपासली, इंटरनेटवर माहिती वाचून काही गोष्टी केल्या, युट्युबवरील व्हिडीओ, एक ना अनेक उपाय केले पण ते वाय-फाय काही प्रतिसाद देईना. संगणक क्षेत्रातील असूनही उत्तर सापडत नसल्याने साहजिकच चीड-चीड वाढली. मोबाईलला, त्यातल्या सॉफ्टवेअरला, एवढच काय, संपूर्ण तंत्रज्ञान क्षेत्राला आणि नेहमीच्या सवयीने शासनाला (!) पण नावं  ठेऊन झाली पण समस्या काही सुटेना. संध्याकाळी घरी आल्यावर माझ्या छोट्या मुलीला हे कळताच ती म्हणाली की बाबा जरा डिस्कनेक्ट करून पुन्हा कनेक्ट करून बघा ना? आणि खरोखरच त्याने ते अडेल-तट्टू वाय-फाय चालू झाल ना! उगाच काहीतरी क्लिष्ट, गुंतागुंतीचे, किचकट करत बसलो पण उत्तर मात्र सोपे होते. ह्यालाच म्हणतात ‘काखेत कळसा गावाला वळसा’. या मेंटल मॉडेलला (मन:प्रारूप) अथवा विचार चित्राला ‘ऑकम्स रेझर’ किंवा साध्या भाषेत ‘सोप्या मार्गाचे शहाणपण’ म्हणू शकतो. २०२३ च्या “भारतीय दूरसंचार विनियामक प्राधिकरण” (टी-आर-ए-आय)  यांच्या अहवालानुसार, भारतातील ८५% पेक्षा जास्त ब्रॉडबँड तक्रारी केवळ मोडेम रीस्टार्ट करून किंवा सैल कनेक्शन तपासून सोडवल्या जातात तरीही बहुतेक लोकं, अधिक जटिल अशा तांत्रिक बिघाडांचा अंदाज लावतात आणि साध्या उपायांकडे दुर्लक्ष करतात. नेहमी खोल विचार करण्याची व गुंतागुंतीची स्पष्टीकरणे देण्याची ही प्रवृत्ती खरंतर एक मानसिक पूर्वग्रह दर्शवते. येथे ‘सोप्या मार्गाचे शहाणपण’ हे विचारचित्र आपल्याला सुधारण्यास मदत करते.

१४व्या शतकातील इंग्लिश तत्त्वज्ञ ‘विल्यम ऑफ ऑकम’ यांच्या नावावरून हे विचारचित्र आले आहे. "रेझर" म्हणजे दाढी करण्याचा ब्लेड. आवश्यक  नसलेली गृहितकं (ऍझमशन्स) दूर करायची, ही यामागची कल्पना आहे. ते आपल्याला सांगते की तुमच्याकडे एखाद्या समस्येसाठी अनेक उत्तरे-पर्याय असतील तर त्यातले असे निवडा की ज्यात सर्वात कमी गृहीतके आहेत किंवा अधिक सोप्या शब्दांत म्हणायचे झाले तर, सर्वात सोपा उपाय बहुधा योग्य असतो. ‘ऑकमचा रेझर’ हे सोपे उत्तर नेहमीच बरोबर असेल याची हमी देत नाही, परंतु ते असे सुचवते की सर्वात कमी गुंतागुंतीच्या स्पष्टीकरणापासून सुरुवात करणे हा समस्येच्या समाधानाकडे जाण्याचा सर्वात कार्यक्षम मार्ग असतो. या जगात जिथे माहिती विपुल आहे आणि जटिलतेचे वैभव दाखवले जाते, तिथे सोपेपणा केवळ सुंदरच नाही तर तो धोरणात्मक सुद्धा असू शकतो. या विचारचित्राची काही उदाहरणे पाहुयात. 

प्रसिद्ध उद्योगपती रतन टाटांना मध्यमवर्गासाठी कार बनवायची होती. सर्वात महत्वाची गोष्ट म्हणजे ती किफायती पाहिजे, त्याकाळात म्हटले तर १ लाख रुपयात. त्यांच्या संशोधक टीमने कार मध्ये आत्यंतिक गरजेचे काय असते तेच ठेवून, इतर कमी महत्वाच्या गोष्टींना फाटा देऊन, इतर किफायतशीर बदल करून, तसे कार मॉडेल बनवून दाखवले आणि ते म्हणजे ‘नॅनो’. 

गूगल कंपनीच्या आधी सुद्धा अनेक प्रसिद्ध शोध-आंतरजाल-स्थळे होती. ती माहितीने गजबजलेली वाटायची. अतिशय नाविन्यपूर्ण आणि कार्यक्षम शोध-प्रणाली बरोबरच गुगलने एक महत्वाचा बदल आणला तो म्हणजे, अतिशय सोप्पे शोध-संकेत-स्थळ (सर्च वेब पेज). टाईप करण्यासाठी एक बॉक्स आणि ‘शोध घे’ असे सांगण्यासाठी १-२ बटन्स, बस्स!! बाकी सगळे कोरे. या संकेत स्थळाच्या जोरावर गुगलने केवढी मोठी भरारी मारलीये. 

आर्थिक व्यवहारांसाठी आपण वापरात असलेल्या भीम-पे अथवा गुगल-पे च्या मागे असलेल्या यु-पी-आय (युनिफाइड पेमेंट्स इंटरफेस) ची जबरदस्त वाढ ओकॅमच्या रेझरचे उदाहरण आहे. आधीच्य ऑनलाईन पेमेंट सिस्टममध्ये अकाउंट नंबर, बँकेचा आय-एफ-एस-सी नंबर इत्यादी बाबी आवश्यक होत्या. पुन्हा प्रत्येक बँकेची संकेत-स्थळे वेगळी. यु-पी-आयने ही सर्व क्लिष्टता काढून एका साध्या क्यू-आर कोड स्कॅनपर्यंत काम कमी केले, अनावश्यक गुंतागुंत दूर केली आणि आता ते दरमहा १० अब्जाहून जास्त व्यवहार करत आहे.

गुंतवणूक क्षेत्रातही बऱ्याच क्लिष्ट रणनीती वापरल्या जातात. त्यातल्या तज्ज्ञांचं ठीक आहे पण सामान्य गुंतवणूकदारही अनेकदा बाजारपेठेतील किमतीच्या चढ-उताराची अचूक वेळ (मार्केट टायमिंग) साधणे, क्लिष्ट सूत्रे-गणितीय प्रणाल्यांसारखी साधने निवडणे किंवा जटिल ट्रेडिंग अल्गोरिदम याचा वापर करताना दिसतात. गरज नसताना, समजत नसतानाही अनावश्यकपणे गुंतागुंतीच्या मार्गावर जातात. खरंतर, बऱ्याच वेळेला असे दिसून येते की साधे इंडेक्स-फंड सुद्धा या जटिल रणनीतींपेक्षा चांगले प्रदर्शन करतात, तेही अल्पशा खर्चात.

आपण उत्पादनांमध्ये अनेक व कदाचित अनावश्य वैशिष्ट्ये (फीचर्स) भरतो, संस्थांमध्ये अति स्तर आणि रिपोर्टींगची गुंतागुंत करतो आणि अनेकदा असे मानतो की अधिक क्लिष्ट म्हणजे फार भारी. खरे सौंदर्य सोपेपणात आणि साधेपणात आहे. मान्य आहे की साधेपणाचाही अतिरेक नको पण अतिआवश्यक ते तर ठेवायलाच पाहिजे. कमीतकमी गोष्टीत जास्तीत जास्त परिणामकारकता कशी आणता येईल हा मुख्य विचार आहे. 

‘ऑकम्स रेझर’ची फक्त उपयुक्तता समस्या-सोडवण्यासाठी आहे असे नाही. ती जीवनाबद्दलची एक तात्त्विक भूमिका आहे. एका अशा जगात जे आपल्याला सातत्याने अधिक गुंतागुंतीकडे ढकलत आहे, तिथे साधेपणा निवडणे हा एक महत्वाचा पर्याय बनतो. 



\chapter{जुनं ते सोनं}

सध्याच्या काळात पैसे गुंतवण्यासाठी अनेक पर्याय उपलब्ध आहेत. बँकेत मुदत-ठेव (फिक्स्ड डिपॉझिट), शेअर बाजार, म्युच्युअल फंड्स, जमीन-घर खरेदी आणि सर्वात आधुनिक मार्ग म्हणजे क्रिप्टो चलन. पण त्यातील एका पर्यायाची लोकप्रियता अगदी अनादी काळापासून जशीच्या तशी आहे, ते म्हणजे, सोनं. मागणी-पुरवठ्याचे नियम त्यालाही लागत असले तरी सर्वसाधारणपणे भाववाढीच्या बरोबर कमी-जास्त होणारे, सुरक्षित, जागतिक स्तरावर विश्वासार्ह आणि तरल (लिक्वीड) म्हणजेच वेळप्रसंगी कधीही मोडून पैसे उभे करता येणारे, ते म्हणजे सोनं. म्हणूच एका पिढीकडून दुसरीकडे, लग्नप्रसंगी एका कुटुंबातून दुसऱ्याकडे देवाणघेवाण होऊन, पै-पै साठवून हे चमकदार ‘धन’ सांभाळले जाते आणि सणासुदीला ‘लक्ष्मी’ स्वरूपात पुजले पण जाते. शेकडो वर्षे झाली पण यात काही बदल नाही. अनेक राजे-राजवाडे होऊन गेले, अनेक तेजी-मंदीचे काळ येऊन गेले पण सोन्याची शान तशीच्या तशीच आहे. अशाप्रकारे चिरकाल जिवंत राहणाऱ्या कल्पनांच्या ‘मेंटल मॉडेल’ला (मन:प्रारूप)‘लिंडी इफेक्ट’ म्हणजेच सोप्या भाषेत ‘जुनं ते सोनं ‘ असे म्हणतात. जे आतापर्यंत, हजारो वर्षे टिकले आहे ते यापुढेही अनंतकाल टिकेल अशी या विचारचित्रामागची भूमिका आहे. 
१९६० च्या दशकात अमेरिकेतील विचारवंतांनी ब्रॉडवे नाटकांच्या दीर्घकालीन अस्तित्वाचे निरीक्षण करत "लिंडी इफेक्ट" चा सिद्धांत मांडला. नंतर, नसीम निकोलस तालेब यांनी त्याला लोकप्रिय केलं. या सिद्धांतानुसार, नष्ट न होणाऱ्या गोष्टी, जसे की कल्पना, तंत्रज्ञान, पुस्तके किंवा परंपरा या जितक्या जुन्या असतात, तितके त्यांचे भविष्य अधिक उज्ज्वल व टिकाऊ असते. साधे उदाहरण सांगायचं तर, जर एखादं पुस्तक ५० वर्षं अस्तित्वात राहिलं असेल, तर ते अजून ५० वर्षं टिकण्याची शक्यता जास्त आहे. शेकडो वर्षांपूर्वी चाणक्य यांनी लिहिलेलं अर्थशास्त्र. त्याच्यातील राज्यकारभार आणि धोरणातील धडे आजही एमबीए वर्गात आणि राजनीतीच्या-धोरणात्मक प्रशिक्षणात गिरवले जातात. का? कारण कालानुरूप असल्याने प्रत्येक पिढीत ते टिकून राहिले आहेत आणि या विचारचित्रानुसार ते भविष्यातही राहतील हा कयास आहे. अशीच काही इतर उदाहरणे पाहुयात. 
भारतातील कोणत्याही शास्त्र-अभियांत्रिकीच्या विध्यार्थ्याला विचारलं की १२वी ला भौतिक-शास्त्राचा अभ्यास कशातून केला तर बहुतेकांचं उत्तर येईल एच.सी. वर्मांचं "कॉन्सेप्ट्स ऑफ फिजिक्स". १९९० च्या दशकातील हे पुस्तक अजूनही टिकून आहे. अशीच प्रत्येक शाखेची, त्यातील विषयांची जुनी-लोकप्रिय पाठ्यपुस्तके (क्लासिक्स) असतात त्यावाचून अभ्यासाचे ‘पान’ हालत नाही. काळानुसार काही बदल होतात पण बहुतांशी मूळ गाभा तोच राहतो. इतर समाज-जीवनातही आजही आपण संत ज्ञानोबारायांचे-तुकोबारायांचे अभंग, दासबोधातील समर्थ रामदास स्वामींच्या ओव्या सहज उद्धृत करीत असतो कारण ते अजून समयोचित (रिलेव्हन्ट) राहिले आहेत. शेवटी, जुनं ते सोनं. 
फॅशन प्रमाणेच आहार विषयात कायम काहीतरी नवीन वारे (ट्रेंड्स) वाहत असतात. कधी किटो, तर कधी लो-कार्ब्स, कधी सतत खा, तर कधी दोनच वेळेला. असे एक ना अनेक. भूक लागेल तेंव्हा, थोडा अवकाश (म्हणजे पोटात जागा) ठेवून, आणि आपली आजी जे खात होती ते साधारणतः आपणही खाल्लं तर ते चांगलं मानवतं, नाही का? हेच शाश्वत तत्व आहे. या क्षेत्रात अजून एक ट्रेंड दिसुन येतो. अमेरिकेतील (आणि आता भारतातही) प्रसिद्ध असलेली कॅफे मध्ये काही पेय (आणि त्याच्या किमती) पहिल्या की तोंडात बोटे (आश्चर्याने!!) जातात. ‘टर्मरिक लॅट्टे’ हा प्रकार भारी लोकप्रिय झाला आहे. खरं म्हणजे, भारतीय घरात वर्षानुवर्षे प्यायले जाणारे ते हळदीचे दूध!!! नवीन उत्पादन म्हणून समोर आणलं आणि प्रसिद्ध झालं शेकडो वर्षे ज्ञात असलेल्या त्याच्या गुणधर्मानेच. 
कितीही ई-कॉमर्स द्वारे व्यवहार होत असले तरी, जवळच्या किराणामालाच्या दुकानात असलेली आपली वही-खातं  अजून चालू आहे. लहान कुटुंबाद्वारे चालवलेली, वेळेला उसने देणारी ही व्यवस्था अजून टिकून आहे, कारण संबंध, स्थानमाहात्म्य आणि विश्वास हे कायम टिकणारे असतात.
आरोग्यक्षेत्रात कितीही प्रगती झाली, नवनवीन तंत्रज्ञान आले, दूरस्थ (रिमोट) पद्धतीने उपचार चालू झाले तरी अनुभव सिद्ध डॉक्टरांच्या रोग-निदानाला, नाडी परीक्षेला अजूनही स्थान आहे. त्यांच्या आश्वासक बोलण्यातच अर्धा आजार पळून जात असावा असे वाटते. त्यामुळे या कृत्रिम बुद्धिमतेच्या (‘आर्टिफिशिअल इंटेलिजन्स’), यंत्रमानवांच्या  जगातही (आणि भविष्यातही) डॉक्टरांचे महत्व अबाधितच राहणार आहे. 
हजारो वर्षांच्या इतिहासाने समृद्ध असलेल्या आपल्या देशात, जागोजागी, पदोपदी 'लिंडी इफेक्ट’ची उदाहरणे सापडतात. मंदिरे, शास्त्र, पाककला, परंपरा आणि जीवन मूल्ये आपण काहीप्रमाणात का होईना जपली आहेत. आपण अनेकदा (गरज-कारण नसताना) नवतेच्या मागे धावतो, काहीतरी पूर्वीपेक्षा वेगळं-हटके करायचं या हट्टापायी. वर्षातून दहा वेळा दिशा बदलणारा (‘पिव्हट’  करणारा) स्टार्टअप. भाषणात दुसऱ्या दिवशी धोरण बदलणारा पलटूराम राजकारणी. रोज नवे ब्रँडिंग करून सल्ला देणारा इन्फ्लुएन्सर, अशी एक ना अनेक उदाहरणे आहेत. 
‘लिंडी इफेक्ट’ हा काही नवीन गोष्टींना नाकारण्याचा सल्ला नक्कीच देत नाही. कोणतीही गोष्ट करताना, जे टिकले आहे त्याचाही विचार करा एवढेच तो सुचवतो. अनेक स्थित्यंतरातही जे शाबूत राहिलं त्याचा आपण खोल विचार करायला हवा. पुढच्या वेळी तुम्हाला निर्णय घ्यायचा असेल,  काय वाचावे, काय खावे, कोणत्या सल्ल्यावर विश्वास ठेवावा, तेंव्हा विचार करा की याने काळाची कसोटी पार केली आहे का?   



\chapter{बुडीत-खात्याचा मोह}

माझ्या कॉलेज काळातील एक आठवण आहे. हिंदी चित्रपटसृष्टीत बलवान आणि पीळदार शरीरयष्टीच्या हिरोंचा बोलबाला सुरु झाला होता. साहजिकच तरुणांमध्ये त्याचे अनुकरण करण्याची ओढ-क्रेझ निर्माण झाली होती. माझ्या काडी-पैलवान मित्राला तर ‘सलमान’ व्हायचं वेडच लागल होतं. आई-बाबांच्या मागे लागून लागून त्याने सुमारे ५०००/- (त्या काळाचे) रुपयांचे जिमचे साहित्य आणून घरातील त्याच्या खोलीत एक वैयक्तिक व्यायामशाळाच थाटली होती. काही दिवस उत्साहात तासंतास व्यायाम केला. नंतर जसे शरीराने गाऱ्हाणे  मांडायला सुरवात केली तसे व्यायामाकडे दुर्लक्ष होऊ लागले. नंतर तर त्या सर्व सामुग्रीची अडचणच वाटू लागली. वर्ष उलटून गेलं आणि व्यायाम तर मात्र २ आठवड्यानंतरच बंद झालेला होता. ‘मी त्यात एवढे पैसे घातले आहेत’ या विचारांनी ते सर्व मिळेल त्या किमतीत (म्हणजे ‘अर्ध्या’) विकायची ही इच्छा होईना. आर्थिक नुकसान होणारच होतं. पण ती सगळी सामग्री तशीच ठेवणं हेच खरं नुकसान नव्हतं का? आपण अनेक वेळा काही गोष्टींमध्ये “आधीच एवढं केलंय” म्हणून अडकून पडतो आणि मग परत फिरता येत नाही. या मेंटल मॉडेल (मन:प्रारूप) ला म्हणजेच विचार-चित्राला ‘संक कॉस्ट फॅलसी’ म्हणजेच ‘बुडीत-खात्याचा मोह’ म्हणू शकतो. हा असा मानसिक सापळा आहे जो तुमच्या वैयक्तिक निर्णयांपासून ते सरकारच्या धोरणांपर्यंत सगळीकडे दिसतो.

या विचार-चित्रात, आपण एखाद्या गोष्टीत वेळ, पैसा किंवा श्रम घातल्यामुळे ती सोडणं आपल्याला कठीण जातं, जरी पुढे काहीही फायदा होणार नसलं, तरीही. आपण म्हणतो, "इतकं केलंय, तर आता सोडणं योग्य नाही." पण वास्तवात मागचं नुकसान लक्षात घेऊन पुढचं नुकसान वाढवणं अजूनच घातक ठरतं. उदाहरणार्थ, सरकार एखादा प्रकल्प हाती घेतं. काही कारणांमुळे किंवा नीट नियोजन न केल्यामुळे जनतेकडून अपेक्षित प्रतिसाद मिळत नाही. वापर कमी होतो पण देखभाल खर्च करावाच लागतो. तरीही सरकार तो प्रकल्प बंद करत नाही. कारण? "आधीच एवढा खर्च झालाय!" अशा प्रकारची अनेक उदाहरणं आपल्याला आजूबाजूला दिसतात, ती पाहूयात.

अनेकदा आपण अशा नात्यांमध्ये अडकून पडतो जे मानसिक त्रास, अपमान किंवा सतत दुःख देत असतात. तरीही आपण ती नाती टिकवतो, कारण आपण आधी खूप काही दिलेलं असतं. वेळ, भावना, कधी कधी आर्थिक मदतसुद्धा. "इतकी वर्षं एकत्र आहोत, आता तोडून कसं चालेल?" हा विचारच आपल्याला अडकवून ठेवतो. पण सत्य असं असतं की, चुकीचं नातं जितकं लवकर संपवता येईल, तितकं उरलेलं आयुष्य आनंदात घालवता येईल हा विचार करणे सुद्धा कधी कधी ‘गुन्ह्या’सारखे वाटत असले तरी त्याचा विचार करावा हेच हे विचार-चित्र सुचवतं. 

एखाद्या छोट्या शहरातल्या दुकानात, जिथे ग्राहक येत नाहीत, विक्री होत नाही, तरी दुकानदार म्हणतो की “मी या दुकानात इतका पैसा घातलाय, आता बंद कसं करू?” पण याचा विचार होत नाही की रोज दुकान उघडून ठेवणं, माल आणणं, वीजबिल भरणं, हे सर्व अजून तोटा करत आहेत. कधी कधी जुना व्यवसाय सोडून काही नवीन सुरू करणं हेच शहाणपणाचं असतं.

एखादा चित्रपट बघायला एखाद्या पॉश चित्रपटगृहात जावं आणि नेमका तो चित्रपट जाम बोअर निघावा. वेळ वाया जातोय, हे कळत असताना सुद्धा  “आधीच एक तास झाला आहे, दीडशे रुपयाचं तिकीट काढलाय,” असं म्हणत आपण तो आता शेवटपर्यंत पाहतो आणि उरलेला वेळही वाया घालवतो. जे आधीच वाया गेलं आहे, त्याची भर अजून वेळ वाया घालवून का करायची?

काही शेअर्स मध्ये गुंतवलेले पैसे दिवसोंदिवस कमी-कमी होत आहेत हे लक्षात येत असूनही लोक त्या शेअर्सना चिकटून राहतात. कारण, “आधीच पैसे घातलेत, आता विकलं तर नुकसान होईल.” पण हे लक्षात घेतलं जात नाही की ज्यामध्ये गुंतवणूक अकार्यक्षम ठरतेय तीचा प्रत्येक दिवस एक ‘नवीन नुकसान’ आणत असतो.

ही सगळी उदाहरणं आपल्याला एक गोष्ट शिकवतात. मागे जे झालं त्यासाठी आपण पुढे चुकीचे निर्णय घेऊ नयेत. जेव्हा आपण फक्त "आधीच इतकं केलंय" म्हणून काही करत राहतो, तेव्हा आपला वेळ, पैसा आणि मानसिक ऊर्जा वाया जाते. ह्या मानसिक सापळ्याची जाणीव ठेवणं गरजेचं आहे.
गीतकार साहिर लुधियानवी यांच्या ह्या ओळी  ‘संक कॉस्ट फॅलसी’ या मेंटल मॉडेलला अगदी चपखल बसतात - “वो अफ़्साना जिसे अंजाम तक लाना न हो मुमकिन, उसे इक ख़ूबसूरत मोड़ दे कर छोड़ना अच्छा।”


\chapter{कमीतून जास्त}

बऱ्याच भारतीय शहरांप्रमाणे पुण्यातही वाहतुकीचा बहुतांश भार काही मोजक्या रस्त्यांवरच असतो. सुमारे २०\% रस्त्यांवर ८०\% वाहतूक केंद्रित असते. बाणेर-हिंजवडी, जे.एम. रोड, एफ.सी. रोड आणि स्वारगेट-टिळक रस्ता हे मुख्य मार्ग विशेषतः गर्दीच्या वेळात प्रचंड ताण सहन करतात. शहरात शेकडो किलोमीटर रस्ते असले तरी, गर्दी आणि विलंब नेहमी या काही मोजक्या मार्गांवरच केंद्रित असतात. सार्वजनिक वाहतुकीच्या बसेस सुद्धा आपला बहुतांश वेळ फक्त याच काही गर्दीच्या रस्त्यांवर अडकून घालवताना दिसतात. याला उपाय म्हणजे पुण्यातील सर्व रस्त्यांपेक्षा हे अतिरहदारीचे रस्ते निवडणे आणि त्यांच्या समस्या प्राधान्यक्रमाने सोडवणे. अशा मोजक्या-सीमित बाबींवर लक्ष केंद्रित करून मोठा बदल घडवून आणण्याच्या मेंटल मॉडेल (मन:प्रारूप) ला अथवा विचारचरित्राला ‘पॅरेटो तत्व’ किंवा ‘८०/२० नियम’ असे म्हणतात, याला सोप्या भाषेत ‘कमीतून जास्त’ असे म्हणू शकतो. हेच तत्त्व वापरून, पुणे महानगरपालिका स्मार्ट सिग्नलिंग, मेट्रो विस्तार आणि निवडक रस्ते रुंदीकरण यांसारख्या उपाययोजनांवर लक्ष केंद्रित करत आहे. उदाहरणार्थ, नळ स्टॉप चौकातील वाहतूक सुलभ केल्यामुळे संपूर्ण मार्गावरील प्रवासाचा वेळ लक्षणीय कमी झाला. पायाभूत सुविधांमध्ये सगळीकडे पैसे उधळण्यापेक्षा, हे काही मोजके "महत्त्वाचे रस्ते" सुधारण्याचे धोरण अधिक परिणामकारक ठरते.

इटालियन अर्थशास्त्रज्ञ ‘विल्फ्रेडो परेटो यांनी १९व्या शतकात हे तत्त्व मांडले. त्यांनी पाहिले की इटलीतील ८०\% जमीन केवळ २०\% लोकांच्या मालकीची आहे. असे चित्र जगभरातील अनेक नैसर्गिक व सामाजिक यंत्रणांमध्ये ही वारंवार दिसते. भारतातही हे तत्त्व अनेक ठिकाणी ठळकपणे दिसते. या तत्वातील ८०\% आणि २०\% या आकड्यांच्या मागे अगदी  फार तंतोतंतपणे पाहू नये तर त्याचा भावार्थ घ्यावा. थोड्याशा कारणांमुळेच बहुतांश परिणाम घडतात, हेच या तत्त्वाचं सार आहे. मुद्दा जास्त मेहनत करण्याचा नाही तर चतुरपणे, योग्य ठिकाणी करण्याचा आहे. याचा उपयोग केवळ वाहतूक-नियंत्रणापुरता मर्यादित नाही. व्यवसायापासून वैयक्तिक आयुष्यापर्यंत, सर्वत्र हे तत्व आपल्याला दिशा दाखवते. त्याची काही उदाहरणे पाहुयात. 

एका स्टार्टअपने पाहिलं की ८०\% तक्रारी २०\% ग्राहकांकडून येतात. त्यांनी त्या मोजक्या ग्राहकांच्या समस्या सोडवल्या आणि संपूर्ण सेवा सुधारली.

जगातील, तसेच भारतातीलही  फार थोडे लोकं इतर सर्व जनतेपेक्षा जास्त कमावतात. अतिशय कमी लोक आयकर भरत असल्याने ते इतर सर्वांच्या कल्याणासाठी भार उचलतात. 

आपण एखाद्या मध्यमवर्गीय कुटुंबाचा खर्च पहिला तर जवळजवळ बहुतांशी भाग एक-दोन मोठ्या गोष्टी, जसे घर भाडे (अथवा कर्जाचा हप्ता) किंवा मुलांच्या शिक्षणात जातो. तेथे नीट-सखोल विचार करून निर्णय घेतला तर योग्य बचत होऊ शकते. 

आरोग्याचा विचार केला तर बहुतेक घरांमध्ये फक्त २०\% अन्नपदार्थ जसे साखर, तेलकट आणि प्रोसेस्ड फूड हे ८०\% आरोग्य समस्यांना कारणीभूत ठरतात. हेच पदार्थ बदलले तर मोठा फरक पडू शकतो. 

तुमच्या मोबाईलमध्ये शेकडो ऍप्स असले तरी ८०\% वेळ तुम्ही त्यातील ठराविक २०\% ऍप्स वरच घालवता, नाही का? त्या मोजक्याच ऍप्स च्या उपयोगावर नियंत्रण ठेवले तर मोठ्याप्रमाणात वेळ वाचू शकतो. 

समाजजीवनात सुद्धा तुमचे आप्त-स्वकीय-मित्र हे मोजकेच असतात. वेळप्रसंगी अर्ध्या रात्री धावून येणारे २०\%, बाकी ८०\% असतात केवळ ओळखी, कामापुरत्या. त्यामुळे कुठली नाती जपून ठेवायची आणि कोणाचा खूप विचार करायचा नाही हे या तत्वामुळे जरूर ओळखता येते. 

‘परेटो तत्व’ हे कोठे लक्ष केंद्रित करायचे ते सांगते. सध्याच्या जगात भारंभार गोष्टी करण्यावर खूप भर असतो. पण त्यासर्वातून अपेक्षित परिणाम मिळतोय का हे तपासून पाहिलं जात नाही, मग होते नुसती दमणूक आणि हात राहतात रिकामेच. आपण जेव्हा कामाच्या भल्या मोठ्या यादीने गुदमरतो, तेव्हा हे लक्षात ठेवा की बहुतेक गोष्टी या आवश्यक-परिणामकारक नसतात. काही मोजक्याच कृती खरंतर परिणाम घडवतात. काय करावं यापेक्षा काय टाळावं हे ओळखणं जास्त मौल्यवान ठरतं. भारतासारख्या देशात  ८०/२० नियम हे एक अमूल्य मार्गदर्शक साधन आहे. आपली ऊर्जा सर्वत्र खर्च करण्याऐवजी, काही मोजक्या महत्त्वाच्या बाबींवर केंद्रित केली, तर परिणाम अधिक आणि टिकाऊ होतो. थोडक्यात सांगायचं तर कमी करून अधिक साध्य करणं ही युक्ती नाही, आळशीपणा नाही तर तीच खरी शक्ती आहे.







\chapter{मनाच्या पूर्वग्रहाचे खेळ}
संगणकात अथवा मोबाईलमध्ये सॉफ्टवेअर-ऍप इन्स्टॉल करताना तुम्ही ‘नियम व अटी’ (टर्म्स अँड कंडिशन्स) आणि घेतल्या जाणाऱ्या ‘परवानग्या’ (पर्मिशन्स) वाचता का? का अशा अटींना होकार देऊन पुढे जाता? एका सर्वेक्षणानुसार केवळ ९% गुंतवणूकदारच संपूर्ण अटी वाचतात, उरलेले सोडतात. हा आकडा केवळ निष्काळजीपणा दाखवत नाही, तर आपल्या मेंदूची एक नैसर्गिक कमतरता देखील दर्शवतो. एवढ्या मोठ्या कंपन्या काही दगाबाजी करणार नाहीत हा पूर्वग्रह (बायस) आपल्या मनात ठाम असतो. अशाच आळसामुळे असो किंवा अंध-विश्वासामुळे, आपण सहज सायबर आक्रमणांना दार किलकिले करून ठेवतो. 

आपल्या मेंदूत अनेक पूर्वग्रह (‘बायसेस’) किंवा शॉर्टकट्स असतात, जे आपल्याला कष्ट घेण्यापासून वाचवतात पण चुकीच्या निर्णयांकडे नेतात. आजच्या धावपळीच्या जीवनात हे शॉर्टकट्स सतत वापरले जातात. पूर्वग्रह अधिक दृढ होतात आणि आपण त्या मनाच्या खेळात अडकतो. अशा नकळत होणाऱ्या फसवणुकीच्या मेंटल मॉडेल (मन:प्रारूपाला) अथवा विचार-चित्रांच्या समूहाला ‘कॉग्निटिव्ह बायसेस’ (मानसिक पूर्वग्रह) किंवा सध्या-सोप्या भाषेत ‘मनाच्या पूर्वग्रहाचे खेळ’ म्हणू शकतो. १९७० च्या दशकात मानसशास्त्रज्ञ डॅनियल काहनेमन आणि आमोस टव्हर्स्की यांनी हे स्पष्ट केलं की, आपले निर्णय बहुतेक वेळा भावनांवर आणि सहजतेवर आधारित असतात, वास्तवावर नव्हे. असे पूर्वग्रह अनेक प्रकारचे असतात. त्याची काही उदाहरणे पाहुयात. 

तुम्ही बाजारात गालिचा घ्यायला गेलात. दुकानदार किंमत सांगतो १२,०००. तुम्ही घासाघीस करून ६,००० वर घेता आणि स्वत:ला हुशार समजता. पण त्या गालिच्याची खरी किंमत कदाचित २,०००च असण्याची शक्यता खूप, नाही का?  हा प्रकार भारतातच नाही तर आशियातील अनेक देशांत, विशेषतः चीनमध्ये दिसतो. मी शांघायमध्ये वस्तू विकत घेताना सुरुवातीला नेहमी सांगितलेल्या किंमतीच्या फक्त तृतीयांश किंमतीत बोलणी सुरु करायचो. नंतर सौदा सुमारे अर्ध्या किमतीत होई. दोघेही खुश!  याला "अ‍ॅंकरिंग बायस" (आधार-पूर्वग्रह) म्हणतात, जिथे सुरुवातीला ऐकलेली किंमत आपल्या निर्णयावर प्रभाव टाकते. 

आपली काही ठाम मते-विचार असतात त्यांनाच आपण कवटाळून बसलेलो असतो. आपल्या अवतीभोवती त्याला पूरक अशाच गोष्टी आपल्याला दिसत राहतात आणि विरोधातली माहिती आपण नकळत दुर्लक्षित करतो. याला ‘कन्फर्मेशन बायस’ (दृढीकरण-पूर्वग्रह) म्हणतात. घरी ‘टाटा पंच’ ही गाडी विकत घेण्याची चर्चा सुरु असेल तर रस्त्यावरून जाताना दर १-२ मिनीटांनी टाटा-पंच दिसायला लागतात, बरोबर ना? समाज-माध्यमात, पत्रकारितेत एकाच विचारधारेला वाहिलेले गट ठळकपणे दिसतात. उदाहरणार्थ, जर एखाद्या नेत्यावर भ्रष्टाचाराचे आरोप झाले, तरी त्याचे समर्थक म्हणतील, “हे विरोधकांचे कारस्थान आहे,” कारण त्यांचा मेंदू आधीच ठरवून बसलेला असतो. त्यामुळे सत्य शोधण्याऐवजी, आपल्या विश्वासाला पाठिंबा देणारी माहितीच खरी वाटते.

कोणतीही घटना मनात ताजी असेल, तर आपण ती अधिक सामान्य-खरी समजतो. हाच ‘अ‍ॅव्हेलेबिलिटी बायस’ (स्मरणसुलभता-पूर्वग्रह). उदाहरणार्थ, जर नुकतेच एखाद्या हवाई अपघाताची बातमी ऐकली असेल, तर विमान प्रवास धोकादायक वाटू लागतो, जरी आकडेवारी वेगळीच असली तरी. आपल्या मेंदूला जे लगेच आठवते तेच खरं वाटते.

जुनी स्थिती कायम ठेवायची मनाची प्रवृत्ती म्हणजे ‘स्टेटस क्वो बायस’ (स्थितीस्थैर्य-पूर्वग्रह). अनेकदा नवीन आणि फायदेशीर पर्याय असतानाही आपण जुन्या सवयींच्या पलीकडे जाण्यापासून घाबरतो. एखाद्या कर्मचाऱ्याला चांगली नवीन नोकरीची ऑफर असली, तरी तो म्हणतो, “सध्या तशी स्थिती ठीक आहे, नवीन नोकरी धोकादायक वाटते.” आपल्या सवयी, सामाजिक अपेक्षा आणि अनिश्चिततेची भीती आपल्याला नवीन संधींचा वापर करायला टाळाटाळ करायला भाग पाडते.

‘लॉस अ‍ॅव्हर्जन’ म्हणजेच ‘नुकसान टाळण्याचा पूर्वग्रह’. आपण नफा मिळण्यापेक्षा नुकसान टाळण्याला अधिक महत्त्व देतो. उदाहरणार्थ, शेअर बाजारात तोटा होण्याची भीती बाळगून अनेक लोक पैसे गुंतवत नाहीत, जरी दीर्घकालीन फायदा होण्याची शक्यता असली तरी.  काही दु:खी कुटुंबे पाहून, संसाराचा गाडा ओढताना होणारी दमछाक बघून अनेकजण लग्न करण्याचे टाळतायत. आपल्याकडे जे आहे (त्यातील सर्वात महत्वाचे म्हणजे ‘स्वातंत्र्य’) ते गमावण्याची भीती आपल्याला पुढे जायची हिंमत देत नाही. 

आपल्या आयुष्यात अनेकदा महत्त्वाचे निर्णय घ्यावे लागतात जसे की शिक्षण, गुंतवणूक, मतदान, व्यवसाय, लग्न इत्यादी. जर आपले निर्णय पूर्वग्रहांनी प्रभावित असतील, तर त्यातून नुकसान होऊ शकते. ते पूर्वग्रह ओळखणे महत्त्वाचे आहे. त्यांचा अभ्यास करणे आणि त्यांचा अडथळा ओळखणे आपल्याला अधिक विवेकपूर्ण बनवू शकते.
दिलेल्या उदाहरणांवरून आपण हे शिकू शकतो की थोडा तटस्थपणा आणि साक्षेपी भाव ठेवण्याची सवय लागल्यास तुम्ही प्रभावी विचार करणारे होऊ शकता.



\chapter{पाऊस पडेल काय?}

प्रसिद्ध हिंदी चित्रपट ‘शोले’तील ‘जय’ (अमिताभ बच्चन) ला कोठलाही निर्णय घेण्यासाठी छापा-का-काटा करण्याची सवय असते, आठवतंय? इथे भोळ्या ‘वीरू’ला वाटत असतं की, आपण जिंकण्याची शक्यता ५०% तरी आहे. तशाच पद्धतीने बुगू-बुगू भोलानाथला ‘पाऊस पडेल काय?’ असं विचारल्यावर त्याने मान डोलावली तर ‘हो’ नाहीतर ‘नाही’. येथेही खात्री नाही तर शक्यतेचा खेळ. क्रिकेट मॅच मध्ये कोण फलंदाजी आधी करणार हे पण नाणेफेकीवर ठरते. दोन्ही संघांना ‘टॉस’ जिंकण्याची समान संधी, म्हणजेच ५०% शक्यता. या सर्व उदाहरणांमध्ये खात्रीलायक काही न कळता शक्यता वर्तवल्या जात आहेत. आयुष्यातही अनेक वेळा निर्णयांमध्ये खात्रीशीर सांगता येत नाही. बहुतेक ठिकाणी असतो तो फक्त अंदाज. उदाहरणार्थ, उद्या पाऊस पडेल का हे अचूक सांगता येत नाही, म्हणून हवामान खातं सांगतं, “८०% शक्यता आहे” किंवा “२०% आहे.” पण आपण या शक्यतेचा वापर आपण कसा करायचा? छत्री न्यायची की नाही? सर्वसाधारणपणे विचार केला तर ५०%च्या वर शक्यता असेल तर ‘न्यायाची’ आणि कमी असेल तर ‘नाही’. पण प्रत्येक वेळी हे सूत्र लागू होत नाही. ते प्रसंगानुरूप बदलू शकते. समजा, तुम्हाला अतिमहत्वाच्या मीटिंगला जायचे आहे, भिजून चालणारच नाहीये तर २०% इतकी कमी शक्यता असूनसुद्धा तुम्ही छत्री न्याल आणि समजा मौजमजेसाठीच-ट्रेकिंगला बाहेर पडत असाल तर ८०% इतकी जास्त शक्यता असून सुद्धा छत्री नेणार नाहीत, बरोबर ना? अशा पद्धतीने संभाव्यतेचा-शक्यतेचा वापर करून निर्णय घेण्याचा मेंटल मॉडेल (मन:प्रारूप) ला म्हणजेच विचारचित्राला ‘प्रोबॅबिलिस्टिक थिंकिंग’ (संभाव्यतेचा विचार) म्हणतात. 

‘संभाव्यतेचा विचार’ वैयक्तिक निर्णयांपुरता मर्यादित नाही. तो सामूहिक निर्णयांमध्येही अत्यंत उपयुक्त ठरतो. उदाहरणार्थ, सरकार पावसाच्या अंदाजावरून पूर नियोजन करतं, विमा कंपन्या पीक वीमा दर ठरवतात, शहरांमध्ये वाहतूक नियोजन होतं, हे सगळं शक्यतांच्या आधारावर. अशाच प्रकारे ‘संभाव्यतेचा विचार’ याची इतर काही उदाहरणे पाहुयात. 

शेअर बाजारात किंवा म्युच्युअल फंडात गुंतवणूक करताना, कोणताही पर्याय खात्रीशीर परतावा देत नाही, हे लक्षात घेतलं पाहिजे. यशस्वी गुंतवणूकदार विविध शक्यतांचा विचार करतात जसे की आर्थिक परिस्थिती, कंपनीची कामगिरी, इत्यादी. यावरून भाव वर-खाली जाण्याच्या शक्यता जोखतात व आपल्या जोखीम-क्षमते (रिस्क ऍपेटाइट) नुसार निर्णय घेतात. 

पालक म्हणून तुम्ही तुमच्या सातवीतील मुलाला कोडिंग क्लासमध्ये घालावे का? यामुळे तो भविष्यात तंत्रज्ञान क्षेत्रातच जाईल याची कोणतीही खात्री नाही, पण जर अगदी कोडिंग नाही तर संगणकीय विचारांची लवकर ओळख झाल्यामुळे भविष्यातील योग्यतेची ‘शक्यता वाढत असेल’, तर तो एक योग्य निर्णय असू शकतो. समजा आधीच कळले की त्याला ते काही आवडत नाहीये किंवा जमत नाहीये तेंव्हा मार्ग बदलून दुसऱ्या मार्गाचा विचार करता येऊ शकतो. 

ऐच्छिक शस्त्रक्रिया किंवा प्रतिबंधात्मक चाचण्या अनेकदा “जोखमीचे घटक” पाहून निवडल्या जातात. जर हस्तक्षेप न केल्यास हृदयविकार १०% वाढण्याची शक्यता असेल तर आताच कृती करण्यासाठी ही जोखीम पुरेशी आहे का? वय जास्त असेल तर गुंतागुंत (कॉम्प्लिकेशन्स) वाढण्याची शक्यता किती? तज्ञ डॉक्टरच हे स्वीकार्य पातळी (थ्रेशोल्ड) च्या खाली का वर आहे ते ठरवून निर्णय घेऊ शकतात. 

नोकरी करावी की स्टार्टअप? यामध्येही संभाव्य यशाच्या शक्यतेचा विचार केला जातो. स्वतः:चा स्वभाव, बाजारपेठ, कौटुंबिक आधार, आणि स्वयंचलन (ऑटोमेशन) ची जोखीम हे सगळे घटक निर्णय ठरवतात.

उत्तर प्रदेशातील मतदान सर्वेक्षण असो वा तामिळनाडूमधील युतीबद्दलचे अंदाज, निवडणुका म्हणजे शक्यतांचाच खेळ असतो. पण मतदार अनेकदा ६०% विजयाच्या अंदाजाला निश्चित निष्कर्ष समजतात आणि  हे विसरून जातात की उरलेली ४०% शक्यता प्रत्यक्षात येऊ शकते आणि अनेकदा येतेही, विशेषतः भारताच्या वैशिष्ट्यपूर्ण (जनतेच्या भल्यासाठी काहीही!!) राजकीय परिस्थितीत.
आपण बरेचदा निश्चिततेच्या शोधात असतो. आपल्याला अचूक उत्तरं हवी असतात. पण जग हे शक्यतांवर आधारित असतं. संभाव्यतेचा विचार आपल्याला चांगले निर्णय घेण्यास मदत करतो, अचूक अंदाजाची हमी देत नाही. तो आपल्याला अवास्तव आत्मविश्वास टाळायला, अपेक्षा व्यवस्थापित करायला आणि जोखीम टाळायला शिकवतो.
जीवन हे हमखास निष्कर्षांची मालिका नसून, संधी आणि अंदाजांवर आधारित एक खेळ आहे. आपल्याला चांगल्या पैजा लावता आल्या पाहिजेत. आणि विशेषतः भारतात, जिथे मान्सून, बाजारपेठा, राजकीय समीकरणे, आणि माणसांचे वर्तन हे सर्व फारच अनिश्चित आहे, तिथे शक्यता समजून निर्णय घेणं ही फक्त शहाणपणाची नाही, तर आवश्यकतेची गोष्ट आहे.
डॉ. योगेश हरिभाऊ कुलकर्णी




\chapter{उथळ पाण्याला खळखळाट फार}

दरवर्षी फेब्रुवारी महिना येऊ लागला की, देशात अर्थतज्ज्ञांची लाट येते. कारण अर्थसंकल्प घोषित होणार असतो म्हणून.  केंद्रीय स्पर्धा परीक्षेची तयारी करणारा रमेशही त्याला कसा अपवाद असेल? तो कररचना, अनुदान आणि वित्तीय धोरणांवर आत्मविश्वासाने बोलायला लागतो, अगदी अर्थमंत्र्यांच्या अविर्भावात. एक-दोन महिन्यात, जेव्हा आयपीएलचा हंगाम असतो तेव्हा रमेश क्रिकेट-विश्लेषक होऊन जातो. टीव्हीवर खेळपट्टी बघून ती कोणाला साथ देईल याचा अंदाज तो छातीठोकपणे सांगतो. जगात कोठेही युद्धाचे वारे वाहायला लागले की रमेश सामरिक-नीती तज्ज्ञ म्हणून तयार. कोणी, कुठून आणि किती सैन्य तैनात केले पाहिजे याचा सखोल आराखडाच तो जाहीर करतो. एवढ्या विविध विषयात मुशाफिरी करणाऱ्या रमेशला त्याच्या वैयक्तिक आयुष्यात ‘प्रीलिम परीक्षा’ काही सुटत नाहीये, अनेक वर्षे झाली तरी. त्याला अभ्यासक्रमातलं विचारलं की ‘वित्तीय तूट आणि व्यापारी तूट यातील फरक काय?’ तर तो अजूनही गडबडतो. त्याची विद्वत्ता व्हॉट्सऍपचे व्हिडीओ फॉरवर्ड करून, चहा पीत त्यावर जोरजोरात चर्चा कारण्यापुरतीच आहे खरे तर, पण आव असा आणतो की विचारू नका. मला नक्की वाटते की आमचा रमेश हे काही अपवादात्मक उदाहरण नाहीये, तुमच्या आसपास असे अनेक ‘बहूआयामी तज्ज्ञ’ तुम्ही पहिले असतील. अशा अति-आत्मविश्वासी पण पोकळ विद्वत्तेच्या उदाहरणास ‘डनिंग-क्रुगर इफेक्ट’ मेंटल मॉडेल (मन: प्रारूप अथवा विचार-चित्र) म्हणतात. सोप्या भाषेत, मराठीतील म्हणीनुसार ‘उथळ पाण्याला खळखळाट फार’. हा एक ‘कॉग्निटिव्ह बायस’चा (पूर्वग्रहदूषित विचारसरणी) प्रकार आहे जेथे अत्यल्प ज्ञान किंवा क्षमता असणाऱ्यांना आपण ‘लयं भारी’ असा जोरदार आत्मविश्वास असतो.
१९९९ साली मानसशास्त्रज्ञ डेव्हिड डनिंग आणि जस्टिन क्रुगर यांनी हे मेंटल मॉडेल जगासमोर आणले. यामागची कहाणी गंमतीशीर आहे.  एका माणसाने चेहऱ्यावर लिंबाचा रस लावून दोन बँका लुटल्या, कारण त्याचा विश्वास होता की त्यामुळे तो सीसीटीव्ही कॅमेर्‍यांना दिसणार नाही! तो ना वेडा होता, ना नशेत होता, फक्त त्याला स्वतःच्या अज्ञानाची अजिबात जाणीव नव्हती. वेगळ्या भाषेत सांगायचं तर, "जितकं कमी माहिती असतं, तितकं आपल्याला माहित नसल्याचंही माहित नसतं." जे लोक नवखे असतात, त्यांच्याकडे स्वतःच्या कौशल्याचं अचूक मूल्यांकन करण्यासाठी पुरेशी माहिती नसते, त्यामुळे ते स्वतःला खूप हुशार समजतात. उलट जे खरे तज्ज्ञ असतात, ते अनेकदा स्वतःच्या क्षमतेबद्दल संकोच करतात, कारण त्यांना वाटतं की इतरांनाही तितकंच ज्ञान आहे.
भारतीय संदर्भात हा परिणाम अनेक ठिकाणी दिसतो. स्पर्धा परीक्षा देणाऱ्यांमध्ये, टीव्ही-सोशल मीडियावरील चर्चांमध्ये, आणि व्हॉट्सअ‍ॅप युनिव्हर्सिटीमधून पदवी घेतलेले लोक भौगोलिक राजकारण, वैद्यकीय शास्त्र आणि संविधान यावर निर्विवाद मतं मांडताना दिसतात. याची काही दैनंदिन आयुष्यात दिसणारी उदाहरणे पाहुयात. 
कोविड-१९ च्या काळात, कितीतरी लोक कोणती औषधं चालतात हे ठामपणे सांगत होते. घरगुती उपाय, ‘आजीबाईच्या बटव्यातील’ औषधे, वेगवेगळे काढे सर्रास सुचवले जात होते, त्यांच्याकडे वैद्यकीय पार्श्वभूमी नसतानाही. 
कंपन्यांमधील नुकतेच रुजू झालेले (फार नावाजलेल्या कॉलेजमधून शिकलेले) कर्मचारी अनेकदा बैठकींमध्ये अतिसोप्या कल्पना फार ठामपणे मांडतात, त्या मागच्या गुंतागुंतीची त्यांना जाणीव नसते. त्याचवेळी अनुभवी व्यवस्थापक मात्र विचारपूर्वक निर्णय घेतात, जोखमी समजून पुढे जातात. बरेचदा निर्णय हे केवळ तांत्रिक अथवा आर्थिक बाबींवर अवलंबून नसतात, मानवी संबंध, कंपनीतील राजकारण हे भारी पडत असते आणि तेथे अनुभवाशिवाय गत्यंतर नसते. 
पहिल्यांदाच व्यवसाय करणारे काही उद्योजक, आपल्या अजमावून न पाहिलेल्या कल्पनांनी गुगल-मायक्रोसॉफ्टसारखी कंपनी उभी करणार असल्याचा आत्मविश्वास बाळगतात. केवळ सादरीकरण आणि दुर्दम्य विश्वास घेऊन ते बाजारात उडी घेतात. ज्याचा अभ्यास केला नसतो काही वर्षे ‘घासली’ नसतात अशा क्षेत्रात केवळ ‘इंप्रेशन’ महत्वाचे ठरत नाही, तर काम दाखवावे लागते. 
डनिंग-क्रुगर इफेक्ट समजून घेण्यामागचा उद्देश इतरांची खिल्ली उडवणे नाही, तर स्वतःकडे डोळसपणे पाहणे हा आहे. आपल्यापैकी प्रत्येकालाच काही ना काही गोष्टी अज्ञात असतात. विशेषतः नवीन क्षेत्रांमध्ये. सगळेच काही ‘इलॉन मस्क’ नसतात की जेथे जाईल त्यात यश मिळवेल. सर्वसामान्यांना यशासाठी गरज असते ती अपार कष्टाची, बौद्धिक क्षमतेची आणि महत्वाचे  म्हणजे नम्रतेची; म्हणजेच "मला माहित नाही" हे स्वीकारण्याची आणि त्यात लाज न वाटण्याची क्षमता.
आपण या मेंटल मॉडेलने ग्रासलो आहे का हे शोधण्यासाठी स्वतःला काही प्रश्न विचारूया. ‘या विषयावर आत्मविश्वासाने बोलायला माझ्याकडे पुरेसा अनुभव आहे का?’, ‘मी अशा लोकांशी चर्चा केली आहे का, जे माझ्याशी असहमत आहेत किंवा ज्यांना अधिक माहिती आहे?’, ‘मी ठाम बोलतोय, पण हे खरंच ज्ञानावर आधारित आहे का’
भारतीय परंपरेतील एक सुंदर विचार आहे,  खरा विद्वान तो नाही की ज्याला सगळं माहित आहे, तर तो आहे ज्याला हे माहित आहे की "आपल्याला अजून खूप काही माहित नाही."
आजच्या सोशल मीडियाच्या युगात प्रत्येकाकडे एक माईक आहे, ऑनलाइन किंवा ऑफलाइन. सर्वांना ‘व्यक्त’ होण्याची मुभा आहे. अशा काळात, डनिंग-क्रुगर इफेक्ट ही केवळ सावधगिरीची गोष्ट नाही, तर दिशा दाखवणारा एक मानसिक होकायंत्र आहे. तो आपल्याला सांगतो की, बोलण्याआधी थांबून विचार कर, लिहिण्याआधी वाच, नेतृत्व करण्याआधी कार्यकर्ता हो आणि सल्ला देण्याआधी अनुभव घे. सरकार असो, कार्यालय, शाळा किंवा घर, चांगले निर्णय घ्यायचे असतील, तर आत्मविश्वास आणि कौशल्य यामधील अंतर ओळखणं आणि भरून काढणं आवश्यक आहे, नाही का?



\chapter{निर्हेतुक चुका, ‘क्षमापात्र’}

आज सकाळची वेळ गडबडीची होती. ८:३० वाजले होते. अर्ध्या तासात एक महत्त्वाची मीटिंग गाठायची होती. मी रिक्षा केली आणि लवकरच लक्षात आले की, रिक्षावाला जवळचा रस्ता न घेता चकवा देतोय. मनात शंका आली की, हा मुद्दाम `फसवतोय` का? मी स्थानिक असूनही मला परगावचा समजून लांबचा रस्ता घेतोय? `मी सांगितलेला रस्ताही त्याला कळत नव्हता.` दोघांत चांगलीच बोलाचाली झाली. ‘चार शब्द’ ऐकवले गेले. पोहोचलो, थोडा उशीर झाला, पण ठीक होते. आमच्या ‘संभाषणात’ असे कळले की तोच परगावाचा होता, कुटुंबाला गावी सोडून आलेला. नवीन शहर आणि मागे राहिलेल्या घराची विवंचना याने त्याचे चित्त थाऱ्यावर नसेल म्हणून `त्याच्याकडून` रस्ता लांबचा घेतला गेला. `नंतर माझ्या लक्षात आले` की, त्याचा काही फसवण्याचा हेतू नव्हता. उगाच बोललो. `सोडून द्यायला हवे होते.` त्याची चूक निर्हेतुक असल्याने `क्षमापात्र` होती, नाही का? माझ्याच जीवाची उगाचच घालमेल झाली. अशा ताण आणणाऱ्या, गैरसमज होणाऱ्या छोट्या-छोट्या गोष्टी `टाळून` जीवन साधे `आणि` सोपे करायला हवे.

आपल्या रोजच्या आयुष्यात अशा घटना वारंवार होतात. मंदिरात कोणी रांगेत घुसतं, ऑफिसमधला सहकारी महत्त्वाच्या मेसेजला उत्तर देत नाही, कॉलनीतील क्रिकेट खेळताना एखादी ‘सिक्स’ तुमच्या खिडकीवर येते, आणि कोणी माफीही मागत नाही. तेव्हा आपले मन `पटकन` निष्कर्ष काढते की, “हे मुद्दामच केले गेले असेल.” पण खरंच असे असते का? बहुतेक वेळा कारण वेगळे असते: विसरभोळेपणा, अकार्यक्षमता, दुर्लक्ष किंवा साधा मूर्खपणा. 

इथे ‘हॅनलॉनचे रेझर’ हे मेंटल मॉडेल (मनःप्रारूप) अथवा विचारचित्र मदतीला येते. हे विचारचित्र रॉबर्ट जे. हॅनलॉन यांनी मांडले. याच आशयाची कल्पना इतिहासात अनेकांनी, अगदी इ.स. १७७४ पासूनच मांडलेली आहे. नेपोलियन बोनापार्टने एकदा म्हटले होते की, ‘`जे अकार्यक्षमतेमुळे घडल्यासारखे वाटते, त्यामागे वाईट हेतू आहे असे समजू नका’. ‘रेझर’ म्हणजे दाढीचे ब्लेड. इथल्या संदर्भात सांगायचे तर, हे असे तत्त्व आहे, जे अनावश्यक कारणे कापून टाकते आणि साधे उत्तर निवडायला सांगते. म्हणजेच, लोकांमध्ये वाईट हेतू शोधण्याऐवजी, आधी साधे कारण शोधा. हे तत्त्व जरी पश्चिमेत जन्माला आले असले, तरी भारतासारख्या १.४ अब्ज लोकांच्या देशात याचे महत्त्व फार मोठे आहे. इथे वाहतुकीत, सरकारी कार्यालयात, एकत्र कुटुंबात आणि समाज माध्यमांमध्ये आपला संयम रोज तपासला जातो. अशा वेळी हे तत्त्व फक्त एक वरवरचा, गमतीशीर विचार नसून, एक महत्त्वाचा, जगण्यासाठी उपयोगी विचार ठरतो. याची काही आणखी उदाहरणे पाहूयात.

तुम्ही बॉसला एक रिपोर्ट पाठवता आणि उत्तरच येत नाही. लगेच मनात येते की, “मी काही चुकलो का?” `पण खरे कारण असू शकते की, बॉस सतत मीटिंगमध्ये असतील, तुमचा मेल स्पॅममध्ये गेला असेल किंवा (आपले आवडते कारण) ते विसरभोळे असल्याने त्यांना पुन्हा आठवण करून द्यावी लागेल.

तुमच्या खास मित्राने तुमचा वाढदिवस लक्षात ठेवला नाही. तुम्हाला वाटेल, “त्याला काही फरकच पडत नाही!” पण शक्यता आहे की, त्याच्या कामाच्या व्यापात किंवा सोशल मीडियावरील शेकडो मेसेजच्या गर्दीत तुमच्या वाढदिवसाची आठवण राहिली नसेल. 

एखाद्या गावात रस्त्यांची आणि नाल्यांची दुर्दशा पाहून आपल्याला वाटते, की हे मुद्दाम करतायत का? एवढा निधी खर्च होऊनही उपयोग का होत नाही? त्यामागे असेही कारण असू शकते की, संबंधित अधिकारी स्थापत्य अभियंता (सिव्हिल इंजिनियर) नसतील. कदाचित कला, इतिहास किंवा भाषा यांसारख्या विषयातून स्पर्धा परीक्षा उत्तीर्ण होऊन त्यांची नियुक्ती या पदावर झालेली असेल. ते खूप प्रयत्न करत असतील, पण तो त्यांचा विषयच नसल्याने कामाचा दर्जा चांगला नसेल.

आपल्या देशात लोकांमध्ये आपुलकी आहे, पण संयम कमी आहे. त्यामुळे छोटेसे दुर्लक्षदेखील आपण मोठा अपमान समजतो. हॅनलॉनचे रेझर आपल्याला विचारायला शिकवते की, “यामागे दुसरे काही साधं कारण असेल का?” हे तत्त्व चुकीचे वागणे माफ करायला सांगत नाही, पण आपल्या मनाचा ताण नक्कीच कमी करायला मदत करते. शक्य तिथे चूक लक्षात आणून द्यावी आणि सरकारी अनास्थेबद्दल आवाज उठवावा, पण परिस्थिती समजून घेण्याचाही प्रयत्न करावा. इतरांच्या चुकीत खोडसाळपणा शोधण्याऐवजी, आपण जर थोडे समजून घेतले, तर आपल्यालाच मनःशांती मिळते. पुढच्या वेळी एखादा रिक्षावाला तुमच्या अपेक्षेपेक्षा वेगळा रस्ता घेत असेल, तेव्हा असा विचार करा, कदाचित त्या रस्त्यावर मेट्रोचे काम सुरू असेल किंवा तुम्हाला माहीत नसलेला एखादा मोठा खड्डा पडला असेल. तो फसवतोय असे नाही, तोही तुमच्यासारखाच, रोजच्या गरजा भागवण्यासाठी धडपडणारा एक सर्वसाधारण माणूस आहे.




\chapter{समग्र विचारांची शक्ती }

पावसाळा सुरू झाला की नेहमीचं दृश्य दिसतं, रस्त्यांवर पाणी साठलेलं. मुंबई, पुणे आणि बंगलोरसारखी मोठी शहरेही याला अपवाद नाहीत. खरंतर या महत्वाच्या आणि जागतिक व्यवहार असलेल्या शहरांकडून अधिक चांगल्या नियोजनाची अपेक्षा असते. पण मग जनतेला 'वॉटर पार्क'चा चकटफू अनुभव कसा मिळणार? स्थानिक नेत्यांना जनतेवर असेलेली कणव दाखवण्यासाठी बिस्किटवाटपाची संधीही कशी मिळणार? खरंतर, नालेसफाई, ड्रेनेज नियोजन आणि अंमलबजावणीवर वेळेवर लक्ष दिलं गेलं असतं, तर अशा गोष्टी टाळता आल्या असत्या. सध्या जनतेचा आक्रोश आणि तंग वातावरण निवळण्यासाठी इकडे-तिकडे थोडी डागडुजी केली जाते, बैठका होतात, जबाबदारी निश्चित करण्याचे आश्वासन दिले जाते, बस्स, नंतर पुढच्या वर्षी, पहिले पाढे पंचावन्न.

या समस्या तितक्या क्लिष्ट नाहीत. नीती, निधी आणि तंत्रज्ञान असूनही त्या का टिकून राहतात? कारण आहे आपल्या तुटक, तात्कालिक विचारसरणीत. आपण समस्यांची लक्षणं वेगवेगळी समजून घेतो, पण त्या मूळ कारणांचा गुंता सोडवायला टाळतो. खोलात जाऊन, बहुअंगी कारणमीमांसा टाळतो कारण त्यासाठी वेळ, बुद्धी आणि समन्वय लागतो. तातडीनं काहीतरी ‘करून दाखवलं’ म्हणून जाहिरात केली जाते आणि समग्र विचार टाळला जातो. पण अनेक वेळा स्पष्ट झालं आहे की, ‘सिस्टिम्स थिंकिंग’ हे मेंटल मॉडेल (मन:प्रारूप) ) अथवा विचारचित्र येथे खूप गरजेचं ठरतं.

सिस्टिम्स थिंकिंग हा जगाकडे पाहण्याचा एक विशिष्ठ दृष्टिकोन आहे. कुठलीही गोष्ट अथवा घटना किंवा एक स्वतंत्र घटक म्हणून नव्हे, तर परस्परांवर परिणाम करणाऱ्या, एकमेकांशी जोडलेल्या घटकांच्या रूपात अभ्यास करणे हे या विचारचित्राचे तत्व आहे. म्हणजे जसं एखादं जंगल समजून घ्यायचं, केवळ एखादे झाडं किंवा एखादा प्राणी अभ्यासून चालणार नाही तर, इतर सर्व घटक, त्यांचे परस्परांवर असलेले परिणामही लक्षात घ्यावे लागतील. बॉस्टन येथील एमआयटीचे जे. फॉरेस्टर यांनी ‘सिस्टम डायनॅमिक्स/थिंकिंग’ या शास्त्रशाखेची मांडणी केली तर चार्ली मंगर यांनी याला मेंटल मॉडेल्सच्या चौकटीत मान्यता दिली. याची आपल्या सभोवतालची काही उदाहरणे पाहुयात. 

भारतामध्ये सिस्टिम्स थिंकिंगचा अभाव दिसतो तो आपल्या नद्यांच्या स्वच्छता मोहिमांमध्ये. उदाहरणार्थ, १९८६ मध्ये सुरू झालेली ‘गंगा अ‍ॅक्शन प्लॅन’ ही योजना काही दशकांनंतर आणि हजारो कोटी रुपयांनंतरही पूर्णत्वास गेलेली दिसत नाही. गंगा अजूनही प्रदूषितच आहे. का? कारण आपण फक्त प्रामुख्याने सांडपाण्याच्या प्रकल्पांवर लक्ष केंद्रित केलं, पण वरच्या पातळीवरचं कचराप्रबंधन, स्थानिक प्रशासन आणि लोकांची सवय बदलणं हे सर्व घटक बऱ्यापैकी दुर्लक्षित केले. ते सगळं एकाच ‘सिस्टम’चा भाग आहे त्या सर्वांचा विचार करून उपाययोजना ठरवल्या पाहिजेत.

ग्रामीण भागात कॉम्पुटर-टॅबलेट वाटप केलं जातं. उद्दिष्ट चांगलं आहे. पण नियमित वीज नाही, शिक्षक प्रशिक्षित नाहीत, आणि अभ्यासक्रम कालबाह्य असल्यामुळे उपयोग होत नाही. बदल हवा असेल, तर पोषण, शिक्षक प्रेरणा, पायाभूत सुविधा आणि पालक साक्षरता यांसह पूर्ण पार्श्वभूमी विचारात घ्यावी लागेल.

वाहतूक कोंडी रोखण्यासाठी उड्डाणपूल बांधले जातात. काहीसा दिलासा मिळतो. पण लोकसंख्येचा वाढता लोंढा आणि वाहनांची वाढ यामुळे ते उपाय अल्पकालीन ठरतात. फक्त रस्ते नव्हे, तर शहर नियोजन, सार्वजनिक वाहतूक, आणि कडक नियमांची अंमलबजावणी गरजेची आहे.

भारतातील शेती आजही मोठ्याप्रमाणात नैसर्गिक पाण्यावर, सरकारी अनुदानावर, सावकारी पतपुरवठ्यावर आणि बाजारभावाच्या अनिश्चिततेवर तोल सांभाळत आहे. थोडे काही चुकले तर गरीब शेतकऱ्यापुढे आत्महत्येशिवाय पर्याय उरत नाही. कर्जमाफी ही तात्पुरती मलमपट्टी आहे. सिस्टिम्स थिंकिंगने विचार केल्यास ‘शेती ही केवळ पेरणी आणि कापणी नाही’, तर एक आर्थिक, सामाजिक, (कधीकधी राजकीय) आणि नैसर्गिक जाळं आहे. त्यासाठी सिंचन, बाजारपेठा, जमीनधारण, आणि पीकविमा अशा सर्व पातळ्यांवर विचार करणं गरजेचं आहे.

आजचा वेगवान काळ आपल्याला त्वरेने निर्णय घ्यायला भाग पाडतो. पण सिस्टिम्स थिंकिंग सांगते की, गडबडीत चुकण्याची शक्यता वाढते. योजना आखताना ‘फीडबॅक लूप्स’ (प्रतिक्रया-पडसाद तपासणी-बदल), ‘डिलेज ‘ (विलंब, कालांतराने होणारे परिणाम), आणि ‘इमर्जन्स‘(अनपेक्षित परिणाम) यांची जाण असणं गरजेचं आहे.

भारतीय जुगाड ही कल्पकता असली, तरी तो समग्र विचारांशिवाय कधी कधी उलट परिणाम देतो. खरी प्रगती हवी असेल, तर तात्पुरते उपाय करणारे नव्हे, तर संपूर्ण चित्र पाहणारे विचारवंत होणं गरजेचं आहे. पुढच्या वेळी एखादी समस्या दिसली, तेव्हा "काय बिघडलंय?" एवढंच न विचारता, "या मागची संपूर्ण प्रणाली काय आहे?" हा प्रश्न स्वतःला विचारणार का? 




\chapter{ताजे स्मरण...निर्णयास कारण }

नुकताच एक मोठा आणि अतिशय दुर्दैवी विमान अपघात झाला. विमानातील एक सोडून सर्वच मृत्युमुखी पडले. विमान जिथे कोसळले, तिथेही काहींचा मृत्यू झाला, अनेक जखमी झाले. हे सर्व पाहून, ऐकून मन विषण्ण होते. जीवन किती क्षणभंगुर आहे, याची जाणीव होते. याचा आणखी एक परिणाम म्हणजे आपल्याला विमान प्रवासाची तीव्र भीती वाटू शकते. तथापि, आकडेवारी पाहता, कार अपघातात मृत्यूची शक्यता विमान अपघातांपेक्षा खूपच जास्त आहे. तरीही, विमान अपघाताची कल्पना अधिक भयावह वाटते. का?

या प्रश्नाचं उत्तर आहे‘अ‍ॅव्हेलेबिलिटी ह्युरिस्टिक’ नावाच्या मेंटल मॉडेल (मन:प्रारूप) अथवा विचारचित्रामध्ये. मनातील विचारांमध्ये 'हाजीर तो वजीर' या तत्त्वानुसार, नुकतेच आलेले विचार अधिक प्रभावी ठरतात. म्हणजेच या मॉडेलचा महत्त्वाचा पैलू आहे,'रिसेन्सी इफेक्ट'. याचा अर्थ, जे अलीकडे घडलंय, त्याचा आपल्या निर्णयांवर अधिक प्रभाव पडतो. आपला मेंदू निर्णय घेताना नेहमी खोलवर विचार करत नाही. अनेकदा लगेच आठवणीत येणाऱ्या गोष्टींवर आधारित निर्णय घेतो. ‘हाजीर तो वजीर’ हे विचारचित्र याच विषयी आहे. जे लक्षात ताजं असतं, ते प्रभाव टाकतं, पण ते आकडेवारीनुसार खरं असतंच असं नाही.

खरंतर हे मेंटल मॉडेल ही एक मानसिक सवय आहे. एखादी घटना लक्षात राहिली की आपण समजतो ती फार महत्त्वाची आहे, नेहमीच घडत असणार. ती डोक्यात खोलवर रुजते आणि पुढचे निर्णय त्यावरच आधारित होतात. आपले मेंदू शक्यता (प्रोबॅबिलिटी) मोजत नाहीत, तर कधी काय ऐकले गेले, हे लक्षात ठेवतात. भभारतासारख्या देशात, जिथे सोशल मीडियावर, कौटुंबिक चर्चांमध्ये आणि व्हॉट्सअ‍ॅप मेसेजेसमधून सतत काही ना काही ऐकायला मिळतं, तिथे अलीकडे ऐकलेल्या गोष्टींचा प्रभाव अधिक असतो. हा प्रभाव वैयक्तिक निवडींपुरताच मर्यादित नसून, धोरणांवरही पडतो. त्याची काही उदाहरणे पाहूया. 

निवडणुका काळाच्या थोडेच आधी सर्व पक्षांकडून लोकप्रिय घोषणांची खैरातच सूरु होते. ती प्रलोभने समाज मनात ताजी असतानाच मत दिलं जात असल्याने फायदा होतो ही यामागची अपेक्षा असते आणि अशातऱ्हेने आधीच्या चारवर्षांच्या अपयशावर सहजपणे पांघरूण टाकता येते.

कोविड काळात शेअर बाजार कोसळला होता. तेव्हा अनेकांनी घाबरून शेअर्स, म्युच्युअल फंड विकले होते. कुणाला हे लक्षात आलं नाही की (अगदी खात्री देता येत नसली तरी) शेअर बाजार तसेच सेन्सेक्स हा सर्वसाधारणपणे दीर्घकाळात सातत्याने वाढत असतो. नजीकच्या काळातील मथळे वाचून निर्णय होतात, शुद्ध आकड्यांच्यावर, पुराव्याच्या माहितीवर (डेटा) नव्हे.

जाहिराती या सतत आपल्या डोळ्यासमोर दिसत अथवा कानावर पडत असल्याने, एखादी गोष्ट विकत घेताना त्या जाहिरातीचीच आठवण होऊन तेच उत्पादन विकत घेतले जाते ना की त्याच्या गुणधर्माचा अभ्यास करुन. आपल्याला कधी कधी त्रास होतो, चीड येते, पण तरीही त्या दाखवल्या जातात कारण त्या नाकारात्मक भावनेतही त्या लक्षात नक्की राहतात आणि याचा उत्पादन विक्रीत फायदा होतो. 

आयआयटी किंवा बिट्समधून शिकणारा एखादा विद्यार्थी कॉलेज सोडून स्टार्टअप करतो आणि यशस्वी होतो. ही कहाणी ऐकून अनेकजण नोकरी-शिक्षण सोडतात. पण यशस्वी उदाहरणं मोजकीच असतात. बहुसंख्य अपयशी प्रयत्न दिसत नाहीत.

नामांकित विद्यापीठांतील विद्यार्थ्यांना मिळणाऱ्या कोट्यवधीच्या पगाराचे आकडे पाहून, पालक आपल्या आठवीतल्या मुलालाही जेईई क्लासला घालतात.मात्र त्या मुलाचा कल, बौद्धिक क्षमता आणि पुढचं वास्तव लक्षात घेतलं जात नाही.

या सगळ्या उदाहरणांमधून एक स्पष्ट धोक्याची घंटा ऐकू येते, ‘नजीकता’ आणि ‘महत्त्व’ यात गोंधळ होण्याची शक्यता. ‘अ‍ॅव्हेलेबिलिटी ह्युरिस्टिक’ हे मॉडेल हा धोका टाळायला शिकवतं. आपण नजीकच्या (पण क्वचित घडणाऱ्या) घटनांवर विश्वास ठेवतो आणि दररोज जीवघेण्या ठरणाऱ्या गोष्टी, जसं की प्रदूषण, मधुमेह, बेदरकार ड्रायव्हिंग यांकडे साफ दुर्लक्ष करतो.

मग यावर उपाय काय? थांबा आणि स्वतःला विचारा की ही गोष्ट खरंच सामान्य आहे का, की फक्त अलीकडे घडल्यामुळे लक्षात राहिली आहे? यावरील माहिती (डेटा) शोधा. भावनांपेक्षा आकड्यांवर आणि पुराव्यावर विश्वास ठेवा. काही गृहितकं (ऍझम्पशन्स) केली आहेत का ते तपासा. कारण जर असं नाही केलं, तर आपण आकाशातून कोसळणाऱ्या विमानाला घाबरू, पण आपल्याच पायाखालच्या रस्त्यातल्या खड्ड्याकडे दुर्लक्ष करू.





\chapter{जावे त्याच्या वंशा तेव्हा कळे}

भारतात लोकांचा सरकारी यंत्रणांवरील विश्वास केवळ अकार्यक्षमतेमुळेच नाही, तर जबाबदारीच्या अभावामुळे देखील खिळखिळा झाला आहे. जरा बारकाईने पाहा. आरोग्यमंत्री स्वतःच्या कुटुंबीयांवर उपचार नामांकित खासगी रुग्णालयांमध्ये करून घेतात, तर करोडो सामान्य लोक सरकारी रुग्णालयांतील गर्दीत वाट पाहत राहतात. शिक्षणमंत्री आपली मुले सरकारी शाळांमध्ये घालण्याचे सोडून, ‘पॉश’ आंतरराष्ट्रीय शाळांमध्ये घालतात. मातृभाषेचा आग्रह धरणे मान्यच आहे पण त्याच्या अंमलबजावणीस हिंसक स्वरूप देणाऱ्यांची मुले कुठल्या माध्यमांच्या शाळेत शिकतात? हे सर्व बघून प्रसिद्ध विचारवंत आणि लेखक नसीम निकोलस तालेब यांनी विचारलेला एक धारधार प्रश्न आठवतो, "जो आर्किटेक्ट-बिल्डर स्वतः निर्मिलेल्या इमारतीत राहायला तयार होत नाही, त्याच्यावर तुम्ही विश्वास ठेवाल का?" उत्तर अगदी स्पष्ट आहे, पण हा साधा नियमही अनेकदा दुर्लक्षित केला जातो. तालेब यांचे तत्त्व आहे की, जो सल्ला देतो किंवा निर्णय घेतो, त्याच्यावर त्या निर्णयांचे प्रत्यक्ष परिणामही यायला हवेत, तरच तो त्याचे काम गांभीर्याने करेल. हीच संकल्पना आहे ‘स्किन इन द गेम’ या मेंटल मॉडेल (मन:प्रारूप) अथवा विचारचित्राची. जर एखाद्याला त्याच्या निर्णयाच्या यशाचं श्रेय मिळत असेल, तर संभाव्य अपयशाची जबाबदारीही त्यानेच घ्यायला हवी. रस्ते विभागाच्या अधिकाऱ्यांना, त्यांनी ‘पास’ केलेल्या रस्त्यांवरून दुचाकीवरून नेहमी प्रवास करायला भाग पाडले तरच रस्त्यांची गुणवत्ता सुधारेल, नाही का? या मेंटल मॉडेलची इतर काही उदाहरणे पाहुयात. 
जेव्हा एखादा संस्थापक स्वतःचे पैसे आपल्या स्टार्टअपमध्ये गुंतवतो, तेव्हा गुंतवणूकदार त्याच्यावर अधिक विश्वास ठेवतात. कारण तेथे ‘स्किन इन द गेम’ हे तत्व लागू पडते. याच्या उलट, काही कंपन्यांचे प्रमुख (सीईओ), धंद्यात कितीही नुकसान होत असलं तरी भरघोस पगार घेत राहतात आणि वेळप्रसंगी कंपनी डुबवून निघूनही जातात कारण त्यांची वैयक्तिक काहीच गुंतवणूक नसते. 
गावोगावी शाळांमध्ये ‘पोषक आहार योजना’ चालवल्या जातात. त्यात अन्न  बनवण्यासाठी कोणी बाहेरचा आचारी नेमला जात नाही. विद्यार्थ्यांच्या आयाच ते काम आळीपाळीने करतात. आपण बनवलेले अन्न  आपलेच मूल खाणार असल्याने चांगला दर्जा राखला जातो. 
काही वित्त-विश्लेषक-सल्लागार फक्त इतरांना सल्ले देतात, पण त्यांनी सुचवलेल्या ठिकाणी स्वतः  पैसे गुंतवत नाहीत. अशा सल्ल्यांचा काय उपयोग? त्यांचं ‘स्किन’ तिथे नसतं. पण जे विश्लेषक स्वतः त्या शेअर्समध्ये गुंतवणूक करतात, त्यांच्या शिफारसी अधिक जबाबदार, विचारपूर्वक असतात. त्यांच्या निर्णयांचा परिणाम त्यांच्या स्वतःच्या खिशावर होतो  त्यामुळे सल्ला देखील प्रामाणिक असतो.
एखादा गैरसरकारी संस्थेचा (एनजीओ) कार्यकर्ता जर शहरातून येऊन ग्रामीण भागात दोन दिवस कार्यक्रम करून निघून जात असेल तर त्याला काय समजणार? पण जो कार्यकर्ता त्या गावात राहतो, तिथले रस्ते, दवाखाने, वीज याच्या चांगल्या-वाईट गोष्टींचा भाग असतो, त्याचे निर्णय त्याच्या योजना अधिक समर्पक असतात.
आपल्या आदरस्थानी असलेले राजे, छत्रपती शिवाजी महाराज, राणा प्रताप, थोरले बाजीराव हे महान योद्धे स्वत: लढाईत अग्रेसर असायचे. त्यांनी आखलेल्या समरनीतीचा, योजनांचा थेट परिणाम त्यांच्या आयुष्यावरही होणार असल्याने निर्णयांची संपूर्ण जबाबदारी त्यांच्यावर असायची. ही असामान्य धडाडी पाहून त्यांचे सैन्य देखील मग जीवावर उदार होऊन साथ द्यायचे. 
आधुनिक भारताचं भविष्य केवळ नवकल्पनांवर आणि वाढीवर अवलंबून नाही, तर विश्वासावर देखील आहे. संस्थांवर, नेतृत्वावर, आणि एकमेकांवर. विश्वास निर्माण होतो तेव्हा, जेव्हा निर्णय घेणारे लोक त्या निर्णयांचे परिणाम स्वतः भोगतात. ‘स्किन इन द गेम’ हे मेंटल मॉडेल आपल्याला एक पुरातन सत्य पुनर्रबिंबित करते ते म्हणजे जेव्हा फायदा तुमचा असेल, तेव्हा नुकसानही तुमचंच असायला हवं. 
‘स्किन इन द गेम’ ही केवळ सरकारी, सामाजिक, व्यावसायिक किंवा आर्थिक उत्तरदायित्व घेण्याची गोष्ट नाही तर ती एक नैतिक जबाबदारीची चौकट आहे. तुम्ही ज्या निर्णयांचा इतरांवर परिणाम करत आहात, त्याचे काही परिणाम तुमच्यावरही हवेत, तेव्हा तुमचं निर्णय घेण्याचं वर्तन पारदर्शक, संवेदनशील आणि विश्वासार्ह होतं. जेव्हा दुसऱ्याला किंमत मोजायला लावून निर्णय घेणे शक्य असतं, तेव्हा खऱ्या नेतृत्वाची, जबाबदार नागरिकत्त्वाची, आणि नैतिक व्यावसायिकतेची खरी कसोटी सुरु होते. या मेंटल मॉडेलची कसोटी एकदम सोपी आहे,  “जर मी चुकलो, तर त्याची शिक्षा मलाच मिळणार आहे का आणि ती भोगण्याची माझी तयारी आहे का?” जोपर्यंत सत्तेच्या स्थानावर असलेले लोक “हो” असं प्रामाणिक उत्तर देत नाहीत, तोपर्यंत कुठेतरी नुकताच बांधलेला पूल पुन्हा कोसळू शकतो!!



\chapter{श्रीखंडाची चौथी वाटी}

लग्नसमारंभात हल्ली प्रचलित असलेल्या ‘बुफे’ पद्धतीपेक्षा मला पंगतीची व्यवस्थाच जास्त आवडते. ‘जड’ प्लेट हातात घेऊन, उभे राहून, ताटकळत खाण्यापेक्षा, पंगतीत आरामात बसून पक्वान्नांचा विशेष आनंद घेता येतो. अशाच एका पंगतीत गोड पदार्थ होता, माझे आवडते श्रीखंड. मग काय विचारता सोय नाही. पहिली वाटी संपवली. फारच भारी. इच्छा संपेना. दुसरी घेतली. छान वाटलं. तिसरी खरंतर नको होती पण घ्यावी लागली, वधू-वरांच्या आग्रहाखातर. आनंद कमी झाला आणि कशीबशी संपवली. आतामात्र बास म्हणायचं असं  ठरवताच समोरच्या ‘पार्टी’तील ओळखीच्यांनी पैज लावली. केवळ ‘इगो’ पायी चौथी वाटी घेतली पण अगदी गळ्याशी येऊन कधीही पुढचा सोपस्कार होईल असे झाले. पहिल्या वाटीला हवेहवेसे वाटणारे श्रीखंड आता बघवतपण नव्हते. हा प्रसंग ‘द लॉ ऑफ डिमीनीशींग रिटर्न्स’ हे मेंटल मॉडेल (मन:प्रारूप) म्हणजेच विचारचित्र दर्शवतो. ह्याला आपण ‘लाभ ह्रासाचा नियम’ म्हणू शकतो. या नियमाचा उगम जरी अर्थशास्त्रातील असला तरी तो आपले काम, नातेसंबंध, आरोग्य आणि जवळजवळ प्रत्येक निर्णयावर लागू होतो. काही गोष्टी सुरुवातीला मोठा फायदा देतात, पण जसजसे आपण त्या अधिक करत जातो, तसतसे त्याचा परिणाम कमी होतो आणि कधी कधी तोटा होऊ लागतो. आज ज्या युगात "जास्त मेहनत = जास्त यश" असं समजलं जातं, तिथे हा नियम समजून घेणं फार गरजेचं आहे. कारण प्रत्येक गोष्टीत अधिक घालून उपयोग होईलच, असं नाही. याची काही उदाहरणे पाहुयात. 

स्वातंत्र्यानंतर भारतातील अन्न-धान्य उत्पादनाची स्थिती बिकट होती. मोठ्याप्रमाणात तुटवडा असल्याने परदेशातून ते मिळवण्यासाठी हात पसरावे लागायचे. मालही पाठवला जायचा तो निकृष्ट दर्जाचा. पण तो खाण्यासाठी वापरण्यावाचून गत्यंतर नव्हते. यावर आमूलाग्र उपाय म्हणून १९६०-७० च्या दशकात भारतात हरित क्रांती झाली. त्यावेळी सिंचन, नवीन बियाणं आणि खतांमुळे शेती उत्पादनात मोठी वाढ झाली. भारत स्वयंपूर्ण झाला. पण पुढील दशकांमध्ये याच पद्धतीने शेती केली गेली, तेव्हा उत्पादन फारसे वाढले नाही आणि खतांचा अति वापर केल्याने जमिनीची गुणवत्ताही खराब झाली. जे उपाय आधी फायदेशीर वाटत होते, तेच नंतर अकारण खर्चिक आणि अपायकारक ठरले.

रोज ३०-४० मिनिटे व्यायाम केल्याने ऊर्जा वाढते, मनःस्थिती सुधारते आणि आरोग्य उत्तम राहते. हा सुरुवातीचा मोठा फायदा आहे. पण तोच व्यायाम दिवसातून तीन-चार तास, कोणत्याही मार्गदर्शनाशिवाय केल्यास शरीराला दुखापत, प्रचंड थकवा आणि मानसिक ताण वाढू शकतो. इथे व्यायामाचा ‘अतिरेक’ फायद्याऐवजी तोट्याचा सौदा ठरतो.

परीक्षेच्या आदल्या रात्री जागून अभ्यास करणे कदाचित फायद्याचे ठरू शकते. पण जर हीच सवय बनली आणि रोजच रात्री जागरण केले, तर हळूहळू स्मरणशक्ती आणि एकाग्रता कमी होऊ लागते. सुरुवातीला फायदेशीर वाटणारी कृती, सवयीची झाल्यावर तिची परिणामकारकता घटते.

नवीन पिढीला तंत्रज्ञान-स्नेही बनवण्यासाठी शाळेत टॅबलेटवर शिकवले जाते. शैक्षणिक ॲप्समुळे विषय खेळकर पद्धतीने समजतात आणि लक्षात राहतात. पण जसजसा टॅबलेटच्या वापराचा वेळ वाढू लागतो, तसतशी त्यावर इतर ॲप्स आणि खेळांचे आक्रमण होते. मग अभ्यासापेक्षा करमणूकच जास्त होऊ लागते आणि मूळ हेतू बाजूला राहतो.

सुरुवातीला एखादे ॲप किंवा सॉफ्टवेअर वापरण्यास अतिशय सोपे असते. पण ते प्रसिद्ध झाल्यावर त्यात नवनवीन 'फीचर्स' टाकण्याची स्पर्धा सुरू होते. हळूहळू ते ॲप इतके किचकट आणि संथ होते की मूळ वापरकर्तेच त्याला कंटाळून नवीन, सोप्या पर्यायाकडे वळतात. इथेही 'जास्त' फीचर्स देणे फायद्याऐवजी तोट्याचे ठरते.

पैसा कोणाला नको आहे. वर्षाला ₹३ लाख कमावणाऱ्या व्यक्तीसाठी ₹१० लाखांचे उत्पन्न आयुष्य बदलून टाकणारे ठरू शकते. कालांतराने १० लाखांचे २० आणि २० लाखांचे ५० लाख झाल्यावर अनेक स्वप्ने साकार होतात, पण त्याचबरोबर खर्च आणि जबाबदाऱ्याही वाढतात. एका टप्प्यानंतर जेव्हा उत्पन्न ₹१ कोटींवरून ₹२ कोटी होते, तेव्हा मिळणाऱ्या आनंदात फारसा फरक पडत नाही; उलट कामाचा ताण आणि जबाबदाऱ्या वाढल्याने सुख कमी होऊ शकते.

एखाद्या मित्राशी आठवड्यातून एखादवेळेस बोलणं चांगलचं, ते संबंध दृढ करतं. गरजेच्या काळात दरदिवशीही कॉल केला तरी ठीक वाटतं. पण दिवसातून पाच वेळा फोन केला, तर कंटाळा येतो, चिडचिड होते, नाही का?

थोडक्यात, ‘लाभ-ह्रासाचा नियम’ हे मॉडेल निराशावादी नसून, ते एक व्यावहारिक सत्य आहे. ते आपल्याला शिकवते की, ‘नेहमीच ‘जास्त’ म्हणजे ‘चांगले’ असे नाही.’ कधीतरी योग्य वेळी थांबणे, हेच अधिक फायदेशीर ठरते. हा नियम आपल्याला "थांबण्याची योग्य मर्यादा कोणती?" हा प्रश्न विचारायला प्रवृत्त करतो.

आजच्या स्पर्धेच्या आणि सतत धावपळीच्या जगात, हा नियम एका ‘ब्रेक’सारखे काम करतो, जो आपल्याला अधिक प्रभावी आणि सुज्ञ बनवतो. प्रत्येक प्रयत्नाला एक मर्यादा असते आणि ती वेळेवर ओळखली, तरच फायदा होतो. कधीकधी, थोडे 'कमी' करणे हाच सर्वात मोठा शहाणपणा असतो, नाहीतर श्रीखंडाप्रमाणेच कोणतीही चांगली गोष्ट 'गळ्याशी' यायला वेळ लागत नाही.



\chapter{मनी वसे ते चहू दिसे }

२०२३ मध्ये सर्वोच्च न्यायालयाने कलम ३७० रद्द करण्याचा सरकारचा निर्णय कायम ठेवला. साहजिकच, सत्ताधारी पक्षाने त्याचे जोरदार स्वागत केले, पण जम्मू आणि काश्मीरमधील नेते आणि अनेक विरोधकांनी त्याला तीव्र विरोध केला. पुढे २०२४ मध्ये, त्याच न्यायालयाने सरकारने सुरू केलेली ‘इलेक्टोरल बाँड्स’ योजना असंवैधानिक ठरवली. यावेळी विरोधक आणि अनेक सामाजिक संघटनांनी “हा लोकशाहीचा विजय आहे,” म्हणत या निर्णयाचे स्वागत केले. पण सत्ताधारी पक्षाच्या समर्थकांनी याला “न्यायालयाचे सीमोल्लंघन (ओव्हररीच)” म्हटले. खरा मुद्दा असा आहे की, सर्वोच्च न्यायालयाचे काम संविधानाचा अर्थ लावणे, पुरावे तपासणे आणि निष्पक्ष निर्णय देणे आहे, जे ते सातत्याने करत असते. मात्र, जनतेच्या आणि राजकीय पक्षांच्या प्रतिक्रिया त्यांच्या निष्ठा व पूर्वग्रहांमधूनच उमटतात. या नैसर्गिक मानसिक प्रवृत्तीला ‘कन्फर्मेशन बायस’ किंवा ‘पुष्टीकरण पूर्वग्रह’ हे मेंटल मॉडेल (मन:प्रारूप) अथवा विचारचित्र म्हटले जाते. हा एका प्रकारचा ‘कॉग्निटिव्ह बायस’ अर्थात मानसिक-वैचारिक पूर्वग्रह आहे.

पुष्टीकरण पूर्वग्रह ही आपल्या मेंदूची एक नैसर्गिक प्रवृत्ती आहे. आपण आपल्या विश्वासांना बळकटी देणारी माहिती शोधतो, ऐकतो, लक्षात ठेवतो आणि विरोधी माहितीकडे दुर्लक्ष करतो. याचाच अर्थ, दोन भिन्न मतांचे लोक एकाच घटनेकडे पाहूनही आपापल्या मताचीच अधिक खात्री बाळगतात. हे विशेषतः राजकीय, धार्मिक किंवा भावनिक यांसारख्या संवेदनशील विषयांमध्ये अधिक दिसून येते. हा एक मानसिक शॉर्टकट आहे, जो काही वेळा उपयुक्त ठरतो. आपल्या जगाविषयीच्या धारणेला किंवा अहंकाराला धक्का लागू नये म्हणून मेंदू ही सोपी वाट निवडतो. गुंतागुंतीच्या जगात प्रत्येक वेळी नव्याने विचार करण्याची ऊर्जा वाचवण्यासाठी ही एक नैसर्गिक संरक्षणप्रणाली आहे. पण अनेकदा, विशेषतः गुंतागुंतीच्या समस्यांमध्ये, तो चुकीच्या निर्णयांकडे नेऊ शकतो. याचे परिणाम म्हणजे खोटी बातमी खरी मानणे, अयोग्य व्यक्तींवर विश्वास ठेवणे किंवा संकुचित विचारसरणीत अडकून राहणे. याची काही उदाहरणे पाहुयात.

शेअर बाजारात अनेक गुंतवणूकदार एखाद्या ‘नावाजलेल्या’ कंपनीवर किंवा उद्योजकावर इतका दृढ विश्वास ठेवतात, की त्या उद्योजकाची पुढची पिढी आली तरी तो कायम राहतो. ते त्या कंपनीबद्दलची केवळ सकारात्मक माहितीच लक्षात घेतात. ते केवळ त्याच आर्थिक सल्लागारांना किंवा वृत्तवाहिन्यांना फॉलो करतात, जे त्यांच्या गुंतवणुकीच्या निर्णयाला दुजोरा देतात. कायदेशीर अडचणी, आर्थिक घसरण आणि कौटुंबिक कलह यांसारख्या नकारात्मक बाबींचा कंपनीच्या भविष्यावर होणारा परिणाम माहीत असूनही त्याकडे दुर्लक्ष केले जाते. शेअरची किंमत घसरली तरी, ते याला "ही तात्पुरती बाब आहे," असेच समजतात आणि अधिक गुंतवणूक करून तोटा वाढवतात.

विवाहाच्या बाबतीतही मुला-मुलीच्या अनुरूपतेपेक्षा ‘घराण्याला’ अधिक महत्त्व दिले जाते. आपला विश्वास असतो की इतक्या प्रसिद्ध, तालेवार घराण्यात सर्व काही चांगलेच असणार. काही ज्ञात अप्रिय घटनांकडे सोयीस्करपणे दुर्लक्ष केले जाते. किरकोळ उणिवांकडे कानाडोळा करण्याच्या नावाखाली मोठ्या समस्याही झाकल्या जातात, ज्याचे पर्यवसान नंतर दुर्दैवी घटनांमध्ये होते. ‘आपला अंदाज चुकला’ हे मान्य करण्याच्या त्रासापेक्षा ‘सगळं ठीक होईल’ या भ्रमात राहणे मेंदूला सोपे वाटते.

बहुतांश पालकांना वाटते की मुला-मुलींनी डॉक्टर किंवा इंजिनीअरच व्हावे, आणि त्यातही कॉम्प्युटर शाखाच निवडावी. काही यशस्वी उदाहरणे पाहून त्यांना खात्री पटते की हाच एकमेव सुरक्षित मार्ग आहे. नामांकित महाविद्यालयांतूनही सर्वांना नोकऱ्या मिळत नाहीत किंवा इतर क्षेत्रांतही उत्तम संधी आहेत, ही वस्तुस्थिती समोर असूनही, या पुष्टीकरण पूर्वग्रहामुळे पाल्याच्या कमी गुणांकडे आणि आवडीनिवडीकडे दुर्लक्ष करून, त्याला कॉम्प्युटर शाखेतच प्रवेश मिळवून दिला जातो.

सोशल मीडिया किंवा टीव्हीवरील चर्चा पाहिल्या की दोन स्पष्ट तट पडलेले दिसतात. तिथे जणू युद्धभूमीचे स्वरूप आलेले असते. पूर्वग्रह इतके तीव्र असतात की एखादी चांगली किंवा वाईट घटना घडली तरी दोन्ही बाजू समोरासमोर उभ्या ठाकलेल्या दिसतात. सोशल मीडिया ॲप्सदेखील संगणक प्रणाल्यांच्या (अल्गोरिदम्स) आधारे असे फीड तयार करतात, की तिथे फक्त आपल्या मतांशी जुळणाऱ्या पोस्ट्स दिसतात. यालाच ‘इको-चेंबर’ म्हणतात, जिथे विरोधी विचारांना प्रवेशच नसतो. यामुळे वैचारिक कट्टरता वाढते आणि समाजात संवादऐवजी संघर्षच अधिक दिसू लागतो. आपणही विरोधी मते दिसल्यास त्यांना ब्लॉक किंवा ट्रोल करतो, नाही का? खरंतर, आपल्या वैचारिक विकासासाठी परस्परविरोधी मते जाणून घेणे आवश्यक असते, ज्याला भारतीय परंपरेत ‘शत्रूबोध’ असे म्हटले आहे. समोरच्याची बाजू नीट अभ्यासली तरच त्यावर सुसंस्कृतपणे मात करता येते, अन्यथा केवळ आरडाओरड आणि शाब्दिक संघर्षच दिसतो.

मग या पुष्टीकरण पूर्वग्रहावर मात कशी करायची? सर्वात आधी, आपल्याला जागरूक व्हायला हवे. स्वतःला विचारा: 'मी एखाद्या विषयाच्या सर्व बाजू तपासत आहे, की केवळ मला पटेल तीच माहिती स्वीकारत आहे? माझी मते पुराव्यांवर आधारित आहेत की केवळ विश्वासांवर?' विरोधी मते सक्रियपणे शोधा आणि वाचा. व्यक्तीऐवजी विचारांवर चर्चा करा. समोरच्या व्यक्तीचा युक्तिवाद शक्य तितक्या चांगल्या प्रकारे समजून घेऊन मगच त्यावर प्रतिक्रिया द्या. यामुळे चर्चेची पातळी उंचावते. “मला यावर पुरेशी माहिती नाही,” असे म्हणण्याची सवय लावा. नेहमी बरोबर असण्याचा हट्ट सोडा. चुका मान्य करायला शिका, कारण त्यातूनच आपण शिकतो. तेव्हाच खऱ्या अर्थाने शहाणपणाचा प्रवास सुरू होतो.




\chapter{बाप दाखव, नाहीतर श्राद्ध कर}

मी कदाचित पहिली-दुसरीत असेन. उन्हाळ्याच्या सुट्टीत सोलापूरजवळ माझ्या मामाच्या गावाला गेलो होतो. मोठ्या, सपाट मैदानी भागात फुटबॉल खेळताना मी माझ्या मोठ्या मामेभावाला विचारलं, “आपलं हे गाव तिथे पुढे दिसतंय तिथपर्यंतच का?” तो हसून म्हणाला, “ती आडवी रेषा दिसतेय, ते क्षितिज आहे. त्यापलीकडेही खूप गावं आहेत. पृथ्वी ही फुटबॉलसारखीच गोल आहे.” मला विश्वासच बसला नाही. जमीन तर मला सपाट दिसतेय. पुढच्याच वर्षीच्या सुट्टीत आत्याकडे कोकणात गेलो होतो. सायंकाळी समुद्रकिनाऱ्यावर असताना, एक जहाज जणू पाण्याबाहेर हळूहळू वर येताना दिसलं. सुरुवातीला केवळ त्याचं टोक, मग हळूहळू ते संपूर्ण जहाज. तेव्हाच भावाचं म्हणणं पटलं. पृथ्वी गोल आहे. हा सिद्धांत एका छोट्या मुलालाही तपासता आला. मीच माझं पूर्वीचं मत खोडून काढलं.
ही घटना जरी अगदीच ‘बालिश’ असली तरी एक महत्त्वाचा मुद्दा अधोरेखित करते. आपण एखाद्या कल्पनेचं मूल्यांकन करताना क्वचितच विचारतो, “ही गोष्ट खोटी ठरवता येईल का?” इथेच ‘फॉल्सीफायेबिलीटी’ मेंटल मॉडेल (मन:प्रारूप) अर्थात, ‘असत्यसिद्धिक्षमता’ म्हणजेच "खोटेपणा सिद्ध करता येण्याची शक्यता" हे विचारचित्र कामास येते. ही कल्पना वैज्ञानिक तत्त्वज्ञ कार्ल पॉपर यांनी मांडली होती. यानुसार, कोणताही दावा वैज्ञानिक किंवा तार्किक मानायचा असेल, तर तो तपासता येणारा आणि गरज पडल्यास खोटा ठरवता येणारा असावा. जर एखादा सिद्धांत कधीच चुकीचा ठरवता येणार नसेल, तर ते विज्ञान नाही, ती अंधश्रद्धा आहे. याची काही उदाहरणे पाहूयात. 
भारतात एका योगी-बाबांनी त्यांचे एक उत्पादन मधुमेह (डायबेटीस) बरा करतं, असा प्रचार व जाहिरात केली होती. हजारो लोकांनी विश्वास ठेवला, मोठा खप झाला, पण काहींना शंका आल्याने त्यांनी पुराव्याची व शास्त्रीय तपासणीची मागणी केली. प्रकरण न्यायालयात गेल्यावर त्या बाबांनी आपण असा दावा केलाच नव्हता, असे सांगितले. नवीन संशोधन किंवा उत्पादन करायला कोणाचीच बंदी नाही, पण इथे काय करायला हवं होतं? सांख्यिकी दृष्ट्या स्वैर (रँडम) चाचण्या घ्यायला हव्या होत्या. उत्पादनाच्या वापराआधी आणि वापरानंतर रक्तातील साखरेचे प्रमाण तपासून मधुमेह कमी झाल्याचा दावा सिद्ध करता आला असता, नाही का?
राजकारणात तर कोण काय दावे करेल आणि आश्वासनं देईल, हे सांगताच येत नाही. ‘कोकणाचा कॅलिफोर्निया’ आणि ‘मुंबईचं शांघाय-सिंगापूर’ कधी होणार, हे देवालाही ठाऊक नसेल. याचप्रमाणे ‘गरिबी हटवू’ किंवा ‘भ्रष्टाचार मिटवू’ यांसारख्या शिळ्या घोषणांकडे आता कोणी लक्षही देत नाही. यावर उपाय म्हणून निवडणुकांपूर्वी प्रसिद्ध केलेला जाहीरनामा-’वचन’नामा कायदेशीर दस्तऐवज बनवायला हवा. त्यातील आश्वासनांना पक्षाचे नेते जबाबदार असायला हवेत आणि त्यांनी त्यावर स्वाक्षरी करायला हवी. या जाहीरनाम्यात मोजता येणाऱ्या आकड्यांचा समावेश असावा. कोणती योजना, किती लोकांना, किती पैशांत देणार, याचं स्पष्ट विवरण हवं. यासाठी पैसा कुठून उभारणार, याचा संपूर्ण ताळेबंद (बॅलन्स शीट) त्यात असावा. या घोषणांची अंमलबजावणी न झाल्यास दंडात्मक कारवाई, अगदी पक्षाची नोंदणी रद्द करण्यापर्यंतची तरतूद, या दस्तऐवजात स्पष्टपणे नमूद असावी. जनतेला त्यांचे दावे तपासता येतील आणि वेळप्रसंगी खोटे ठरवता येतील. मगच त्यांच्या बोलण्यात सत्यता येईल.
आपल्या मतांची सत्यताही तपासता यायला हवी. उदाहरणार्थ, बर्‍याच पालकांचं मत असतं, की मुलांना मोबाईल दिला तर ते जास्त ‘ढ’ होतात. तर काहींना वाटतं, की विविध गोष्टी शिकल्याने ते ‘स्मार्ट’ होतात. यातलं खरं काय? अशा वेळी वर्तणूकशास्त्रातील अभ्यासक वेगवेगळ्या गटांवर प्रयोग करून हे तपासतात. यातून जे निष्कर्ष निघतील, तेच खरे मानायला हवेत.
एखादा स्टार्टअप संस्थापक म्हणतो, “हे अ‍ॅप पुढचं झोमॅटो होईल”. पण हा दावा तपासता येईल का? त्याला विचारायला हवं, “म्हणजे नक्की काय?” जर तो म्हणाला, “मी ३ महिन्यांत १०,००० रोजचे ग्राहक मिळवेन”, तर हे मात्र तपासता येईल. अशी तपासता येणारी उद्दिष्टं ठरवल्याशिवाय ‘फॉल्सीफायेबिलीटी’ हे मेंटल मॉडेल प्रभावीपणे वापरता येत नाही.
भारत एक आध्यात्मिक-धार्मिक देश आहे. हे खूप सुंदर आहे, पण कधीकधी श्रद्धेचा गैरवापर होतो. “ही पूजा केली की श्रीमंत व्हाल,” असा दावा तपासता येतो का? तर, नाही, पण जर ‘श्रीमंत’ म्हणजे किती पैसे, हे आधीच ठरवलं तर मात्र तपासता येतो! ‘फॉल्सीफायेबिलीटी’ श्रद्धेवर हल्ला करत नाही, तर ती श्रद्धाळूंना फसवणुकीपासून वाचवते.
‘जागतिक भूक निर्देशांक’ (ग्लोबल हंगर इंडेक्स) मध्ये भारताचा क्रमांक बांगलादेशपेक्षाही खालचा आहे. हे ऐकून आश्चर्य वाटतं. मग इथे विचारायला हवं की, यामागे मोजमाप काय आहे? ही आकडेवारी कशावर आधारित आहे? हे वास्तव आहे की कोणत्यातरी जागतिक संस्थेचा खोडसाळपणा, हे सिद्ध करता येतं. 
प्रत्येक वेळी कोणी कोणताही दावा केला, की आपण विचारायला हवं: “कशामुळे हे खोटं ठरू शकेल?” जर याचं उत्तरच मिळत नसेल, तर त्या गोष्टीवर विश्वास ठेवण्यात अर्थ नाही. भारताला फक्त स्वप्नं पाहणारे कोणत्याही आश्वासनांवर विश्वास, नेत्यांवर अंध-श्रद्धा ठेवणारे नागरिक नकोत; त्याला विचार करणारे, चाचणी करणारे आणि प्रश्न विचारणारे नागरिक हवेत. ‘फॉल्सीफायेबिलीटी’ हे केवळ वैज्ञानिकांचं हत्यार न राहता, प्रत्येक भारतीयाच्या विचारसरणीचं साधन बनलं पाहिजे. जी श्रद्धा तपासणी मानत नाही, ती श्रद्धा काय कामाची? 


\chapter{सर सलामत तो पगडी पचास}

दुसऱ्या महायुद्धाच्या काळात अमेरिकेच्या लष्करासमोर एक मोठे संकट उभे राहिले. त्यांची अनेक बॉम्बर विमाने शत्रूच्या हल्ल्यात नष्ट होत होती. जी काही विमाने परत येत, त्यांची तपासणी केली असता असे दिसून आले की, बहुतेक गोळ्या पंखांवर आणि शेपटीवर लागल्या होत्या, तर इंजिन सुस्थितीत होते. त्यामुळे लष्करी तज्ज्ञांच्या मते, जिथे गोळ्या लागल्या आहेत, ते भाग अधिक मजबूत करायला हवेत.
पण, अब्राहम वॉल्ड नावाच्या एका हुशार गणितज्ञाने वेगळाच मुद्दा मांडला. त्याने स्पष्ट केले की, आपण फक्त परत आलेली विमाने पाहत आहोत. ही विमाने गोळ्या लागूनही परत आली आहेत, म्हणजेच त्या जागांवर गोळ्या लागूनही ती वाचली आहेत. याचाच अर्थ, त्या जागा फारशा धोकादायक नाहीत. खरी धोकादायक जागा तर ती होती, जिथे गोळी लागल्यामुळे विमान परत येऊच शकले नाही. आणि म्हणूनच आपल्याला त्या भागांवर कोणतीही इजा दिसत नाही. वॉल्डने सल्ला दिला की, ज्या भागांवर काहीच झाले नाही असे वाटते, म्हणजेच इंजिन, त्यालाच अधिक संरक्षण द्यायला हवे. त्याच्या या विचारानुसार योग्य बदल करण्यात आले आणि अनेक वैमानिकांचे प्राण वाचले.
ही गोष्ट केवळ युद्धापुरती मर्यादित नाही, तर आपल्या दैनंदिन विचार करण्याच्या पद्धतीबद्दलही आहे. आपण बहुतेक वेळा फक्त यशस्वी गोष्टी पाहतो आणि त्यावरूनच निष्कर्ष काढतो. याउलट, जे अपयशी ठरले, त्यांचा अभ्यास करणे अधिक महत्त्वाचे ठरते.
या विचारसरणीच्या मेंटल मॉडेलला (मन:प्रारूप) अथवा विचारचित्राला ‘सर्व्हायवरशीप बायस’ म्हणतात. यालाच मराठीत ‘उत्तरजिवीत्व पूर्वग्रह’ असे म्हणू शकतो. ‘सर सलामत तो पगडी पचास’ ही हिंदी म्हण याला चपखल बसते. याचा अर्थ, पगडी घालण्याची हौस खुप असली तरी डोकं असलं तरच, म्हणजे जगलो तरच ते शक्य आहे.  आपण केवळ यशस्वी आणि टिकून राहिलेल्या गोष्टी पाहतो. आपण नेहमीच यशाच्या कथा ऐकतो, त्या सांगितल्या जातात, छापल्या जातात, पण अपयशाच्या कथा सहसा पडद्याआड राहतात. याच कारणामुळे आपण अनेकदा चुकीचे निष्कर्ष काढतो. याची काही उदाहरणे पाहूया.
तरुण पिढीचे आदर्श असलेले बिल गेट्स, मार्क झकरबर्ग आणि स्टीव्ह जॉब्स यांनी शिक्षण पूर्ण न करता उद्योग सुरू केला. नंतर ते आर्थिक आणि जागतिक प्रभावाच्या दृष्टीने प्रचंड यशस्वी झाले. मग काही जण विचार करतात की, “त्यांनी केले, मग मी का नाही करू शकत?” पण याच वाटेवर जाऊन ज्यांनी सर्वस्व गमावले, त्यांच्याकडे आपले लक्ष जात नाही. त्यांनीही तितकाच प्रयत्न केला होता, पण त्यांची गोष्ट कोणीच सांगत नाही.
हेच शेअर मार्केटच्या बाबतीतही घडते. ‘एखाद्या स्टॉकने दहा वर्षांत शंभरपट परतावा दिला, त्यात गुंतवणूक केली असती तर तुम्ही आज करोडपती असता,’ अशा जाहिराती आपण पाहतो. पण कोणी आपल्याला हे सांगत नाही की, हजारो स्टॉक्सपैकी फार थोडेच टिकून राहतात. या काळात इतर अनेक कंपन्या बंद पडल्या आणि गुंतवणूकदारांचे पैसे बुडाले. पण त्यांची चर्चा कुठेच होत नाही.
बॉलिवूडमध्येही हेच चित्र आहे. “शाहरुख खान काहीही नसताना मुंबईत आला आणि सुपरस्टार झाला,” ही कथा सर्वांना माहीत आहे. हे उदाहरण डोळ्यासमोर ठेवून असंख्य तरुण-तरुणी आपले भविष्य आजमावायला मायानगरीत येतात. त्या सर्वांचे काय होते? त्यांचे स्वप्नही शाहरुख सारखेच होते, पण त्यांच्या वाट्याला वेगळे वास्तव आले.
आज अनेक तरुण यूट्यूब चॅनेल सुरू करतात किंवा इंस्टाग्रामवर व्हिडीओ-रील्स बनवतात. अनेकदा सुमार दर्जाचा ‘कन्टेन्ट’ असूनही, काहींचे व्हिडीओ व्हायरल होतात आणि त्यांना प्रचंड प्रसिद्धी व पैसा मिळतो. अशा गोष्टींचा बोलबाला होतो. हे पाहून अनेक तरुण या ‘इझी मनी’च्या मागे धावत आहेत. पण सगळे थोडेच कमाई करू शकतात. त्यांच्याकडे कोणी बघतही नाही. आपण फक्त यशस्वी झालेल्यांची उदाहरणे पाहतो, बाकीच्यांचे अपयश कुठेच समोर येत नाही.
मग यावर उपाय काय? प्रत्येक यशस्वी गोष्टीकडे बघताना स्वतःला एक प्रश्न जरूर विचारावा, "ह्याच गोष्टी करताना जे अपयशी झाले, त्यांचे काय?" कुठल्याही यशामागे अपयशाचे प्रमाण किती आहे, हे शोधण्याची सवय लागली पाहिजे. यशस्वी व्यक्तींना देव मानण्याऐवजी, त्यांच्या केवळ अंतिम यशाकडे न पाहता त्यांच्या प्रयत्नांचा आदर करा. आणि हेही लक्षात ठेवा की, कधीकधी नशिबाचीही साथ मिळालेली असते. हे यश अनेकदा अपवादात्मक असते हे लक्षात घ्या. आपण ‘सरासरी’ निकालांवर लक्ष केंद्रित करून आपली योजना आखली पाहिजे.
आपण भारतीय समाज म्हणून यशाचे मोठे कौतुक करतो आणि यशस्वी व्यक्तींना देव मानतो. पण ही उदाहरणे केवळ त्या स्पर्धेत टिकून राहिलेल्यांची आहेत. ‘सर्व्हायवरशीप बायस’ म्हणजे फक्त वाचलेल्या लोकांच्या कहाण्या ऐकून आपला दृष्टिकोन ठरवणे. म्हणूनच, यशाची प्रत्येक कहाणी ऐकताना त्यामागे लपलेल्या अपयशाचाही विचार करायला हवा. कायम लक्षात ठेवायला हवं की, जी विमाने परत आलीच नाहीत, त्यांचे काय झाले?


\chapter{सोडी सोन्याचा पिंजरा}

सुरेश चांगल्या गुणांनी उत्तीर्ण होऊन ‘डॉक्टर’ झाला होता. एका मोठ्या खाजगी रुग्णालयात त्याला कनिष्ठ पदावर नोकरी मिळाली होती. त्याने दिवस-रात्र काम केले. वरिष्ठ डॉक्टरांच्या मार्गदर्शनाखाली मन लावून काम केल्याने त्याचे निदान, औषधयोजना आणि प्रत्यक्ष शस्त्रक्रिया अधिक अचूक आणि विना गुंतागुंतीच्या होऊ लागल्या. वरिष्ठ पदे मिळत गेली. त्याचे नाव केवळ डॉक्टरांमध्ये नव्हे तर रुग्णांमध्येही आदराने घेतलं जायचं. त्यांच्या हातातली स्थिरता, शस्त्रक्रियेतलं कौशल्य, आणि माणुसकीनं ओतप्रोत भरलेली रुग्णसेवा यामुळे तो रुग्णालयाचा आधारस्तंभ झाला. साहजिकच, जेव्हा वैद्यकीय संचालकपद रिक्त झालं, तेव्हा कोणतीही शंका न घेता त्याची निवड झाली. पण काही महिन्यांतच परिस्थिती बदलू लागली. विभागीय मिटिंग्स, अनेक प्रकारचे रिपोर्ट्स आणि व्यवस्थापनाने दिलेली ‘टार्गेट्स’ पूर्ण करण्याच्या नादात त्याचे रुग्णांकडे वेळ देणं कमी व्हायला लागलं. आजारावर सखोल चिंतन करण्याऐवजी तो तपासण्या 'उरकू' लागला. मग अनेक विभागांमध्ये कुरबुरी सुरू झाल्या. जो सुरेश तणावातही शांतपणे अचूक निर्णय घ्यायचा, तोच आता गोंधळलेला आणि हतबल दिसू लागला. काही परिचारिका कुजबुजत म्हणू लागल्या, “डॉक्टर म्हणून ते भारी होते, पण ही नवीन जबाबदारी त्यांच्यासाठी नाहीये”. डॉ. सुरेश यांच्या बाबतीत जे घडले, ते दुर्मिळ नाही. हे 'पीटर्स प्रिन्सिपल' (पीटरचे तत्त्व) या मेंटल मॉडेलचे (मन:प्रारूप) अथवा विचारचित्राचे  उत्तम उदाहरण आहे. सोप्या भाषेत याला ‘पदोन्नतीमुळे आलेली अकार्यक्षमता’ म्हणता येईल.

ही संकल्पना शिक्षणतज्ज्ञ लॉरेन्स जे. पीटर यांनी १९६९ मध्ये मांडली. त्यानुसार, "प्रत्येक संस्थेत कर्मचारी अखेरीस अशा पदावर पोहोचतो, ज्यासाठी लागणारी कौशल्ये त्याच्याकडे नसतात." याचा अर्थ, पदोन्नती देताना व्यक्तीच्या आजवरच्या कामगिरीचा विचार होतो, पण नव्या जबाबदारीसाठी तो सक्षम आहे की नाही, हे पाहिले जात नाही. त्यामुळे कर्मचारी अशा पदावर पोहोचतो, जिथे त्याची कौशल्ये अपुरी पडतात आणि तो तिथेच अडकतो. परिणामी, संस्थेची गती मंदावते. याची काही इतर उदाहरणे पाहुयात. 

असेच बँकांमध्ये घडते. खात्यांचे बिनचूक काम आणि मोठ्या व्यवहारांची जबाबदारी यशस्वीपणे सांभाळल्यामुळे एखाद्या कर्मचाऱ्याला शाखा व्यवस्थापक बनवले जाते. पण तिथे कर्जवाटप, कर्मचारी व्यवस्थापन आणि ग्राहक संबंध सांभाळण्यासाठी पूर्णपणे वेगळी कौशल्ये लागतात. याचा परिणाम म्हणून अनेक शाखा केवळ नेतृत्वाची कमतरता असल्यामुळे अडचणीत येतात.

उत्तम शिक्षक जेव्हा शाळेचा मुख्याध्यापक बनतो, तेव्हा त्याचे लक्ष शिकवण्याऐवजी कागदपत्रे, बजेट आणि प्रशासकीय कामांमध्ये अडकते. परिणामी, शाळेची गुणवत्ता घसरते आणि मुलांना नवनवीन पद्धतीने शिकवण्याचे त्याचे मूळ आवडते काम बाजूलाच राहते.

क्रिकेटमध्येही अनेक उत्तम-स्टार खेळाडूंना कर्णधार बनवले जाते. पण संघाचे व्यवस्थापन, खेळाडूंची मनस्थिती समजून घेण्याची क्षमता आणि माध्यमांना सांभाळणे यात ते कमी पडतात. याचा परिणाम संघाच्या तसेच त्यांच्या वैयक्तिक कामगिरीवरही होतो.

राजकारणातही असेच होते. कोणाच्याही हाकेला अर्ध्या रात्रीत धावून जाणारा कार्यकर्ता-नेता जेंव्हा राज्याचा मंत्री होतो तेंव्हा त्याची भूमिका बदलणे अपेक्षित असते. प्रशासकीय कौशल्याअभावी त्याला धोरणे ठरवणे किंवा विभाग चालवणे जमत नाही. काम करवून घेण्याच्या पद्धतीत बदल झालेले असतात. ते न समजल्याने यामुळे नाराजी वाढते आणि संपूर्ण विभाग अकार्यक्षम होतो.

टेक स्टार्टअपमध्ये एखादा प्रतिभावंत कोडर जेव्हा टीम लीड बनतो, तेव्हा त्याला माणसे हाताळावी लागतात. त्याला वरिष्ठांची आणि प्रसंगी विक्री विभागाचीही मर्जी सांभाळावी लागते. प्रमोशन मिळते, पण ते पद 'सोन्याचा पिंजरा' होऊन जाते.

मग यातून मार्ग काय? आपल्याला पदोन्नतीचा अर्थ नव्याने समजून घ्यावा लागेल. वरिष्ठ पद देणे हेच एकमेव बक्षीस नसावे; त्याऐवजी स्वायत्तता, उद्दिष्ट आणि कौशल्य वाढवण्याची संधी देणे अधिक परिणामकारक ठरू शकते. कर्मचाऱ्याला पुढच्या पदासाठी तयार करा, तो आपोआप तयार होईल असे गृहीत धरू नका. वैयक्तिक पातळीवरही विचार करा, "मी या नवीन भूमिकेत स्थिरावतोय की फक्त निभावतोय?". पुढच्या पदासाठी प्रयत्न करण्याआधी, ती जबाबदारी तुमच्या कौशल्यांना साजेशी आहे का, हे तपासा.

नेतृत्व करणाऱ्यांनी अशी संस्कृती निर्माण करायला हवी, जिथे पदोन्नती नाकारणे कमीपणाचे नव्हे, तर शहाणपणाचे लक्षण मानले जाईल. केवळ उच्च पद म्हणजे यश, हे समीकरण चुकीचे आहे. आपल्या समाजात पद, पगार आणि मोठे केबिन हेच यशाचे मापदंड आहेत. पण ‘पीटरचे तत्त्व’ आपल्याला शिकवते की, क्षमतेपलीकडची वाढ ही प्रगती नसून अधोगती असते. सोन्याचा असला तरी तो पिंजरा असू शकतो.


\chapter{माझं झालं की झालं}

जगभरात हवामान बदलाचा प्रश्न दिवसेंदिवस गंभीर होत आहे. तापमानवाढ, वाढती समुद्रपातळी, जंगलतोड आणि पाणीटंचाई ही संकटे आता कोणत्या एका देशापुरती मर्यादित राहिलेली नाहीत. या ह्रासाला प्रामुख्याने विकसित देश जबाबदार आहेत, ज्यांनी औद्योगिक क्रांतीपासून निसर्गाचा बेसुमार वापर करून समृद्धी साधली. त्यांनी पृथ्वीच्या संसाधनांचा स्वार्थासाठी वापर केला, प्रदूषण केले आणि आता तेच विकसनशील राष्ट्रांना पर्यावरणाचे ज्ञानामृत पाजत आहेत.
वातावरण, हवामान आणि समुद्र ही कोणाच्याही मालकीची नसून ‘जागतिक सामायिक (सार्वजनिक) संपत्ती’ आहे, हे ते सोयीस्करपणे विसरले. त्यामुळे तिच्या वापराची जबाबदारीही जागतिक असायला हवी. पण स्वार्थाने घेतलेले निर्णय आणि “माझं झालं की झालं” या वृत्तीने प्रदूषणाचा भार इतरांवर ढकलला. यामुळेच आज आपण ‘ट्रॅजेडी ऑफ कॉमन्स’ म्हणजेच ‘सार्वजनिक संसाधनांची शोकांतिका’ या मेंटल मॉडेलच्या (मन:प्रारूप, विचारचित्र) परिणामांचा सामना करत आहोत. या मॉडेलनुसार, जेव्हा व्यक्ती किंवा देश सार्वजनिक संसाधनाचा वापर करताना फक्त स्वतःच्या फायद्याचा विचार करतात, तेव्हा ते संसाधन हळूहळू संपते आणि अंतिमतः सर्वांचेच नुकसान होते.
या संकल्पनेचा पाया १८३३ मध्ये अर्थतज्ज्ञ विल्यम फॉर्स्टर लॉयड यांनी घातला. पुढे १९६८ मध्ये उत्क्रांतीशास्त्रज्ञ गॅरेट हार्डिन यांनी ती अधिक सविस्तरपणे मांडली. या सिद्धांतानुसार, जेव्हा प्रत्येक जण सार्वजनिक हिताचा विचार न करता सामायिक संसाधनातून जास्तीत जास्त फायदा घेण्याचा प्रयत्न करतो, तेव्हा समस्या निर्माण होतात. हार्डिन यांनी एका सार्वजनिक कुरणाचे (ज्याला इंग्रजीत “कॉमन्स” म्हटलं जातं) उदाहरण दिले. जर प्रत्येक गुराख्याने मर्यादित गुरे चारली, तर कुरण टिकून राहील. पण प्रत्येकाने स्वतःच्या फायद्यासाठी जास्त गुरे आणल्यास, कुरण काही काळातच नष्ट होईल. या शोकांतिकेचे मूळ म्हणजे, आपण दीर्घकालीन नुकसानाकडे दुर्लक्ष करून तात्काळ फायद्याचा विचार करतो. याची काही उदाहरणे पाहुयात. 
भारतात याची अनेक उदाहरणे दिसतात. घराबाहेर कचरा फेकणे, सार्वजनिक जागेवर अतिक्रमण करणे किंवा विजेची चोरी करणे. सार्वजनिक शौचालये असूनही ती वापरण्यायोग्य नसतात. कारण “मी एकट्याने स्वच्छ ठेवून काय होणार?” हा विचार प्रत्येक जण करतो आणि अंतिमतः कोणीच जबाबदारी घेत नाही. माझं झालं की झालं.
एखाद्या व्यक्तीने रस्त्यावर नको त्या ठिकाणी गाडी पार्क केली की, त्याचे अनुकरण इतरही करतात. लवकरच, चालण्यासाठी जागाच उरत नाही आणि हाच ‘नवा नियम’ (न्यू नॉर्मल) बनतो.
शेतीतील पाण्याचा अतिवापर हे आणखी एक उदाहरण. जास्त उत्पादनासाठी शेतकरी भूगर्भातील पाण्याचा प्रचंड उपसा करतात. यामुळे पाण्याची पातळी खोल जाते, बोरवेल कोरड्या पडतात. याचे पर्यवसान दुष्काळ, स्थलांतर आणि कृषी संकटात होते.
अनेक ठिकाणी उपजीविकेच्या नावाखाली बेकायदेशीर जंगलतोड चालते. जेव्हा शेकडो लोक तोच विचार करतात, तेव्हा जंगल नाहीसे होते, पर्जन्यचक्र बिघडते आणि मानवी जीवन उद्ध्वस्त होते.
डिजिटल जगातही हेच घडते. एकाच व्हिडिओ-सेवेचे अकाऊंट अनेक जण वापरतात, ऑनलाइन कोर्सेसचे लॉगिन शेअर केले जातात. यामुळे कंपनीचे उत्पन्न घटते आणि सेवेचा दर्जा खालावतो.
यावर उपाय आहे का? होय, नक्कीच. जिथे मालकी स्पष्ट असते, तिथे संसाधनांची जपणूक होते. खासगी मालमत्ता सार्वजनिक मालमत्तेपेक्षा जास्त जपली जाते, कारण तिथे जबाबदारी निश्चित केलेली असते. राजस्थानमधील काही गावांमध्ये पंचायतींनी सामाजिक नियम घालून विहिरींच्या वापरावर नियंत्रण ठेवले. ‘आपलं गाव, आपली जबाबदारी’ यासारखी तत्त्वे संसाधनांचा शाश्वत वापर सुनिश्चित करतात.
दुसरा उपाय म्हणजे प्रोत्साहन किंवा दंड. जपानमध्ये कचऱ्याच्या वर्गीकरणासाठी कठोर नियम आहेत. नियम मोडल्यास दंड आणि शिस्तबद्ध पालनासाठी बक्षीस दिले जाते. भारतातही अशी प्रभावी वर्तनप्रणाली आणायला हवी.
या सगळ्यातून आपण काय शिकायचे? जे काही ‘सार्वजनिक’ आहे, गावाचे, देशाचे आणि जगाचे, ते जपणे ही आपली सर्वांची जबाबदारी आहे. “कोणीच जबाबदार नाही” अशी भूमिका घेतल्यास ती साधनसंपत्ती नष्ट होते. भारतासारख्या प्रचंड लोकसंख्या आणि मर्यादित संसाधने असलेल्या देशात ही शोकांतिका टाळण्याचा एकच मार्ग आहे: लोकांनी एकत्र येऊन, नियम पाळून आणि स्वार्थ बाजूला ठेवून सार्वजनिक संपत्तीची काळजी घेणे. जेव्हा आपण वाहतूककोंडी पाहतो तेव्हा ठरवूया अशी कोंडी मीच गाडी दामटल्याने कधीही होणार नाही. सगळ्यांनीच नियम  पळाले, पोलिसांना, वॉर्डन्सना सहकार्य केले तर आपणच सर्व लवकर या कोंडीतून सुटू, नाही का?   




\chapter{देणाऱ्याचे हात घ्यावे}


२०२१ मध्ये अफगाणिस्तानमध्ये तालिबानने सत्ता घेतली आणि काबूलमध्ये गोंधळ माजला. तेव्हा भारताने "ऑपरेशन देवी शक्ती" नावाची एक मोठी मोहीम राबवली. यात ८०० हून अधिक भारतीय नागरिक आणि अफगाण अल्पसंख्यांकांना सुरक्षितपणे बाहेर काढण्यात आले. पण हे सहज शक्य नव्हते. थेट विमान पाठवणे धोकादायक होते. म्हणून भारताने उझबेकिस्तान आणि ताजिकिस्तान यांसारख्या जवळच्या देशांकडून मदत मागितली, त्यांच्या आकाशमार्गाचा वापर, लँडिंगची परवानगी, इंधन व इतर तांत्रिक सुविधा, यांसाठी. या देशांनी भारताला मोठ्या मनाने मदत केली. का? अगदी नेमके बोट ठेवता येत नसले तरी भारताचा जगन्मान्य उदारपणा त्याला कारणीभूत असावा. काही वर्षांपूर्वी भारताने, कोविड काळात, "व्हॅक्सीन मैत्री" उपक्रमांतर्गत बऱ्याच छोट्या, विकसनशील देशांना लस, संरक्षक किट्स आणि औषधे भेट दिली होती. जेव्हा ते अडचणीत होते, तेव्हा भारत त्यांच्यामागे उभा होता. म्हणून जेव्हा भारत अडचणीत होता, तेव्हा या देशांनीही आपल्यामागे उभे राहणे केवळ नैतिकच नव्हे तर आपुलकीचे मानले. हेच आहे ‘रेसिप्रोसिटी’ म्हणजेच परस्परतेचे मेंटल मॉडेल (मन:प्रारूप, विचारचित्र). जेव्हा तुम्ही मदत करता, तेव्हा तीच भावना तुम्हाला संकटात सावरायला येते. कधी कधी उलटेही होऊ शकते. जर तुम्ही कोणाला त्रास दिला तर त्याची व्याजासकट परतफेड होण्याची शक्यता जास्त असते. ‘पेरले तसे उगवते’ हेच खरे. याची काही उदाहरणे पाहुयात.
महाराष्ट्रातील हिवरे बाजार या छोट्याशा गावाने १९९०च्या दशकात एक वेगळाच इतिहास घडवला. इथे ना कोणी मोठे भांडवल आणले, ना सरकारने फार मोठ्या योजना दिल्या. पण गावकऱ्यांनी एकमेकांना साथ दिली. कोणी बांध घातले, कोणी विहिरी खोदल्या, ज्याच्याकडे ट्रॅक्टर होता, त्याने तो वापरायला दिला आणि जे अन्न शिजवू शकले, त्यांनी श्रमिकांसाठी जेवण तयार केले. इथे कुठलाही करार नव्हता, ना कुणी शासकीय आदेश दिला होता. फक्त एक अदृश्य नियम होता. "आज मी तुझी मदत करतो, उद्या तू माझी कर." हीच परस्परता! ही आपली संस्कृतीच आहे, आणि म्हणून ती आपल्याला फार सहजपणे समजते.
आपण साड्यांच्या दुकानात गेल्यावर दुकानदार आपल्याला चहा (उन्हाळ्यात कोल्ड्रिंक) देतो. हा केवळ पाहुणचार म्हणून नाही, तर ते एक प्रभावी परस्परतेचे तंत्र आहे. आपण चहा घेतला, म्हणजे काहीतरी खरेदी करण्यास आपोआप अजून प्रवृत्त होतो. रिकाम्या हाताने बाहेर पडणे मनाला खटकते. तसेच नात्यांमध्ये. जर एखादा समोरच्याचे नीट ऐकून घेतो आणि समजून घेतो तर समोरचाही (सर्वसाधारण परिस्थितीत) तशाच पद्धतीने वागतो, नाही का? ऑफिसमध्ये, जेव्हा मॅनेजर आपल्या टीमचे कौतुक करतो, जास्त वेळ थांबून तोही काम करतो, तेव्हा ती टीम अटीतटीची (डेडलाइनची) वेळ आल्यावर त्याच्यासाठी दिवस-रात्र मेहनत घेते. जे शिक्षक विद्यार्थ्यांना अभ्यासाबरोबरच आयुष्यबद्दल, करियरबद्दल आपुलकीने मार्गदर्शन करतात, त्यांना मुलांकडून फक्त शैक्षणिक यशच नाही, तर आजन्म आदरही मिळतो.
नाही म्हणायला, बऱ्याच वेळेला चांगले वागण्यात फायदाही दडलेला असतो. इंटरनेटच्या जगात, अनेक सोशल मीडिया इन्फ्लुएन्सर मोफत सल्ले देतात, उदाहरणार्थ फिटनेस टिप्स, आर्थिक योजना. जर मनापासून, कळकळीने आणि अनुभवसिद्ध सल्ले सांगितले तर लोकांमध्ये विश्वास निर्माण होतो. नंतर तेच लोक त्यांचे कोर्स खरेदी करतात.
परस्परतेची सावलीही असते. प्रत्येक गोष्टीची एक दुसरी बाजूही असते. जसा चांगुलपणा परत येतो तसा वाईटपणाही. तीही ‘परस्परताच’ म्हणावी. जर आपण कुणाकडे दुर्लक्ष केले, तर तेही आपल्याला दुर्लक्षित करतात. जर आपण कुणाला त्रास दिला, तर तेही कधी ना कधी तसाच परत देतात. हेच तत्त्व राष्ट्रांमध्येही लागू पडते. एक देश निर्बंध लावतो, तर दुसरा देशही त्याचे आपल्या पद्धतीने (अगदी लगेच नाही तरी योग्य वेळ आल्यावर) बरोबर उत्तर देतो. समर्थ रामदास स्वामींनी दासबोधात म्हटले आहे की “धटासी आणावा धट । उधटासी पाहिजे उधट । खटनटासी खटनट । अगत्य करी ॥” म्हणजेच, दांडग्यासाठी दुसरा दांडगा योजावा. उर्मट माणसास दुसरा उर्मटच योजला पाहिजे. लबाड असेल त्यास दुसऱ्या लबाड माणसासमोर उभे करावे. याप्रमाणे जशास तसे अगत्य योजावे.
एवढे सर्व समजल्यानंतर आपण कशाला कोणाशी मुद्दामून वाईट वागू, नाही का? चांगुलपणाचे मार्गच रास्त. आज तुम्ही जर कुणासाठी चांगले करत असाल, तर उद्या ते चांगले कुठल्या ना कुठल्या रूपात परत तुमच्याकडे येणार आहे. हीच खरी परस्परतेची शक्ती आहे. तिचा उपयोग करा, ती तुमचे जग बदलू शकते. प्रसिद्ध कवी विंदांनी म्हटल्याप्रमाणे “ देणार्‍याने देत जावे, घेणार्‍याने घेत जावे, घेता घेता एक दिवस, देणार्‍याचे हात घ्यावे”.



\chapter{मी करोडपती झालो तर … }

सध्या यूएई (संयुक्त अरब अमिराती) संदर्भातील एक बातमी केवळ भारतातल्या धनिक वर्गाचेच नव्हे, तर उच्च मध्यमवर्गाचेही लक्ष वेधून घेत आहे. कारण यूएईने आपला प्रतिष्ठेचा ‘गोल्डन व्हिसा’ अवघ्या ₹२३ लाखांत देण्याची घोषणा केली आहे. त्याद्वारे तुम्ही तेथे तहहयात वास्तव्य करू शकता आणि नागरी सोयींचा लाभ घेऊ शकता. यापूर्वी या व्हिसासाठी काही कोटी रुपयांची गुंतवणूक आवश्यक होती. त्यामुळे भारतातील अनेक उच्च उत्पन्न गटातील व्यक्ती याकडे आकर्षित होत आहेत. या आकर्षणाचे सर्वात महत्त्वाचे कारण म्हणजे यूएईमध्ये व्यक्तिगत आयकर (डायरेक्ट पर्सनल इन्कम टॅक्स) नाही. पण बदल्यात मिळतात जागतिक दर्जाच्या सुविधा. कोण विचार करणार नाही?

याउलट भारतात काय परिस्थिती आहे? जे नागरिक प्रामाणिकपणे आयकर भरतात, त्यांना काय मिळते? अनेकदा मोडकळीस आलेली सार्वजनिक यंत्रणा, रस्त्यांवरील खड्डे, आणि आरोग्याच्या प्राथमिक सुविधाही नाहीत. असं वाटण्याचं कारण म्हणजे, भारतात अप्रत्यक्ष कर जरी सर्वांना लागू होत असला, तरी प्रत्यक्ष वैयक्तिक आयकर केवळ २-३% लोकच भरतात. उरलेले बरेच लोक खरेच गरीब असल्याने करमुक्त मर्यादेत येतात, तर काही जरी उत्पन्न अधिक असले, तरी त्यांच्या व्यवसायामुळे त्यांना ‘सरसकट’पणे  थेट आयकरभरावा लागत नाही. मग प्रश्न उभा राहतो की, या मोजक्या करदात्यांनीच बाकीच्यांचा भार वाहायचा का?

या विचारांतून एक धाडसी कल्पना निर्माण होते की, भारतानेही जर यूएईप्रमाणे थेट वैयक्तिक आयकर पूर्णपणे बंद केला, तर काय होईल? अर्थव्यवस्था कोसळेल का? की लोक त्यांचे वाचलेले पैसे व्यवसायांत गुंतवतील, जास्त खर्च करतील आणि त्या वाढलेल्या व्यवहारांमधून मिळणाऱ्या अप्रत्यक्ष करांमधून सरकारला जास्त महसूल मिळेल? याचे उत्तर आपण ही योजना प्रत्यक्षात आणण्याआधी ‘विचारप्रयोगांच्या’ (थॉट एक्सपेरिमेंट्स) मेंटल मॉडेलमधून म्हणजेच मन:प्रारूप अथवा विचारचित्रातून  शोधू शकतो.

विचारप्रयोग म्हणजे एखाद्या कल्पनेला वास्तवात अंमलात न आणता, मनातच त्या कल्पनेचे परिणाम, शक्यता आणि धोके तपासणे. हे जणू कल्पनाशक्तीच्या साहाय्याने घेतलेलं वैचारिक प्रतिरूप (सिम्युलेशन) आहे. यात "जर असं झालं, तर काय?" असे प्रश्न विचारून निर्णय घेण्याची स्पष्टता वाढवली जाते. हे विचारचित्र म्हणजे फक्त तात्त्विक चर्चा नाही, तर ते एक प्रभावी विचारसाधन आहे. यासाठी प्रयोगशाळा, निधी, किंवा सरकारची परवानगी लागत नाही तर लागतो तो केवळ कल्पकतेचा उपयोग, थोडं तर्कशास्त्र आणि विचारांचं धाडस.

विचारप्रयोगांचा वापर प्राचीन काळापासून आहे. बृहदारण्यक उपनिषदात ऋषी याज्ञवल्क्य आणि राजा जनक यांच्यात आत्म्याच्या स्वरूपाबाबत संवाद आहे, तोही प्रत्यक्ष अनुभव न घेता, केवळ वैचारिक कल्पनांमधून.

आजही, जेव्हा न्यायालय एखाद्या कायद्याच्या संभाव्य दुरुपयोगाबाबत चिंता व्यक्त करतं, तेव्हा ते ‘या कलमाचा गैरवापर झाला, तर काय?’ असा विचारप्रयोग करते. वास्तवात काही घडलं नसतानाही, फक्त कल्पनेतून सर्व शक्यता तपासून पाहणं केलं जातं. 

आपल्याला वाटेल की विचारप्रयोग हे केवळ विचारवंत किंवा वैज्ञानिक करतात, पण तसे नाही. ते आपल्या रोजच्या आयुष्यातही लपलेले असतात. त्याची काही उदाहरणे पाहुयात:

एक पिता आपल्या मुलीला परदेशात शिक्षणासाठी पाठवायचं की नाही, हे ठरवताना विचार करतो की, “ती यशस्वी झाली तर? अपयशी झाली तर? मी तिला थांबवलं आणि ती दुःखी झाली तर?” या प्रक्रियेतून तो निर्णय घेण्यास तयार होतो.

धोनी, एखाद्या वर्ल्ड कप सामन्यात फलंदाजीच्या क्रमवारीत स्वतःला वर पाठवायचं का, हे ठरवताना, मनातल्या मनात पिचची गती, सामन्याचं दडपण, विरोधकांची मनःस्थिती हे सगळं काही क्षणात तपासतो, फायदे-तोटे मोजतो आणि मग त्वरित निर्णय घेतो. हाच विचारप्रयोग.

एका पुण्यातील उद्योजकाने दुसऱ्या शहरात विस्तार करताना विचार केला की, “जर वितरण यंत्रणा फसली, तर काय? पण जर यशस्वी झाली, तर?” दोन्ही बाजूंचा तौलनिक अभ्यास करून मगच तो पुढचं पाऊल टाकतो.

काही अर्थतज्ज्ञ यूबीआय (युनिव्हर्सल बेसिक इनकम) लागू करण्याची शिफारस करतात. यात प्रत्येक नागरिकाला, त्याच्या उत्पन्नाची पातळी काहीही असो, सरकारकडून एक ठराविक रक्कम नियमितपणे दिली जाते, जीवन जगण्यासाठी आवश्यक किमान आर्थिक आधार म्हणून. इथे विचारप्रयोग असा असतो की, “जर प्रत्येक भारतीयाला ₹१०,००० महिना दिले, तर लोक त्यातील किती साठवतील, किती खर्च करतील? त्या खर्चामुळे महागाई वाढेल का?” या प्रश्नांची उत्तरं शोधण्यासाठी सांख्यिकीसारख्या शास्त्रांचा वापर करून विचारप्रयोग सुरू होतात. सखोल विश्लेषणानंतरच अशा योजना पुढे आणल्या जातात.

विचारप्रयोग महत्त्वाचे का? आपण अनेकदा काहीतरी बिघडल्यानंतरच दुरुस्ती करतो. विचारप्रयोग उलट सांगतात की, आधीच कल्पनांमध्ये सर्व ‘क्रॅश टेस्ट’ करा, जेणेकरून वास्तवात धक्का बसणार नाही. महत्वाचे म्हणजे, केवळ नुकसान टाळण्यासाठीच विचारप्रयोग करायचे असं नाही बरं  का, तर सुखद शक्यतांचाही विचार करता येतो, जसं की, “मी करोडपती झालो तर …”


\chapter{आपला तो बाब्या …}

एका लहानशा खेड्यात, उन्हाने कोरड्या पडलेल्या जमिनीवर पहिल्या पावसाच्या सरी कोसळत आहेत. शेतकरी आपल्या शेतात उभा आहे, दोन्ही हात पसरलेले, डोळे मिटलेले, त्याचं चेहरा आनंदाने फुललेला. हा पाऊस म्हणजे त्याच्यासाठी जीवन आहे. आता त्याच्या कपाशीच्या बियाण्यांत जीव येईल,पीक चांगलं येईल, छान भावाला विकलं गेलं तर मागाच्यावर्षीसकटचं कर्ज फिटेल, त्याच्या कुटुंबाचं वर्ष थोडं सोपं जाईल. त्याच्या समोरचं चित्र, ओलसर, उज्वल आणि आशेचं. पण काहीशा अंतरावर, एक शिक्षक, त्याच पावसाकडे बघत असतो, चिंता घेऊन. गेल्या दोन पावसाळ्यांपासून त्याच्या एकल शाळेच्या खोलीचं छप्पर गळतंय. आता गावापासून येणारा रस्ता चिखलमय होणार, आणि मुलं काही दिवस आजारपणाने अथवा गणवेश सुकला नाही म्हणून येणारच नाहीत. मुलांप्रमाणेच त्याचा भविष्याची चिंता. त्याच्या समोरचं चित्र, ओलसर, काळं आणि उदास. तेच गाव. तोच पाऊस. पण दोन अगदी वेगळी वास्तव. यालाच ‘रेलेटिव्हिटी’ मेंटल मॉडेल म्हणजेच सापेक्षता (मन:प्रारूप) म्हणतात. ती सापेक्षता जी दररोज आपल्या आयुष्यात प्रकट होते. जरी सर्वसाधारणपणे सगळीच लहान मुलं द्वाड-खोडकर असली तरी आपल्याला वाटतं की नाही, “आपला तो बाब्या आणि दुसर्‍याचं ते कार्ट”, तसंच काहीसं. 
आइन्स्टाईनच्या सापेक्षतावादाने आपल्याला शिकवलं की अवकाश आणि काळ-वेळ स्थिर नसतात. ते पाहणाऱ्याच्या गतीवर आणि स्थानावर अवलंबून असतात. एक जण जर रेल्वे प्लॅटफॉर्मवर उभा असेल आणि दुसरा त्या गाडीमध्ये असेल, तर दोघंही तीच गाडी पाहतात, पण अनुभव वेगळा असतो. एकाला गाडी गाडी पुढे जाताना दिसेल तर दुसऱ्याला प्लॅटफॉर्म मागे जाताना आणि दोघंही बरोबर असतात! हे फक्त विज्ञान नाही तर हे जीवनाचं तत्त्वज्ञान आहे. जीवनात ही सापेक्षता अनेक ठिकाणी दिसते. त्याची काही उदाहरणे पाहुयात. 
खेडेगावातील एखाद्या कुटुंबासाठी ₹५०० म्हणजे आठवड्याचं राशन. पण मुंबईच्या समुद्रकिनाऱ्याच्या रेस्टॉरंटमध्ये हे ₹५०० एका स्टार्टरसाठीही पुरणार नाहीत. रक्कम एकच, पण जगण्याची वास्तव वेगवेगळी.
आदिवासी भागातील विद्यार्थ्याने जर ६०% गुण मिळवले, तर अधिक महत्वाचे. तो अनेक अडचणींवर मात करून आलेला असतो. भाषेची अडचण, वीज नाही, शिकवण्या नाहीत. त्याच वेळी एका शहरातील मुलाला ८५% गुण असूनही अपेक्षेपेक्षा कमीच मानायला हवं, नाही का? इथे केवळ आकडे पूर्ण गोष्ट सांगत नाहीत. परीक्षा एकच असली तरी गुणांवरून चिकाटी, जिद्द मोजता येत नाही. 
आपल्याकडे कोठल्याही कौटुंबिक समारंभात "संध्याकाळी ७” ही वेळ असेल तर त्यावेळी यजमानही त्या ठिकाणी हजर असण्याची शक्यता कमीच. पण कॉर्पोरेट मीटिंगमध्ये ५ मिनिट उशीरही चालत नाही. बॉस त्याची नोंद ठेवतो आणि मग त्याचा वापर कोठे होतो ते सांगायलाच नको. वेळेचं महत्व सुद्धा सापेक्षच.
नाटकात नाट्यमय (‘लाऊड’) अभिनय गरजेचा असतो आणि तो प्रेक्षकांना थेट भिडतो, पण तसाच अभिनय एखाद्या चित्रपटात अथवा वेबसिरीजमध्ये केल्यास ‘ओव्हर अ‍ॅक्टिंग’ वाटू शकते. 
एका कलात्मक चित्रपटाला आंतरराष्ट्रीय महोत्सवात पारितोषिक मिळतं, पण स्थानिक चित्रपटगृहामध्ये प्रेक्षक तो अर्धवट सोडून निघून जातात. त्याचवेळी एखाद्या मसाला चित्रपटाला टाळ्यांचा कडकडाट होतो, पण समीक्षकांकडून त्याला ‘दर्जाहीन’ म्हटलं जातं. चांगल्या सिनेमाचं मोजमापही आपापल्या दृष्टिकोनावर ठरतं.
एखाद्या राष्ट्रासाठी रशियासोबत व्यापार करणे ‘आर्थिक गरज’ असते, तर दुसऱ्या राष्ट्रासाठी तेच ‘धोका’ मानले जाते. एकाच धोरणाचं मूल्यांकन वेगवेगळ्या देशांकडून त्यांच्या स्वतःच्या सुरक्षेच्या आणि हितसंबंधांच्या चष्म्यातून केलं जातं.
कधी एखाद्या क्रिकेटपटूची ३० धावांची खेळी सामना जिंकवणारी ठरते, कारण ती योग्य वेळेस आली असते. तर दुसऱ्याच सामन्यात त्याच्या ७० धावाही कमी लेखल्या जातात, कारण त्या कूर्मगतीने काढलेल्या असतात. आकड्यांपेक्षा वेळ आणि परिस्थिती खेळींचे महत्व ठरवतात.
सापेक्षता मेंटल मॉडेल का समजून घ्यायचे? ते आपल्याला थांबून विचार करायला शिकवते. सत्य हे एकाच कोनातून दिसणारं नसतं. अनेक वाद, विशेषतः समाज-माध्यमात, राजकारणात आणि अगदी कुटुंबातसुद्धा, यामुळे होतात की दोघंही स्वतःच्या जागेवरून योग्य असले तरी त्यांचे विचार विरुद्ध असतात.
सापेक्षतेचा स्वीकार म्हणजे सत्याला नाकारणं नव्हे तर त्याची अनेक बाजुंनी ओळख करून घेणं. थोडंसं स्थान बदललं की नवीन दृष्टिकोन समोर येतो. म्हणून पुढच्यावेळी कोणी जग वेगळ्या प्रकारे पाहत असेल, तर त्याला लगेच चुकीचं ठरवू नका. स्वतःला विचारा: "तो असं काय पाहतोय, की जे मला दिसत नाहीये?" मग सत्य कोणाचं? ते तुम्ही कुठे उभे आहात, यावर ठरतं. प्लॅटफॉर्मवर की गाडीच्या आत. 


\chapter{वाढता वाढता वाढे …}


एखाद्या मोठ्या शहरात सकाळच्या वेळेस तुम्ही सिग्नलवर थांबला असाल, तर एक गोष्ट नक्की जाणवते की, सिग्नल सुटताच मोठा गोंधळ उडतो. एकाच वेळी सगळ्यांना पुढे जायचे असते; रिक्षा उजवीकडून शिरते, दुचाकी मधून वाट काढतात आणि एखादी गाडी आडवी घुसते. याच वेळेला पादचाऱ्यांनाही रस्ता ओलांडायचा असतो. कुणीच हा गोंधळ मुद्दाम करत नाही, पण तरीही तो घडतोच. कारण प्रत्येक जण स्वतंत्रपणे निर्णय घेत असतो आणि पोलीस असले तरी त्यांना हा गोंधळ आवरता येईल व इतके सर्व लोक त्यांचे ऐकतील, याची शक्यता नसते. ही विस्कळीतता म्हणजेच एंट्रॉपी. थर्मोडायनॅमिक्सचा (उष्मागतिकी शास्त्र) हा दुसरा नियम आहे. तो सांगतो की, कोणत्याही व्यवस्थेत बाह्य ऊर्जेचा वापर करून शिस्त न लावल्यास, ती आपोआपच गोंधळाकडे झुकते आणि हा गोंधळ दिवसेंदिवस वाढतच जातो, ‘वाढता वाढता वाढे’ या पद्धतीने. ‘आमच्या वेळेस असं नव्हतं’ हे म्हणूनच खरं ठरतं. शिस्त लावण्यासाठी आणि टिकवण्यासाठी ऊर्जा लागते. ही ऊर्जा खर्च करण्याची तयारी नसल्याने सिग्नलसारखे अराजक हमखास घडते.
थर्मोडायनॅमिक्स हा मेकॅनिकल इंजिनियरिंग (यंत्र-अभियांत्रिकी) किंवा भौतिकशास्त्रातील (फिसिक्स) केवळ एक विषय नसून, ते एक प्रभावी ‘मेंटल मॉडेल’ (मन:प्रारूप) देखील आहे. आयुष्यातील निर्णय, नातेसंबंध, दिनचर्या, कामाची शिस्त आणि सामाजिक संबंध यांसारख्या अनेक गोष्टी थर्मोडायनॅमिक्सच्या नियमांतून समजून घ्यायला मदत होते.
थर्मोडायनॅमिक्सचा पहिला नियम सांगतो की ऊर्जा निर्माण करता येत नाही आणि ती नाहीशीही होत नाही; ती केवळ एका प्रकारातून दुसऱ्या प्रकारात बदलते.
आपण खाल्लेलं अन्न, दिवसभर काम करण्यासाठी वापरलं जातं. शाररिक तसेच मानसिक प्रक्रियांसाठी ही ऊर्जा वापरली जाते. काही खाल्लेच नाही तर ऊर्जा येणारच नाही. ऊर्जा खर्च केली नाही तर मेद-स्वरूपात ती साठून राहते. 

‘वेळ’ हीसुद्धा अशीच एक गोष्ट आहे जी निर्माण करता येत नाही. प्रत्येक व्यक्तीकडे दररोज समान, म्हणजेच २४ तास असतात, मग ती सामान्य व्यक्ती असो की नोबेल पुरस्कार विजेता. पण या वेळेचे काय करायचे, ती कुठे आणि कशात गुंतवायची, हे मात्र पूर्णपणे आपल्या निवडीवर आणि शहाणपणावर अवलंबून असते.
त्याचप्रमाणे शेअर बाजारात ‘जोखीम’ कमी केली म्हणजे ती नाहीशी होत नाही; ती केवळ एका स्वरूपातून दुसऱ्या स्वरूपात रूपांतरित होते. मोठ्या गुंतवणूकदारांनी एका ठिकाणी जोखीम कमी केली, की ती संपूर्ण व्यवस्थेत दुसरीकडे कुठे तरी निर्माण होतेच.
थर्मोडायनॅमिक्सच्या दुसऱ्या नियमानुसार, विस्कळीतता (एंट्रॉपी) कायम वाढत असते, अगदी सिग्नलवर होणाऱ्या गोंधळाप्रमाणे. आपण घर झाडून स्वच्छ करतो, पण नियमित साफसफाई न केल्यास काही दिवसांतच धूळ साचते. ऑफिसमधील कामाच्या जबाबदाऱ्या वेळोवेळी पार पाडल्या नाहीत, तर कामाचा ढीग (आता संगणकात) साचतो. म्हणजेच, आपण काहीच केले नाही, तर गोष्टी बिघडतच जातात. त्यामुळे ऊर्जा हे केवळ वापराचे साधन नसून, ती आपल्या अस्तित्वाचा आणि प्रगतीचा आधार आहे.
थर्मोडायनॅमिक्समधील उष्णतेच्या प्रवाहाची कल्पना आपल्या सामाजिक वागणुकीतही दिसून येते. उष्णता जशी गरम वस्तूपासून थंड वस्तूकडे वाहते, त्याचप्रमाणे एखाद्या टीममधील एका सकारात्मक व्यक्तीचा प्रभाव इतरांवर पडतो. याउलट, सतत तक्रार करणारी व्यक्ती सर्वांची मनःस्थिती खराब करू शकते. त्यामुळे आपण कोणाच्या संगतीत वेळ घालवतो, याचा थेट परिणाम आपल्या ऊर्जेवर होतो.
एखाद्या देशाचा किंवा राज्याचा प्रमुख नेता जर स्वच्छ प्रतिमेचा, नीतिमान, पारदर्शक आणि लोकहितासाठी काम करणारा असेल, तर त्याची ही ऊर्जा, म्हणजेच त्याची मूल्यव्यवस्था, वर्तनशैली आणि सार्वजनिक आचारधर्म, हळूहळू संपूर्ण व्यवस्थेत झिरपतो. त्याच्या मंत्रिमंडळातील सदस्य, खालच्या पातळीवरील प्रशासक, अधिकारी आणि स्थानिक राजकारणी हे सर्व त्या वरच्या पातळीवरून वाहणाऱ्या नैतिकतेच्या दबावाखाली आपले वर्तन सुधारतात. त्यांना ठाऊक असते की चुकीच्या कृत्यांना वरून संरक्षण मिळणार नाही, उलट जबाबदारीची अपेक्षा आहे. त्यामुळे एक संस्थात्मक शिस्त आणि आंतरिक नियंत्रण निर्माण होते. ‘यथा राजा तथा प्रजा’ (काही वेळेस उलटेही सत्य होते!!) हा नियम दिसून येतो.
थर्मोडायनॅमिक्सचे नियम आपल्याला हे शिकवतात की जग एका विशिष्ट पद्धतीने चालते. ऊर्जा मर्यादित आहे, गोंधळ (एंट्रॉपी) नैसर्गिकरीत्या वाढतो, आणि व्यवस्था टिकवणे आव्हानात्मक असते. ही केवळ शास्त्राची नव्हे, तर जीवनशैलीची शिकवण आहे. “ऊर्जा जपा, विस्कळीततेला ओळखा आणि संतुलन राखा” ही त्रिसूत्री आपल्या वैयक्तिक आणि सामाजिक आयुष्यासाठीही तितकीच महत्त्वाची आहे.



\chapter{कडू कारलं, साखरेत घोळलं}

१९८०च्या दशकाच्या उत्तरार्धात आणि १९९०च्या सुरुवातीला भारत एका मोठ्या बदलाच्या उंबरठ्यावर उभा होता. इतर प्रगत देशांप्रमाणेच भारतानेही डिजिटल युगात प्रवेश करणे गरजेचे होते, पण हे ओळखणारे फारच कमी होते. त्याकाळचे पंतप्रधान राजीव गांधी यांनी जेव्हा सरकारी कार्यालयांमध्ये आणि सार्वजनिक संस्थांमध्ये संगणक व माहिती तंत्रज्ञान यांचा वापर सुरू करण्याचा प्रयत्न केला, तेव्हा या प्रयत्नांना जोरदार विरोध झाला. राजकारणाच्या रंगमंचावर समाजवादी पक्ष आणि डाव्या विचारसरणीच्या अनेक गटांनी या निर्णयावर टीका करत, संगणक हे “नोकऱ्या हिरावून घेणारे यंत्र” असल्याचा प्रचार सुरू केला. त्या काळात “संगणक हटाओ, रोजगार बचाओ” हा नारा रस्तोरस्ती ऐकू येत होता. सरकारी कार्यालयांतील टायपिस्ट, लिपिक आणि इतर कर्मचाऱ्यांच्या नोकऱ्या धोक्यात येतील अशी भीती अनेकांच्या मनात होती. संगणक म्हणजे माणसांना विस्थापित करणारी थंड, बेरकी यंत्रं असा समज झाला होता. पण ज्या गोष्टीला त्या वेळी धोका मानले गेले, तीच गोष्ट काही वर्षांनी भारताच्या आर्थिक व तांत्रिक विकासाची नांदी ठरली.

१९९०च्या दशकाच्या अखेरीस भारताने माहिती तंत्रज्ञानाच्या क्षेत्रात जागतिक स्तरावर आपली छाप पाडली. बंगळूर, पुणे आणि हैदराबादसारख्या शहरांनी सॉफ्टवेअर, आउटसोर्सिंग आणि सेवा उद्योगांमध्ये झपाट्याने प्रगती केली. लाखो युवकांना चांगल्या पगाराच्या नोकऱ्या मिळाल्या, त्या थंड, बेरकी संगणकांमुळेच. ज्या तंत्रज्ञानाला आधी विरोध करण्यात आला होता, त्याचे रूपांतर एका संधीमध्ये झाले.

येथे ‘‘इनर्शिया’ (जडत्व) हे मेंटल मॉडेल (मन:प्रारूप) आपल्याला अशा विरोधांमागचं मूळ कारण उलगडून दाखवतं. भौतिकशास्त्रानुसार, स्थिर वस्तू स्थिरच राहते आणि गतिमान वस्तू गतिमानच राहते, जोपर्यंत एखादी बाह्य असंतुलित शक्ती तिच्यावर कार्य करत नाही. हीच संकल्पना आपल्या मनोवृत्ती, वर्तणूक आणि सामाजिक व्यवस्थांनाही लागू पडते. याची काही उदाहरणे पाहूया.

आजही आपल्या देशातील अनेक शाळांमध्ये जुन्या अभ्यासक्रमावर आणि पाठांतरावर आधारित परीक्षा पद्धतीवर भर दिला जातो. "हे तर नेहमीच असंच चालत आलंय," अशी एक मानसिकता तयार होते. शिक्षण पद्धतीत आमूलाग्र बदल करायचा झाल्यास प्रस्थापित पद्धतींना आव्हान द्यावे लागते, जे अनेकांना रुचत नाही.

राजकीय व्यवस्थांमध्येही अशीच एक ठरावीक जडत्वता दिसून येते. मतदार अनेकदा एखाद्या पक्षाला पाठिंबा देतात, जरी त्या पक्षाचा कारभार फारसा समाधानकारक नसला तरी, त्या पक्षाचे नवीन पिढीतील नेते अतिशय सुखवस्तू, विचारहीन व भरकटलेले असले तरी. केवळ परंपरेने, किंवा “नेहमीच यांनाच मत दिलंय,” या विचाराने निर्णय घेतला जातो. अशा वेळेस विचार न करता सवयीने निर्णय होतो आणि समाजाचा विकास खुंटतो.

सरकारी संस्था, विशेषतः सार्वजनिक क्षेत्रातील बँका, वर्षानुवर्षे जुन्याच पद्धती वापरत आहेत. गोष्टी ‘ऑनलाईन’ झाल्या असल्यातरी तक्रार निवारण खुद्द संस्थेत गेल्याशिवाय, ‘वाटाघाटी’ केल्याशिवाय होत नाही. बदलाची गरज असल्याचे सगळ्यांनाच ठाऊक आहे. तरीही त्या जड प्रक्रियांमध्ये हालचाल होत नाही, कारण “आजवर असंच चाललंय” ही मानसिकता आणि स्थिर-स्थावर झालेल्या प्रक्रियेतून मिळणाऱ्या लाभांवर पाणी सोडता येत नाही.  बरं , काही चुकले तर जबाबदारी कोण घेणार या भीतीपायीसुद्धा बदल टाळले जातात. म्हणूनच फायदे स्पष्ट असले तरी त्या बदलाची सुरुवात करणे कठीण जाते.

जडत्व ही मानसिकता केवळ सरकारपुरती मर्यादित नाही. ती आपल्या प्रत्येकाच्या रोजच्या जगण्यात दिसून येते. उदाहरणार्थ, एखादी जुनी सवय मोडणे, जसे की साखरयुक्त चहा पिणे. त्याचा त्याग किती कठीण वाटतो, नाही का? आरोग्यासाठी ते हानिकारक ठरत असले, तरी त्या सवयीचा मोह आपण टाळू शकत नाही. हेच ते जडत्व.

मात्र जडत्व हा शत्रू नसून एक तटस्थ शक्ती आहे. तिचा परिणाम नकारात्मक तेव्हाच होतो, जेव्हा आपण विचार न करता तिला आपल्या निर्णयांवर स्वार होऊ देतो. सवय, परिचय आणि स्थैर्य या गोष्टी चांगल्या असल्या तरी, बदलाची गरज असताना त्या अडथळा ठरू शकतात.

उपाय आहे, “सुरुवात करणे”, अगदी छोट्या पावलांनी. जसे एखादी नवीन सवय फक्त दोन मिनिटांसाठी सुरू करणे, किंवा एक छोटा निर्णय घेणे. हीच कृती हळूहळू गती निर्माण करते. एकदा का सुरुवात झाली, की पुढचा प्रवास आपोआप सुलभ होतो.

भारतात प्रतिभा आणि नवनवीन कल्पनांची कमतरता नाही. अडथळा आहे तो बदलाला सामोरे जाण्याचा. प्रस्थापित व्यवस्थेला केवळ प्रश्न विचारून फायदा नसतो तर एक समांतर बदल यशस्वी करून दाखवावा लागतो.  बदल न होण्यामागे लोकांची असमर्थता हे कारण नसून, बदलाला होणारा मानसिक विरोध आणि पर्याय देण्यात लागणारी धडाडी नसणे हे कारणीभूत ठरते.

म्हणून पुढच्या वेळी जर एखादी गोष्ट आपण “नेहमीच असं केलंय” म्हणून करत असाल, तर थांबा, आणि स्वतःला विचारा की, “हे खरंच प्रभावी आहे का?” जर उत्तर ‘नाही’ असेल, तर पहिले पाऊल टाका, त्या पहिल्या पावलाच्या प्रतिकाराला सामोरे जा. कारण एकदा का धक्का मिळाला, गाडी हलली, ‘मोसम’ (मोशन) पकडला, की तिचे पुढे जाणे अटळ असते.



\chapter{थांबा, पुढे गतिरोधक आहे}

अनेक मोठ्या खाजगी कंपन्यांमध्ये भरतीची प्रक्रिया वरवर पाहता पारदर्शक आणि योग्यता (मेरिट) आधारित वाटते. ऑनलाइन अर्ज करा, मुलाखत द्या आणि निवड झाली तर ऑफर मिळवा. पण या सगळ्या प्रक्रियेच्या मागे काही छुपे अडथळे कार्यरत असतात. अनेक वेळा अर्ज केलेले फॉर्म कुठे गेले हे कळत नाही. आपली पुढच्या फेरीसाठी निवड झाली आहे का, हे विचारण्यासाठी वेबसाईटवर दिलेल्या ईमेलला प्रतिसाद मिळत नाही. कोणी फोन उचलत नाही. कालांतराने कळते की जागा भरली गेली आहे. समजा, पुढच्या फेरीसाठी बोलावणे आले, मुलाखती झाल्या. तुम्हाला वाटते की छान उत्तरे दिली, पण पुन्हा निर्णयासाठी प्रतीक्षा. एचआर (ह्युमन रिसोर्स, मानव संसाधन विभाग) चा नंबर मिळवून विचारता नकार कळतो. कारण? ते दिले जात नाही. आपल्याला कशामुळे नाकारले, याचे स्पष्टीकरण मिळत नाही. हा सर्व केवळ देखावा तर नाही ना, अशी शंका येते. मुद्दाम उशीर करणे, कृत्रिम अडथळे आणणे, जेणेकरून उमेदवार कंटाळून अर्ज करण्याचा विचारच सोडून देतील. मग ती पदे शांतपणे अंतर्गत ओळखींतून भरली जातात. सर्वसामान्य उमेदवाराला जाणवणारा हा संघर्ष कधी मुद्दामहून तर कधी बेपर्वाई, अंतर्गत राजकारण, विस्कळीत प्रणाली आणि एकमेकांपासून वेगळे ठेवलेले अंतर्गत विभाग यांमुळे निर्माण झालेला असतो. भरती पारदर्शक नसल्याने आणि वेळेत प्रतिसाद मिळत नसल्याने उमेदवारांना 'फ्रिक्शन' (घर्षण, संघर्ष) व 'व्हिस्कॉसिटी' (अतरलता) या मेंटल मॉडेल (मन:प्रारूप) याचा अनुभव येतो. 
भौतिकशास्त्रात, 'फ्रिक्शन' म्हणजे दोन घन पृष्ठभाग एकमेकांवर घासले गेल्याने निर्माण होणारा प्रतिकार. यामुळे गती कमी होते, उष्णता निर्माण होते आणि पुढे जाण्यासाठी जास्त ऊर्जा लागते. तर ‘व्हिस्कॉसिटी‘ म्हणजे द्रवपदार्थाच्या वाहण्याला होणारा अडथळा. जसे मध पाण्याच्या तुलनेत हळू वाहतो, कारण त्याची 'व्हिस्कॉसिटी' (अतरलता) जास्त असते. या दोन्ही संकल्पना गतिरोधकता दर्शवणारे मेंटल मॉडेल म्हणून वापरता येतात. सामाजिक, संस्थात्मक किंवा वैयक्तिक पातळीवर कोणतीही 'प्रतिकारक शक्ती' असेल, तर ती गतिरोधकतेच्या स्वरूपात दिसते. उदाहरणार्थ, सामन (रावस) माशाचा प्रवास हा प्रवाहाविरुद्ध असतॊ, कष्टप्रद पण गंतव्याकडे पोहोचण्याची आस कधी ढळू न देता तो परिस्थितीला तोंड देत प्रवास चालू ठेवतो. अशी गतिरोधकता नेहमीच्या जीवनात केवळ अडथळ्यांच्या स्वरूपातच नाही तर संघर्षांच्या, कधीकधी लढ्याच्या रूपात प्रकट होते, अगदी, तुकोबारायांनी म्हणल्याप्रमाणे ‘रात्रंदिन आम्हा युद्धाचा प्रसंग’. अशाप्रकारे या मन:प्रारूपाची दैनंदिन जीवनात दिसणारी काही उदाहरणे पाहूया. 
विमान प्रवास घ्या. तिकीट बुक करणे सोपे वाटते, पण त्यात बदल करायचा झाल्यास ‘फ्रिक्शन’ सुरू होते. दडवलेले बदल-शुल्क, गोंधळात टाकणाऱ्या अटी आणि ग्राहक सेवेतील प्रतीक्षा. एअरलाईनच्या अंतर्गत प्रणाली, आरक्षण, ग्राहक सेवा आणि वेळापत्रक विभाग हे एकमेकांशी समन्वय ठेवत नाहीत आणि त्यामुळे आपल्या वाट्याला वाईट अनुभव येतो. ज्या कंपन्या ह्या गोष्टी सुधारतात, त्या प्रवाशांचा विश्वास जिंकतात.
व्हिसा प्रक्रिया हे आणखी एक उदाहरण. अर्ज करताना लागणारी कागदपत्रे, महिनोंमहिने अपॉइंटमेंट न मिळणे आणि कॉन्स्यूलेटच्या संथ कामकाजामुळे, पूर्ण प्रक्रिया संथ व अतरल दिसते. ज्या देशांनी हे डिजिटल केले, प्रक्रिया स्पष्ट केली आणि ट्रॅकिंग प्रणाली उपलब्ध केली, तिथे पर्यटक, कौशल्य आणि गुंतवणूक वाढली.
वैयक्तिक जीवनातही हे लागू होते. अस्ताव्यस्त टेबल, मंद झालेला संगणक किंवा एकाच वेळी अनेक टॅब उघडणे हे गतिमान कामातील अडथळेच आहेत. एका वेळेस एक काम नीट न करता सतत वेगवेगळ्या कामांमध्ये उडी मारणे, हे कार्यक्षमतेत अडथळा आणते. यावर उपाय म्हणजे कार्यक्षेत्र साफ ठेवणे, प्रत्येक कामाला ठराविक वेळ राखून ठेवणे आणि चांगली साधने वापरणे. ही गतिरोधकता संपली तर कार्यक्षमता वेगाने वाढू शकते.
या सर्व उदाहरणांवरून आपल्याला वाटेल की गतिरोधकता (‘फ्रिक्शन’ आणि ‘व्हिस्कॉसिटी’) ही नकारात्मकच गोष्ट असून ती टाळणे, कमी करणे गरजेचे आहे. तर, तसे नाही. या दोन्ही गोष्टी कधीकधी आवश्यक असतात. टायरला रस्ता धरून ठेवण्यासाठी 'फ्रिक्शन' लागतेच. इंजिन ऑईल जर पाण्यासारखे वाहायला लागले, तर त्याचा उपयोग नाही. अशाचप्रमाणे, द्वि-चरण तपासणी (टू-स्टेप व्हेरिफिकेशन) सारखा पर्याय 'फ्रिक्शन' वाटत असला तरी सुरक्षा वाढवतो. निर्णय प्रक्रियेमध्ये 'व्हिस्कॉसिटी' जर विचारपूर्वक असेल, तर ती फायद्याची असते. शरीरातील रक्त खूप पातळ किंवा घट्ट झालेले चालत नाही; त्यात समतोल असणे महत्त्वाचे आहे. 
म्हणून, पुढच्या वेळी जेव्हा तुम्हाला एखाद्या प्रक्रियेत अडथळा जाणवेल, तेव्हा तो लगेच कमी करण्याचा प्रयत्न करू नका. भौतिकशास्त्रज्ञासारखा विचार करा. 'फ्रिक्शन' कुठे आहे, 'व्हिस्कॉसिटी' कुठे आहे, ती कशामुळे आहे हे बघा आणि मग ती जर हानिकारक असेल तरच कमी करण्यासाठी कृती करा.


\chapter{वेगे वेगे धावू }

रमेश नुकताच पदवीधर झाला होता. त्याने व्यवसायात उडी घ्यायचे ठरवले आणि गुंतवणूक व विमा एजन्सी सुरू केली. आवश्यक परीक्षा उत्तीर्ण होऊन त्याने परवानग्यादेखील मिळवल्या. आता ग्राहक मिळवण्यासाठी त्याची धावपळ सुरू झाली. दिवसभर व्यस्त वेळापत्रक; कधी बैठका, कधी फोन कॉल्स, कधी ईमेल्स, तर कधी स्वतःच ठरवलेली ‘टार्गेट्स’ पूर्ण करण्याची घाई. वरवर पाहता हे सर्व प्रगतीसारखे दिसत होते. पण थोडे खोलात पाहिल्यास लक्षात येत होते की, या धावपळीला कुठल्याही ठोस उद्दिष्टाची दिशा नव्हती. व्यवसाय विशेष वाढत नव्हता. असलेले ग्राहक टिकवण्यातच वेळ खर्च होत होता. नवीन गुंतवणूक संधींचा अभ्यास करून त्या ग्राहकांना समजावून सांगितल्यास अधिक ग्राहक जोडले जातील, पण त्यासाठी त्याला वेळच मिळत नव्हता. ना कोणती योजना, ना स्पष्ट दिशा; केवळ पळणे सुरू होते. हे जणू ट्रेडमिलवर वेगाने धावताना बाहेरील दृश्य स्थिर राहण्यासारखे होते.
इथे ‘व्हेलॉसिटी’ (वेग) हे मेंटल मॉडेल (मन:प्रारूप) आपल्याला योग्य दृष्टी देते. हे मॉडेल भौतिकशास्त्रातून घेतलेलं असलं, तरी याचा उपयोग आपल्या वैयक्तिक आणि व्यावसायिक आयुष्यातही तितकाच आहे. ‘व्हेलॉसिटी’ म्हणजे केवळ ‘स्पीड’ (गती) नव्हे, तर दिशेसह असलेली गती. केवळ किती गतीने चाललो आहोत हे महत्त्वाचे नाही, तर आपण योग्य दिशेने चाललो आहोत की नाही, हे समजून घेणे पण महत्त्वाचे आहे. आपण निर्णय घेताना, करिअर निवडताना किंवा व्यवसाय करताना, ही दिशा निर्णायक ठरते. आपण करत असलेल्या कष्टांमुळे आपण खरोखरच आपल्या ध्येयाकडे जात आहोत की केवळ थकत आहोत? हे समजून घेण्यासाठी हे मॉडेल उपयुक्त आहे. याची काही उदाहरणे पाहूया.
आज अनेकजण नवनवीन कोर्स करत राहतात, सतत नोकऱ्या बदलतात किंवा ओव्हरटाईम करतात. पण हे सर्व करताना त्यांचे अंतिम ध्येय काय आहे, याचा स्पष्ट विचार नसतो. त्यांना नेता बनायचे आहे की उद्योजक, की एखाद्या विशिष्ट कौशल्यात पारंगत व्हायचे आहे? हा विचारच स्पष्ट नसतो. म्हणूनच, ‘व्हेलॉसिटी’ हा प्रश्न विचारतो: तुमचा प्रवास तुम्हाला खरोखर तुमच्या ध्येयाजवळ नेत आहे का? की ‘एक ना धड, भाराभर चिंध्या’ असे प्रयत्न सुरू आहेत?
काही कंपन्या केवळ ट्रेंड्सच्या मागे धावतात. आज एआय (कृत्रिम बुद्धिमत्ता), उद्या ई-कॉमर्स, परवा पर्यावरणपूरक योजना. परंतु, जर यामागे त्यांच्या मूळ उद्दिष्टांचा विचार नसेल, तर अशा हालचालींमुळे केवळ लक्ष विचलित होते आणि ऊर्जा वाया जाते. याउलट, ‘व्हेलॉसिटी’ असलेले व्यवसाय त्यांच्या ध्येयाशी आणि मिशनशी सुसंगत असलेल्या गोष्टींवरच लक्ष केंद्रित करतात.
सरकारे अनेकदा केवळ कृती दाखवण्यासाठी एकामागून एक योजना सुरू करतात. एकाच रस्त्याचे उदाहरण घ्या: आधी डांबरीकरण, मग पाण्याच्या पाईपलाईनसाठी खोदकाम, त्यानंतर काँक्रिटीकरण आणि मग केबलसाठी पुन्हा खोदकाम. या सगळ्यानंतर मेट्रोचे काम सुरू होतेच. सरतेशेवटी नागरिकांच्या पदरात काय पडते, हे आपण पाहतोच. नुसताच पैशाचा व्यय. मग ‘सारे काही (नवीन) टेंडरसाठी’ तर नाही ना असे वाटायला लागते. खरी प्रगती तेव्हाच होते, जेव्हा या योजना नीट अभ्यास करून, योग्य क्रम ठरवून आणि दीर्घकालीन उद्दिष्टांशी जोडून राबवल्या जातात. 
आपल्या समाजात ‘हसल’ (धडपड) म्हणजेच प्रगती, असा एक गैरसमज रूढ झाला आहे. लवकर उठा, काम करा, जास्त काम करा. पण जर आपण दहा वेगवेगळ्या गोष्टींचा पाठलाग करत असू आणि त्यांच्यामागे का धावत आहोत हेच माहिती नसेल, तर आपण काहीच साध्य करू शकणार नाही. ‘व्हेलॉसिटी’ आपल्याला विचार करायला भाग पाडते की, मी कुठे चाललो आहे? माझ्या कृती त्या दिशेला पूरक आहेत का? माझ्या मार्गातील कोणते अडथळे मी दूर करू शकेन?
या मॉडेलचा एक महत्त्वाचा भाग म्हणजे नियमितपणे आपल्या मार्गाचे मूल्यमापन करून तो सुधारणे (‘कोर्स करेक्शन’). जसे एखादे क्षेपणास्त्र प्रक्षेपणानंतर वेळोवेळी आपला मार्ग तपासते आणि सुधारते, तसेच आपणही बदलत्या परिस्थितीनुसार, नवीन माहितीनुसार आणि आपल्या उद्दिष्टांनुसार आपल्या दिशेमध्ये आवश्यक बदल करायला हवेत. ‘व्हेलॉसिटी’ आपल्याला आठवण करून देते की आयुष्य ही शर्यत नसून एक अर्थपूर्ण यात्रा आहे. केवळ वेग असून चालत नाही, तर योग्य दिशाही हवी. अन्यथा, हाती फक्त थकवा, निसटलेल्या संधी आणि पश्चात्तापच उरतो. याउलट, जर आपल्या कृती ध्येयाशी सुसंगत असतील, तर लहान पावलांमधूनही मोठे परिवर्तन घडू शकते. म्हणून, पुढच्या वेळी आपल्या जीवनाच्या बोटीचे इंजिन किती शक्तिशाली आहे याकडेच लक्ष न देता, आपले होकायंत्र (कम्पास) योग्य दिशा दाखवत आहे की नाही, हेदेखील तपासा. कारण खरे यश केवळ गतीत नाही, तर योग्य दिशेने साधलेल्या वेगात आहे.


\chapter{अवकाशात एक जागा नि तरफ द्या…}

रमेश आणि सुरेश दोघेही जिवलग मित्र. दोघेही एका नामांकित अभियांत्रिकी महाविद्यालयातून पदवीधर झालेले. दोघेही समान हुशार आणि मेहनती होते. सुरेशला यंत्रांमध्ये रस होता, म्हणून तो एका वाहननिर्मिती कंपनीत उत्पादन विभागात कामाला लागला. रमेशला संगणक आवडत असल्याने त्याने एका सॉफ्टवेअर कंपनीत काम सुरू केले. दोघेही दिवसाचे १०-१२ तास कष्ट करायचे, शिस्तबद्ध आणि निष्ठावान होते. पण दहा वर्षांनंतर त्यांचे आयुष्य वेगवेगळ्या मार्गांनी पुढे गेले.
वाहन उद्योगातील सुरेश एक स्थिर पगार मिळवत राहिला, त्याने सहकाऱ्यांचा आदर मिळवला आणि तो हळूहळू पदोन्नती घेत गेला. रमेशने काही काळानंतर, चांगला अनुभव घेतल्यानंतर धडाडी करून स्वतःची कंपनी सुरु केली. त्याला माहिती झालेली समस्या सोडवण्यासाठी हुशारीने व मेहनतीने एक सॉफ्टवेअर (संगणक प्रणाली) तयार केले. त्याने ते जागोजागी प्रदर्शित करून बऱ्याच लोकांपर्यत पोहोचवले. लोकांना त्याचा फायदा दिसू लागला. त्याचे ते उत्पादन आंतरराष्ट्रीय स्तरावर पोहोचले. डॉलरमध्ये उत्पन्न मिळू लागले. एकामागोमाग एक कंत्राटे मिळत रमेशच्या उत्पन्नातही प्रचंड वाढ झाली. इथे दोघा मित्रांच्या उत्पन्नातील फरक फार मोठा झाला. तो फरक त्यांच्या मेहनतीत नव्हता, तर त्यांनी ती मेहनत कशावर घेतली यात होता. कमी किंवा त्याच मेहनतीत जास्त फायदा मिळवणे, यालाच म्हणतात लेव्हरेज (प्रवर्धन) मेंटल मॉडेल (मन:प्रारूप). भौतिकशास्त्रात ‘लेव्हरेज’ म्हणजे कमी ताकद वापरून मोठे ओझे उचलण्याचा उपाय. नेहमीच्या जीवनात लेव्हरेज म्हणजे कमीत कमी साधने, तंत्रज्ञान, नातेसंबंध किंवा भांडवल वापरून आपल्या तुलनेने कमी कामाचे परिणाम अनेक पटींनी वाढवणे. काही लोक हे कौशल्य शिकून आपल्या कष्टांना गुणतात, तर काही जण तेवढीच मेहनत करूनही फारसे पुढे जाऊ शकत नाहीत. फरक केवळ लेव्हरेजचा असतो.
आधुनिक तत्त्वचिंतक नवल रविकांत यांनी लेव्हरेजचे तीन प्रमुख प्रकार सांगितले आहेत. पहिला प्रकार आहे 'श्रम' (लेबर), म्हणजेच इतरांकडून काम करून घेणे किंवा एक टीम तयार करणे. दुसरा प्रकार आहे 'भांडवल' (कॅपिटल), म्हणजेच पैशातून अधिक पैसा कमावणे, अर्थात गुंतवणूक. आणि तिसरा, सर्वात प्रभावी प्रकार म्हणजे 'परवानगी-मुक्त प्रवर्धन' (पर्मिशनलेस लेव्हरेज). यात अशा गोष्टी येतात, ज्या एकदा तयार केल्या की त्यांची प्रतिकृती बनवायला नगण्य खर्च येतो. उदाहरणार्थ, सॉफ्टवेअर, पुस्तके किंवा ऑनलाइन व्हिडीओ. हे एकदा तयार झाले की विशेष खर्च न करता लाखो लोकांपर्यंत पोहोचू शकतात आणि प्रचंड नफा मिळवून देऊ शकतात. याची काही इतर उदाहरणे पाहूया.
एका शिक्षिकेचा विचार करूया. त्या दररोज ५० विद्यार्थ्यांना शिकवतात. त्यांचे शिकवणे प्रभावी असले, तरी त्याची पोहोच मर्यादित आहे. पण त्याच शिक्षिकेने आपले व्याख्यान रेकॉर्ड करून ऑनलाइन उपलब्ध केल्यास, हजारो विद्यार्थी देशभरातून ते पाहू शकतात. काम तेच, पण पोहोच प्रचंड वाढली. नंतर काही व्याख्याने सशुल्क केल्याने पैसे ही मिळाले. हेच आहे तंत्रज्ञानाचे लेव्हरेज.
अर्थ-जगतात हे अधिक स्पष्टपणे दिसून येते. एक गुंतवणूकदार एखाद्या स्टार्टअपमध्ये ₹१ लाख गुंतवतो. कंपनी यशस्वी झाली, तर ही रक्कम शंभरपट वाढू शकते. त्याचे स्वत:चे कोणतेही अतिरिक्त श्रम नाहीत, फक्त भांडवलाचे योग्य नियोजन. हेच भांडवलाचे लेव्हरेज.
सरकारी क्षेत्रातही हे दिसून येते. एखाद्या अधिकाऱ्याने नागरिकांच्या अर्ज-सेवांसाठी एक ऑनलाइन पोर्टल तयार केल्यास, हजारो सेवा रोज अत्यल्प मनुष्यबळ वापरून हाताळल्या जाऊ शकतात. एकदाच केलेले काम, सतत परिणाम देत राहते.
पण लेव्हरेज ही दुधारी तलवारीसारखी आहे. चुकीच्या पद्धतीने वापरल्यास ती नुकसान करू शकते. उदा., वित्तीय लेव्हरेजचा अतिरेक केल्यास कर्जाच्या गर्तेत अडकण्याची शक्यता असते. तसेच, आपण आपल्या पदाचा किंवा प्रभावाचा गैरवापर केला, तर नातेसंबंध, विश्वास आणि प्रतिष्ठा या सगळ्यांचे नुकसान होऊ शकते. म्हणूनच लेव्हरेज वापरताना विवेकबुद्धी आणि दूरदृष्टी आवश्यक असते.
भारतीय तरुणांसाठी तंत्रज्ञानाचे लेव्हरेज ही संधी मोठी आहे. पूर्वी केवळ प्रस्थापितांच्या कवेत असणारे तंत्रज्ञान आता सर्वसामान्यांना उपलब्ध आहे. ज्ञान आणि कल्पकता असलेल्यांकडे आता मोठे लेव्हरेज आहे, केवळ श्रीमंत अथवा सत्ताधीशांकडे नव्हे.
दोन हजार वर्षांपूर्वी अर्किमिडीजने म्हटले होते, "मला उभे राहण्यासाठी अवकाशात एक जागा आणि एक बाजू लांब असलेली एक तरफ द्या, मी संपूर्ण पृथ्वी उचलून दाखवेन." आजच्या जगात ती तरफ म्हणजेच वेगवेगळ्या प्रकारचे लेव्हरेजेस. योग्य जागा निवडा, योग्य कौशल्ये वापरा आणि अशा गोष्टी तयार करा, ज्याच्या लेव्हरजने तुमचा प्रभाव आणि उत्पन्न वाढवत राहतील. कारण यश केवळ ताकदीत नाही, तर योग्य ठिकाणी लेव्हरेज वापरण्यात आहे.



\chapter{ल्युडोतील सहाचं दान}

एकोणिसाव्या शतकाच्या उत्तरार्धात जेव्हा ब्रिटिश साम्राज्य भारतावर राज्य करत होते, तेव्हा अनेक ब्रिटिश अधिकारी भारतीय लोकांकडे आणि त्यांच्या उद्यमशीलतेकडे संशयाने व तुच्छतेने पाहत असत. त्यांना वाटत असे की भारतातील लोक छोटे-छोटे व्यवसाय करतील, पण मोठा उद्योग चालवण्यास सक्षम नाहीत. विशेषतः पोलादनिर्मितीसारखा आधुनिक आणि महत्त्वाकांक्षी उद्योग चालवणे हे त्यांच्या मते भारतासाठी अशक्य होते. पण जमशेदजी टाटा यांना हा अपमान जिव्हारी लागला. त्यांनी ब्रिटिशांच्या या संकुचित विचारसरणीला केवळ शब्दांनी नव्हे, तर कृतीने उत्तर देण्याचा आणि त्यातूनच एक स्वतंत्र, आत्मनिर्भर भारत घडवण्याचा संकल्प केला. ब्रिटिशांच्या शंका, वंशभेद आणि राजकीय अडथळे झुगारून, त्यांनी १९०७ मध्ये टाटा आयर्न अँड स्टील कंपनी (आजची टाटा स्टील) सुरू केली. जमशेदजींनंतरसुद्धा त्यांचा हा संकल्प त्यांच्या वारसांनी पुढे नेला आणि भारताच्या औद्योगिक परिवर्तनाचा पाया रचला. जमशेदजींची  कहाणी केवळ उद्योगाची नसून, ती माणसाच्या जिद्दीची, ताकदीची आहे. त्यांच्या जीवनातून आपण एक मेंटल मॉडेल (मन:प्रारूप) शिकू शकतो, ते म्हणजे ऍक्टिव्हेशन एनर्जी, अर्थात ‘स्फुल्लिंग ऊर्जा’. ही संज्ञा विज्ञानातून आली असली तरी, ती आयुष्यातही तितकीच लागू होते.
रसायनशास्त्रात, एखादी रासायनिक प्रक्रिया सुरू होण्यासाठी ज्या किमान ऊर्जेची गरज असते, तिला ऍक्टिव्हेशन एनर्जी म्हणतात. ही ऊर्जा मिळाली नाही, तर काहीच घडत नाही. जसे ल्युडो (सारिपाटाचे एक रुपडे) खेळात जोपर्यंत सहा या आकड्याचे दान पडत नाही, तो पर्यंत तुम्हाला खेळाला सुरुवातच करता येत नाही. आपल्या दैनंदिन जीवनातही नेमके हेच घडते. आपल्याकडे वेळ असतो, साधने असतात, ज्ञान असते, पण तरीही सुरुवात करायला एक “ठिणगी” लागते. हाच तो क्षण असतो, जेव्हा आपण स्थैर्यातून क्रियाशीलतेकडे वळतो. एखादे काम सुरू करणे कठीण वाटते, पण एकदा का सुरुवात झाली, की पुढे जाणे तुलनेने सोपे होते. हीच गोष्ट आपली टाळाटाळ, आळस आणि नवीन सवयी लावण्यातील अपयश यामागील कारण स्पष्ट करते. याची काही उदाहरणे पाहूया.
रमेश परीक्षा जवळ आलेली असूनही अभ्यास सुरू करत नव्हता. पण एक दिवस त्याचा मित्र (आणि प्रतिस्पर्धी) सुरेशशी बोलताना त्याला कळले की, सुरेशचा आठपैकी पाच धड्यांचा अभ्यास पूर्ण झाला आहे. हे ऐकताच रमेश खडबडून जागा होतो, आपले टेबल आवरतो, मोबाईल दुसऱ्या खोलीत ठेवतो, पुस्तक उघडतो आणि तासन्‌तास अभ्यासाला लागतो. सुरुवात करणे हेच सर्वात कठीण होते. अशा वेळी, एक छोटीशी ठिणगीही पुरेशी ठरते.
एका छोट्या व्यवसायालाही हे तत्त्व लागू होते. एका स्टार्टअपने एक उत्तम उत्पादन बनवले होते, पण विक्री मात्र होत नव्हती. एके दिवशी त्याच्या गुंतवणूकदाराचा फोन आला आणि त्याने थेट धमकी दिली की, कामगिरी सुधारली नाही, तर पुढचा निधी मिळणार नाही. यामुळे मोठी धावपळ उडाली. तातडीने सोशल मीडियावर मार्केटिंग सुरू झाले, एका प्रसिद्ध इन्फ्लुएन्सरला उत्पादनाबद्दल पोस्ट करण्यास सांगितले आणि ती पोस्ट व्हायरल झाली. बघता बघता उत्पादनाची विक्री वाढू लागली आणि गोष्टी पुन्हा रुळावर आल्या. एका फोनकॉलमधील धमकीने सर्व चित्र पालटले.
सुरुवात करणे कठीण असते. म्हणूनच आपण गुंतवणूक पुढे ढकलतो, कठीण संभाषण टाळतो किंवा एखादे ध्येय सतत लांबणीवर टाकतो. आणि आपण जेवढा वेळ सुरुवात न करता थांबतो, तेवढी ती आवश्यक ऊर्जा अधिकच जास्त वाटू लागते. पण याउलट, सुरुवातीचा अडथळा कमी केला की कामाला वेग येतो. 
ऍक्टिव्हेशन एनर्जी कमी करणे हे प्रगतीसाठी उपयोगी असले तरी, व्यवसायात काही वेळा प्रतिस्पर्ध्यांना रोखण्यासाठी ती मुद्दाम वाढवावी लागते. यालाच "बॅरियर टू एंट्री" म्हणजेच ‘प्रवेशातील अडथळा’ म्हणतात. जेव्हा एखादी कंपनी आपल्या व्यवसायात मोठी गुंतवणूक, विशेष तांत्रिक कौशल्य किंवा कठोर परवाना नियमांची गरज निर्माण करते, तेव्हा नवीन स्पर्धकांना प्रवेश करणे अत्यंत कठीण होते. उदाहरणार्थ, एक औषधनिर्माण कंपनी नवीन औषध बाजारात आणण्यासाठी अनेक वर्षे संशोधन, परवाने आणि मान्यता प्रक्रियेत घालवते. एकदा का त्या औषधाला पेटंट मिळाले की, इतर कंपन्यांना ते बनवता येत नाही आणि त्या उद्योगातील स्पर्धा आपोआप कमी होते.
आपण सभोवताली पाहतो की सर्वच यशस्वी लोक किंवा संस्था अत्यंत बुद्धिमान नसतात, पण ते ‘सुरुवात’ करतात. ‘ऍक्टिव्हेशन एनर्जी’ हे मन:प्रारूप आपल्याला सांगते की यशासाठी सर्वात महत्त्वाची गोष्ट बुद्धिमत्ता किंवा वेळ नसून, ती म्हणजे ‘सुरुवात’ करणे. छोट्याने सुरुवात करा. पोषक वातावरण तयार करा. स्वतःला कृतीसाठी उद्युक्त करा. कारण सवय असो, उद्योग असो किंवा एखादी सामाजिक चळवळ असो, प्रत्येकाची सुरुवात एका ठिणगीनेच होते. एखादा अपमान जसा मोठा उद्योगसमूह निर्माण करतो, त्याचप्रमाणे जुलमी शासनाविरुद्ध उचललेले मूठभर मीठसुद्धा स्वातंत्र्याची एक विराट चळवळ उभी करू शकते.



\chapter{दुग्ध-शर्करा योग }


दरवर्षी भारतात हजारो तरुण पदवीधर होतात. पण या पदव्या असूनही अनेकांना चांगली नोकरी मिळवण्यात अपयश येते. याच वेळी कंपन्यांच्या तक्रारी वेगळ्याच असतात. उपलब्ध पदवीधरांमध्ये त्यांना हवे तसे कौशल्य मिळतच नाही. हे केवळ ज्ञानाच्या अभावाचे नव्हे, तर कौशल्यांबरोबरच्या योग्य मिश्रणाच्या अभावाचे चित्र आहे. कोड (संगणक प्रणाली) लिहिता येतो म्हणून केवळ चालत नाही, संवादकौशल्यही तितकेच गरजेचे असते. कालांतराने नेतृत्व करणाऱ्याला सहवेदना (एम्पथी) नसेल, तर काही खरं  नाही. खरी कमतरता आहे ती योग्य मिश्रणाची. म्हणजेच ज्ञानाच्या, योग्य कौशल्यांच्या आणि दृष्टिकोनांच्या (अ‍ॅटिट्यूड) मिश्रणातून तयार होणाऱ्या संमिश्रणाची, 'अ‍ॅलॉय'ची.
धातुविज्ञानात (मेटॅलर्जी) 'अ‍ॅलॉयिंग' म्हणजे दोन किंवा अधिक धातू एकत्र करून त्यांचे गुणधर्म अजून वाढवणे. लोखंड आणि कार्बन एकत्र करून तयार होणारे स्टील, हे त्याचे प्रमुख उदाहरण. पण जेव्हा आपण या संकल्पनेला 'मेंटल मॉडेल' (मन:प्रारूप) म्हणून पाहतो, तेव्हा 'अ‍ॅलॉयिंग' म्हणजे फक्त धातूंचे मिश्रण नव्हे, तर कल्पना, कौशल्ये आणि दृष्टीकोन यांचे सर्जनशील एकत्रीकरण होय.
'अ‍ॅलॉयिंग' या संकल्पनेचा गाभा 'संश्लेषण' (सिंथेसिस) आहे. केवळ एकाच विषयात पारंगत होणे आज पुरेसे नाही; विविध क्षेत्रांचा समन्वय अधिक फलदायी ठरतो. जसे तांबे आणि कथील (टिन) यांचे मिश्रण 'कांस्य' (ब्रॉन्झ) तयार करते, तसेच दोन वेगळ्या शास्त्रांची युती नवी दिशा निर्माण करते. उदाहरणार्थ, अर्थशास्त्र आणि मानसशास्त्र एकत्र केल्यामुळे ‘वर्तणूक अर्थशास्त्र' (बिहेवियरल इकॉनॉमिक्स) जन्माला आले. हे म्हणजे समन्वय (सिनर्जी) जिथे एकत्र मिळालेला परिणाम वेगवेगळ्या भागांच्या बेरजेपेक्षा अधिक असतो, म्हणजेच दुग्ध-शर्करा योग म्हणायचा, अथवा  “दो और दो, पाँच”. याची काही उदाहरणे पाहूयात.
शिक्षण आणि तंत्रज्ञान या दोन वेगळ्या क्षेत्रांतील समन्वयातून 'एड-टेक' म्हणजेच शैक्षणिक तंत्रज्ञान क्षेत्र जन्माला आले. इथे शिक्षकाचे ज्ञान आणि एआय (आर्टिफिशिअल इंटेलिजन्स, कृत्रिम बुद्धिमत्ता) यांची शक्ती एकत्र येते, आणि त्यातून वैयक्तिक पातळीवर झेपेल तसे शिकवण्याची क्षमता (पर्सनलाइज्ड एज्युकेशन) निर्माण होते. केवळ शिक्षक किंवा केवळ संगणक यापैकी कोणतेही एकटे हे साध्य करू शकत नाही.
भारतीय शास्त्रीय संगीत आणि पाश्चात्य संगीत यांचे फ्युजन हे असेच उदाहरण. एक पारंपरिक तबलावादक आणि एक जॅझ वादक एकत्र काम करतात. प्रारंभी त्यांचे संगीत परस्परविरोधी वाटते, पण ज्या क्षणी त्या दोघांमध्ये सर्जनशीलतेबरोबर परस्पर आदर आणि समजून घेण्याची तयारी येते, त्याक्षणी एक नवा संगीतप्रकार तयार होतो. जुगलबंदी अफलातून होते. प्राचीन परंपरा आणि आधुनिक प्रयोगशीलता यांचा हा सुरेल मिलाफ.
लोकशाही धोरणे बनवताना जर केवळ आकडेवारीवर आधारित निर्णय घेतले, तर ते कधी कधी वास्तवात चालत नाहीत. पण जेव्हा अभ्यासक शास्त्रीय माहितीबरोबरच स्थानिक समाजाचा अनुभव, ज्ञान यांनाही समजून घेतात, तेव्हा धोरणे अधिक वास्तववादी आणि टिकाऊ ठरतात. हासुद्धा एक प्रकारचा अ‍ॅलॉयच.
उद्योजकतेमध्येही हेच लागू होते. केवळ नफा-तोट्याची भाषा समजणारा उद्योजक जर गावातील गरजांपासून दूर असेल, तर त्याचे उत्पादन अपुरे-तोकडे राहील. पण जर तो एखाद्या अशा सहकाऱ्याशी भागीदारी करतो, ज्याला समाजाची प्रत्यक्ष ओळख आहे, तर तयार होणारे उत्पादन केवळ विकले जात नाही, ते जीवनही बदलते. इथे नफा आणि उद्दिष्ट दोन्ही हातात हात घालून चालतात.
क्रिकेटसारख्या खेळातही आज 'फक्त अनुभव' पुरेसा नाही. एक यशस्वी प्रशिक्षक आता केवळ सराव आणि भावना नव्हे, तर आकडेवारी, डेटा, आणि विश्लेषण यांचा वापर करून निर्णय घेतो. खेळाडूंच्या फिटनेस डेटावर आधारित रणनीती आखली जाते, पूर्वीच्या सामन्यांचा अभ्यास करून फलंदाजी क्रम ठरवला जातो. हा आहे पारंपरिक जाणिवा आणि आधुनिक तंत्रज्ञानाचा अ‍ॅलॉय.
पण एक महत्त्वाची गोष्ट लक्षात ठेवायला हवी की, कोणतेही धातू मनात येईल तसे एकत्र करता येत नाही. संशोधनात चाचणी केली जाते, काही मिश्रणं उपयोगी ठरतात, तर काही अपयशी. विचारांचंही तसंच आहे. योग्य वेळ, योग्य प्रमाण आणि योग्य समज हवी. चुकीच्या लोकांचा एकत्रित प्रयत्न, किंवा तांत्रिक कौशल्याच्या अतिरेकात मानवी संवेदना हरवली, तर नुकसान पदरी पडते.
आपल्या समाजात अनेकदा “शुद्धते”चे महत्त्व अधोरेखित केले जाते. जात, परंपरा, विचारपद्धती या सर्वांमध्ये. पण अ‍ॅलॉयिंग आपल्याला सांगते की खरी ताकद एकसंधतेत नव्हे, तर विविधतेत आहे. एखादे उत्पादन, धोरण, नातेसंबंध किंवा संघटना हे सगळं फक्त “शुद्ध” रूपात टिकत नाही, तर योग्य मिश्रणात भरभराटीला येते. म्हणूनच, ‘एकाला चालो रे’  होण्याऐवजी आपण एखाद्या गटाचा भाग व्हावं, जिथे एकमेकांपासून शिकत नव्या शक्यतांचा शोध घेतला जातो. विचार करा आपणास कोण व्हायचं आहे, २४ कॅरेट शुद्ध पण जरा ठिसूळ सोनं, की थोडे तांबे मिश्रित पण मजबूत-तरीही-लवचिक २२ कॅरेट (‘अ‍ॅलॉय’) सोनं.



\chapter{सुसंगती सदा घडो … }

एका लहानशा गावात किरण नावाच्या उद्यमशील तरुणाने सौरऊर्जेवर चालणाऱ्या, स्वस्त आणि प्रभावी दिव्यांची रचना आणि निर्मिती केली. आता तो त्यांच्या विक्रीचा प्रयत्न करत होता. उत्पादन चांगले असूनही आणि किंमत योग्य असूनही, विक्री काही केल्या वाढत नव्हती. मग एक दिवस, राष्ट्रीय दूरदर्शनवर हवामान बदलावर आधारित एक माहितीपट दाखवण्यात आला. त्यात त्या दिव्यांचा थेट उल्लेख नव्हता, पण लोकांना केरोसीनवर चालणाऱ्या दिव्यांचे दुष्परिणाम कळले आणि मनोमन पटले. गावातील शाळेतही मुलांना ही चित्रफीत दाखवल्याने, घरोघरी अक्षय ऊर्जेचे (रिन्यूएबल एनर्जी) महत्त्व पोहोचले. मग काय विचारता? दुसऱ्या दिवसापासून किरणच्या सौर-दिव्यांची मागणी अचानक वाढली. गावकरी पारंपरिक केरोसीनऐवजी पर्यायी ऊर्जास्रोतांकडे वळू लागले आणि विक्री झपाट्याने वाढली. हे त्या उद्योजकाने अधिक कष्ट घेतल्यामुळे घडले नाही, तर त्या माहितीपटाने उत्प्रेरकाचे (कॅटालिस्ट) काम केल्यामुळे घडले. अद्नान सामीच्या लोकप्रिय गीतातील ओळीप्रमाणे किरणच्या व्यवसायाला थोडी ‘लिफ्ट’ मिळाली आणि त्याची जोरात प्रगती सुरु झाली. 
हेच आहे 'कॅटालिस्ट' किंवा उत्प्रेरकाचे मेंटल मॉडेल (मन:प्रारूप). यात, एक असा बाह्य घटक असतो जो थेट तुमच्या प्रयत्नांत प्रत्यक्ष सहभागी नसतो, पण तरीही तो तुमच्या परिणामांमध्ये केवळ सानिध्याने वेगाने बदल घडवून आणतो. रसायनशास्त्रात उत्प्रेरक रासायनिक क्रिया अधिक वेगाने घडवतो आणि स्वतःमध्ये काहीही बदल न होऊ देता ती क्रिया पूर्ण करतो, तसेच जीवनातील काही प्रसंग, व्यक्ती, कल्पना किंवा साधने मोठा बदल घडवून आणू शकतात, अगदी 'जादू' केल्यासारखे.
आपण अनेकदा असे गृहीत धरतो की यश मिळवण्यासाठी खूप मेहनत, वेळ आणि धडपड करावी लागते. पण खरी प्रगती अनेकदा योग्य वेळ, योग्य संधी आणि योग्य ‘संगतीवर’ अवलंबून असते. मोरोपंतांच्या ‘सुसंगती सदा घडो ..’ या केकावली प्रमाणे. हेच तत्व 'कॅटालिस्ट' या मेंटल मॉडेलमधून समजते, जिथे केवळ सान्निध्य मोठा परिणाम घडवते. काही वेळा एक सल्ला, एक पुस्तक, एक संवाद, किंवा योग्य वेळी मिळालेली प्रेरणा आपल्या जीवनात नवे वळण आणू शकते. हे घटक तुमच्यासाठी थेट काही करत नाहीत, पण तुमचे काम अधिक प्रभावी बनवतात. याची काही उदाहरणे पाहूया.
एखादा विषय मुलांना कंटाळवाणा वाटत असतो, पण अचानक एक असे शिक्षक येतात, जे अत्यंत उत्साही आणि कल्पक असतात. तोच विषय, तीच पुस्तके, पण शिकवण्याची पद्धत आणि त्यातील ऊर्जा मुलांमध्ये कुतूहल निर्माण करते. त्या शिक्षकामुळे एखाद्या विद्यार्थ्याचा त्या विषयाकडे पाहण्याचा दृष्टिकोनच बदलतो आणि पुढे तोच विद्यार्थी विज्ञानात करिअर करतो. येथे ‘कॅटालिस्ट’ ठरतो, तो शिक्षक.
कोविड-१९ च्या काळात विविध माध्यमांतून मास्क वापरण्याची गरज सांगितली जात होती, पण खेड्यापाड्यांत तो संदेश प्रभावीपणे पोहोचत नव्हता. नंतर एका तरुणाने स्थानिक बोलीभाषेत, साध्या शब्दांत आणि विनोदी शैलीत मास्क वापरण्यावरचा एक व्हिडीओ तयार केला. तो सोशल मीडियावर व्हायरल झाला, लाखो लोकांनी पाहिला आणि गावोगावी मास्क वापरणे सुरू झाले. तोच संदेश आधीही दिला गेला होता, पण योग्य भाषा, माध्यम आणि सादरीकरणामुळे तो अधिक प्रभावी ठरला.
अनेकदा आपण एखाद्या नोकरीत कंटाळलेलो असतो, पण ती सोडण्याचे धाडस होत नाही. मग एक दिवस, मित्रासोबत सहज बोलताना तो म्हणतो, "तू सतत त्रासात दिसतोस, तू स्वतःसाठी काहीतरी चांगले शोधायला हवे." त्या एका वाक्यामुळे विचारांना चालना मिळते. आपण निर्णय घेतो, नोकरी बदलतो किंवा नवीन काहीतरी सुरू करतो. इथेही, कॅटालिस्ट ठरतो, तो संवाद.
प्रत्येक संगतीचा चांगलाच किंवा फलदायीच परिणाम होईल असे नाही. जसे मराठीतील प्रसिद्ध वाक्प्रचार आहे, “ढवळ्या शेजारी बांधला पवळा, वाण नाही पण गुण लागला’ हा संगतीने चुकीचेही घडू शकते, दुर्गुणही लागू शकतात याविषयी आहे, काहीसं तसंच. म्हणून ‘कॅटालिस्ट’ हा मदतनीस सिद्ध होत आहे ना याची खात्री करून घ्यावी लागते. 
या सर्व उदाहरणांमध्ये एक गोष्ट स्पष्ट होते की, बदल घडवण्यासाठी प्रत्येक वेळी मोठा धक्का देण्याची गरज नसते; कधीकधी एखादी संगती अथवा योग्य हस्तक्षेप पुरेसा असतो. आपण प्रत्येक परिणामावर नियंत्रण ठेवू शकत नाही, पण एखादी प्रक्रिया कोठे मागे पडत आहे का, हे ओळखून आपण योग्य ठिकाणी योग्य हस्तक्षेप करू शकतो. प्रसंगी स्वत:ला माझ्या एका छोट्याशा कृतीने मोठा बदल घडेल का? कारण आयुष्यात, अगदी रसायनशास्त्राप्रमाणेच, कधीकधी सर्वात मोठे परिणाम अगदी केवळ सानिध्याने घडतात; फक्त ती सुसंगती कोणाची हे कळणे महत्त्वाचे असते.



\chapter{... तेथे लव्हाळे वाचती}

हजारो वर्षांपूर्वी जिराफांचे पूर्वज आजच्या जिराफांपेक्षा बरेच वेगळे होते. त्या काळात, आफ्रिकेच्या सॅव्हाना जंगलात झाडांची पाने हा अन्नाचा एक महत्त्वाचा स्रोत होता. जमिनीवरील गवत व झुडपे इतर प्राणीही खात असल्यामुळे स्पर्धा वाढली. अशा वेळी ज्या जिराफांची मान थोडी लांब होती, त्यांना उंच झाडांच्या फांद्यांवरील पाने सहज मिळू लागली. त्यामुळे त्यांना अधिक अन्न मिळाल्याने ते अधिक निरोगी राहिले आणि त्यांची संततीही अधिक संख्येने टिकून राहिली. हळूहळू, पिढ्यानपिढ्या, लांब मान असलेले जिराफच अधिक प्रमाणात टिकले आणि त्यांची संख्या वाढली. यालाच ‘नॅचरल सिलेक्शन’ म्हणजेच नैसर्गिक निवड म्हणतात. जेव्हा एखादे वैशिष्ट्य एखाद्या जीवाला त्याच्या पर्यावरणात टिकून राहण्यासाठी फायदेशीर ठरते, तेव्हा ते वैशिष्ट्य काळानुरूप अधिक प्रबळ होत जाते. यामुळेच आज दिसणारा जिराफ त्याच्या लांब मानेसह उत्क्रांत झाला आहे. ही संकल्पना "सर्व्हायव्हल ऑफ द फिटेस्ट" म्हणजेच 'जो जुळवून घेतो, तोच टिकतो' (‘उत्क्रांतीची नैसर्गिक निवड’) याचे अत्यंत प्रभावी उदाहरण आहे.
‘उत्क्रांतीची नैसर्गिक निवड’ ही मूलतः जीवशास्त्रातील संकल्पना असली तरी, ती आपल्या रोजच्या जगण्यालाही तंतोतंत लागू होते. म्हणूनच, याकडे एक मेंटल मॉडेल (मन:प्रारूप) म्हणून पाहता येते. जे बदलाशी जुळवून घेतात, ते टिकतात; जे घेत नाहीत, ते कालांतराने नाहीसे होतात. तुकोबारायांच्या ओळी "महापुरे झाडे जाती, तेथे लव्हाळे वाचती" हेच दर्शवतात. ‘उत्क्रांतीची  नैसर्गिक निवड’ हे मन:प्रारूप तीन महत्त्वाच्या आधारांवर उभे आहे. एक म्हणजे ‘वैविध्य’ (डायव्हर्सिटी), म्हणजेच वेगवेगळ्या प्रकारच्या कल्पना, पद्धती किंवा कौशल्यांचे अस्तित्व. दुसरे म्हणजे ‘निवड दबाव’, म्हणजे असे सामाजिक, आर्थिक किंवा पर्यावरणीय घटक जे काही विशिष्ट गोष्टींना इतरांपेक्षा जास्त फायदेशीर ठरवतात. आणि तिसरे म्हणजे ‘प्रसार’ म्हणजेच यशस्वी कल्पना किंवा वर्तनाचा इतरांमध्ये होणारा प्रसार किंवा मोठ्या प्रमाणात वापर. याची काही उदाहरणे पाहूया.
भारतामध्ये डिजिटल पेमेंट्सचा वेगाने झालेला प्रसार हे याचे ठळक उदाहरण आहे. जे पारंपरिक व्यवसाय केवळ रोख व्यवहारांवर अवलंबून राहिले किंवा बदलास तयार नव्हते, ते मागे पडले. पण ज्या दुकानदारांनी क्यू-आर कोड, ई-कॉमर्स आणि इन्व्हेंटरी मॅनेजमेंट सॉफ्टवेअर वापरायला सुरुवात केली, ते मात्र बदलाशी जुळवून टिकले.
टायपिंग किंवा शॉर्टहँडसारख्या एकेकाळच्या लोकप्रिय कौशल्यांना आज फारशी मागणी नाही. आता तर आवाजाचे थेट लेखनात रूपांतर (डिक्टेशन) करणारे तंत्रज्ञान आले आहे. ज्या लोकांनी स्वतःला अशा डिजिटल साधनांमध्ये प्रशिक्षित केले, ते आजही केवळ टिकूनच नाहीत, तर त्यांना मोठी मागणी आहे. कारण तंत्रज्ञानाच्या दुनियेत "स्थिर" असे काहीच नसते.
फोटो काढणे एके काळी केवळ फिल्मवर शक्य होते. पण जेव्हा डिजिटल तंत्रज्ञान आले, तेव्हा ज्या कंपन्यांनी ते स्वीकारले, त्या यशस्वी झाल्या. ‘कोडॅक’सारखी कंपनी, जी फिल्मवरच अवलंबून राहिली, तिने आपला बाजार गमावला. अगदी याचप्रमाणे मोबाईल उद्योगात नोकियाचा एक काळ होता. पण स्मार्टफोन आणि टचस्क्रीन तंत्रज्ञान आले, तेव्हा नोकिया वेळेवर बदल करू शकली नाही. त्याचवेळी ॲपल आणि अँड्रॉइडने बाजार काबीज केला.
घोकंपट्टीवर आधारलेली पारंपरिक शिक्षणपद्धती आता टीकेचा विषय ठरली आहे. त्या तुलनेत ज्या शिक्षणसंस्था मुलांमध्ये समस्या सोडवण्याची क्षमता, सर्जनशीलता आणि आंतरविद्याशाखीय  (इंटर डिसिप्लिनरी)  दृष्टिकोन रुजवतात, त्या अधिक यशस्वी ठरत आहेत. या बदलत्या जगात कोणत्याही शिक्षणपद्धतीला टिकून राहायचे असेल, तर तिला बदल स्वीकारावेच लागतील.
एकेकाळी युद्धातील सर्वात प्रभावशाली शक्ती म्हणजे घोडेस्वार सैन्य. ते वेगवान, शक्तिशाली आणि थेट हल्ल्यांसाठी वापरले जात असे. पण विसाव्या शतकात टँक, मशीनगन आणि हवाई दलाचा उदय झाला. दुसऱ्या महायुद्धात पोलंडचे पारंपरिक घोडेस्वार सैन्य जर्मनीच्या टँक-आधारित 'ब्लिट्झक्रिग' रणनीतीपुढे टिकू शकले नाही. आता तर चित्र पूर्णपणे बदलले आहे; केवळ शेकडो ड्रोन्स आणि क्षेपणास्त्रे पाठवून शत्रू राष्ट्राला नामोहरम करता येते. जे देश जुन्या पद्धतींमध्ये अडकले, त्यांना युद्धात मोठे नुकसान सहन करावे लागले.
नैसर्गिक निवडीचे वैशिष्ट्य हेच आहे की, ती कोणाचे यश, शक्ती, परंपरा किंवा निष्ठा पाहून निर्णय घेत नाही. ती फक्त हे पाहते की, कोण ‘जुळवून घेऊ शकते’. या मेंटल मॉडेलचा खरा अर्थ असा आहे की, यश कधीच कायमस्वरूपी नसते आणि टिकून राहण्याची कोणतीही हमी नसते.
जर आपल्याला टिकायचे असेल, तर ‘नामशेष होणे’ या संकल्पनेकडे अपयश म्हणून नव्हे, तर 'आता काहीतरी बदलायला हवे' अशी सूचना म्हणून पाहावे लागेल. हे जितके लवकर उमगेल, तितक्या लवकर आपण स्वतःमध्ये आणि आपल्या सभोवतालच्या जगात आवश्यक बदल घडवू शकू. कुणी सांगावे, ज्या गरजेमुळे जिराफाची मान लांब झाली, तशीच भविष्यात अधिक विचार करण्याच्या गरजेमुळे मानवाचे डोके मोठे होईल?




\chapter{बे एके बे, बे दुणे चार }

१९८० आणि ९०च्या दशकात, चीनने आपल्या आर्थिक प्रगतीची सुरुवात पाश्चिमात्य औद्योगिक मॉडेल्सची हुबेहुब नक्कल करून आणि तंत्रज्ञानाची उलट-अभियांत्रिकी (रिव्हर्स इंजिनिअरिंग) करून केली. इलेक्ट्रॉनिक्सपासून ते अवजड यंत्रसामग्रीपर्यंत अनेक उत्पादनांची चीनने बारकाईने प्रतिकृती केली. परदेशी कंपन्यांशी संयुक्त भागीदारी करत चीनने तंत्रज्ञान, उत्पादन प्रक्रिया आणि पुरवठा साखळीचे कौशल्य आत्मसात केले. चीनला "जगाचा कारखाना" असं म्हटलं जाऊ लागलं, कारण त्याचा भर स्वस्त उत्पादनावर आणि नक्कल करण्यावर होता.

पण हळूहळू चित्र बदलू लागलं. चीनने संशोधन आणि विकास (आर-अँड-डी), शिक्षण आणि स्थानिक प्रतिभेमध्ये मोठ्या प्रमाणात गुंतवणूक केली. २०१०च्या दशकात ‘हुवावे’, ‘शाओमी’ आणि ‘बीवायडी’ सारख्या कंपन्या केवळ नक्कल करणाऱ्या राहिल्या नाहीत, तर अनेक तंत्रज्ञानात आघाडीच्या ठरल्या. ही परिवर्तनकथा ‘रेप्लिकेशन’ या मेंटल मॉडेलचे (मन:प्रारूप) उत्कृष्ट उदाहरण आहे. याचा अर्थ असा की, केवळ जशीच्या-तशी नक्कल करणं आणि समजून घेतलेली पुनरावृत्ती यात मोठा फरक आहे. विचारपूर्वक केलेली पुनरावृत्ती ही परिणामकारक प्रगतीची रणनीती ठरू शकते, पण आंधळेपणाने केलेली नक्कल नेहमी चेष्टेचा विषय बनते आणि अपयशाकडे नेते.

जीवशास्त्रात, ‘रेप्लिकेशन’ म्हणजे गुणसूत्रांची (डीएनए) स्वतःची प्रत तयार करण्याची प्रक्रिया. जीवसृष्टीच्या सातत्यासाठी अत्यावश्यक असलेली ही प्रक्रिया केवळ पेशींपुरती मर्यादित नाही. ही संकल्पना एक मेंटल मॉडेल म्हणून शिकवते की यशस्वी उत्पादन, प्रक्रिया किंवा प्रणाली प्रभावीपणे पुन्हा वापरता येते. या संकल्पनेचं तत्त्व सोपं आहे, जे काही यशस्वी ठरतं, ते प्रथम समजून घ्या, त्यामागचं "का" जाणून घ्या, आणि मग ते शहाणपणाने पुन्हा राबवा. आपण प्रत्येक वेळी नव्याने चाकाचा शोध घेण्याची गरज नाही (“व्हाय रिइन्व्हेन्ट द व्हील?”). ‘बे एके बे’ झाले की नीट समजून घ्या, बरोबर झाले असेल तर ‘बे दुणे चार” आणि पुढे. याची काही उदाहरणे पाहुयात.

शिक्षणाचंही तसंच आहे. गुरुकुलाची पद्धत काळानुसार बदलत गेली. मॅकॉलेने इंग्रजांना हवी तशी शिक्षणपद्धती आणली तशी ती स्वातंत्र्योत्तर भारतातातील शाळांमध्ये बऱ्यापैकी तशीच उतरली. पण  ‘देश, काल, पात्र’ यांचा विचार न करता केल्याने, तर ती फारशी यशस्वी झाली नाही. भारताने यशस्वी कारकून, कामगार बनवले पण नवोन्मेषी शोधांमध्ये आपण मागे पडलो. त्यामुळे स्थानिक गरजांनुसार योग्य बदल करतच पुनरावृत्ती करावी लागते.

तसेच भारतीय स्टार्टअप्सपैकी अनेकांनी परदेशी यशस्वी मॉडेल्सची नक्कल केली. काहींनी त्यात यश मिळवलं कारण त्यांनी स्थानिक गरजांशी जुळवून घेतलं (उदाहरण: यु-पी-आय मुळे डिजिटल पेमेंट्समध्ये झालेली क्रांती). पण अनेक कंपन्या अपयशी ठरल्या कारण त्या मॉडेल्सना स्थानिक सामाजिक आणि आर्थिक वास्तवाशी मेळ घालता आला नाही.

आपल्या वैयक्तिक आयुष्यातही हे लागू होतं. इंटरनेटवर प्रसिद्ध झालेले "मॉर्निंग मिरॅकल" किंवा उत्पादकतेचे हॅक्स जेव्हा आपण आपल्या गरजा, सवयी आणि ध्येय न समजता अनुकरण करतो, तेव्हा ते फार काळ टिकत नाहीत. थेट नक्कल न करता, आपल्या गरजेनुसार बदल करूनच त्याचा उपयोग होतो.

‘रेप्लिकेशन’ आपल्याला एक अमूल्य पाठ शिकवतं, प्रगती अनेकदा नक्कलांमधूनच सुरू होते. लहान मूल मोठ्यांची नक्कल करून बोलायला शिकतं. शिकाऊ चित्रकार जुनी-प्रसिद्ध पाहून चित्र काढायला शिकतात. सुप्रसिद्ध गायक-गायिका त्यांच्या सुरुवातीच्या काळात जुन्या गायकांसारखेच गाताना वाटतात. पण कालांतराने जे स्वतःची शैली आणि आवाज निर्माण करतात, तेच टिकतात. मात्र, यामध्ये सावधगिरी आवश्यक आहे. संपूर्ण संदर्भ न समजता केवळ नक्कल केल्यास परिणाम शून्य होतात. उदाहरणार्थ, एक यशस्वी प्रयोग दुसऱ्या ठिकाणी करायचा असेल, तर स्थानिक परिस्थिती, संस्कृती आणि संसाधनांचा विचार करावा लागतो. अन्यथा अशी नक्कल उथळ ठरते आणि पुनरावृत्तीचा अयोग्य वापर हानिकारकही ठरू शकतो.

आजच्या जगात "एका तासात यशस्वी व्हा" अशा टेम्प्लेट्स, फॅड्स आणि प्रसिद्ध व्यावसायिक कहाण्या सर्वदूर पसरलेल्या आहेत. त्यामुळे काही यशस्वी गोष्ट पाहिल्यावर आपल्याला ती तशीच नक्कल करावीशी वाटते. पण विचार न करता नक्कल करणं म्हणजे मूळ मजकूर न वाचता फोटोकॉपी काढणं, जिथे चुका वाढतात, अर्थ हरवतो आणि मूळ उद्देशही गमावतो. म्हणून, काहीही नक्कल करण्याआधी स्वतःला विचारा: हे का यशस्वी झालं? आणि गरज पडल्यास मी ते बदलून घेण्यास तयार आहे का? ‘रेप्लिकेशन’ ही एक दुधारी तलवार आहे. ती एकाच वेळी उत्कृष्टतेचा प्रचार करू शकते किंवा सुमारतेचा प्रसार. निवड आपली आहे, यशस्वी चित्रपटाचे दाक्षिणात्य ‘रिमेक’ आपण किती दिवस काढत राहणार?


\chapter{परस्पर संबंधांचे महाजाल }

अलीकडेच एक मोठा व्यापारी संघर्ष उफाळून आला, जेव्हा अमेरिकेने चीनवर मोठ्या प्रमाणात आयात-कर (टॅरिफ) लादले. त्याला प्रत्युत्तर देताना, चीनने दुर्मीळ खनिजांच्या निर्यातीवर निर्बंध घातले. ही खनिजे अशी आहेत की त्यांचा उपयोग स्मार्टफोनपासून सौरऊर्जा पॅनेलपर्यंत अनेक आधुनिक तंत्रज्ञानांत होतो. या संघर्षात भरडून निघाला तो भारताचा नुकताच उगवता इलेक्ट्रिक वाहन उद्योग. गरजेच्या खनिजांचा तुटवडा निर्माण झाला, उत्पादन रखडले आणि रोजगाराला खीळ बसली. अमेरिकेच्या एका निर्णयाचे परिणाम हजारो मैल दूर भारतात जाणवतात. हेच आहे ‘इकोसिस्टम’ नावाचे मेंटल मॉडेल (मन:प्रारूप). ते आपल्याला शिकवते की, जग वेगवेगळ्या, अलिप्त गोष्टींनी बनलेले नाही, तर या सर्व गोष्टी एकमेकांशी जोडलेल्या आहेत. परस्पर संबंधांचे महाजाल. एक घटक हलला की त्याचे पडसाद दुसरीकडे उमटतात. हे अगदी ‘फुलपाखराच्या फडफडीसारखे’ (बटरफ्लाय इफेक्ट) आहे, ज्यात एका कोपऱ्यातील फडफडणारे पंख दुसऱ्या कोपऱ्यात वादळ (केऑस) निर्माण करू शकतात.
आपण अनेकदा घटनांकडे एक स्वतंत्र घडामोड म्हणून पाहतो. पण आजचे जग तसे चालत नाही. तैवानमधील एका तांत्रिक धक्क्याचे परिणाम उत्तर प्रदेशातील मोबाईल दुकानदारांना, पुण्यातील वाहन उद्योगाला, आसाममधील विद्यार्थ्यांना आणि तामिळनाडूमधील लॉजिस्टिक कंपन्यांना जाणवतात. हे केवळ जागतिकीकरणाचे परिणाम नाहीत, तर हे एका परस्परसंबंधित जगाचे, म्हणजेच ‘इकोसिस्टम’चे चित्र आहे. ती स्थिर नसते; ती सतत बदलत असते, स्वतःला सुधारते किंवा कधीकधी कोसळतेही. त्यात सकारात्मक आणि नकारात्मक प्रतिक्रियांचा प्रतिसाद (फीडबॅक लूप्स) असतो, जो परिणामांना कधी वाढवतो किंवा स्थिर करतो. या व्यवस्थेत स्पर्धा आणि सहकार्य एकत्र नांदतात. ‘इकोसिस्टम’ ही कल्पना मुळात निसर्गशास्त्रातून आली आहे, जिथे झाडे, प्राणी, हवामान आणि जमीन यांचा परस्परसंबंध अभ्यासला जातो. पण आज हीच चौकट आपण व्यवसाय, आरोग्य, तंत्रज्ञान आणि शिक्षण या क्षेत्रांतही वापरू शकतो. याची काही उदाहरणे पाहूया.
कोरोना विषाणूचा उगम चीनमधील एका बाजारातून झाल्याची शक्यता वर्तवली गेली. या एका स्थानिक घटनेने संपूर्ण जगातील आरोग्य, शिक्षण, अर्थव्यवस्था आणि दैनंदिन जीवनाला धक्का दिला. भारतातही लॉकडाउन, बेरोजगारी, ऑनलाइन शिक्षण याचे दूरगामी परिणाम झाले. 
त्याचप्रमाणे, रशिया-युक्रेन युद्धामुळे भारतातील सामान्य ग्राहकांपर्यंत थेट परिणाम पोहोचला, कारण दोन्ही देश भारताला सूर्यफूल तेलाचा पुरवठा करणारे मोठे देश होते. तेलाच्या किमती वाढल्या आणि भारतीय स्वयंपाकघरांपर्यंत महागाईची झळ पोहोचली.
एखाद्या कृषी शास्त्रज्ञाने प्रयोगशाळेत विकसित केलेले गव्हाचे एक नवीन, उच्च-उत्पादन देणारे वाण (एक लहान बदल) भारतात 'हरित क्रांती' घडवते. यामुळे केवळ देशाची अन्नसुरक्षाच साधली जात नाही, तर सिंचन पद्धती, खतांचा वापर आणि ग्रामीण अर्थकारण यांवरही दूरगामी परिणाम होतात. एका बियाणात झालेला बदल संपूर्ण सामाजिक-आर्थिक व्यवस्थेला नवी दिशा देतो.
शहराच्या विकास आराखड्यात, एखाद्या निवासी भागाजवळ एक नवीन 'मेट्रो स्टेशन' मंजूर करण्याचा निर्णय (एक स्थानिक बदल) वरवर पाहता छोटा वाटू शकतो. पण त्यामुळे त्या परिसरातील घरांच्या किमती गगनाला भिडतात, लहान-मोठे व्यवसाय वाढीस लागतात, वाहतुकीची पद्धत बदलते आणि लोकांचे जीवनमान उंचावते. काहीवेळा यामुळे जुन्या रहिवाशांना विस्थापितही व्हावे लागते. एका स्टेशनचा निर्णय संपूर्ण परिसराचे चित्र बदलून टाकतो.
एखाद्या सोशल मीडिया ॲपवर कोणीतरी खोडसाळपणे पसरवलेली एक छोटीशी अफवा (एक डिजिटल फडफड) काही तासांतच एखाद्या कंपनीचे लाखो रुपयांचे नुकसान करते किंवा समाजात मोठा तणाव निर्माण करते. इथे तंत्रज्ञानाच्या परिसंस्थेतील एक नगण्य वाटणारी घटना वास्तवात मोठे वादळ निर्माण करते आणि तिचे परिणाम आर्थिक किंवा सामाजिक क्षेत्रात भोगावे लागतात.
‘इकोसिस्टम’ हे मानसिक प्रारूप आपल्याला सांगते की, सर्व काही आपल्या नियंत्रणात नसते आणि जे आपल्याला वरवर दिसते, ते संपूर्ण सत्य नसते. आपण उचललेल्या एका पावलाचे दूरगामी परिणाम होऊ शकतात, कधी फायदेशीर तर कधी हानिकारक. हे प्रारूप आपल्याला घाईगडबडीत निर्णय घेण्यापासून रोखते आणि विचारपूर्वक कृती करायला शिकवते. आपण ज्या जगात राहतो, ते एका विशाल, अदृश्य जाळ्यासारखे आहे; आपण एक धागा ओढतो आणि कंपने भलत्याच ठिकाणी जाणवतात.
ही विचारसरणी आपल्याला अधिक सजग नागरिक आणि माणूस बनवते. म्हणूनच, आपण सिग्नल मोडायचा नाही, हा एक छोटासा निश्चयसुद्धा सामाजिक शिस्तीच्या मोठ्या चळवळीची सुरुवात ठरू शकतो. सुरुवात तर करा. 


\chapter{एकमेवाद्वितीय }

अनेक शहरांमध्ये सकाळी-सकाळी एक नेहमीचे चित्र दिसते, ते म्हणजे भाजीबाजाराचे. अनेक विक्रेते एकाच वेळी ओरडून आपापल्या भाजीपाल्याकडे ग्राहकांचे लक्ष वेधण्याचा प्रयत्न करत असतात. सगळे जण थोड्याफार प्रमाणात तेच विकत असतात, तेही त्याच लोकांना, त्याच पद्धतीने. पण या गोंधळात, एका शांत कोपऱ्यात एक छोटासा, साधा ठेला-स्टॉल असतो, जो काहीतरी वेगळे करत असतो. इथे फक्त एकच गोष्ट विकली जाते: विदेशी मशरूम्स. ना ओरडणे, ना घासाघीस. त्याचा ग्राहकवर्ग ठरलेला आहे: शेफ्स, आरोग्याबाबत जागरूक लोक आणि काही जिज्ञासू खरेदीदार, जे मुद्दामहून इथेच येतात. इतर विक्रेते जिथे कमी नफ्यावर झगडत असतात, तिथे हा विक्रेता मात्र भरभराटीस आलेला दिसतो. कारण तो स्वस्त विकत नाही, गोंगाट करत नाही, तर तो इतरांपेक्षा वेगळा आहे.
हे काही योगायोगाने घडलेले नाही. हे आहे ‘नीश’ (अतिविशेषता) या मेंटल मॉडेलचे (मनःप्रारूप) एक जिवंत उदाहरण. ही कल्पना मूळची पर्यावरणशास्त्रातील असली, तरी आजच्या जगात आपले काम, व्यवसाय आणि आयुष्य समजून घेण्यासाठी ती अत्यंत उपयुक्त ठरते.
निसर्गात ‘नीश’ म्हणजे एखाद्या जीवाचे पर्यावरणातील अतिविशिष्ट (स्पेशलाइज्ड) स्थान. तो काय खातो, कुठे राहतो, आणि इतर जीवांशी कसा संवाद साधतो, हे सर्व त्यात येते. दोन प्रजाती एकाच नीशमध्ये फार काळ टिकत नाहीत; स्पर्धेमुळे एकाला स्वतःत बदल करावाच लागतो. आपल्या मानवी जगातही असेच नीश असतात. हे नीश म्हणजे काय? तर, एक असे विशिष्ट स्थान किंवा अढळपद, जे एखादी व्यक्ती आपल्या कौशल्यांवर आणि अनुभवावर आधारित मिळवते. बाजारातील एक अनोखे उत्पादन, करिअरमधील दुर्मिळ स्पेशलायझेशन, ही सगळी ‘नीश’ मेंटल मॉडेलचीच उदाहरणे आहेत. आपल्याकडेही याचे अनेक दाखले आहेत त्यातील एक आपल्या जवळचे आहे ते म्हणजे, स्वर्गीय लता मंगेशकर. पुलंनी एकदा त्यांच्याविषयी म्हटले होते, ‘आकाशात सूर्य आहे, चंद्र आहे… आणि लताचा स्वर आहे!’ केवढे यथार्थ वर्णन, नाही का? सुप्रसिद्ध वचन, "झाले बहू, होतील बहू, परी या सम हा", म्हणजे ‘नीश ‘च. एकमेवाद्वितीय. या संकल्पनेची काही अजून उदाहरणे पाहूयात. 
अनेक वस्त्रोद्योगातील कंपन्या सर्वसामान्य ग्राहकांच्या मागे धावत असतात. पण ज्या कंपन्या एका विशिष्ट गरजेला लक्ष्य करतात, त्या अनेकदा अधिक यशस्वी ठरतात. उदाहरणार्थ, एक भारतीय स्टार्टअप जो फक्त बाळांसाठी रसायनमुक्त लंगोट (डायपर्स) बनवते. ही कंपनी विशेषतः अशा पालकांसाठी आहे, जे आपल्या मुलांच्या त्वचेसाठी अतिशय जागरूक आहेत. मुख्य प्रवाहातील ‘सरासरी’ फॅशनच्या गर्दीत सामील न होता, ही कंपनी बाजारात स्वतःचे अढळ स्थान निर्माण करते. साहजिकच, स्पर्धा कमी असल्याने किमतीवर पकड राहते.
सर्वसाधारण वकील घरगुती किंवा नेहमीच्या अपराधिक तंट्यांच्या केसेस घेतात. पण याउलट, एखादी वकील जर फक्त सायबर क्राईम आणि डिजिटल प्रायव्हसीसारख्या अत्याधुनिक आणि तुलनेने अपरिचित विषयांवर काम करत असेल, तर ती नक्कीच जास्त फी आकारू शकते. भारतात ही क्षेत्रे अजूनही विकसित होत आहेत. हजारो सामान्य वकिलांच्या स्पर्धेत न पडता, ती स्वतःची जागा मजबूत करते. हेच ‘नीश’.
शेतीतही हेच तत्त्व लागू होते. एखादा शेतकरी पारंपरिक ऊस, गहू न पिकवता, काळ्या तांदळाची जुनी जात किंवा इतर दुर्मीळ धान्ये पिकवतो. त्याची पैदावार कमी असली, तरी आरोग्याबद्दल जागरूक असलेले ग्राहक ते पीक प्रीमियम दराने खरेदी करतात. असे करून तो अस्थिर कृषी-व्यवस्थेत स्वतःसाठी स्थैर्य आणि संपत्ती निर्माण करतो.
डिजिटल युगात, ‘सर्वांसाठी सर्वकाही’ या जागतिक स्पर्धेत, केवळ 'इतरांपेक्षा चांगले' (बीइंग बेटर) असून भागत नाही, तर 'इतरांपेक्षा वेगळे' (बीइंग डिफरंट) असणे पण महत्त्वाचे ठरते.
पण, ‘नीश’ निवडणे म्हणजे कायमस्वरूपी यशाची हमी नव्हे. बदलत्या काळात एखादा ‘नीश’ कालबाह्य किंवा नष्टही होऊ शकतो. ग्राहकांच्या सवयी बदलतात, तंत्रज्ञान मागे पडते. म्हणूनच आपल्या नीशमध्ये तरबेज असलेल्यांनीसुद्धा बदलांसाठी तयार रहावे लागते. मशरूम विक्रेत्याला ग्राहकांच्या बदलत्या चवीनुसार. सायबर वकिलांना तंत्रज्ञानाशी संबंधित नवनवीन कायदे समजून घ्यावे लागतील. थोडक्यात, ‘विशेषतेसोबतच लवचिकता’ हाच ‘नीश’ या मेंटल मॉडेलचा गाभा आहे.
हे मेंटल मॉडेल नवउद्योजक आणि विचारवंत नवल रविकांत यांच्या ‘विशिष्ट ज्ञान’ (स्पेसिफिक  नॉलेज) या कल्पनेशी मिळतेजुळते आहे. तुमचे ज्ञान-कौशल्य असे असावे, की जे अतिविशिष्ट, दुर्मिळ आणि सहजासहजी शिकता येण्यासारखे नसेल. त्यामुळे जागतिक पटलावर तुम्ही एकमेवाद्वितीय ठरू शकता.
अशा प्रकारे, ‘नीश’ हे केवळ स्पर्धा टाळण्याचे साधन नाही, तर ते स्वतःचे एक वेगळे अस्तित्व निर्माण करण्याचा मार्ग आहे. हे जगापासून दूर जाणे नव्हे, तर आपण निवडलेल्या क्षेत्रात स्वतःला केंद्रस्थानी प्रस्थापित करणे आहे. “परी या सम हा" अशी स्तुती आपल्या वाट्याला आली, तर आयुष्याचे सार्थक झाले असे म्हणता येईल, नाही का?



\chapter{पिंजऱ्याचे दार उघडावे}

या लेखमालेचे शीर्षक ‘तिसरा मेंदू’ हे बाह्य किंवा पूरक बुद्धिमत्तेचे प्रतीक आहे. म्हणूनच कृत्रिम बुद्धिमत्ता असो वा मानवी बुद्धिमत्तेला पूरक ठरणारी मेंटल मॉडेल्स (मनःप्रारूपे) असोत, योग्य प्रसंगी योग्य पद्धतीने विचार करणे किंवा करवून घेणे, हेच खरे परिपूर्ण बुद्धिमत्तेचे लक्षण ठरते. या प्रक्रियेमुळे आपली विचारक्षमता व्यापक होते आणि कोणत्याही प्रश्नाचा सर्वांगीण विचार करता येतो. लौकिकदृष्ट्या ‘कृत्रिम बुद्धिमत्ता आणि मेंटल मॉडेल्स’ या विषयांवरील लेखमालारूपी हा विचारप्रवाह जरी येथे थांबत असला तरी, ‘राम-राम’ करण्याआधी मात्र, थोडे ‘सिंहावलोकन’ आणि थोडे ‘पुढे काय?’ यासाठी हा लेखप्रपंच. त्यातही प्रामुख्याने ‘मेंटल मॉडेल्स’ विषयी. 
‘मेंटल मॉडेल्स’ ही केवळ तात्त्विक संकल्पना किंवा बौद्धिक कसरत नाही, तर जगाकडे पाहण्याची एक विशिष्ट पद्धत आहे. ही एक अशी विचाराची चौकट (फ्रेमवर्क) आहे, जणू एखादा चष्मा, ज्यातून आपण घटना, संधी आणि धोके अधिक स्पष्टतेने पाहू शकतो. सुप्रसिद्ध गुंतवणूकतज्ञ आणि तत्वज्ञ चार्ली मंगर यांनी म्हटल्याप्रमाणे, वेगवेगळ्या क्षेत्रांतील मॉडेल्सचे लॅटिसवर्क (म्हणजेच विचारांचे जाळे) तयार करणे हेच खरे कौशल्य आहे. कारण अशा पद्धतीनेच आपण वास्तवाकडे विविध दृष्टिकोनांतून पाहू शकतो, आणि एकांगी विचारसरणीपासून स्वतःचे संरक्षण करू शकतो.
योग्य मेंटल मॉडेल्स कशी निवडावीत?
आपल्याला उपलब्ध असलेल्या मेंटल मॉडेल्सचे विश्व अक्षरशः अमर्याद आहे. त्यांतील काही मॉडेल्स परस्परविरोधी भासतात किंवा इतरांपेक्षा उलट सल्ले देतात. येथे सारासार विवेक अत्यंत महत्त्वाचा ठरतो. कोणत्या परिस्थितीत काय वापरायचे, हे अनुभवानेच जमेल. सुरुवात अशा मॉडेल्सने करा, जी अनेक परिस्थितींमध्ये उपयोगी पडतात. उदाहरणार्थ: ‘इन्वर्जन’ म्हणजे उलट पद्धतीने विचार करणे, ‘ऑपॉर्च्युनिटी कॉस्ट’ म्हणजे एक गोष्ट निवडल्याने दुसरी कोणती संधी गमावली हे लक्षात घेणे, किंवा ‘सेकंड-ऑर्डर थिंकिंग’ म्हणजे निर्णयांच्या दूरगामी परिणामांचा विचार करणे.
आपल्या विचारशैलीला आव्हान देणारी आणि आपले अंधबिंदू (ब्लाइंड स्पॉट्स) उघड करणारी मॉडेल्स प्राधान्याने समजून घ्या. जसजशी अनुभवाची भर पडेल, तसतसे भौतिकशास्त्र, जीवशास्त्र, अर्थशास्त्र, मानसशास्त्र अशा विविध क्षेत्रांतील मॉडेल्स आपल्या विचारसंपदेत सामील करा. हे लक्षात ठेवा की, सर्व विषयांत तज्ज्ञ होणे आवश्यक नाही, परंतु गुंतागुंतीच्या समस्यांकडे पाहण्यासाठी भक्कम आणि विश्वसनीय चौकटी (फ्रेमवर्क) निर्माण करणे महत्त्वाचे आहे.
काय करावे आणि काय टाळावे?
ही मॉडेल्स घाईघाईने एकामागून एक गोळा करण्यापेक्षा, त्यांची निवडक आणि सखोल समज आवश्यक आहे. गुणवत्ता ही संख्येपेक्षा अधिक महत्त्वाची आहे. मात्र, कोणत्याही एका मॉडेलला अंतिम सत्य मानू नका. प्रत्येक मॉडेल हे वास्तवाचे केवळ सरलीकरण असते, ते संपूर्ण सत्य नव्हे. एकाच मॉडेलचा अतिरेक टाळा; अतिरेकी आसक्ती विचारांना मर्यादित करते, कारण ‘ज्याच्या हातात हातोडा असतो, त्याला प्रत्येक गोष्ट खिळ्यासारखी दिसू लागते.’
आपल्या ज्ञानाची मर्यादा ओळखा. कुठे आपली समज मजबूत आहे आणि कुठे कमकुवत, हे प्रामाणिकपणे स्वीकारा. गरज असेल तिथे मार्गदर्शन घेण्याची किंवा अधिक शिकण्याची तयारी ठेवा. सर्वात महत्त्वाचे म्हणजे, स्वतःला नेहमी प्रश्न विचारत राहा. काळ, परिस्थिती आणि समाज सतत बदलत असतो. त्यामुळे आपली मेंटल मॉडेल्सदेखील वेळोवेळी तपासून, सुधारून किंवा बाजूला ठेवण्याची तयारी ठेवावी लागते.
पुढे काय?
मेंटल मॉडेल्सचे शिक्षण ही आयुष्यभर चालणारी प्रक्रिया आहे. केवळ आपल्या आवडीच्या क्षेत्रातच नव्हे, तर वेगवेगळ्या विषयांत वाचन करा. त्यातून नवे दृष्टिकोन मिळतील. इतरांशी चर्चा करा, कारण शिकवण्याने आणि वादविवादाने आपली समज अधिक खोल होते.
स्वतःच्या विचारांची चिकित्सा करा. जेव्हा तुम्ही एखादा निर्णय घ्याल, तेव्हा त्याचे परिणाम कसे झाले, हे लिहून ठेवा. यामुळे कोणती मॉडेल्स आपल्याला उपयुक्त ठरतात, हे समजण्यास मदत होईल. वास्तवातील प्रश्नांवर ही मॉडेल्स तपासून पाहा, ते तुमच्यासाठी जिवंत प्रयोगशाळा ठरतील. 
शेवटचे आणि महत्त्वाचे
ही लेखमाला तुमच्यासाठी एक सुरुवात ठरावी. अथ मनःप्रारूपानुशासनम्!! आता मनःप्रारूप वापरण्याची शिस्त अंगीकारावी. खरी वाटचाल तेव्हा सुरू होते, जेव्हा तुम्ही या मॉडेल्सना तुमच्या आयुष्यात उतरवता, त्यांना स्वतःच्या अनुभवांनुसार घडवता आणि सतत जगाकडे नव्या नजरेने पाहण्याचा प्रयत्न करता. कारण शहाणपण म्हणजे सर्व उत्तरे माहीत असणे नव्हे, तर योग्य प्रश्न विचारण्याची आणि अधिक स्पष्टतेने पाहण्याची क्षमता मिळवणे होय.
आपण सर्वजण नकळतपणे कोणत्यातरी विचारपद्धतीच्या, सवयींच्या आणि पूर्वग्रहांच्या पिंजऱ्यात राहात असतो. तो पिंजरा सोन्याचा, म्हणजेच सुरक्षिततेचा, ओळखीचा आणि आरामाचा असतो. पण तो कितीही आकर्षक असला तरी, आपली विचारशक्ती, जिज्ञासा आणि जाणिवांचा संकोच करू शकतो. त्या सोयीच्या चौकटीबाहेर पाऊल ठेवावे, आपले पूर्वग्रह तपासावेत आणि विचारांच्या विशाल आकाशात झेप घ्यावी, जिथे नवीन ज्ञान, नवे दृष्टिकोन आणि खरी मोकळीक मिळते. म्हणूनच, आता नक्कीच “पिंजऱ्याचे दार उघडावे”..
