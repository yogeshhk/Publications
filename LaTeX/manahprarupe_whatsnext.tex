\chapter{`मेंटल मॉडेल्स' शिकण्याचे मार्ग}

मन : प्रारूपे अथवा विचारचित्रे (`मेंटल मॉडेल्स') म्हणजे आपल्याला जग समजून घेण्यासाठी आणि निर्णय घेण्यासाठी उपयोगी पडणाऱ्या कल्पनांची चौकट. उदाहरणार्थ: `कारण-परिणाम ', ८०/२० नियम, इत्यादी.

\section{`मेंटल मॉडेल्स' का शिकाव्यात?}
	\begin{itemize}
		\item जगाकडे समजूतदारपणे बघण्याची दृष्टी मिळते
		\item  चांगले निर्णय घेता येतात
		\item  चुका कमी होतात
		\item  विविध क्षेत्रांत लागू करता येतात – व्यवसाय, वैयक्तिक आयुष्य, नेतृत्व, शिक्षण
	\end{itemize}

\section{ शिकण्याची तयारी: मूलभूत गोष्टी}

\subsection{वाचनाची तयारी करा}

	\begin{itemize}
		\item सोप्या भाषेतले वैचारिक लेख वाचा
		\item  थोडे अर्थशास्त्र, तर्कशास्त्र, मनोविज्ञान, आणि भौतिकशास्त्र या विषयांचा परिचय घ्या
	\end{itemize}

\subsection{ही पुस्तके किंवा स्रोत वापरा (मराठीत/हिंदीत उपयुक्त पर्याय असतील तेही शोधा)}

	\begin{itemize}
		\item Poor Charlie’s Almanack – चार्ली मंगर
		\item Farnam Street Blog (fs.blog)
		\item Super Thinking – Gabriel Weinberg
	\end{itemize}


\section{शिकण्याची पद्धत}

	\begin{itemize}
		\item  टप्पा १: एका वेळी एक संकल्पना -  उदाहरणार्थ: फक्त "कारण-परिणाम" याचा एक आठवडा अभ्यास करा
		\item  टप्पा २: वाचा ,  विचार करा,  लिहा, वापरा
		\item  टप्पा ३: नियमित पुनरावलोकन:   दर महिन्याला शिकलेल्या संकल्पनांची उजळणी करा,  मित्रांबरोबर चर्चा करा
	\end{itemize}

\section{वापरण्याचे ठिकाण}

	\begin{itemize}
		\item निवडी: पैसे गुंतवणे, नोकरी/व्यवसायातील निर्णय
		\item  संपर्क: माणसांशी व्यवहार करताना
		\item  प्रेरणा: स्वतःला आणि इतरांना प्रेरित करताना
		\item  चुका ओळखणे: पूर्वग्रह किंवा भावनिक निर्णय टाळणे
	\end{itemize}


\section{अंतिम सल्ला}

{\em संकल्पना शिकणे म्हणजे अनेक चष्मे तयार करणे, जेंव्हा जगाकडे बघाल, तेव्हा योग्य चष्मा लावल्यास गोष्टी स्पष्ट दिसतील.]

	\begin{itemize}
		\item  तुम्हाला प्रत्येक गोष्ट समजली पाहिजे असं नाही, पण तुम्ही वेगळ्या संकल्पनांद्वारे विचार करण्याची सवय लावली पाहिजे.
		\item संकल्पना वापरून विचार करणे ही एक वाढणारी सवय आहे.
	\end{itemize}
