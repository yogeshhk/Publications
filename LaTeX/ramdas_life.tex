Chronology
\begin{itemize}
\item Born on Ram-navami (1608) at Jamb, Jalna District, Maharashtra, India
\item Full Name: Narayan Suryajipant Kulkarni Thosar (जमदग्नी गोत्र, देशस्थ ऋग्वेदी ब्राह्मण)
\item Ran away at marriage at 12
\item 12 years (1620-32) upasana at the banks of Godavari, at Takali, Panchavati - Nasik, 3 hours jap daily, standing in the river.
\item 12 years (1632-44) of traveling all over India to observe the social situation and state of mankind. Started establishing several Ashrams (Mutths)
\item Died on 1682 (at 74 years)
\end{itemize}

Disciples
\begin{itemize}
\item \textbf{Shivaji}: अनुग्रह   on 1649.
\item 
\end{itemize}

Writings
\begin{itemize}
\item \textbf{Shri Manāche Shlok/Manobodh} advises ethical behaviour and love for God, on 1678
\item \textbf{Dasbodh} provides advice on both spiritual and practical topics. Being written from 1632-80. 
\item \textbf{Shri Māruti Stotra} a poem in praise of Hanuman, 
\item \textbf{AatmaaRaam} 11-Laghu Kavita
\item \textbf{Raamayan} Marathi-Teeka
\item \textbf{Ganapati aarati}: Sukhkartā Dukhhartā Vārtā Vighnāchi.
\item Anandavanbhuvan
\item Shivakalyanaraja
\item Letters
\end{itemize}

Principles
\begin{itemize}
\item Practiced physical yoga exercises as well as meditation. The Surya Namaskara or Sun Salutation was his favourite as it involves every part of the body. It is said that he would do 1,200 Sun Salutations every day, believing that no spirituality could be attained if the physical body was not strong.
\item  Started building Hanuman temples (based on his devotion to Lord Ram), to encourage youth to visit the temple, develop physical strength
\item Built Ram temple at Chaphal, in a cemetery. Threw stones regarded as Gods, in the river – form of education against blind faith.
\item Approached villagers with the intention of educating young children, to teach them Sanskars and develop them as better citizens. For the kids, 100 Surya Namaskar daily. Goal for the year – 1000 Surya Namaskar. If your body is not strong, you can't serve others.
\item Everyone is the same, caste, religion is no barrier.
\item What to pray for? What to ask God? Wisdom, Capability, nothing more.
\item Several Muslims were followers. Usage of Urdu was not uncommon.
\item Responsible for construction of social and human foundation.
\item Key characteristics: Self confidence, Faith in Lord Ram.
\item Changed the image of a Saint at the time.
\item  He encouraged women to participate in religious work and gave them positions of authority. He had 18 staunch women disciples. Wennabai took care of the study centre at Miraj, Akkabai took charge of Chafal and Sajjangad. He strongly reprimanded an old man who was against women participation by saying that everyone came from a woman’s womb and those who did not understand their importance were not worthy of being called a man. 
\end{itemize}

