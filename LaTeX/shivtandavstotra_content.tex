
%%%%%%%%%%%%%%%%%%%%%%%%%%%%%%%%%%%%%%%%%%%%%%%%%%%%%%%%%%%
\begin{frame}[fragile]\frametitle{}

जटाटवीगलज्जलप्रवाहपावितस्थले गलेवलम्ब्य लम्बितां भुजङ्गतुङ्गमालिकाम् | 

डमड्डमड्डमड्डमन्निनादवड्डमर्वयं चकार चण्डताण्डवं तनोतु नः शिवः शिवम् || ०१


मराठी – जटांच्या जंगलातून वाहणा-या (गंगेच्या) प्रवाहाने विशुद्ध झालेल्या ठिकाणी, गळ्यात लोंबणारी सापांची जाडजूड माळ अडकवून डमरूतून डमड्डमड्डमड्डम असा आवाज काढणारा व महाभयंकर तांडव करणारा शंकर आमचे कल्याण करो.. 
\end{frame}

%%%%%%%%%%%%%%%%%%%%%%%%%%%%%%%%%%%%%%%%%%%%%%%%%%%%%%%%%%%
\begin{frame}[fragile]\frametitle{}
जटाकटाहसम्भ्रमभ्रमन्निलिम्पनिर्झरी- -विलोलवीचिवल्लरीविराजमानमूर्धनि | 

धगद्धगद्धगज्ज्वलल्ललाटपट्टपावके किशोरचन्द्रशेखरे रतिः प्रतिक्षणं मम || ०२ 


मराठी- ज्याच्या मस्तकावरील जटांच्या विस्तीर्ण डोहात गोंधळून फिरतांना अस्थिर चंचल लाटांची वेलबुट्टी करणारी गंगा आरूढ झाली आहे, ज्याच्या कपाळपट्टीवरील अग्नी धगधगत आहे, ज्याच्या मस्तकावर चंद्रकोर आहे अशा (शंकरा) बद्दल मला नेहेमी प्रेम वाटते. 
\end{frame}

%%%%%%%%%%%%%%%%%%%%%%%%%%%%%%%%%%%%%%%%%%%%%%%%%%%%%%%%%%%
\begin{frame}[fragile]\frametitle{}
धराधरेन्द्रनन्दिनीविलासबन्धुबन्धुर स्फुरत् दृगन्तसन्ततिप्रमोदमानमानसे | 

कृपाकटाक्षधोरणीनिरुद्धदुर्धरापदि क्वचिद्दिगम्बरे मनो विनोदमेतु वस्तुनि || ०३ 


मराठी- पृथ्वीवरील श्रेष्ठ पर्वतराजाच्या कन्येच्या क्रीडेचा नित्य साथीदार असणा-या, तिच्या चंचल नेत्रांच्या कोप-यातून येणा-या कटाक्षांनी मनाला आनंद होणा-या, कृपापूर्ण दृष्टिक्षेपांमधून अवघड संकटांना दूर ठेवणा-या, कधीकधी दिगंबरावस्थेत असणा-या शंकरात माझे मन रमो. 

टीप- येथे काही आवृत्तीत ‘दृगन्त’ ऐवजी ‘दिगन्त’ असा पाठभेद आढळतो. परंतु तो घेतल्यास संपूर्ण ओळीचा अर्थ लावणे दुरापास्त ठरते. 
\end{frame}

%%%%%%%%%%%%%%%%%%%%%%%%%%%%%%%%%%%%%%%%%%%%%%%%%%%%%%%%%%%
\begin{frame}[fragile]\frametitle{}
जटाभुजङ्गपिङ्गलस्फुरत्फणामणिप्रभा कदम्बकुङ्कुमद्रवप्रलिप्तदिग्वधूमुखे | 

मदान्धसिन्धुरस्फुरत्त्वगुत्तरीयमेदुरे मनो विनोदमद्भुतं बिभर्तु भूतभर्तरि || ०४ 


मराठी- जटांना वेढणा-या विटकरी नागाच्या सळसळणा-या फड्यावरील मण्याच्या तेजात, कदंब व कुंकवाच्या द्रावणाचा रंग दिशारूपी वधूच्या वदनावर पसरवणारा, मदमस्त हत्तीचे चमचमणारे जाड कातडे पांघरणारा भूतनाथ मनास आत्यंतिक आनंदाचा अनुभव देवो. 
\end{frame}

%%%%%%%%%%%%%%%%%%%%%%%%%%%%%%%%%%%%%%%%%%%%%%%%%%%%%%%%%%%
\begin{frame}[fragile]\frametitle{}
सहस्रलोचनप्रभृत्यशेषलेखशेखर प्रसूनधूलिधोरणी विधूसराङ्घ्रिपीठभूः | 

भुजङ्गराजमालया निबद्धजाटजूटक श्रियै चिराय जायतां चकोरबन्धुशेखरः || ०५ 


मराठी- सहस्र डोळे असलेला इंद्र व इतर झाडून सगळ्या देवांच्या किरीटातील फुलांमधील परागांमुळे ज्याचे पायदान धुळकट झाले आहे, सर्पश्रेष्ठाच्या माळेने ज्याच्या जटांची गाठ बांधली आहे, असा चकोराचा मित्र (चंद्र) ज्याच्या शिरावर आहे असा (चंद्रशेखर) अखंड संपदेला कारणीभूत होवो. 
\end{frame}

%%%%%%%%%%%%%%%%%%%%%%%%%%%%%%%%%%%%%%%%%%%%%%%%%%%%%%%%%%%
\begin{frame}[fragile]\frametitle{}
ललाटचत्वरज्वलद्धनञ्जयस्फुलिङ्गभा- -निपीतपञ्चसायकं नमन्निलिम्पनायकम् | 

सुधामयूखलेखया विराजमानशेखरं महाकपालिसम्पदेशिरोजटालमस्तु नः || ०६ 


मराठी- कपाळाच्या चौथ-यावर धगधगणा-या अग्नीने ज्याने मदनाला वैराण करून टाकले (जाळून टाकले), ज्याला देवांचा नेता (इंद्र) वंदन करतो, ज्यातून जणू अमृतकिरणच स्रवतात असा चंद्र मस्तकी धारण करणारा व हाती मोठे भिक्षापात्र घेतलेला (गळ्यात नरमुंडांचा हार घातलेला) शिव आमचा होवो. 

टीप- येथे `कपालिसंपदे’ चा अर्थ अभ्यासकांनी विविध पद्धतीने लावलेला दिसतो. कोणी ‘Wealth of Siddhis’ तर कोणी शिवाच्या गळ्यातील कवट्यांचा हार असा घेतला आहे. कपालि = शंकर असा अर्थ घेतल्यास महाकपालि = महादेव अशा अर्थाने महाकपालिसंपद याचा अर्थ महादेवाच्या विविध सिद्धी असा अर्थ लावता येईल.शिव पुराणातील कथेचा संदर्भ घेतल्यास तो शिवाच्या हातातील भिक्षापात्र असाही होऊ शकेल. 
\end{frame}

%%%%%%%%%%%%%%%%%%%%%%%%%%%%%%%%%%%%%%%%%%%%%%%%%%%%%%%%%%%
\begin{frame}[fragile]\frametitle{}

करालभालपट्टिकाधगद्धगद्धगज्ज्वल- द्धनञ्जयाधरीकृतप्रचण्डपञ्चसायके |

धराधरेन्द्रनन्दिनीकुचाग्रचित्रपत्रक- प्रकल्पनैकशिल्पिनि त्रिलोचने मतिर्मम || ०७ 


मराठी- प्रचंड कपाळपट्टीवर धगधगीत जळणा-या विक्राळ अग्नीच्या प्राशनासाठी ज्याने पाच बाण ज्याचे शस्त्र आहेत अशा मदनाला दाखल केले (मदनाची आहुती दिली), पर्वतश्रेष्ठ हिमालयाच्या कन्येच्या स्तनाग्रांवर शोभिवंत रेषांची नक्षी काढणारा एकमेव चित्रकार असणा-या त्रिनेत्र शंकराचे ठायी माझी बुद्धी एकवटो. 
\end{frame}

%%%%%%%%%%%%%%%%%%%%%%%%%%%%%%%%%%%%%%%%%%%%%%%%%%%%%%%%%%%
\begin{frame}[fragile]\frametitle{}
नवीनमेघमण्डली निरुद्धदुर्धरस्फुरत्- कुहूनिशीथिनीतमः प्रबन्धबन्धुकन्धरः | 

निलिम्पनिर्झरीधरस्तनोतु कृत्तिसिन्धुरः कलानिधानबन्धुरः श्रियं जगद्धुरन्धरः || ०८ 


मराठी- अमावास्येच्या रात्री रोखून धरलेल्या काळ्याकभिन्न नवीन ढगांच्या माळांमुळे दिसणा-या अंधारासारखी ज्याची मान तेजस्वी दिसते, ज्याने (आपल्या शिरावर) देवांची नदी (गंगा) धारण केली आहे, ज्याने हत्तीचे कातडे पांघरले आहे, जो सर्व कलांचे एकत्रित स्थान आहे, जो सर्व जगाचा भार वाहतो, (असा शंकर) आमची सुखे वृद्धिंगत करो. 
\end{frame}

%%%%%%%%%%%%%%%%%%%%%%%%%%%%%%%%%%%%%%%%%%%%%%%%%%%%%%%%%%%
\begin{frame}[fragile]\frametitle{}

प्रफुल्लनीलपङ्कजप्रपञ्चकालिमप्रभा- -विलम्बिकण्ठकन्दलीरुचिप्रबद्धकन्धरम् | 

स्मरच्छिदं पुरच्छिदं भवच्छिदं मखच्छिदं गजच्छिदान्धकच्छिदं तमन्तकच्छिदं भजे || ०९ 


मराठी- ज्याची पूर्ण उमललेल्या कृष्णकमळाच्या, विश्वाच्या (पातकांच्या) समान, काळ्या रंगाची ग्रीवा मानेवर लोंबणा-या कानशीलांच्या सौंदर्याशी निगडीत झाली आहे, ज्याने मदनाचा, तीन नगरांचा, जगताच्या बंधनांचा, यज्ञाचा, हत्तीचा, अंधकासुराचा नाश केला आहे, त्या यमावर नियंत्रण ठेवणा-या (शंकरा) ची मी उपासना करतो. 
\end{frame}

%%%%%%%%%%%%%%%%%%%%%%%%%%%%%%%%%%%%%%%%%%%%%%%%%%%%%%%%%%%
\begin{frame}[fragile]\frametitle{}

अखर्वसर्वमङ्गलाकलाकदम्बमञ्जरी रसप्रवाहमाधुरी विजृम्भणामधुव्रतम् |

स्मरान्तकं पुरान्तकं भवान्तकं मखान्तकं गजान्तकान्धकान्तकं तमन्तकान्तकं भजे || १० 


मराठी- महत्त्वपूर्ण, सर्वांसाठी कल्याणमय कलारूपी कदंबाच्या मोहोरातील मधुर रसाचा विरंगुळा असलेल्या भ्रमराची, ज्याने मदनाचा, तीन नगरांचा, जगताच्या बंधनांचा, यज्ञाचा, हत्तीचा, अंधकासुराचा नाश केला आहे, त्या यमावर नियंत्रण ठेवणा-या (शंकरा) ची मी उपासना करतो. 

टीप- येथे अगर्व / अखर्व असाही पाठभेद आढळतो. तथापि त्यामुळे श्लोकाचा अर्थ बदलू शकतो. (अगर्व- निगर्वी,अहंकाररहित; अखर्व- महत्त्वपूर्ण,महान). 
\end{frame}

%%%%%%%%%%%%%%%%%%%%%%%%%%%%%%%%%%%%%%%%%%%%%%%%%%%%%%%%%%%
\begin{frame}[fragile]\frametitle{}
जयत्वदभ्रविभ्रमभ्रमद्भुजङ्गमश्वस- -द्विनिर्गमत्क्रमस्फुरत्करालभालहव्यवाट् | 

धिमिद्धिमिद्धिमिध्वनन्मृदङ्गतुङ्गमङ्गल ध्वनिक्रमप्रवर्तित प्रचण्डताण्डवः शिवः || ११ 


मराठी- प्रचंड तीव्रतेने फिरणा-या सर्पाच्या बाहेर पडणा-या निश्वासातून कपाळी धगधगणारा वन्ही पसरताना, धिमिद धिमिद असा मृदंगाचा मंगलमय उच्च आवाज चढत जाताना क्रमाक्रमाने उग्र तांडव करणारा शंकर विजयी होवो. 
\end{frame}

%%%%%%%%%%%%%%%%%%%%%%%%%%%%%%%%%%%%%%%%%%%%%%%%%%%%%%%%%%%
\begin{frame}[fragile]\frametitle{}

दृषद्विचित्रतल्पयोर्भुजङ्गमौक्तिकस्रजोर्- -गरिष्ठरत्नलोष्ठयोः सुहृद्विपक्षपक्षयोः | 

तृणारविन्दचक्षुषोः प्रजामहीमहेन्द्रयोः समं प्रवर्तयन्मनः कदा सदाशिवं भजे || १२



 मराठी- विभिन्न प्रकारच्या शय्या (आसने), (गळ्यातील) सर्पमाला किंवा मोत्यांची माळ, श्रेष्ठ रत्न किंवा मातीचे ढेकूळ, मित्र किंवा विरोधक, गवत किंवा कमळासारखे नेत्र, सामान्य जन वा पृथ्वीचा राजा, सर्वांना सारखेपणाने वागवणा-या शंकराचे मी केव्हां पूजन करू शकेन. 
 

 टीप- येथे ‘दृषद्’ च्या ऐवजी ‘स्पृषद्’ असाही पाठभेद आढळतो. तो घेतल्यास समान भावनेने विविध आसने, सर्प, मोत्यांचा हार इत्यादींना स्पर्श करणारा असा अर्थ घ्यावा लागेल. 
\end{frame}

%%%%%%%%%%%%%%%%%%%%%%%%%%%%%%%%%%%%%%%%%%%%%%%%%%%%%%%%%%%
\begin{frame}[fragile]\frametitle{}

 कदा निलिम्पनिर्झरीनिकुञ्जकोटरे वसन् विमुक्तदुर्मतिः सदा शिरःस्थमञ्जलिं वहन् |

 
 विमुक्तलोललोचनो ललाटभाललग्नकः शिवेति मन्त्रमुच्चरन् सदा सुखी भवाम्यहम् || १३ 
 
 
 मराठी- गंगाकाठी कुटीरात राहून, वाईट विचारांपासून सुटका झालेला, नित्य डोक्यावर जोडलेले हात धरून, (स्त्रीच्या) चंचल नेत्रांपासून मुक्तता मिळवून, (शिवाच्या) कपाळीच्या तेजाचा बंधक मी, ‘शिव शिव’ असा मंत्र जपत केव्हां सुखी होईन ?
\end{frame}

%%%%%%%%%%%%%%%%%%%%%%%%%%%%%%%%%%%%%%%%%%%%%%%%%%%%%%%%%%%
\begin{frame}[fragile]\frametitle{}
 
 इमं हि नित्यमेवमुक्तमुत्तमोत्तमं स्तवं पठन्स्मरन्ब्रुवन्नरो विशुद्धिमेति सन्ततम् | 
 
 
 हरे गुरौ सुभक्तिमाशु याति नान्यथा गतिं विमोचनं हि देहिनां सुशङ्करस्य चिन्तनम् || १४ 
 
 
 मराठी- अशा या सांगितलेल्या अतीव उत्कृष्ट स्तवनाचे जो स्मरण करतो, वाचतो, उच्चारतो तो कायमचा पवित्र होतो. अन्य कोणत्याही मार्गाने न जाता तो श्रेष्ठ शंकराचा भक्त होतो. शंकराच्या केवळ चिन्तनाने जीवाला मुक्तता मिळते. 
\end{frame}

% %%%%%%%%%%%%%%%%%%%%%%%%%%%%%%%%%%%%%%%%%%%%%%%%%%%%%%%%%%%
% \begin{frame}[fragile]\frametitle{}
 
 % पूजावसानसमये दशवक्त्रगीतं यः शम्भुपूजनपरं पठति प्रदोषे | 
 
 
%  तस्य स्थिरां रथगजेन्द्रतुरङ्गयुक्तां लक्ष्मीं सदैव सुमुखिं प्रददाति शम्भुः || १५ 
 
 
 
 % मराठी- जो (हे) शंकराचे स्तवनपर रावणाचे गीत सायंकाळी पूजेच्या वेळी पठन करतो, महादेव त्याला नित्य रथ, श्रेष्ठ हत्ती, घोडे यांच्यासह सुंदर स्थायी लक्ष्मी दान करतो. 
 

 % टीप- हा १५ वा श्लोक माझ्या मते प्रक्षिप्त असावा. त्याचे वृत्त (वसंततिलका), त्यातील भाव बाकीच्या श्लोकांशी सुसंगत नाहीत. या श्लोकात हे स्तोत्र पठन केल्याने लक्ष्मी प्राप्त होते असे रावणाने आवर्जून सांगण्याचे काही कारण दिसत नाही. काही आवृत्तींमध्ये दोन अधिक श्लोक दिसतात. परंतु ते साधारण अशाच स्वरूपाचे असल्याने येथे घेतलेले नाहीत. 
 
% \end{frame}

%%%%%%%%%%%%%%%%%%%%%%%%%%%%%%%%%%%%%%%%%%%%%%%%%%%%%%%%%%%
\begin{frame}[fragile]\frametitle{}
 । इति श्रीरावणकृतम् शिवताण्डव स्तोत्रम् सम्पूर्णम्  । 
 \end{frame}