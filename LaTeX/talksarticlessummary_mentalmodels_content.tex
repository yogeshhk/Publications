%%%%%%%%%%%%%%%%%%%%%%%%%%%%%%%%%%%%%%%%%%%%%%%%%%%%%%%%%%%%%%%%%%%%%%%%%%%%%%%%%%
\begin{frame}[fragile]\frametitle{}
\begin{center}
{\Large Introduction to Mental Models}

{\tiny Farnam Street and Miscellaneous Authors }


\end{center}
\end{frame}

%%%%%%%%%%%%%%%%%%%%%%%%%%%%%%%%%%%%%%%%%%%%%%%%%%%%%%%%%%%
\begin{frame}[fragile]\frametitle{What Are Mental Models?}
\begin{itemize}
    \item A mental model is a compression of how something works.
    \item Any idea, belief, or concept can be distilled down.
    \item Like a map, mental models reveal key information while ignoring irrelevant details.
    \item Models concentrate the world into understandable and useable chunks.
    \item Mental models help us understand the world.
\end{itemize}
\textbf{Examples:}
\begin{itemize}
    \item Velocity: understanding that both speed and direction matter.
    \item Reciprocity: understanding how going positive and going first gets the world to do most of the work for you.
    \item Margin of Safety: understanding that things don’t always go as planned.
    \item Relativity: showing us we have blind spots and how a different perspective can reveal new information.
\end{itemize}
\end{frame}

%%%%%%%%%%%%%%%%%%%%%%%%%%%%%%%%%%%%%%%%%%%%%%%%%%%%%%%%%%%
\begin{frame}[fragile]\frametitle{Eliminating Blind Spots}
\begin{itemize}
    \item The person with the fewest blind spots wins in life and business.
    \item Blind spots are the source of all poor choices.
    \item Perfect information would lead to perfect decisions.
    \item Reducing blind spots requires changing perspective.
    \item Different perspectives reveal critical information and new solutions.
\end{itemize}
\textbf{Examples:}
\begin{itemize}
    \item Seeing all players' cards in poker ensures perfect play.
    \item Changing the angle of a photograph to capture a better shot.
\end{itemize}
\end{frame}

%%%%%%%%%%%%%%%%%%%%%%%%%%%%%%%%%%%%%%%%%%%%%%%%%%%%%%%%%%%
\begin{frame}[fragile]\frametitle{Using Multiple Mental Models}
\begin{itemize}
    \item Each model acts as a lens to view the world differently.
    \item Combining lenses reveals more information than each one individually.
    \item Specific models are numerous, but only a handful of general ones from big disciplines are crucial.
    \item Understanding general models helps reduce errors and uncover hidden insights.
    \item Better actions result from applying these diverse perspectives.
\end{itemize}
\end{frame}

%%%%%%%%%%%%%%%%%%%%%%%%%%%%%%%%%%%%%%%%%%%%%%%%%%%%%%%%%%%
\begin{frame}[fragile]\frametitle{Mental Model Toolbox}
\begin{itemize}
    \item "You’ve got to have models in your head and you’ve got to array your experience – both vicarious and direct – onto this latticework of mental models." - Charlie Munger
    \item Worldly wisdom involves understanding the consequences of actions.
    \item Wisdom aligns facts with reasoning.
    \item Disciplines are interconnected in the real world.
    \item Insights from various fields like physics can enhance understanding and performance in business.
\end{itemize}
\end{frame}

%%%%%%%%%%%%%%%%%%%%%%%%%%%%%%%%%%%%%%%%%%%%%%%%%%%%%%%%%%%
\begin{frame}[fragile]\frametitle{A Latticework of Mental Models}
\begin{itemize}
    \item The world is interconnected like a latticework.
    \item Knowledge isn't confined to distinct disciplines.
    \item For instance, business can benefit from physics principles like velocity and kinetic energy.
    \item Teachers might not integrate big ideas from all disciplines, but self-learning is possible.
    \item Connecting insights across fields leads to better understanding and decision-making.
\end{itemize}
\end{frame}

%%%%%%%%%%%%%%%%%%%%%%%%%%%%%%%%%%%%%%%%%%%%%%%%%%%%%%%%%%%%%%%%%%%%%%%%%%%%%%%%%%
\begin{frame}[fragile]\frametitle{}
\begin{center}
{\Large Core/Generic Mental Models}

{\tiny Farnam Street and Miscellaneous Authors }


\end{center}
\end{frame}

%%%%%%%%%%%%%%%%%%%%%%%%%%%%%%%%%%%%%%%%%%%%%%%%%%%%%%%%%%%
\begin{frame}[fragile]\frametitle{The Map is Not the Territory}
\begin{itemize}
    \item The map of reality is not reality itself.
    \item Maps are reductions and inherently imperfect.
    \item A perfect map would be as large as the territory itself and thus impractical.
    \item Maps can represent outdated snapshots of reality.
    \item Recognizing this helps avoid over-reliance on models and maps.
\end{itemize}
\textbf{Examples:}
\begin{itemize}
    \item A GPS map shows the route but not the current traffic conditions.
    \item Business forecasts are useful but can’t predict unexpected market shifts.
\end{itemize}
\end{frame}

%%%%%%%%%%%%%%%%%%%%%%%%%%%%%%%%%%%%%%%%%%%%%%%%%%%%%%%%%%%
\begin{frame}[fragile]\frametitle{Circle of Competence}
\begin{itemize}
    \item Understand your strengths and limits.
    \item Knowledge of your competence provides a competitive edge.
    \item Awareness of blind spots allows for better learning and growth.
    \item Ego-driven actions outside your competence lead to mistakes.
    \item Honest self-assessment improves decision-making.
\end{itemize}
\textbf{Examples:}
\begin{itemize}
    \item An investor sticks to industries they understand.
    \item A software developer specializes in a particular programming language.
\end{itemize}
\end{frame}

%%%%%%%%%%%%%%%%%%%%%%%%%%%%%%%%%%%%%%%%%%%%%%%%%%%%%%%%%%
\begin{frame}[fragile]\frametitle{First Principles Thinking}
\begin{itemize}
    \item Break down complex problems to their basic components.
    \item Separate underlying facts from assumptions.
    \item Build new knowledge from fundamental truths.
    \item Encourages creativity and innovation.
    \item Essential for solving novel and complicated problems.
\end{itemize}
\textbf{Examples:}
\begin{itemize}
    \item Elon Musk rethinking battery costs by analyzing raw materials.
    \item A chef creating a new recipe by understanding the basics of flavors.
\end{itemize}
\end{frame}

%%%%%%%%%%%%%%%%%%%%%%%%%%%%%%%%%%%%%%%%%%%%%%%%%%%%%%%%%%%
\begin{frame}[fragile]\frametitle{Thought Experiment}
\begin{itemize}
    \item Use imagination to explore possibilities.
    \item Investigate the nature of things without real-world constraints.
    \item Learn from hypothetical scenarios to avoid future mistakes.
    \item Evaluate consequences and re-examine historical decisions.
    \item Helps clarify what we truly want and how to achieve it.
\end{itemize}
\textbf{Examples:}
\begin{itemize}
    \item Schrödinger's cat in quantum mechanics.
    \item Imagining the impact of a business decision on future market trends.
\end{itemize}
\end{frame}

%%%%%%%%%%%%%%%%%%%%%%%%%%%%%%%%%%%%%%%%%%%%%%%%%%%%%%%%%%%
\begin{frame}[fragile]\frametitle{Second-Order Thinking}
\begin{itemize}
    \item Consider the long-term effects of actions.
    \item Go beyond immediate consequences.
    \item Think holistically about cascading impacts.
    \item Prevent unintended side effects.
    \item Differentiate yourself by thinking further ahead.
\end{itemize}
\textbf{Examples:}
\begin{itemize}
    \item Implementing a policy that encourages short-term sales might hurt long-term customer loyalty.
    \item Environmental regulations affecting not just industry costs but public health and future compliance costs.
\end{itemize}
\end{frame}

%%%%%%%%%%%%%%%%%%%%%%%%%%%%%%%%%%%%%%%%%%%%%%%%%%%%%%%%%%%
\begin{frame}[fragile]\frametitle{Probabilistic Thinking}
\begin{itemize}
    \item Estimate the likelihood of outcomes.
    \item Use math and logic to improve decision accuracy.
    \item Account for complex, multifactorial influences.
    \item Identify most likely scenarios and plan accordingly.
    \item Enhance precision in uncertain environments.
\end{itemize}
\textbf{Examples:}
\begin{itemize}
    \item Evaluating the risk of an investment based on historical data.
    \item Deciding whether to carry an umbrella based on weather forecasts.
\end{itemize}
\end{frame}

%%%%%%%%%%%%%%%%%%%%%%%%%%%%%%%%%%%%%%%%%%%%%%%%%%%%%%%%%%%
\begin{frame}[fragile]\frametitle{Inversion}
\begin{itemize}
    \item Approach problems from the opposite end.
    \item Identify obstacles to success and eliminate them.
    \item Think backward to avoid common pitfalls.
    \item Helps in discovering unseen perspectives.
    \item Effective in problem-solving and strategy formulation.
\end{itemize}
\textbf{Examples:}
\begin{itemize}
    \item Instead of asking how to succeed, ask how to avoid failure.
    \item Designing a user-friendly product by first identifying user frustrations.
\end{itemize}
\end{frame}

%%%%%%%%%%%%%%%%%%%%%%%%%%%%%%%%%%%%%%%%%%%%%%%%%%%%%%%%%%%
\begin{frame}[fragile]\frametitle{Occam’s Razor}
\begin{itemize}
    \item Simpler explanations are usually correct.
    \item Avoid unnecessary complexity in problem-solving.
    \item Focus on explanations with the fewest assumptions.
    \item Save time by not disproving unlikely scenarios.
    \item Enhance clarity and confidence in decisions.
\end{itemize}
\textbf{Examples:}
\begin{itemize}
    \item Diagnosing an illness with common symptoms rather than rare diseases.
    \item Troubleshooting a computer issue by checking simple fixes first.
\end{itemize}

\end{frame}

%%%%%%%%%%%%%%%%%%%%%%%%%%%%%%%%%%%%%%%%%%%%%%%%%%%%%%%%%%%
\begin{frame}[fragile]\frametitle{Hanlon’s Razor}
\begin{itemize}
    \item Don’t attribute to malice what can be explained by stupidity.
    \item Avoid paranoia and ideological biases.
    \item Look for simpler, benign explanations for actions.
    \item Recognize that people make mistakes.
    \item Seek reasonable explanations before assuming ill intent.
\end{itemize}
\textbf{Examples:}
\begin{itemize}
    \item A colleague's rude email might be due to stress, not personal animosity.
    \item A missed appointment could be a simple scheduling error, not disrespect.
\end{itemize}
\end{frame}


%%%%%%%%%%%%%%%%%%%%%%%%%%%%%%%%%%%%%%%%%%%%%%%%%%%%%%%%%%%%%%%%%%%%%%%%%%%%%%%%%%
\begin{frame}[fragile]\frametitle{}
\begin{center}
{\Large Physics and Chemistry Mental Models}

{\tiny Farnam Street and Miscellaneous Authors }


\end{center}
\end{frame}

%%%%%%%%%%%%%%%%%%%%%%%%%%%%%%%%%%%%%%%%%%%%%%%%%%%%%%%%%%%
\begin{frame}[fragile]\frametitle{Relativity}
\begin{itemize}
    \item An observer cannot truly understand a system of which he is a part.
    \item A man inside an airplane does not feel movement, but an outside observer can see it.
    \item This concept affects social systems similarly.
    \item Perspective influences understanding of any system.
    \item Recognizing our position helps us understand our limitations.
\end{itemize}
\textbf{Example:}
\begin{itemize}
    \item In a team, an individual may not see the group's progress as an outsider does.
\end{itemize}
\end{frame}

%%%%%%%%%%%%%%%%%%%%%%%%%%%%%%%%%%%%%%%%%%%%%%%%%%%%%%%%%%%
\begin{frame}[fragile]\frametitle{Reciprocity}
\begin{itemize}
    \item Pushing on a wall results in the wall pushing back with equivalent force.
    \item In biological systems, actions are reciprocated in kind.
    \item Humans demonstrate intense reciprocity in their interactions.
    \item Understanding reciprocity helps predict reactions in social systems.
    \item Reciprocity is a foundation of social and biological interactions.
\end{itemize}
\textbf{Examples:}
\begin{itemize}
    \item Helping a colleague often leads to them helping you in return.
    \item In nature, animals reciprocate grooming behaviors.
\end{itemize}
\end{frame}

%%%%%%%%%%%%%%%%%%%%%%%%%%%%%%%%%%%%%%%%%%%%%%%%%%%%%%%%%%%
\begin{frame}[fragile]\frametitle{Thermodynamics}
\begin{itemize}
    \item The laws of thermodynamics describe energy in a closed system.
    \item Useful energy is constantly being lost.
    \item Energy cannot be created or destroyed.
    \item These laws underlie the physical world.
    \item Applying these principles to social systems can be profitable.
\end{itemize}
\textbf{Examples:}
\begin{itemize}
    \item In business, resources are finite and must be managed efficiently.
    \item In personal energy management, rest and recovery are crucial.
\end{itemize}
\end{frame}

%%%%%%%%%%%%%%%%%%%%%%%%%%%%%%%%%%%%%%%%%%%%%%%%%%%%%%%%%%%
\begin{frame}[fragile]\frametitle{Inertia}
\begin{itemize}
    \item An object in motion wants to continue moving in the same direction.
    \item This principle applies to individuals, systems, and organizations.
    \item Inertia allows for energy minimization.
    \item It can also lead to destruction or erosion if not managed.
    \item Recognizing inertia helps in predicting and influencing behavior.
\end{itemize}
\textbf{Examples:}
\begin{itemize}
    \item A company with a successful product line may resist innovation.
    \item Personal habits are hard to change due to inertia.
\end{itemize}
\end{frame}

%%%%%%%%%%%%%%%%%%%%%%%%%%%%%%%%%%%%%%%%%%%%%%%%%%%%%%%%%%%
\begin{frame}[fragile]\frametitle{Friction and Viscosity}
\begin{itemize}
    \item Friction opposes the movement of objects in contact.
    \item Viscosity measures the difficulty of fluid movement.
    \item Higher viscosity leads to higher resistance.
    \item These concepts illustrate how environments impede movement.
    \item Understanding them helps in navigating and reducing resistance.
\end{itemize}
\textbf{Examples:}
\begin{itemize}
    \item Bureaucratic processes create friction in organizations.
    \item Thick social networks can increase viscosity, making change harder.
\end{itemize}
\end{frame}

%%%%%%%%%%%%%%%%%%%%%%%%%%%%%%%%%%%%%%%%%%%%%%%%%%%%%%%%%%%
\begin{frame}[fragile]\frametitle{Velocity}
\begin{itemize}
    \item Velocity is speed plus vector: direction matters.
    \item Speed alone does not determine effective movement.
    \item Velocity helps in assessing actual progress.
    \item Consider both speed and direction in practical life.
    \item Properly directed effort is more productive than speed alone.
\end{itemize}
\textbf{Examples:}
\begin{itemize}
    \item A company growing rapidly but in the wrong market lacks positive velocity.
    \item A project moving fast but with constant changes in direction shows no effective progress.
\end{itemize}
\end{frame}

%%%%%%%%%%%%%%%%%%%%%%%%%%%%%%%%%%%%%%%%%%%%%%%%%%%%%%%%%%%
\begin{frame}[fragile]\frametitle{Leverage}
\begin{itemize}
    \item Leverage allows a small input force to create a great output force.
    \item Many engineering marvels were achieved with leverage.
    \item Understanding leverage can lead to significant success.
    \item Apply leverage in the human world for great impact.
    \item Proper use of leverage multiplies effectiveness.
\end{itemize}
\textbf{Examples:}
\begin{itemize}
    \item Using financial leverage to multiply investment returns.
    \item Delegating tasks to a team to achieve larger goals.
\end{itemize}
\end{frame}

%%%%%%%%%%%%%%%%%%%%%%%%%%%%%%%%%%%%%%%%%%%%%%%%%%%%%%%%%%%
\begin{frame}[fragile]\frametitle{Activation Energy}
\begin{itemize}
    \item Chemical reactions require a critical level of activation energy.
    \item Combustible elements need activation energy to react.
    \item Two elements alone are not enough without activation energy.
    \item This concept applies to initiating actions in social systems.
    \item Overcoming initial resistance is crucial for progress.
\end{itemize}
\textbf{Examples:}
\begin{itemize}
    \item Starting a new habit requires overcoming initial resistance.
    \item Launching a new product needs significant initial effort to gain traction.
\end{itemize}
\end{frame}

%%%%%%%%%%%%%%%%%%%%%%%%%%%%%%%%%%%%%%%%%%%%%%%%%%%%%%%%%%%
\begin{frame}[fragile]\frametitle{Catalysts}
\begin{itemize}
    \item A catalyst initiates or maintains a chemical reaction without being consumed.
    \item Reactions may slow or stop without catalysts.
    \item Social systems can also benefit from catalysts.
    \item Catalysts can accelerate processes and drive change.
    \item Identifying and utilizing catalysts can lead to success.
\end{itemize}
\textbf{Examples:}
\begin{itemize}
    \item A strong leader can act as a catalyst for organizational change.
    \item Introducing new technology can catalyze productivity improvements.
\end{itemize}
\end{frame}

%%%%%%%%%%%%%%%%%%%%%%%%%%%%%%%%%%%%%%%%%%%%%%%%%%%%%%%%%%%
\begin{frame}[fragile]\frametitle{Alloying}
\begin{itemize}
    \item Combining elements creates new, often stronger substances.
    \item The result can be greater than the sum of its parts.
    \item Alloying principles apply to social systems and individuals.
    \item The right combination of elements can create significant synergies.
    \item 2+2 can equal 6 in the right context.
\end{itemize}
\textbf{Examples:}
\begin{itemize}
    \item A well-balanced team outperforms the sum of its individual members.
    \item Combining complementary skills in a project leads to superior outcomes.
\end{itemize}
\end{frame}

%%%%%%%%%%%%%%%%%%%%%%%%%%%%%%%%%%%%%%%%%%%%%%%%%%%%%%%%%%%%%%%%%%%%%%%%%%%%%%%%%%
\begin{frame}[fragile]\frametitle{}
\begin{center}
{\Large Biology Mental Models}

{\tiny Farnam Street and Miscellaneous Authors }


\end{center}
\end{frame}


%%%%%%%%%%%%%%%%%%%%%%%%%%%%%%%%%%%%%%%%%%%%%%%%%%%%%%%%%%%
\begin{frame}[fragile]\frametitle{Evolution Part One: Natural Selection and Extinction}
\begin{itemize}
    \item Evolution by natural selection was proposed by Charles Darwin and Alfred Russel Wallace.
    \item Species evolve through random mutation and differential survival rates.
    \item Natural selection: Mother Nature decides the success or failure of a mutation.
    \item Those best suited for survival tend to be preserved.
    \item Conditions change, leading to continuous evolution.
\end{itemize}
\textbf{Example:}
\begin{itemize}
    \item Animal breeding is artificial selection; nature's version is natural selection.
\end{itemize}
\end{frame}

%%%%%%%%%%%%%%%%%%%%%%%%%%%%%%%%%%%%%%%%%%%%%%%%%%%%%%%%%%%
\begin{frame}[fragile]\frametitle{Evolution Part Two: Adaptation and The Red Queen Effect}
\begin{itemize}
    \item Species adapt to their surroundings for survival.
    \item Adaptations made in an individual’s lifetime are not passed down genetically.
    \item Evolution by natural selection leads to an arms race among species.
    \item Competing species must evolve advantageous adaptations or face extinction.
    \item The Red Queen Effect: running as fast as possible to stay in the same place.
\end{itemize}
\textbf{Example:}
\begin{itemize}
    \item Predators and prey constantly evolve to outcompete each other.
\end{itemize}
\end{frame}

%%%%%%%%%%%%%%%%%%%%%%%%%%%%%%%%%%%%%%%%%%%%%%%%%%%%%%%%%%%
\begin{frame}[fragile]\frametitle{Ecosystems}
\begin{itemize}
    \item Ecosystems: groups of organisms coexisting with the natural world.
    \item Diverse life forms take on different survival approaches.
    \item Social systems can be seen as ecosystems.
    \item Similar pressures lead to varying behaviors in both.
    \item Understanding ecosystems helps in understanding social dynamics.
\end{itemize}
\textbf{Example:}
\begin{itemize}
    \item Business ecosystems with competing companies and consumers.
\end{itemize}
\end{frame}

%%%%%%%%%%%%%%%%%%%%%%%%%%%%%%%%%%%%%%%%%%%%%%%%%%%%%%%%%%%
\begin{frame}[fragile]\frametitle{Niches}
\begin{itemize}
    \item Organisms find a niche: a method of competing and surviving.
    \item Species select niches for which they are best adapted.
    \item Multiple species competing for the same niche can cause extinction.
    \item Limited resources drive niche competition.
    \item Understanding niches helps in strategic positioning.
\end{itemize}
\textbf{Example:}
\begin{itemize}
    \item Different companies targeting specific market segments.
\end{itemize}
\end{frame}

%%%%%%%%%%%%%%%%%%%%%%%%%%%%%%%%%%%%%%%%%%%%%%%%%%%%%%%%%%%
\begin{frame}[fragile]\frametitle{Self-Preservation}
\begin{itemize}
    \item Self-preservation instinct is strong in all organisms.
    \item Without it, organisms would tend to disappear over time.
    \item Self-preservation can cause violent, erratic, and destructive behavior.
    \item Cooperation is also important but secondary to self-preservation.
    \item Recognizing self-preservation instincts helps predict behaviors.
\end{itemize}
\textbf{Examples:}
\begin{itemize}
    \item Animals defending their territory.
    \item Human survival instincts in dangerous situations.
\end{itemize}
\end{frame}

%%%%%%%%%%%%%%%%%%%%%%%%%%%%%%%%%%%%%%%%%%%%%%%%%%%%%%%%%%%
\begin{frame}[fragile]\frametitle{Replication}
\begin{itemize}
    \item High-fidelity replication is a fundamental building block of life.
    \item DNA provides a blueprint for offspring.
    \item Various replication methods: sexual and asexual.
    \item Replication ensures the continuation of species.
    \item Understanding replication helps in genetics and breeding.
\end{itemize}
\textbf{Examples:}
\begin{itemize}
    \item Bacteria replicating asexually.
    \item Plants reproducing through seeds.
\end{itemize}
\end{frame}

%%%%%%%%%%%%%%%%%%%%%%%%%%%%%%%%%%%%%%%%%%%%%%%%%%%%%%%%%%%
\begin{frame}[fragile]\frametitle{Cooperation}
\begin{itemize}
    \item Cooperation is as important as competition in biological systems.
    \item The cooperation of a bacterium and a cell created the first complex cell.
    \item No group survives without cooperation.
    \item Cooperation gives rise to complex organizational structures.
    \item The Prisoner’s Dilemma illustrates the importance of cooperation.
\end{itemize}
\textbf{Examples:}
\begin{itemize}
    \item Symbiotic relationships in nature.
    \item Teamwork in human organizations.
\end{itemize}
\end{frame}

%%%%%%%%%%%%%%%%%%%%%%%%%%%%%%%%%%%%%%%%%%%%%%%%%%%%%%%%%%%
\begin{frame}[fragile]\frametitle{Hierarchical Organization}
\begin{itemize}
    \item Complex organisms often organize hierarchically.
    \item Humans feel the hierarchical instinct strongly.
    \item The Stanford Prison Experiment showed human bias towards authority.
    \item Leaders have a responsibility to act well.
    \item Hierarchies provide structure and order in social systems.
\end{itemize}
\textbf{Examples:}
\begin{itemize}
    \item Corporate organizational structures.
    \item Social hierarchies in animal groups.
\end{itemize}
\end{frame}

%%%%%%%%%%%%%%%%%%%%%%%%%%%%%%%%%%%%%%%%%%%%%%%%%%%%%%%%%%%
\begin{frame}[fragile]\frametitle{Incentives}
\begin{itemize}
    \item All creatures respond to incentives to stay alive.
    \item Constant incentives lead to consistent behavior.
    \item Human incentives can be hidden or intangible.
    \item Understanding incentives helps in predicting behaviors.
    \item The rule of life is to repeat what is rewarded.
\end{itemize}
\textbf{Examples:}
\begin{itemize}
    \item Employees working harder for bonuses.
    \item Animals hunting for food.
\end{itemize}
\end{frame}

%%%%%%%%%%%%%%%%%%%%%%%%%%%%%%%%%%%%%%%%%%%%%%%%%%%%%%%%%%%
\begin{frame}[fragile]\frametitle{Tendency to Minimize Energy Output}
\begin{itemize}
    \item Organisms minimize energy usage for survival.
    \item Wasteful energy usage is a disadvantage.
    \item Behavior is governed by minimizing energy usage.
    \item Efficient energy usage ensures better survival chances.
    \item Understanding this tendency helps in optimizing processes.
\end{itemize}
\textbf{Examples:}
\begin{itemize}
    \item Animals conserving energy during hibernation.
    \item Streamlining business operations to reduce costs.
\end{itemize}
\end{frame}

%%%%%%%%%%%%%%%%%%%%%%%%%%%%%%%%%%%%%%%%%%%%%%%%%%%%%%%%%%%%%%%%%%%%%%%%%%%%%%%%%%
\begin{frame}[fragile]\frametitle{}
\begin{center}
{\Large Systems Thinking Mental Models}

{\tiny Farnam Street and Miscellaneous Authors }


\end{center}
\end{frame}

%%%%%%%%%%%%%%%%%%%%%%%%%%%%%%%%%%%%%%%%%%%%%%%%%%%%%%%%%%%
\begin{frame}[fragile]\frametitle{Feedback Loops}
\begin{itemize}
    \item Complex systems are subject to positive and negative feedback loops.
    \item A causes B, which influences A, creating higher-order effects.
    \item Homeostatic systems use feedback loops to maintain balance (e.g., human body temperature).
    \item Runaway feedback loops describe self-catalyzing reactions.
\end{itemize}
\textbf{Examples:}
\begin{itemize}
    \item Organizational culture behaviors.
    \item Economic market corrections.
\end{itemize}
\end{frame}

%%%%%%%%%%%%%%%%%%%%%%%%%%%%%%%%%%%%%%%%%%%%%%%%%%%%%%%%%%%
\begin{frame}[fragile]\frametitle{Equilibrium}
\begin{itemize}
    \item Homeostasis is the process of self-regulation to maintain equilibrium.
    \item Systems often over or undershoot and must keep adjusting.
    \item Everything within a system contributes to maintaining equilibrium.
    \item Understanding the range of equilibrium is crucial.
\end{itemize}
\textbf{Examples:}
\begin{itemize}
    \item Body's response to temperature changes.
    \item Economic market stability.
\end{itemize}
\end{frame}

%%%%%%%%%%%%%%%%%%%%%%%%%%%%%%%%%%%%%%%%%%%%%%%%%%%%%%%%%%%
\begin{frame}[fragile]\frametitle{Bottlenecks}
\begin{itemize}
    \item A bottleneck stops the flow, constraining movement.
    \item A small bottleneck can have a large impact if in the critical path.
    \item Bottlenecks can inspire alternative pathways to success.
\end{itemize}
\textbf{Examples:}
\begin{itemize}
    \item Production process delays.
    \item Traffic jams on critical routes.
\end{itemize}
\end{frame}

%%%%%%%%%%%%%%%%%%%%%%%%%%%%%%%%%%%%%%%%%%%%%%%%%%%%%%%%%%%
\begin{frame}[fragile]\frametitle{Scale}
\begin{itemize}
    \item Systems are sensitive to scale.
    \item Properties and behaviors change when scaled up or down.
    \item Always quantify the scale when observing or analyzing systems.
\end{itemize}
\textbf{Examples:}
\begin{itemize}
    \item Business operations from startup to enterprise.
    \item Ecological impacts from local to global.
\end{itemize}
\end{frame}

%%%%%%%%%%%%%%%%%%%%%%%%%%%%%%%%%%%%%%%%%%%%%%%%%%%%%%%%%%%
\begin{frame}[fragile]\frametitle{Margin of Safety}
\begin{itemize}
    \item Engineers add a margin for error into calculations.
    \item A margin of safety prevents failures in unknown conditions.
    \item Robust margins lead to better long-term outcomes.
\end{itemize}
\textbf{Examples:}
\begin{itemize}
    \item Structural engineering standards.
    \item Financial investment strategies.
\end{itemize}
\end{frame}

%%%%%%%%%%%%%%%%%%%%%%%%%%%%%%%%%%%%%%%%%%%%%%%%%%%%%%%%%%%
\begin{frame}[fragile]\frametitle{Churn}
\begin{itemize}
    \item Churn is the slow leak of customers or users.
    \item It can be hidden, unlike sudden crises.
    \item Some churn is inevitable and can present opportunities for improvement.
\end{itemize}
\textbf{Examples:}
\begin{itemize}
    \item Customer turnover in businesses.
    \item Staff turnover in organizations.
\end{itemize}
\end{frame}

%%%%%%%%%%%%%%%%%%%%%%%%%%%%%%%%%%%%%%%%%%%%%%%%%%%%%%%%%%%
\begin{frame}[fragile]\frametitle{Algorithms}
\begin{itemize}
    \item Algorithms are step-by-step processes for specific outcomes.
    \item They filter noise and focus on signals.
    \item Thinking algorithmically means finding reliable processes.
\end{itemize}
\textbf{Examples:}
\begin{itemize}
    \item Manufacturing processes.
    \item Software development routines.
\end{itemize}
\end{frame}

%%%%%%%%%%%%%%%%%%%%%%%%%%%%%%%%%%%%%%%%%%%%%%%%%%%%%%%%%%%
\begin{frame}[fragile]\frametitle{Critical Mass}
\begin{itemize}
    \item Critical mass is when a system shifts from slow to explosive growth.
    \item It is the tipping point for self-sustaining reactions.
    \item Understanding critical mass helps design and implement changes.
\end{itemize}
\textbf{Examples:}
\begin{itemize}
    \item Viral marketing campaigns.
    \item Social movements gaining momentum.
\end{itemize}
\end{frame}

%%%%%%%%%%%%%%%%%%%%%%%%%%%%%%%%%%%%%%%%%%%%%%%%%%%%%%%%%%%
\begin{frame}[fragile]\frametitle{Emergence}
\begin{itemize}
    \item Emergence is when combinations create new properties.
    \item New results are more than the sum of parts.
    \item Acknowledge and leverage emergent properties.
\end{itemize}
\textbf{Examples:}
\begin{itemize}
    \item Innovation from diverse teams.
    \item Ecosystems developing new traits.
\end{itemize}
\end{frame}

%%%%%%%%%%%%%%%%%%%%%%%%%%%%%%%%%%%%%%%%%%%%%%%%%%%%%%%%%%%
\begin{frame}[fragile]\frametitle{Irreducibility}
\begin{itemize}
    \item Irreducibility is about the essence that can't be broken down.
    \item Emergent properties arise from complex systems.
    \item Focus on the big picture and embrace complexity.
\end{itemize}
\textbf{Examples:}
\begin{itemize}
    \item Cultural phenomena.
    \item Organizational behaviors.
\end{itemize}
\end{frame}

%%%%%%%%%%%%%%%%%%%%%%%%%%%%%%%%%%%%%%%%%%%%%%%%%%%%%%%%%%%
\begin{frame}[fragile]\frametitle{Law of Diminishing Returns}
\begin{itemize}
    \item Initial efforts yield the most gains.
    \item Further optimization requires more effort for less gain.
    \item Allocate resources where they have the biggest impact.
\end{itemize}
\textbf{Examples:}
\begin{itemize}
    \item Agricultural productivity.
    \item Efficiency improvements in operations.
\end{itemize}
\end{frame}

%%%%%%%%%%%%%%%%%%%%%%%%%%%%%%%%%%%%%%%%%%%%%%%%%%%%%%%%%%%%%%%%%%%%%%%%%%%%%%%%%%
\begin{frame}[fragile]\frametitle{}
\begin{center}
{\Large Numeracy Mental Models}

{\tiny Farnam Street and Miscellaneous Authors }


\end{center}
\end{frame}

%%%%%%%%%%%%%%%%%%%%%%%%%%%%%%%%%%%%%%%%%%%%%%%%%%%%%%%%%%%
\begin{frame}[fragile]\frametitle{Distributions}
\begin{itemize}
    \item Normal distribution leads to the bell curve, with a central average.
    \item Many processes, especially in social systems, do not follow this pattern.
    \item Contrast with power law distributions.
\end{itemize}
\textbf{Examples:}
\begin{itemize}
    \item Human height and weight.
    \item Economic wealth distribution.
\end{itemize}
\end{frame}

%%%%%%%%%%%%%%%%%%%%%%%%%%%%%%%%%%%%%%%%%%%%%%%%%%%%%%%%%%%
\begin{frame}[fragile]\frametitle{Compounding}
\begin{itemize}
    \item Exponential growth from small consistent gains over time.
    \item Long-term thinking about knowledge, experiences, and relationships.
    \item The majority of success comes from steady, patient accumulation of efforts.
\end{itemize}
\textbf{Examples:}
\begin{itemize}
    \item Financial investments.
    \item Personal development.
\end{itemize}
\end{frame}

%%%%%%%%%%%%%%%%%%%%%%%%%%%%%%%%%%%%%%%%%%%%%%%%%%%%%%%%%%%
\begin{frame}[fragile]\frametitle{Sampling}
\begin{itemize}
    \item Extracting information about a population by studying a sample.
    \item More measurements lead to more accurate results.
    \item Small sample sizes can produce skewed results.
\end{itemize}
\textbf{Examples:}
\begin{itemize}
    \item Opinion polls.
    \item Quality control in manufacturing.
\end{itemize}
\end{frame}

%%%%%%%%%%%%%%%%%%%%%%%%%%%%%%%%%%%%%%%%%%%%%%%%%%%%%%%%%%%
\begin{frame}[fragile]\frametitle{Randomness}
\begin{itemize}
    \item Many events are random, non-sequential, and non-ordered.
    \item Human tendency to attribute causality to random events.
    \item Course-correct for the fooled-by-randomness effect.
\end{itemize}
\textbf{Examples:}
\begin{itemize}
    \item Stock market fluctuations.
    \item Weather patterns.
\end{itemize}
\end{frame}

%%%%%%%%%%%%%%%%%%%%%%%%%%%%%%%%%%%%%%%%%%%%%%%%%%%%%%%%%%%
\begin{frame}[fragile]\frametitle{Regression to the Mean}
\begin{itemize}
    \item Long deviations from average tend to return to that average.
    \item Often mistaken for causal relationships.
    \item Be cautious not to confuse statistically likely events with causal ones.
\end{itemize}
\textbf{Examples:}
\begin{itemize}
    \item Sports team performance.
    \item Medical treatments.
\end{itemize}
\end{frame}

%%%%%%%%%%%%%%%%%%%%%%%%%%%%%%%%%%%%%%%%%%%%%%%%%%%%%%%%%%%
\begin{frame}[fragile]\frametitle{Multiplying by Zero}
\begin{itemize}
    \item Multiplying any number by zero results in zero.
    \item A failure in one area can negate efforts in other areas.
    \item Fixing the "zero" often has a greater effect than enlarging other areas.
\end{itemize}
\textbf{Examples:}
\begin{itemize}
    \item Project management failures.
    \item Organizational efficiency.
\end{itemize}
\end{frame}

%%%%%%%%%%%%%%%%%%%%%%%%%%%%%%%%%%%%%%%%%%%%%%%%%%%%%%%%%%%
\begin{frame}[fragile]\frametitle{Equivalence}
\begin{itemize}
    \item Algebra demonstrates that two seemingly different things can be the same.
    \item Manipulating symbols to show equivalence led to engineering advancements.
    \item Understanding algebra basics is crucial for various results.
\end{itemize}
\textbf{Examples:}
\begin{itemize}
    \item Engineering design equations.
    \item Financial modeling.
\end{itemize}
\end{frame}

%%%%%%%%%%%%%%%%%%%%%%%%%%%%%%%%%%%%%%%%%%%%%%%%%%%%%%%%%%%
\begin{frame}[fragile]\frametitle{Surface Area}
\begin{itemize}
    \item Surface area is the space on the outside of a three-dimensional object.
    \item More surface area means more contact with the environment.
    \item Desirable in some situations but not in others.
\end{itemize}
\textbf{Examples:}
\begin{itemize}
    \item Biological structures like lungs and intestines.
    \item Cybersecurity strategies.
\end{itemize}
\end{frame}

%%%%%%%%%%%%%%%%%%%%%%%%%%%%%%%%%%%%%%%%%%%%%%%%%%%%%%%%%%%
\begin{frame}[fragile]\frametitle{Global and Local Maxima}
\begin{itemize}
    \item Maxima and minima are the largest and smallest values of a function.
    \item Global maxima vs. local maxima help identify peaks and potential.
    \item Sometimes, going down is necessary to go back up.
\end{itemize}
\textbf{Examples:}
\begin{itemize}
    \item Optimization problems in mathematics.
    \item Business performance analysis.
\end{itemize}
\end{frame}


%%%%%%%%%%%%%%%%%%%%%%%%%%%%%%%%%%%%%%%%%%%%%%%%%%%%%%%%%%%%%%%%%%%%%%%%%%%%%%%%%%
\begin{frame}[fragile]\frametitle{}
\begin{center}
{\Large Microeconomics Mental Models}

{\tiny Farnam Street and Miscellaneous Authors }


\end{center}
\end{frame}

%%%%%%%%%%%%%%%%%%%%%%%%%%%%%%%%%%%%%%%%%%%%%%%%%%%%%%%%%%%
\begin{frame}[fragile]\frametitle{Opportunity Costs}
\begin{itemize}
    \item Doing one thing means giving up another.
    \item Living in a world of trade-offs.
    \item Summarized as "there is no such thing as a free lunch."
\end{itemize}
\textbf{Examples:}
\begin{itemize}
    \item Choosing between work and leisure.
    \item Allocating time between different activities.
\end{itemize}
\end{frame}

%%%%%%%%%%%%%%%%%%%%%%%%%%%%%%%%%%%%%%%%%%%%%%%%%%%%%%%%%%%
\begin{frame}[fragile]\frametitle{Creative Destruction}
\begin{itemize}
    \item Coined by economist Joseph Schumpeter.
    \item Describes the capitalistic process in a free-market system.
    \item Entrepreneurs push to best one another, destroying old ideas and replacing them with newer technology.
\end{itemize}
\textbf{Examples:}
\begin{itemize}
    \item Evolution of technology.
    \item Disruption of traditional industries by startups.
\end{itemize}
\end{frame}

%%%%%%%%%%%%%%%%%%%%%%%%%%%%%%%%%%%%%%%%%%%%%%%%%%%%%%%%%%%
\begin{frame}[fragile]\frametitle{Comparative Advantage}
\begin{itemize}
    \item Introduced by economist David Ricardo.
    \item Two entities can benefit from trading even if one is better at everything.
    \item Applied opportunity cost.
\end{itemize}
\textbf{Examples:}
\begin{itemize}
    \item International trade agreements.
    \item Outsourcing in business.
\end{itemize}
\end{frame}

%%%%%%%%%%%%%%%%%%%%%%%%%%%%%%%%%%%%%%%%%%%%%%%%%%%%%%%%%%%
\begin{frame}[fragile]\frametitle{Specialization (Pin Factory)}
\begin{itemize}
    \item Introduced by economist Adam Smith.
    \item Advantages gained in a free-market system by specialization.
    \item Each worker specializing in one aspect of production.
\end{itemize}
\textbf{Examples:}
\begin{itemize}
    \item Division of labor in manufacturing.
    \item Professional specialization in career fields.
\end{itemize}
\end{frame}

%%%%%%%%%%%%%%%%%%%%%%%%%%%%%%%%%%%%%%%%%%%%%%%%%%%%%%%%%%%
\begin{frame}[fragile]\frametitle{Seizing the Middle}
\begin{itemize}
    \item Strategy of controlling the middle for maximum potential moves and control.
    \item Demonstrated by historical business successes like Rockefeller and Microsoft.
\end{itemize}
\textbf{Examples:}
\begin{itemize}
    \item Strategic positioning in competitive markets.
    \item Market dominance through innovation.
\end{itemize}
\end{frame}

%%%%%%%%%%%%%%%%%%%%%%%%%%%%%%%%%%%%%%%%%%%%%%%%%%%%%%%%%%%
\begin{frame}[fragile]\frametitle{Trademarks, Patents, and Copyrights}
\begin{itemize}
    \item Protect creative work, promoting the creative-destruction model of capitalism.
    \item Provide incentives for innovation and creativity.
\end{itemize}
\textbf{Examples:}
\begin{itemize}
    \item Patenting new inventions.
    \item Registering trademarks for brand protection.
\end{itemize}
\end{frame}

%%%%%%%%%%%%%%%%%%%%%%%%%%%%%%%%%%%%%%%%%%%%%%%%%%%%%%%%%%%
\begin{frame}[fragile]\frametitle{Double-Entry Bookkeeping}
\begin{itemize}
    \item Introduced in the 14th century.
    \item Acts as a check on potential accounting errors.
    \item Allows for accurate records and behavior by firm owners.
\end{itemize}
\textbf{Examples:}
\begin{itemize}
    \item Financial accounting in businesses.
    \item Budgeting and financial planning.
\end{itemize}
\end{frame}

%%%%%%%%%%%%%%%%%%%%%%%%%%%%%%%%%%%%%%%%%%%%%%%%%%%%%%%%%%%
\begin{frame}[fragile]\frametitle{Utility (Marginal, Diminishing, Increasing)}
\begin{itemize}
    \item Marginal utility: value of one additional unit.
    \item Diminishing marginal utility: utility diminishes with additional units.
    \item Some cases exhibit critical points where utility function jumps.
\end{itemize}
\textbf{Examples:}
\begin{itemize}
    \item Consumption of goods and services.
    \item Decision-making in resource allocation.
\end{itemize}
\end{frame}

%%%%%%%%%%%%%%%%%%%%%%%%%%%%%%%%%%%%%%%%%%%%%%%%%%%%%%%%%%%
\begin{frame}[fragile]\frametitle{Bribery}
\begin{itemize}
    \item Often ignored in mainstream economics.
    \item Central to human systems, neutralizing rule enforcers.
    \item Seen as a form of arbitrage.
\end{itemize}
\textbf{Examples:}
\begin{itemize}
    \item Corruption in government.
    \item Influence in business transactions.
\end{itemize}
\end{frame}

%%%%%%%%%%%%%%%%%%%%%%%%%%%%%%%%%%%%%%%%%%%%%%%%%%%%%%%%%%%
\begin{frame}[fragile]\frametitle{Arbitrage}
\begin{itemize}
    \item Profitably exploiting price differences in different markets.
    \item Utilized in various industries for financial gain.
\end{itemize}
\textbf{Examples:}
\begin{itemize}
    \item Currency arbitrage in forex markets.
    \item Price differentials in global commodities markets.
\end{itemize}
\end{frame}

%%%%%%%%%%%%%%%%%%%%%%%%%%%%%%%%%%%%%%%%%%%%%%%%%%%%%%%%%%%
\begin{frame}[fragile]\frametitle{Supply and Demand}
\begin{itemize}
    \item Basic equation of economic life: limited supply, competition for goods.
    \item Equilibrium point: supply and demand are equal.
    \item Dynamic and changing in practical life.
\end{itemize}
\textbf{Examples:}
\begin{itemize}
    \item Pricing of goods and services.
    \item Market responses to changes in supply and demand.
\end{itemize}
\end{frame}

%%%%%%%%%%%%%%%%%%%%%%%%%%%%%%%%%%%%%%%%%%%%%%%%%%%%%%%%%%%
\begin{frame}[fragile]\frametitle{Scarcity}
\begin{itemize}
    \item Describes situations of conflict, limited resources, and competition.
    \item Decisions based on limited resources and time.
    \item Traditional game theory may overestimate human rationality.
\end{itemize}
\textbf{Examples:}
\begin{itemize}
    \item Allocation of resources in a business.
    \item Policy decisions in government.
\end{itemize}
\end{frame}

%%%%%%%%%%%%%%%%%%%%%%%%%%%%%%%%%%%%%%%%%%%%%%%%%%%%%%%%%%%
\begin{frame}[fragile]\frametitle{Mr. Market}
\begin{itemize}
    \item Introduced by investor Benjamin Graham.
    \item Represents the vicissitudes of the financial markets.
    \item Investor's job is to take advantage of market fluctuations.
\end{itemize}
\textbf{Examples:}
\begin{itemize}
    \item Stock market investments.
    \item Timing of buying and selling assets.
\end{itemize}
\end{frame}

%%%%%%%%%%%%%%%%%%%%%%%%%%%%%%%%%%%%%%%%%%%%%%%%%%%%%%%%%%%
\begin{frame}[fragile]\frametitle{Conclusion}
\begin{itemize}
    \item Mental models in microeconomics provide insights into economic behavior and decision-making.
    \item Understanding these models can inform strategic planning and resource allocation.
    \item Application of these models can lead to better outcomes in business and personal finance.
\end{itemize}
\end{frame}


%%%%%%%%%%%%%%%%%%%%%%%%%%%%%%%%%%%%%%%%%%%%%%%%%%%%%%%%%%%%%%%%%%%%%%%%%%%%%%%%%%
\begin{frame}[fragile]\frametitle{}
\begin{center}
{\Large Military/War Mental Models}

{\tiny Farnam Street and Miscellaneous Authors }


\end{center}
\end{frame}

%%%%%%%%%%%%%%%%%%%%%%%%%%%%%%%%%%%%%%%%%%%%%%%%%%%%%%%%%%%
\begin{frame}[fragile]\frametitle{Seeing the Front}
\begin{itemize}
    \item Personally observing the situation instead of relying solely on reports and advisors.
    \item Provides firsthand information and improves quality of secondhand information.
    \item Leaders benefit from "seeing the front" before making decisions.
\end{itemize}
\textbf{Examples:}
\begin{itemize}
    \item A general visiting troops on the battlefield.
    \item A CEO touring production facilities and talking to workers.
\end{itemize}
\end{frame}

%%%%%%%%%%%%%%%%%%%%%%%%%%%%%%%%%%%%%%%%%%%%%%%%%%%%%%%%%%%
\begin{frame}[fragile]\frametitle{Asymmetric Warfare}
\begin{itemize}
    \item Weaker side uses unconventional or guerilla tactics against a stronger opponent.
    \item Creates disproportionate fear or impact relative to their limited resources.
\end{itemize}
\textbf{Examples:}
\begin{itemize}
    \item Insurgency using terrorism against a conventional military force.
    \item A startup disrupting an established industry through innovative strategies.
\end{itemize}
\end{frame}

%%%%%%%%%%%%%%%%%%%%%%%%%%%%%%%%%%%%%%%%%%%%%%%%%%%%%%%%%%%
\begin{frame}[fragile]\frametitle{Two-Front War}
\begin{itemize}
    \item Fighting on multiple fronts divides and weakens the impact on each front.
    \item Applicable in organizational conflicts and competition.
\end{itemize}
\textbf{Examples:}
\begin{itemize}
    \item Germany fighting on Eastern and Western fronts in World War II.
    \item A company facing competition from rivals and internal disputes simultaneously.
\end{itemize}
\end{frame}

%%%%%%%%%%%%%%%%%%%%%%%%%%%%%%%%%%%%%%%%%%%%%%%%%%%%%%%%%%%
\begin{frame}[fragile]\frametitle{Counterinsurgency}
\begin{itemize}
    \item Strategies to combat asymmetric warfare and insurgencies.
    \item Tit-for-tat leads to a feedback loop of insurgency and counterinsurgency tactics.
\end{itemize}
\textbf{Examples:}
\begin{itemize}
    \item General Petraeus' tactics against insurgents in Iraq.
    \item A company adopting defensive measures against disruptive competitors.
\end{itemize}
\end{frame}

%%%%%%%%%%%%%%%%%%%%%%%%%%%%%%%%%%%%%%%%%%%%%%%%%%%%%%%%%%%
\begin{frame}[fragile]\frametitle{Mutually Assured Destruction}
\begin{itemize}
    \item Strengthening opponents may decrease the likelihood of conflict.
\end{itemize}
\textbf{Examples:}
\begin{itemize}
    \item Nuclear powers avoiding direct confrontation due to fear of catastrophic consequences.
    \item Companies refraining from price wars to avoid harming the industry.
\end{itemize}
\end{frame}

%%%%%%%%%%%%%%%%%%%%%%%%%%%%%%%%%%%%%%%%%%%%%%%%%%%%%%%%%%%%%%%%%%%%%%%%%%%%%%%%%%
\begin{frame}[fragile]\frametitle{}
\begin{center}
{\Large Human Nature and Judgment Models}

{\tiny Farnam Street and Miscellaneous Authors }


\end{center}
\end{frame}

%%%%%%%%%%%%%%%%%%%%%%%%%%%%%%%%%%%%%%%%%%%%%%%%%%%%%%%%%%%%%%%%%%%%%%%%%%%%%%%%%%
\begin{frame}[fragile]\frametitle{Trust}
\begin{itemize}
    \item Modern world operates on trust (familial, professional, etc.)
    \item Trusting system tends to work efficiently
    \item Rewards of trust are extremely high
\end{itemize}
\textbf{Examples:}
\begin{itemize}
    \item Trusting family members from a young age
    \item Trusting service providers (chefs, clerks, drivers, etc.)
\end{itemize}
\end{frame}

%%%%%%%%%%%%%%%%%%%%%%%%%%%%%%%%%%%%%%%%%%%%%%%%%%%%%%%%%%%%%%%%%%%%%%%%%%%%%%%%%%
\begin{frame}[fragile]\frametitle{Bias from Incentives}
\begin{itemize}
    \item Humans are highly responsive to incentives
    \item Incentives can distort thinking in self-interest
    \item Example: Salesperson believing their product improves lives
\end{itemize}
\textbf{Examples:}
\begin{itemize}
    \item Salespeople promoting their products
    \item Professionals justifying high fees for their services
\end{itemize}
\end{frame}

%%%%%%%%%%%%%%%%%%%%%%%%%%%%%%%%%%%%%%%%%%%%%%%%%%%%%%%%%%%%%%%%%%%%%%%%%%%%%%%%%%
\begin{frame}[fragile]\frametitle{Pavlovian Association}
\begin{itemize}
    \item Humans can respond to associated objects, not just direct incentives
    \item Positive and negative emotions towards intangible objects
\end{itemize}
\textbf{Examples:}
\begin{itemize}
    \item Feeling hungry when seeing food advertisements
    \item Feeling anxious when exposed to certain sounds or smells
\end{itemize}
\end{frame}

%%%%%%%%%%%%%%%%%%%%%%%%%%%%%%%%%%%%%%%%%%%%%%%%%%%%%%%%%%%%%%%%%%%%%%%%%%%%%%%%%%
\begin{frame}[fragile]\frametitle{Envy \& Jealousy}
\begin{itemize}
    \item Tendency to feel envious of those receiving more
    \item Desire to "get what is theirs" can drive irrational behavior
    \item Envy is an old and powerful human trait
\end{itemize}
\textbf{Examples:}
\begin{itemize}
    \item Envy towards colleagues who receive promotions or raises
    \item Jealousy in personal relationships
\end{itemize}
\end{frame}

%%%%%%%%%%%%%%%%%%%%%%%%%%%%%%%%%%%%%%%%%%%%%%%%%%%%%%%%%%%%%%%%%%%%%%%%%%%%%%%%%%
\begin{frame}[fragile]\frametitle{Liking/Disliking Bias}
\begin{itemize}
    \item Tendency to distort thinking based on liking or disliking
    \item Overrating things we like, underrating things we dislike
    \item Missing crucial nuances in the process
\end{itemize}
\textbf{Examples:}
\begin{itemize}
    \item Favoring a sports team or political party
    \item Disliking a particular ethnicity or culture
\end{itemize}
\end{frame}