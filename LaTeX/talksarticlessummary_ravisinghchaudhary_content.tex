%%%%%%%%%%%%%%%%%%%%%%%%%%%%%%%%%%%%%%%%%%%%%%%%%%%%%%%%%%%%%%%%%%%%%%%%%%%%%%%%%%
\begin{frame}[fragile]\frametitle{}
\begin{center}
{\Large Jottings from Talks, Articles}

{\small Ravi Singh Choudhary}


\end{center}
\end{frame}

%%%%%%%%%%%%%%%%%%%%%%%%%%%%%%%%%%%%%%%%%%%%%%%%%%%%%%%%%%%%%%%%%%%%%%%%%%%%%%%%%%
\begin{frame}[fragile]\frametitle{}
\begin{center}
{\Large आर्टिफिशियल इंटेलिजेंस बनाम ऋषि प्रज्ञा | रवि सिंह चौधरी }

{\small SangamTalks\_Hindi}


\end{center}
\end{frame}

%%%%%%%%%%%%%%%%%%%%%%%%%%%%%%%%%%%%%%%%%%%%%%%%%%%%%%%%%%%
\begin{frame}[fragile]\frametitle{इंद्रिय निग्रह}
\begin{itemize}
	\item इंद्रिय निग्रह (उपनिषद कहता है ) निग्रह (उपनिषद कहता है ) वह रथ की तरह
	\item ७  इंद्रिय उसके घोडे ,  रथ में जो बैठा हुआ है वह है आपकी आत्मा.
	\item अब घोड़े को ट्रेनिंग नहीं मिलेगा तो रथ को तो कहीं ले नहीं जाएगा जहां चना दिखेगा वहां घोड़ा चला जाएगा 
	\item सारथी को बुद्धि कहा जाता है सारथी को मैप दे दिया गया फिर भी मैप देने से वो जाएगा नहीं रथ तो हम कहते कि लगाम जो घोड़े और सारथी के बीच में है लगाम उसको हम लोग कहते हैं मन
	\item पैसेंजर सही जगह तब जाएगा जब लगाम कंट्रोल में होगा ठीक और मन को कंट्रोल करने के लिए प्राण कंट्रोल करना पड़ता है और प्राण को कंट्रोल करने के लिए शवास कंट्रोल करना पड़ता है
\end{itemize}

\end{frame}

%%%%%%%%%%%%%%%%%%%%%%%%%%%%%%%%%%%%%%%%%%%%%%%%%%%%%%%%%%%
\begin{frame}[fragile]\frametitle{ज्ञान के मान्य साधन}
\begin{itemize}
	\item प्रत्यक्ष प्रमाण  (Direct Perception)
	\item अनुमान (Inference)
	\item उपमान (analogy)
	\item शब्द (testimonials)
\end{itemize}

\end{frame}

%%%%%%%%%%%%%%%%%%%%%%%%%%%%%%%%%%%%%%%%%%%%%%%%%%%%%%%%%%%
\begin{frame}[fragile]\frametitle{Dimensions of Mind}
अंत : करण (Inner instruments)
\begin{itemize}
	\item मन  (Finding options, creates confusion)
	\item बुद्धी  (Decision making)
	\item चित्त (Use Past impressions)
	\item अहंकार (Ownership, I-ness, if more then that's ego)
\end{itemize}

Outer instruments: sense organs, legs, arms etc

मन and बुद्धी are not enough instruments to understand anything. चित्त calming, thoughtlessness also brings knowledge.
\end{frame}

%%%%%%%%%%%%%%%%%%%%%%%%%%%%%%%%%%%%%%%%%%%%%%%%%%%%%%%%%%%
\begin{frame}[fragile]\frametitle{दर्शन Darshan}
\begin{itemize}
	\item न्याय :  गौतम :  Logic, 16 rules of debate (only one example should be given, and that should be understood by 3 types (born talented, studied hard and the one who is not both, a layman)
	\item  वैशेषिक :  कणाद : Physics (observation changes output), Particle is not first, consciousness came first which affects outcome.
	\item गणित :  भास्कराचार्य , आर्यभट्ट ,  माधवाचार्य 
	\item सांख्य (Cosmology): पुरुष (Consciousness) + प्रकृति (Nature), then comes Intelligence, Ego, Mind ... reversing and going back to पुरुष is मोक्ष 
\end{itemize}

\end{frame}

%%%%%%%%%%%%%%%%%%%%%%%%%%%%%%%%%%%%%%%%%%%%%%%%%%%%%%%%%%%
\begin{frame}[fragile]\frametitle{Ancient Indian Philosophies}
\begin{itemize}
	\item Philosophies are bound by space and time, whereas Darshan are not-bound, they are conclusive and valid always.
	\item Westerns need hypothesis and proofs; whereas Darshan depends on observations and experiences.
	\item संस्कृती :  भाषा ,  भुषा , भजन ,  भोजन ,  भुवन 
\end{itemize}

\end{frame}

