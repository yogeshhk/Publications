\chapter{डिजिटल कर्मचाऱ्यांची फौज}

कोणत्याही मोठ्या शहरात सकाळी अनेक ठिकाणी मजूर-अड्डे भरलेले दिसतात. मुकादम येऊन आवश्यक कौशल्यांप्रमाणे मजूर निवडतो. ही कौशल्ये शाररिक-श्रम स्वरूपाची असतात. कल्पना करा की संगणकीय म्हणजेच ‘डिजिटल’ कामांसाठीपण असे अड्डे भरले तर आणि त्यातील मजूर सदेह  नसून आभासी असतील तर? हा केवळ कल्पनाविलास  नाही. असे ‘डिजिटल’ कर्मचारी आता प्रत्यक्षात उपलब्ध होऊ लागले आहेत. यांनाच ‘कृत्रिम बुद्धिमत्ता कर्मचारी’ (एआय एजन्ट्स) म्हणतात. 

आपण चॅटजिपीटीसारख्या संवाद-आधारित प्रणाल्या (ऍप्स, चॅटबॉट्स) पहिल्या असतील. पण मग एआय एजन्ट्स त्यापेक्षा वेगळे कसे? एआय एजन्ट्स चॅटबॉट्सच्याही पुढची पायरी आहे. ते केवळ संभाषणच नाही तर प्रत्यक्षात कृती देखील करू शकतात. उदाहरणार्थ, तुम्हाला काश्मीरची सहल आयोजित करायची असल्यास चॅटबॉट द्वारे विमानसेवा, राहण्यासाठीची हॉटेल्स, प्रेक्षणीय स्थळे यांची माहिती घेऊ शकता एवढेच. पण एआय एजन्ट्स फक्त माहितीच देत नाहीत, तर ते सहलीचे तुमच्या बजेट नुसार पूर्ण नियोजन करणे, बुकिंग करणे, अगदी तिकिटे काढण्यापर्यंत सर्व कामे करतात. अगदी खऱ्याखुऱ्या प्रवास-सल्लागार (ट्रॅव्हल एजन्ट) प्रमाणे. 

एआय एजन्ट्स  आता विविध क्षेत्रांत उपयोगी ठरत आहेत. संगणकीय प्रणाली (ऍप्स) बनवणे, विक्री-विपणनाची प्राधान्यक्रम ठरवणे, जोखीम ओळखणे, डेटा (माहिती, विदा) यांचे पृथ:करण करणे, विविध प्रक्रियांमध्ये स्वयंचलितता आणणे, नव-नवीन कल्पनांच्या सृजनासाठी पर्याय सुचवणे, यासारख्या अनेक निर्णय आणि कृती प्रक्रियांमध्ये ते उपयोगी ठरू लागले आहेत. केवळ पुनरावुत्तीच्या, तोच-तोच पणाच्या कामांमध्येच नाही तर बुद्धी व निर्णयक्षमता लागणाऱ्या कामांमध्ये सुद्धा एआय एजन्ट्स शिरकाव करीत आहेत. अशा ‘डिजिटल’ कर्मचाऱ्यांची फौजच निर्माण होत आहे. २४ तास, ७ ही दिवस, न कंटाळता, न संप करता, अचूक काम करणाऱ्यांची. 

चॅटजिपीटी बनवणाऱ्या ‘ओपन एआय’ चे प्रमुख सॅम ऑल्टमन यांचे म्हणणे आहे की स्व:प्रेरणेने निर्णय घेणारे, लक्ष्य ठरवणारे आणि कमीत कमी मानवी हस्तक्षेपात कामे पूर्ण करणारे एआय एजन्ट्स आता मोठ्या संख्येने बनणार आहेत. २०२५ पासून ते तुमच्या कंपनीच्या टीमचा भागही होतील. त्यांच्या असामान्य कौशल्यामुळे कंपन्यांची उत्पादकता मोठ्या प्रमाणात वाढेल. 

मायक्रोसॉफ्टच्या सत्य नडेला यांचे म्हणणे आहे सध्या जसे ऍप्स वापरून आपण माहिती मागवतो, विदा-साठ्याची  (डेटा बेस) कामे करतो ते सर्व जाऊन आता एआय एजन्ट्स  ही डेटाची कामे, त्यांच्यातल्या व्यवसाय-सूत्रांच्या (बिझिनेस लॉजिक) आधारे, तडक-प्रत्यक्षपणे करतील. ऍप्सची गरजच नाही. आणि कोणत्याही प्रकारचा डेटाबेस असला तरी. ही एक मोठी क्रांती होणार आहे.

आता प्रश्न निर्माण होतो की, हे एआय एजन्ट्स नक्की काम कसे करतात? ते पाहू. एआय एजन्ट्सकडे जेंव्हा एखादे काम येते, ते बृहत भाषा प्रारूपे (लार्ज लँग्वेज मॉडेल्स, एल-एल-एम्स) यांचा वापर करून, तर्कशुद्धपणे समजून घेऊन (रिझनिंग) कामाची यादी बनवतात. प्रत्येक कामासाठी त्यांच्याकडच्या कार्यप्रणालींची (टूल्स) रचना करतात, माहिती कमी पडत असेल तर अंतराजालातून (इंटरनेट) मागवू शकतात आणि सर्व नियोजनबद्ध पणे घडवून आणतात. आपण केलेले काम बरोबर आहे की नाही ते तपासून कार्यप्रणालीत बदल घडवून आणतात किंवा ग्राहकाच्या अभिप्रायानुसार (फीडबॅक) ते आपल्यात बदलही करतात. ही सर्व कामे ते बऱ्यापैकी स्वयंचलित पद्धतीने आणि अचूक करीत असल्याने आपण निवांतपणे त्यांच्यावर कामे सोपवू शकतो. 

मायक्रोसॉफ्ट, गुगल, सेल्सफोर्स सारख्या बलाढ्य कंपन्याच नाही तर छोट्या छोट्या कंपन्यासुद्धा या एआय एजन्ट्सच्या मागणीचा लाभ घेऊ शकतात. त्याला लागणाऱ्या एजन्ट-प्रणाल्या (एजंटीक सिस्टिम्स) मुक्त स्वरूपात (ओपन सोर्स) उपलब्ध आहेत. त्यामागे असलेली (काही)  एल-एल-एम्स सुद्धा मुक्त स्वरूपात मिळतात. त्यामुळे संकल्पना-प्रमाण (प्रूफ ऑफ कन्सेप्ट) तरी बनवता येतील. अशा एआय एजन्ट्सची गरज इतर जगभरात तर आहेच पण विशेषकरून भारतात सुद्धा आहे. सर्वसामान्य जनतेला भारतीय भाषांमध्ये संवाद साधणारे एआय एजन्ट्स  फारच उपयोगी ठरतील. कल्पना करा हे एआय एजन्ट्स सांगकाम्याप्रमाणे आपली बँकेची, खाद्यपदार्थ मागवण्याची, वस्तू विकत घेण्याची, इमेल-रिपोर्ट लिहिण्याची, कार्यालयाची कामे फटाफट करू लागली तर कोणाला नको आहे. मोठी संधी आहे भारतीय प्रतिभेला. गरज आहे ती बाजाराभिमुख कल्पकतेची आणि ते सत्यात उतरवण्यासाठी लागणाऱ्या मेहनतीची. जशी ही मोठी संधी आहे तर ती मोठ्या जनसंख्येसाठी ही धोक्याची घंटा सुद्धा आहे. विशेष करून ‘डिजिटल’ सेवा क्षेत्रातल्या नोकऱ्या जाणे हे नजीकच्या काळातच सुरु होऊ शकते.  त्यावरील उपायांच्या कार्यान्वयावर तातडीने समाजात व सरकार-दरबारी चर्चा होणे गरजेचे आहे. 