\documentclass[12pt]{article}

\usepackage{graphicx}
\usepackage{adjustbox}
\usepackage{enumitem} % to get rid-off extra spacing and more control

% Line spacing
\usepackage{setspace}
\setstretch{1.15}

% Times New Roman font
\usepackage[T1]{fontenc}
\usepackage[utf8]{inputenc}
\usepackage{mathptmx}

% Margins
\usepackage[top=0.8in, bottom=0.8in, left=0.8in, right=0.5in]{geometry}

% Shift Title up
\usepackage{titling}
\setlength{\droptitle}{-5em}   % This is your set screw

% Header and footer
\usepackage{fancyhdr}
\pagestyle{fancy}
\lhead{\small Idea Generation Competition}
\rhead{\small "Use of Technology and Innovations in promoting Good Governance"}
\lfoot{\rule{\linewidth}{0.25pt} \small Yogesh Kulkarni (41 yrs), College of Engineering Pune.}
\rfoot{\rule{\linewidth}{0.25pt} \small  PhD Student, yogeshkulkarni@yahoo.com, +919890251406}
\pagenumbering{gobble}

\title{``Not more laws, but technology is the key''} % Your article title
\author{Yogesh Kulkarni\\College of Engineering Pune, India} % Your name
\date{} % Add a date here if you would like one to appear underneath the title block

%----------------------------------------------------------------------------------------

\begin{document}

\maketitle

%\thispagestyle{empty}

\begin{flushright}
{\em ``If people are good, only because they fear punishment, \\and hope for reward, then we are a sorry lot indeed.'' \\– Albert Einstein}
\end{flushright}


It is easier to make laws than to provide good governance. That's why there are more of them, the complicated ones, and with loopholes designed in them. It is like a highly-elaborate house plan. But not many ensure that the house gets built properly. The builder (the enforcement agency) is simply interested in the income without bothering  about the quality. Thus, the house turns to shambles irrespective of all the checks and balances provided in the design plan. 

Out of many possible reasons why good governance is not a reality, some are  as follows:

\begin{itemize}[label=\textbullet, noitemsep,nolistsep]
\item Too many laws force the common man to take an easier route. You need hundreds of permissions to start a business. The enforcement agency, having powers and knowledge of the loopholes,  bends only after a commission. \textbf{Simplification} of procedures is the key to address this problem.

\item Different laws at different places and for different situations adds to the confusion. Taxation, for example, varies from place to place, making the system so opaque that you need Chartered Accountants to tackle them. \textbf{Uniformity} is the key to address this problem.

\item Hiding information gives power and breeds corruption. Let the common man be the judge.  \textbf{Easy access to information} is the key to address this problem.
\end{itemize}

\section*{Simplification}
Laws can be simplified, and technology can be used to percolate them in the society. Technology can also be used to plug the loopholes by exposing the decisions taken. Following are some possibilities:

\begin{itemize}
\item \textbf{Black Money}: Unaccounted, untraceable money, mostly in the form of cash. Solution is to discourage the use of cash, or even better is to invalidate it. Cancel all the currency notes above Rs. 100.  Technological solution here is to replace paper notes by plastic ones. They will last longer and won't be easy to counterfeit. It won't be worth for our ``neighbors'' to print Rs. 100 plastic notes.  Use of credit or debit cards and mobile transactions have not reached the masses yet. Ease of user interface, use of regional languages on websites and apps will address this problem.
\item \textbf{Tax Evasion}: Very few people pay taxes, not because they don't want to (in most cases) but because it is easy to evade them. Solution is to levy only indirect tax on transactions. So, a technology platform which overlooks all the transactions, like GST (Goods and Services Tax),  bank transactions, etc, can deduct some percentage at each stage.
\end{itemize}

\section*{Uniformity}
Being  One-nation, we need to have uniformity across all states, for various functions, but that's not the case currently. 

\begin{itemize}
\item \textbf{Single ID}: Forgoing all the rest like PAN, Sr. Citizen, License, etc, {\em Aadhar} card should be the only ID. Against this, lots of technological applications can be built. Similar to SSN (Social Security Number) in the USA, all records, transactions, etc of an individual can be linked to this ID. With easy access to the relevant information, we won't need to provide myriad of paperwork. Say, if students' records are available, an employer can easily access them without the need of notarized certificates, etc.

\item \textbf{Single Education Board}: Single curriculum having enough flexibility for local customization. Subjects like Maths, Science, etc can be standard across the country, including religious schools. Examinations for common subjects can be held nation/state wide at the same time, using an e-platform.  This will ensure portability and common benchmarking as seen in exams like CAT or IIT-JEE.
\end{itemize}



\section*{Easy access to information}
RTI (Right to Information) is a perfect example of empowering the common man to be vigilant,  by having easy access to information. But it is still not used by the  common man on a regular basis. We need to give ready access to government databases using technological solutions. Let all that information be open for access through website, mobile applications and via voice service like {\em Sarathi}; this way, citizens will be able to keep a watch on the governance. Some potential applications can be for:
\begin{itemize}
\item \textbf{Illegal schools}: Each school should display its registration number. Using this, one should be able to access all the information available from the  government database, like, legal documents, teachers, facilities, RTE admissions, fees, location, pictures, etc. If one finds discrepancy, s/he can complain easily. This way, one does have to depend solely on the external inspectors, as parents too can keep a watch on illegality.
\item \textbf{Illegal Hoardings}: Mushrooming of hoardings all over the city not only harms the beauty but also causes revenue loss in case of illegal ones. Each hoarding should display its registration number. With this, one should be able to access all the information available from the government database, like, legal documents, fees paid, size, content, period, etc. If one finds discrepancy, s/he can complain easily.
\end{itemize}
These are basically applications which access information via the displayed ID. There is enormous potential  for such applications apart from the ones mentioned above. Other than the classified defense-related information, all the rest can be exposed via easy apps. 

Technology is ready;  we are now only waiting for the political will and/or a push from  the citizens.


\end{document}