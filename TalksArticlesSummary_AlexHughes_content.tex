%%%%%%%%%%%%%%%%%%%%%%%%%%%%%%%%%%%%%%%%%%%%%%%%%%%%%%%%%%%%%%%%%%%%%%%%%%%%%%%%%%
\begin{frame}[fragile]\frametitle{}
\begin{center}
{\Large Atomic Habits – by James Clear}

{\small Review by Alex Hughes}


\end{center}
\end{frame}


%%%%%%%%%%%%%%%%%%%%%%%%%%%%%%%%%%%%%%%%%%%%%%%%%%%%%%%%%%%
\begin{frame}[fragile]\frametitle{My Notes}

\begin{itemize}
\item ``To write a great book, you must first become the book.'' - Naval Ravikant 
\item Automatic Habits + Deliberate Practice = Mastery 
\end{itemize}
\end{frame}

%%%%%%%%%%%%%%%%%%%%%%%%%%%%%%%%%%%%%%%%%%%%%%%%%%%%%%%%%%%
\begin{frame}[fragile]\frametitle{Self-improvement}

\begin{itemize}
\item 1\% better each day for one year = 37x better 

\item ``Habits are the compound interest of self-improvement.'' 
\item Your outcomes are a lagging measure of your habits. Knowledge is lagging measure of your learning habits.  
\end{itemize}
\end{frame}

%%%%%%%%%%%%%%%%%%%%%%%%%%%%%%%%%%%%%%%%%%%%%%%%%%%%%%%%%%%
\begin{frame}[fragile]\frametitle{Nonlinearity}

\begin{itemize}
\item Ice cube example warming from 26 degrees, one degree at a time, to 32 when it finally begins to melt. But no visible progress from 26-31. 
\item Breakthrough moments = culmination of actions leading up to that point.  
\item Habits need to persist long enough to break through plateau where you don’t see tangible results or ``success” as you’ve envisioned it.  
\item Sorites Paradox: Can one coin make someone rich? No, but as you keep adding/stacking coins (habits), at a certain point one coin makes the difference. 
\end{itemize}
\end{frame}

%%%%%%%%%%%%%%%%%%%%%%%%%%%%%%%%%%%%%%%%%%%%%%%%%%%%%%%%%%%
\begin{frame}[fragile]\frametitle{Goals vs. Systems}

\begin{itemize}
\item Goals are good for setting direction, systems best for making progress. 
\item ``The purpose of setting goals is to win the game. The purpose of building systems is to continue playing the game.” Refinement, improvement and commitment to the process. 
\item Goal is not to read a book, it’s to become a reader. Not to learn an instrument, it’s to become a musician. 
\end{itemize}
\end{frame}

%%%%%%%%%%%%%%%%%%%%%%%%%%%%%%%%%%%%%%%%%%%%%%%%%%%%%%%%%%%
\begin{frame}[fragile]\frametitle{Identity and behavior change}

\begin{itemize}
\item Who is the type of person that could get the outcome I want? If it’s a person who could write a book, that means consistent, reliable, etc. 
\item Decide the type of person you want to be and prove it to yourself with small wins. 
\item ``Does this behavior help me become the type of person I wish to be? Does this habit cast a vote for or against my desired identity?'' 
\item At a certain point, the identity itself becomes the reinforcer. Behavior becomes automatic because it’s who you are.  
\end{itemize}
\end{frame}

%%%%%%%%%%%%%%%%%%%%%%%%%%%%%%%%%%%%%%%%%%%%%%%%%%%%%%%%%%%
\begin{frame}[fragile]\frametitle{Keep your identity small}
\begin{itemize}
\item Tighter you cling to an identity, harder it is to grow beyond it and less capable you are of adapting when life challenges you. 
\item ``When you cling too tightly to one identity, you become brittle. Lose that one thing and you lose yourself.” 
\item Redefine yourself so you keep important aspects of your identity even when your role changes. Instead of ``I’m the CEO,” ``I’m the type of person who builds and creates things.'' 
\item Identity should work with changing circumstances, rather than against them.  
\end{itemize}
\end{frame}

%%%%%%%%%%%%%%%%%%%%%%%%%%%%%%%%%%%%%%%%%%%%%%%%%%%%%%%%%%%
\begin{frame}[fragile]\frametitle{Discipline}
\begin{itemize}
\item ``It is only by making the fundamentals in life easier that you can create the mental space needed for free thinking and creativity.'' 
\item ``‘Disciplined’ people are better at structuring their lives in a way that does not require heroic willpower and self-control. In other words, they spend less time in tempting situations.'' 
\item Create a disciplined environment —> easier to practice self-restraint when you don’t have to use it often. 
\item Environmental design: Remove friction, make doing the right thing as easy as possible. Inversion: add friction to make bad behaviors more difficult. 
\end{itemize}
\end{frame}

%%%%%%%%%%%%%%%%%%%%%%%%%%%%%%%%%%%%%%%%%%%%%%%%%%%%%%%%%%%
\begin{frame}[fragile]\frametitle{Clarity}
\begin{itemize}
\item Don’t mistake lack of clarity for lack of motivation, make it obvious. 
\item Be specific about what you want and how you will achieve it. When you’re vague about your dreams it’s easy to ignore the specifics you need to do to succeed. 
\end{itemize}
\end{frame}

%%%%%%%%%%%%%%%%%%%%%%%%%%%%%%%%%%%%%%%%%%%%%%%%%%%%%%%%%%%
\begin{frame}[fragile]\frametitle{Imitation}
\begin{itemize}
\item Proximity has powerful effect on our behavior (both physical and social environments). Running against the grain requires extra effort. 
\item Surround yourself with people who have the habits you want to have yourself. 
\item ``When changing your habits means challenging the tribe, change is unattractive. When changing your habits means fitting in with the tribe, change is very attractive.'' 
\end{itemize}
\end{frame}

%%%%%%%%%%%%%%%%%%%%%%%%%%%%%%%%%%%%%%%%%%%%%%%%%%%%%%%%%%%
\begin{frame}[fragile]\frametitle{Motion vs. Action}
\begin{itemize}
\item Motion = planning, strategizing, learning. Important, but don’t produce a result. Allows you to feel like you’re making progress without risk of failure. Ex) Making a list of 20 articles to write. 
\item Action = behavior that will deliver an outcome. Ex) Actually sitting down to write an article. 
\item Start with repetition, not perfection. Habits form based on frequency, not time. 
\end{itemize}
\end{frame}



%%%%%%%%%%%%%%%%%%%%%%%%%%%%%%%%%%%%%%%%%%%%%%%%%%%%%%%%%%%
\begin{frame}[fragile]\frametitle{Time inconsistency (hyperbolic discounting)}
\begin{itemize}
\item The way the brain evaluates rewards is inconsistent across time. From an evolutionary perspective, you naturally value present more than future 
\item Costs of good habits are felt today. Costs of bad habits are felt in the future. 
\item 
``Most people will spend all day chasing hits of quick satisfaction. The road less traveled is the road of delayed gratification. If you’re willing to wait for the rewards, you’ll face less competition and often get a bigger payoff. As the saying goes, the last mile is always the least crowded.'' 
\end{itemize}
\end{frame}

%%%%%%%%%%%%%%%%%%%%%%%%%%%%%%%%%%%%%%%%%%%%%%%%%%%%%%%%%%%
\begin{frame}[fragile]\frametitle{Consistency}
\begin{itemize}
\item Always show up, even on your bad days. Lost days hurt you more than successful days help you. 
\item \$100 grows 50\% to $150. Only takes a 33\% loss to take you back to \$100. Avoiding 33\% loss just as valuable at 50\% gain.  
\item Don’t enter games you’re not willing to play:
\item Maximize your odds by choosing right field of competition.  
\item Think about where you achieve greater returns than the average person and the type of work that hurts you less than it hurts others.  
\item Flow = 4\% beyond your current ability. 
\end{itemize}
\end{frame}

%%%%%%%%%%%%%%%%%%%%%%%%%%%%%%%%%%%%%%%%%%%%%%%%%%%%%%%%%%%
\begin{frame}[fragile]\frametitle{Checking progress/reflection}
\begin{itemize}
\item Annual review, EOY: 1) What went well this year? 2) What didn’t go so well this year? 3) What did I learn? https://jamesclear.com/annual-review 
\item Integrity report, mid-year: 1) What are the core values that are driving my life and work? How am I living and working with integrity right now? How can I set a higher standard for the future? 
\end{itemize}
\end{frame}


