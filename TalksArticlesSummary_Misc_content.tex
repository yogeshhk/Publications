%%%%%%%%%%%%%%%%%%%%%%%%%%%%%%%%%%%%%%%%%%%%%%%%%%%%%%%%%%%%%%%%%%%%%%%%%%%%%%%%%%
\begin{frame}[fragile]\frametitle{}
\begin{center}
{\Large Jottings from Talks, Tweets}

{\small Miscellaneous Authors }


\end{center}
\end{frame}

%%%%%%%%%%%%%%%%%%%%%%%%%%%%%%%%%%%%%%%%%%%%%%%%%%%%%%%%%%%
\begin{frame}[fragile]\frametitle{ Randolph's Rules for success }

	\begin{itemize}
	\item Do at least 10% more than you are asked
	\item Never, ever, to anybody present as fact opinions on things you don’t know. Take great care and discipline.
	\item Be courteous and considerate always–up and down
	\item Don’t knock, don’t complain–stick to constructive, serious criticism
	\item Don’t be afraid to make decisions when you have the facts on which to make them
	\item Quantify where possible
	\item Be open-minded but skeptical
	\item Be prompt
	\end{itemize}

{\tiny (Ref: https://thefutureorganization.com/that-will-never-work-netflix-co-founder-on-how-they-beat-the-odds/)}

\end{frame}


%%%%%%%%%%%%%%%%%%%%%%%%%%%%%%%%%%%%%%%%%%%%%%%%%%%%%%%%%%%
\begin{frame}[fragile]\frametitle{ Kapil Gupta with Naval Ravikant}

	\begin{itemize}
	\item Truth of life is that everybody has problems 
	\item Your problems are different than mine
	\item Your solutions are, thus, different 
	\item Thus how-tos or prescriptions are useless after basics
	\item Go deep to solve problems, honestly 
	\begin{itemize}
		\item Weak shabby body: wts, run, diet, yog
		\item anxious insecure mind: meditate
		\item Find patterns robustly: research
	\end{itemize}
	\item Expose yourself to truth, be in the environment 
	\end{itemize}

{\tiny (Ref: https://youtu.be/sBtuqpNZwio)}

\end{frame}

%%%%%%%%%%%%%%%%%%%%%%%%%%%%%%%%%%%%%%%%%%%%%%%%%%%%%%%%%%%
\begin{frame}[fragile]\frametitle{ Some of the Little Ideas \ldots}

	\begin{itemize}
	\item Base Rates: The success rate of everyone who’s done what you’re about to try.
	\item System Justification Theory: Inefficient systems will be defended and maintained if they serve the needs of people who benefit from them – individual incentives can sustain systemic stupidity.
	\item Three Men Make a Tiger: People will believe anything if enough people tell them it’s true.
	\item Pareto Principle: The majority of outcomes are driven by a minority of events.
	\item Cumulative advantage: Social status snowballs in either direction because people like associating with successful people, so doors are opened for them, and avoid associating with unsuccessful people, for whom doors are closed.
	\end{itemize}

{\tiny (Ref: 100 Little Ideas by Morgan Housel)}

\end{frame}

%%%%%%%%%%%%%%%%%%%%%%%%%%%%%%%%%%%%%%%%%%%%%%%%%%%%%%%%%%%
\begin{frame}[fragile]\frametitle{ Some of the Little Ideas \ldots}

	\begin{itemize}
	\item Impostor Syndrome: Fear of being exposed as less talented than people think you are
	\item Ringelmann Effect: Members of a group become lazier as the size of their group increases. Based on the assumption that “someone else is probably taking care of that.”
	\item Semmelweis Reflex: Automatically rejecting evidence that contradicts your tribe’s established norms. 
	\item Boomerang Effect: Trying to persuade someone to do one thing can make them more likely to do the opposite, because the act of persuasion can feel like someone stealing your freedom and doing the opposite makes you feel like you’re taking your freedom back.
	\end{itemize}

{\tiny (Ref: 100 Little Ideas by Morgan Housel)}

\end{frame}


%%%%%%%%%%%%%%%%%%%%%%%%%%%%%%%%%%%%%%%%%%%%%%%%%%%%%%%%%%%
\begin{frame}[fragile]\frametitle{ Some of the Little Ideas \ldots}

	\begin{itemize}
	\item McNamara Fallacy: A belief that rational decisions can be made with quantitative measures alone, when in fact the things you can’t measure are often the most consequential. 
	\item Ludic Fallacy: Falsely associated simulations with real life.
	\item Actor-Observer Asymmetry: We judge others based solely on their actions, but when judging ourselves we have an internal dialogue that justifies our mistakes and bad decisions.
	\item Fredkin’s Paradox: Confronted with two equally good options, you struggle to decide, even though your decision doesn’t matter because both options are equally good. 
	\end{itemize}

{\tiny (Ref: 100 Little Ideas by Morgan Housel)}

\end{frame}

%%%%%%%%%%%%%%%%%%%%%%%%%%%%%%%%%%%%%%%%%%%%%%%%%%%%%%%%%%%
\begin{frame}[fragile]\frametitle{ Some of the Little Ideas \ldots}

	\begin{itemize}
	\item Poisoning the Well: Presenting irrelevant adverse information about someone in a way that makes everything else that person says seem untrustworthy.
	\item Golem Effect: Performance declines when supervisors/teachers have low expectations of your abilities.
	\item Hanlon’s Razor: “Never attribute to malice that which can be adequately explained by stupidity.”
	\item Braess’s Paradox: Adding more roads can make traffic worse because new shortcuts become popular and overcrowded.
	\end{itemize}

{\tiny (Ref: 100 Little Ideas by Morgan Housel)}

\end{frame}

%%%%%%%%%%%%%%%%%%%%%%%%%%%%%%%%%%%%%%%%%%%%%%%%%%%%%%%%%%%
\begin{frame}[fragile]\frametitle{ Some of the Little Ideas \ldots}

	\begin{itemize}
	\item Non-Ergodic: When group probabilities don’t apply to singular events. If 100 people play Russian Roulette once, the odds of dying might be, say, 10\%. But if one person plays Russian Roulette 100 times, the odds are dying are practically 100\%.
	\item Abilene Paradox: A group decides to do something that no one in the group wants to do because everyone mistakenly assumes they’re the only ones who object to the idea and they don’t want to rock the boat by speaking up.
	\item Reflexivity: When cause and effect are the same. People think Tesla will sell a lot of cars, so Tesla stock goes up, which lets Tesla raise a bunch of new capital, which helps Tesla sell a lot of cars.
	\end{itemize}

{\tiny (Ref: 100 Little Ideas by Morgan Housel)}

\end{frame}

%%%%%%%%%%%%%%%%%%%%%%%%%%%%%%%%%%%%%%%%%%%%%%%%%%%%%%%%%%%
\begin{frame}[fragile]\frametitle{ Some of the Little Ideas \ldots}

	\begin{itemize}
	\item Peter Principle: Good workers will continue to be promoted until they end up in a role they’re bad at.
	\item Hedonic Treadmill: Expectations rise with results, so nothing feels as good as you’d imagine for as long as you’d expect.
	\item Ironic Process Theory: Going out of your way to suppress thoughts makes those thoughts more prominent in your mind.
	\item Nonlinearity: Outputs aren’t always proportional to inputs, so the world is a barrage of massive wins and horrible losses that surprise people.
	\item Moderating Relationship: The correlation between two variables depends on a third, seemingly unrelated variable.
	\end{itemize}

{\tiny (Ref: 100 Little Ideas by Morgan Housel)}

\end{frame}


%%%%%%%%%%%%%%%%%%%%%%%%%%%%%%%%%%%%%%%%%%%%%%%%%%%%%%%%%%%
\begin{frame}[fragile]\frametitle{ Some of the Little Ideas \ldots}

	\begin{itemize}
	\item Denomination Effect: One hundred \$1 bills feels like less money than one \$100 bill.
	\item Inversion: Avoiding problems can be more important than scoring wins.
	\item Principle of Least Effort: When seeking information, effort declines as soon as the minimum acceptable result is reached.
	\item Dunning-Kruger Effect: Knowing the limits of your intelligence requires a certain level of intelligence, so some people are too stupid to know how stupid they are.
	\item The Middle Ground Fallacy: Falsely assuming that splitting the difference between two polar opposite views is a healthy compromise.
	\end{itemize}

{\tiny (Ref: 100 Little Ideas by Morgan Housel)}

\end{frame}

%%%%%%%%%%%%%%%%%%%%%%%%%%%%%%%%%%%%%%%%%%%%%%%%%%%%%%%%%%%
\begin{frame}[fragile]\frametitle{ Some of the Little Ideas \ldots}

	\begin{itemize}
	\item Ostrich Effect: Avoiding negative information that might challenge views that you desperately want to be right.
	\item Bounded Rationality: People can’t be fully rational because your brain is a hormone machine, not an Excel spreadsheet.
	\item Fluency Heuristic: Ideas that can be explained simply are more likely to be believed than those that are complex, even if the simple-sounding ideas are nonsense. It occurs because ideas that are easy to grasp are hard to distinguish from ideas you’re familiar with.
	\item Fact-Check Scarcity Principle: Don’t believe everything you read.
	\end{itemize}

{\tiny (Ref: 100 Little Ideas by Morgan Housel)}

\end{frame}


%%%%%%%%%%%%%%%%%%%%%%%%%%%%%%%%%%%%%%%%%%%%%%%%%%%%%%%%%%%
\begin{frame}[fragile]\frametitle{Musashi’s 21 Precepts for Life (1)}

	\begin{itemize}
	\item Accept everything just the way it is.
	\item Do not seek pleasure for its own sake.
	\item Do not, under any circumstances, depend on a partial feeling.
	\item Think lightly of yourself and deeply of the world.
	\item Be detached from desire your whole life long.
	\item Do not regret what you have done.
	\item Never be jealous.
	\end{itemize}

\end{frame}

%%%%%%%%%%%%%%%%%%%%%%%%%%%%%%%%%%%%%%%%%%%%%%%%%%%%%%%%%%%
\begin{frame}[fragile]\frametitle{Musashi’s 21 Precepts for Life (2)}

	\begin{itemize}
	\item Never let yourself be saddened by a separation.
	\item Resentment and complaint are appropriate neither for oneself or others.
	\item Do not let yourself be guided by the feeling of lust or love.
	\item In all things have no preferences.
	\item Be indifferent to where you live.
	\item Do not pursue the taste of good food.
	\item Do not hold on to possessions you no longer need.
	\end{itemize}

\end{frame}

%%%%%%%%%%%%%%%%%%%%%%%%%%%%%%%%%%%%%%%%%%%%%%%%%%%%%%%%%%%
\begin{frame}[fragile]\frametitle{Musashi’s 21 Precepts for Life (3)}

	\begin{itemize}
	\item Do not act following customary beliefs.
	\item Do not collect weapons or practice with weapons beyond what is useful.
	\item Do not fear death.
	\item Do not seek to possess either goods or fiefs for your old age.
	\item Respect Buddha and the gods without counting on their help.
	\item You may abandon your own body but you must preserve your honor.
	\item Never stray from the way.
	\end{itemize}

\end{frame}



%%%%%%%%%%%%%%%%%%%%%%%%%%%%%%%%%%%%%%%%%%%%%%%%%%%%%%%%%%%
\begin{frame}[fragile]\frametitle{Leo Babauta}

	\begin{itemize}
	\item Set your 3 MITs (Most Imprtant Tasks) each morning
	\item Single Task - focus your attention
	\item Zero Inbox (Work In Progress mails ok)
	\item Work while disconnected
	\item Keep your desk de-cluttered
	\item Stick to 5 sentences emails (make words powerful)
	\item Say NO to any commitment outside your goals
	\item Low information diet: emails + twice a day, 
	\end{itemize}

\end{frame}

%%%%%%%%%%%%%%%%%%%%%%%%%%%%%%%%%%%%%%%%%%%%%%%%%%%%%%%%%%%
\begin{frame}[fragile]\frametitle{Sam Harris: Waking Up}

	\begin{itemize}
	\item Religion is primarily for social conduct, may have personal enhancement
	\item Spirituality is for humanity, universal, away from religion, deeper experience
	\item Mindfulness is a state of being clear, non-judgmental and un-distracted attention to the contents of consciousness (mind) whether pleasant or unpleasant.
	\item Mindfulness gives better, more skilled, perspective to look at things
	\end{itemize}

\end{frame}


%%%%%%%%%%%%%%%%%%%%%%%%%%%%%%%%%%%%%%%%%%%%%%%%%%%%%%%%%%%
\begin{frame}[fragile]\frametitle{Dr BM Hegde}

	\begin{itemize}
	\item One should consume natural food in moderation 
	\item Sleep in pitch dark room 
	\item Electronic devices such as Television should be switched off and mobile phones shouldn’t be charged while sleeping. This is to avoid the presence of electromagnetic waves. 
	\item One should get up before sunrise 
	\item One should eat the last meal before 7pm
	\end{itemize}

\end{frame}

%%%%%%%%%%%%%%%%%%%%%%%%%%%%%%%%%%%%%%%%%%%%%%%%%%%%%%%%%%%
\begin{frame}[fragile]\frametitle{Dr BM Hegde References}
	\begin{itemize}
	\item Secrets revealed: BP, Sugar, Cancer, Heart Stroke Healthy Lifestyle for Better Life : https://www.youtube.com/watch?v=r0rRgvhNy1U 
	\item The End of Capitalism - William Creeck (Article), Are We Nearing The End Of Capitalism?  https://www.youtube.com/watch?v=IE3t77AMITU  (not William Creeck, though)
	\item The World Is Too Much with Us https://en.wikipedia.org/wiki/The\_World\_Is\_Too\_Much\_with\_Us   
	\item The Golem : What You Should Know About Science - Harry Collins and Trevor Pinch (Book) https://www.researchgate.net/publication/238342346\_The\_Golem\_What\_You\_Should\_Know\_about\_Science 
	\item "Doctors going on strike will improve society's health for sure" - British Medical Journal (Article) https://www.ncbi.nlm.nih.gov/pmc/articles/PMC1127364/ 
	\item "Is U.S. health really the best in the world?" - Barbara Starfield (Article) [JAMA - 2000: V. 284; Pg. 483] https://www.jhsph.edu/research/centers-and-institutes/johns-hopkins-primary-care-policy-center/Publications\_PDFs/A154.pdf 
	\item "The end of the disease era" - Mary Tinetty (Article) [Archives of Internal Medicine - 2004: V. 116; Pg. 179] https://stallseniormedical.com/wp-content/uploads/The-End-of-the-Disease-Era-Tinetti.pdf 
	\end{itemize}

\end{frame}

%%%%%%%%%%%%%%%%%%%%%%%%%%%%%%%%%%%%%%%%%%%%%%%%%%%%%%%%%%%
\begin{frame}[fragile]\frametitle{Dr BM Hegde References}
	\begin{itemize}
	\item "Matter is not made out of matter" - Hans Peter Durr (Article)
	\item Uncertainty Principle - Werener Heisenberg (Theory)
	\item Occult Chemistry - Annie Besant (Book) https://ia800301.us.archive.org/29/items/occultchemistryc00besa/occultchemistryc00besa.pdf 
	\item Molecules of Emotion - Candace Pert (Book) https://www.youtube.com/playlist?list=PL7aNzLFRfhm3c4V\_SfPsr0DRNPE1XAgAR 
	\item Quantum Doctor - Amit Goswami (Book) https://www.youtube.com/watch?v=eIpSui-wjPk 
	\item Inventing the AIDS virus - Peter H. Deusberg (Book) https://en.wikipedia.org/wiki/Inventing\_the\_AIDS\_Virus  https://www.youtube.com/watch?v=8FLXoDu\_ZDs 
	\item When Breath Becomes Air - Paul Kalanidhi (Book) https://www.youtube.com/watch?v=nTrMgBPZJHA 
	\item Affluent Society - John Kenneth Galbraith (Book)  https://en.wikipedia.org/wiki/The\_Affluent\_Society https://www.youtube.com/watch?v=FhMq6oPD1eM 
	\item Anatomy of an Illness - Norman Cousins (Book) https://www.youtube.com/watch?v=0LwKd68S15I 
	\item The Placebo Effect - Bingel et al (Article) [Science Translational Medicine - 2011: V. 3; Pg. 70] http://stm.sciencemag.org/content/3/70/70ra14 
	\item What Doctors Don't Get to Study in Medical School - B.M. Hegde (Book) https://www.youtube.com/watch?v=yA\_zy1lnlN4 
	\item Dementia: A drug-induced crime on mankind - Grace Elizabeth Jackson (Book)  https://beyondmeds.com/2009/06/08/drug-induced-dementia/   https://www.youtube.com/watch?v=lpsjwe69QFE 
	\item Pure, White and Deadly - John Yudkin (Book)
	\item Closing of an American Mind - Alan Bloom (Book) https://en.wikipedia.org/wiki/The\_Closing\_of\_the\_American\_Mind  https://www.youtube.com/watch?v=-15fOEovI0o 
	\item Science without Sense - Steven Milloy (Book)  https://ia600807.us.archive.org/14/items/milloy\_risky/milloy\_risky.pdf 
	\item  www.archimedesmodel.com
	\end{itemize}

\end{frame}

%%%%%%%%%%%%%%%%%%%%%%%%%%%%%%%%%%%%%%%%%%%%%%%%%%%%%%%%%%%
\begin{frame}[fragile]\frametitle{Make Time: ??}
	\begin{itemize}
	\item Remove distracting apps (from your phone)
	\item Declutter Phone/Laptop screen
	\item What Will Be the Highlight of Your Day?
	\item For five days, a team would cancel all meetings and focus on solving a single problem, following a specific checklist of activities
	\item Something magic happens when you start the day with one high-priority goal.
	\item We got more done when we banned devices
	\item A healthy lunch, a quick walk, frequent breaks, and a slightly shorter workday helped maintain peak energy
	\item Highlight: Start Each Day by Choosing a Focus
	\item Laser: Beat Distraction to Make Time
	\item Energize: Use the Body to Recharge the Brain
	\item Reflect: Adjust and Improve Your System
	\item Newspaper: Schedule time once a week , emails twice a day 10am/5pm
	\item Get Better Sleep
	\end{itemize}

\end{frame}

%%%%%%%%%%%%%%%%%%%%%%%%%%%%%%%%%%%%%%%%%%%%%%%%%%%%%%%%%%%
\begin{frame}[fragile]\frametitle{Make Time: ??}
	\begin{itemize}
	\item Walk to Work
	\item Eat Fat for Breakfast
	\item Stand Most of the Time
	\item Take Breaks Without Screens
	\item Do a Quick But Intense Workout
	\item Do the Chores
	\item One Thing at a Time
	\item Focus: focus on just one thing per month, or per year, we can make big changes to our health, career, relationships, finances, or any other department of life.
	\item Stuck? Switch to Play Mode
	\item Put your most important project on the front burner
	\item Put your second most important project on the back burner
	\item Make a kitchen sink for the rest
	\item World class performance comes after 10,000 hours of deliberate practice, 12,500 hours of deliberate rest, and 30,000 hours of sleep.
	\end{itemize}

\end{frame}

%%%%%%%%%%%%%%%%%%%%%%%%%%%%%%%%%%%%%%%%%%%%%%%%%%%%%%%%%%%
\begin{frame}[fragile]\frametitle{Gut Health - Ben Warren's top 10 tips for a healthy gut.}

	\begin{itemize}
	\item Food/dairy intolerance causing Back pain, difficultly nose breathing, 
	\item IGG IGE test for whey protein, lactose/milk intolerance
	\item Most of the immune system is in stomoch as it has to prottect from outside food.
	\item It decides which protein to keep (food) and which to destroy (virus/bacteria are proteins too!!)
	\item Gluten in some folks gets identified as virus and immune reactions happen.
	\item Leaky gut affects body
	\item Auto immune: immune system goes hyper and attacks good parts of internal body
	\item Gut-brain axis : 3 way intersection of brain, gut and microbiome.
	\item For digestion, mental relaxation is needed. Control breathing, diaphragmatic, while eating.
	\item Chew food to liquid
	\item Apple cider before meal (if there isn't strong acid, this will help. If you already have it, then you can stop and take bicarbonate soda) same in case acid reflux.
	\item Start day with warm water + lemon. Gets stomach ready for the day
	\item If at all you need dairy products, take as raw as possible. Avoid if possible.
	\item Gluten (wheat, maida, etc) and Lactose (dairy) are to be avoided.
	\item Eat fermented food daily, probiotic
	\end{itemize}

\end{frame}

	

%%%%%%%%%%%%%%%%%%%%%%%%%%%%%%%%%%%%%%%%%%%%%%%%%%%%%%%%%%%
\begin{frame}[fragile]\frametitle{Chris Bailey}

	\begin{itemize}
	\item Schedule time to completely disconnect from work
	\item Use visualizations to become more productive
	\item Capture all of the open loops in your head
	\item When you meet someone, shut off phone completely
	\end{itemize}

\end{frame}


%%%%%%%%%%%%%%%%%%%%%%%%%%%%%%%%%%%%%%%%%%%%%%%%%%%%%%%%%%%
\begin{frame}[fragile]\frametitle{Avinash Dharmadhikari}

	\begin{itemize}
	\item Know thy-self - Ongoing
	\item Chose career of favorite topic - CAD
	\item Excellence and Creativity in the chosen topic
	% \item \textmarathi{विषय कळलेला पाहिजे ज्ञानावर पकड}. 
	% \item Yog \textmarathi{आसने, सूर्यनमस्कार, प्राणायाम, ध्यान}
	% \item \textmarathi{स्वाधाय : कोणताही विषय समोर आला तरी त्याचा अभ्यास करण्याचे कौशल्य}. Notes, Ref Cards
	\item Good command over English, Bilungual, Finance.
	\item Awareness of latest trends of your work.
	\item Entrepreneur in my field.  Teacher prepares students
	\item Creator of wealth (value addition to the topic)  
	% \item \textmarathi{वैश्विकता}: Do things at par with global standards
	\end{itemize}

\end{frame}


%%%%%%%%%%%%%%%%%%%%%%%%%%%%%%%%%%%%%%%%%%%%%%%%%%%%%%%%%%%
\begin{frame}[fragile]\frametitle{The book of YOGA - Christina Brown}

	\begin{itemize}
	\item Yoga is a state of mind. Stillness fosters awareness.
	\item Physical exercises still the body. Breath to focus the mind
	\item Relaxation to quieten the body and mind
	\item Chanting to arouse and then calm emotions
	\item Meditation to center the spirits.
	\item Yoga postures bring strength to the weak areas and soften the tight spots.
	% \item \textmarathi{अभ्यास }: Repeated study/effort, then becomes effortless
	% \item  \textmarathi{आसन} builds and controls \textmarathi{प्राण}- the vital force
	% \item Mind is the master of senses and the breath is the master of the Mind.
	% \item When the breath flows steadily your \textmarathi{आसन }becomes closer to perfect
	\item Frequently check your breathing
	\item Finding the quietness that lies within yourself
	\item Push your boundaries every time. Don't overdo it.
	\end{itemize}

\end{frame}


% %%%%%%%%%%%%%%%%%%%%%%%%%%%%%%%%%%%%%%%%%%%%%%%%%%%%%%%%%%%
% \begin{frame}[fragile]\frametitle{The book of YOGA - Christina Brown}

	% \begin{itemize}
	% \item \textmarathi{यम }: Compassion for all beings. \textmarathi{अहिंसा }(tolerance, kindness), \textmarathi{सत्य, अस्तेय} (stealing), \textmarathi{ब्रह्मचर्य} (moderation) \textmarathi{अपरिग्रह}(non-accumulation, simplicity)
	% \item \textmarathi{नियम : शौच}(purity), \textmarathi{संतोष} (contentment), तप (dedication/fire-determiniation), \textmarathi{स्वाध्याय, ईश्वर-प्रणिधान}
	% \item \textmarathi{आसन}  (postures)
	% \item \textmarathi{प्राणायाम }(breath control)
	% \item  \textmarathi{प्रत्याहार} (withdrawal of senses)
	% \item \textmarathi{धारणा} (concentration, mindfulness, paying attention)
	% \item \textmarathi{ध्यान} (meditation)
	% \item \textmarathi{समाधी}
	% \item Start and End any activity with relaxation
	% \item Let it go, detachment. Tense and then relax parts one by one.
	% \item Stay Still and Don't worry
	% \end{itemize}

% \end{frame}


%%%%%%%%%%%%%%%%%%%%%%%%%%%%%%%%%%%%%%%%%%%%%%%%%%%%%%%%%%%
\begin{frame}[fragile]\frametitle{Maria Popova}

	\begin{itemize}
	\item  The secret of happiness: ``Find something more important than you are'' - philosopher Dan Dennett
	\item Don't go after prestigious positions, HoD, on committees etc. Waste of time.
	\item Steve Jobs:  And the only way to do great work is to love, what you do. If you haven't found it yet, keep looking. Don't settle
	\item The whole future lies in uncertainty: live immediately.
	\item No activity can be successfully pursued by an individual who is preoccupied 
	\item 7 Learnings:
		\begin{itemize}
		\item Allow yourself the uncomfortable luxury of changing your mind.
		\item Do nothing for prestige or status or money or approval alone. Have goal more than the promotion!!
		\item Be generous. With your time.
		\item Build pockets of stillness into your life. Sleep.
		\item Don't believe what others say about you
		\item Presence is far more intricate and rewarding an art than productivity.
		\item Expect anything worthwhile to take a long time
		\end{itemize}
	\end{itemize}

\end{frame}

%%%%%%%%%%%%%%%%%%%%%%%%%%%%%%%%%%%%%%%%%%%%%%%%%%%%%%%%%%%
\begin{frame}[fragile]\frametitle{Uma Naidoo}


	\begin{itemize}
\item  Theme: Nutritional Psychiatry (Food Mood)
\item  Anxiety: Foods to avoid: gluten, sugar, processed veg oil (fried stuff outside-home), artificial sweeteners (cold drinks/soda)
\item  Hypoglycemia: keto diet may help
\item  Dementia/Brain-fog: Foods to have: 
	\begin{itemize}

	\item  Turmeric+black pepper, Cinnamon, saffron, rosemary, ginger
	\item  Olive oil + onions, garlic ( good pre-biotics)
	\item  Juniper berries, sage ,thyme, celery seed, dried Mexican Oregano, peppermint
	\end{itemize}


\end{itemize}

{\tiny (Ref: ``Harvard Nutritional Psychiatrist Shares the Key Foods for Incredible Mental Health | Dr. Uma Naidoo | Andrew Huberman'' https://www.youtube.com/watch?v=LtHlw7QJT7Y)}

\end{frame}


%%%%%%%%%%%%%%%%%%%%%%%%%%%%%%%%%%%%%%%%%%%%%%%%%%%%%%%%%%%
\begin{frame}[fragile]\frametitle{Uma Naidoo}

	\begin{itemize}
\item  Sodium-salt but not in excess, from processed food.
\item  Fast food, like French fries, has sugar (no taste though)
\item  Obsessive Compulsive Disorder (OCD): Avoid MSG (Mono Sodium Glutamate), Parmesan cheese, oyster/tomato sauce
\item  Depression/Anxiety: Mediterranean diet helps: avocado, lean protein, olive oil, seafood, whole grains
\item  Aggression: increases with trans-fats
\item  Have:
	\begin{itemize}

	\item  Pre-biotic: onion, garlic
	\item  Fermented food: Kafir, Kimchi
	\item  Salads: Color of rainbow vegetables (avoid unhealthy toppings), seeds, beans, lemon, herbs
	\end{itemize}

\end{itemize}


{\tiny (Ref: ``Harvard Nutritional Psychiatrist Shares the Key Foods for Incredible Mental Health | Dr. Uma Naidoo | Andrew Huberman'' https://www.youtube.com/watch?v=LtHlw7QJT7Y)}

\end{frame}

%%%%%%%%%%%%%%%%%%%%%%%%%%%%%%%%%%%%%%%%%%%%%%%%%%%%%%%%%%%
\begin{frame}[fragile]\frametitle{Andrew Huberman}


	\begin{itemize}
\item  Nervous system functions : Sensation, Perception (spotlight/attention), Feeling, Thoughts, Actions, Triggering immune system
\item  Brain has abstractions of everything around
\item  Abstractions converge at physical phenomenon easily bit also at things like rewards, punishments
\item  Brain outside skull+spinal chord: neural retina, behind eye, not attached to brain. See sunrise and sunset to trigger activities.
\item  Growth Mindset:
	\begin{itemize}
	\item  Dopamine: Pursuit of goal and not much at the achievement of it. Attachment of rewards to efforts
	\item  Serotonin/Oxytocin: Attachment of rewards to goals achievement
	\end{itemize}

\end{itemize}


{\tiny (Ref: ``This Neuroscientist Shows You the Secrets to Obtaining A Growth Mindset | Andrew Huberman'' https://www.youtube.com/watch?v=OGa\_jt3IncY)}

\end{frame}

%%%%%%%%%%%%%%%%%%%%%%%%%%%%%%%%%%%%%%%%%%%%%%%%%%%%%%%%%%%
\begin{frame}[fragile]\frametitle{Andrew Huberman}


	\begin{itemize}
\item  Fear: Act in the fearful situation, lateral eye movements are associated with forward-movement, and thus, action.
\item  (my wording) Even if you get external assistance (luck) in your success, you become confident enough to win future battles.
\item  Solution for stress-fear is Forward movement (gets dopamine)
\item  Reward incremental steps, say, run around a block.
\item  Tom Bilyeu: Success is not guaranteed but the struggle is.

\end{itemize}

{\tiny (Ref: ``This Neuroscientist Shows You the Secrets to Obtaining A Growth Mindset | Andrew Huberman'' https://www.youtube.com/watch?v=OGa\_jt3IncY)}


\end{frame}

%%%%%%%%%%%%%%%%%%%%%%%%%%%%%%%%%%%%%%%%%%%%%%%%%%%%%%%%%%%
\begin{frame}[fragile]\frametitle{Andrew Huberman}


	\begin{itemize}

\item  Hypnosis: relaxing nervous system and opens possibility of nuro plasticity, in contrast, in high alert situation (talking, planing, brain is linear, if then then that). For re-wiring, need deep rest
\item  Deep rest: diaphragmatic breathing, tells brain to be active or rest accordingly (phrenic nerve).
\item  Heightened states of focus by Heavy breathing (Wim Hof), then calm, with double-inhale and long exhale
\item  One change to make: 2-10 minutes of bright light at sunrise
\end{itemize}

{\tiny (Ref: ``This Neuroscientist Shows You the Secrets to Obtaining A Growth Mindset | Andrew Huberman'' https://www.youtube.com/watch?v=OGa\_jt3IncY)}


\end{frame}

%%%%%%%%%%%%%%%%%%%%%%%%%%%%%%%%%%%%%%%%%%%%%%%%%%%%%%%%%%%
\begin{frame}[fragile]\frametitle{Aaron Alexander}

\begin{itemize}

\item  Theme: Align method (better movements)
\item  Dance (Music + Movement), Pose-stance for better state
\item  Smile (even fake), but if actual and with eyes, changes mood
\item  Exercise: While walking look at others color of eyes. Magic of the moment.
\item  Bring child like aspects: flexibility, directly related to nature, compassion, honesty.
\item  Be more on ground-seated. Look up to trees, Sun, nature. 
\item  Micro-opportunities to do movements, squatting, Walk, 2/3 minutes hang on pull-up bar.
\item  Mind movements: Stillness!! vipassana meditation
\item  Accept yourself. Be better as well. Talk about it.
\end{itemize}

{\tiny (Ref: ``The Real Secret to a Healthy Mind and Body | Aaron Alexander on Health Theory'' https://www.youtube.com/watch?v=YpmfoNTTuNU)}


\end{frame}

%%%%%%%%%%%%%%%%%%%%%%%%%%%%%%%%%%%%%%%%%%%%%%%%%%%%%%%%%%%
\begin{frame}[fragile]\frametitle{Sam Altman}

\begin{itemize}

\item ``Minimize your own cognitive load from distracting things that don’t really matter. It’s hard to overstate how important this is, and how bad most people are at it.'' (Essay ``The days are long but the decades are short'')
\item ``Sleep seems to be the most important physical factor in productivity for me. Exercise is probably the second most important physical factor. The third area is nutrition.'' (Essay: ``Productivity'')
\end{itemize}

{\tiny (Ref: ``Sam Altman on Choosing Projects, Creating Value, and Finding Purpose'' https://www.youtube.com/watch?v=uEl2KUZ3JWA)}


\end{frame}

%%%%%%%%%%%%%%%%%%%%%%%%%%%%%%%%%%%%%%%%%%%%%%%%%%%%%%%%%%%
\begin{frame}[fragile]\frametitle{Sam Altman}

\begin{itemize}

\item ``If what you are doing is not important, and if you don’t think it is going to lead to something important, why are you at Bell Labs working on it?’'' (Essay ``You and Your Research by Richard Hamming'')
\item ``Things in life are rarely as risky as they seem. Most people are too risk-averse, and so most advice is biased too much towards conservative paths.'' (Essay: ``The Days Are Long But The Decades Are Short'')
\end{itemize}

{\tiny (Ref: ``Sam Altman on Choosing Projects, Creating Value, and Finding Purpose'' https://www.youtube.com/watch?v=uEl2KUZ3JWA)}
\end{frame}


%%%%%%%%%%%%%%%%%%%%%%%%%%%%%%%%%%%%%%%%%%%%%%%%%%%%%%%%%%%
\begin{frame}[fragile]\frametitle{Sam Altman}

\begin{itemize}

\item ``My system has three key pillars: “Make sure to get the important shit done”, “Don’t waste time on stupid shit”, and “make a lot of lists”'' (Essay ``Productivity'')
\item ``He who works with the door open gets all kinds of interruptions, but he also occasionally gets clues as to what the world is and what might be important.'' (Essay: ``You and Your Research by Richard Hamming'')
\end{itemize}

{\tiny (Ref: ``Sam Altman on Choosing Projects, Creating Value, and Finding Purpose'' https://www.youtube.com/watch?v=uEl2KUZ3JWA)}
\end{frame}


%%%%%%%%%%%%%%%%%%%%%%%%%%%%%%%%%%%%%%%%%%%%%%%%%%%%%%%%%%%
\begin{frame}[fragile]\frametitle{Sam Altman}

\begin{itemize}

\item ``Our self-worth is so based on our intelligence that we believe it must be singular and not slightly higher than all the other animals on a continuum. Perhaps the AI will feel the same way and note that differences between us and bonobos are barely worth discussing.'' (Essay ``The Merge'')
\item “Surrounding yourself with people that will make you more ambitious”
\item Take time to reflect on your purpose, life and what you are really accomplishing. Ask yourself if you are doing something important or something that matters. It's easy to get caught up in things e.g. office politics, status or power games that don't matter. 
\end{itemize}

{\tiny (Ref: ``Sam Altman on Choosing Projects, Creating Value, and Finding Purpose'' https://www.youtube.com/watch?v=uEl2KUZ3JWA)}
\end{frame}

%%%%%%%%%%%%%%%%%%%%%%%%%%%%%%%%%%%%%%%%%%%%%%%%%%%%%%%%%%%
\begin{frame}[fragile]\frametitle{Sam Altman}

\begin{itemize}

\item Take time and explore different things. Learn about different areas. Meet, help and learn from people who are working on different things. Understand that most things won’t work out. 
\item On what to work on: Figuring out what’s working and not after exploring different things. This involves following your intuition, brutal honesty, and focusing on the project that's really working for a sufficient amount of time. Young founders make the mistake of bouncing from projects to projects without giving a project enough time for it to have measurable results. Knowing what to focus on and when to give up are especially hard things to do.  
\end{itemize}

{\tiny (Ref: ``Sam Altman on Choosing Projects, Creating Value, and Finding Purpose'' https://www.youtube.com/watch?v=uEl2KUZ3JWA)}
\end{frame}

%%%%%%%%%%%%%%%%%%%%%%%%%%%%%%%%%%%%%%%%%%%%%%%%%%%%%%%%%%%
\begin{frame}[fragile]\frametitle{Sam Altman}

\begin{itemize}

\item  (Not for everyone) Angel investing and poker are great ways to learn about business, psychology and life in general.
\item  Sleep, exercise and nutrition are important for physical productivity. Find out what are your most productive hours of the day and focus on your work don't let others interrupt you during these hours.
\item Shift your perspectives. Look into things from different angles. Surround yourself with people who make you more ambitious.
\end{itemize}

{\tiny (Ref: ``Sam Altman on Choosing Projects, Creating Value, and Finding Purpose'' https://www.youtube.com/watch?v=uEl2KUZ3JWA)}
\end{frame}

%%%%%%%%%%%%%%%%%%%%%%%%%%%%%%%%%%%%%%%%%%%%%%%%%%%%%%%%%%%
\begin{frame}[fragile]\frametitle{Sam Altman}

\begin{itemize}

\item Read biographies of people who did amazing things like the Apollo program to motivate yourself and provide you with new perspectives on purpose and risks. 
\item An interesting way to increase your productivity is to have a group of friends who are founders who check in on each other frequently.
\item  Doing anything worthwhile takes a long time and emotional trauma (getting rejected over and over again) and if you aren’t willing to do that you probably won’t succeed. 
\end{itemize}

{\tiny (Ref: ``Sam Altman on Choosing Projects, Creating Value, and Finding Purpose'' https://www.youtube.com/watch?v=uEl2KUZ3JWA)}
\end{frame}


%%%%%%%%%%%%%%%%%%%%%%%%%%%%%%%%%%%%%%%%%%%%%%%%%%%%%%%%%%%
\begin{frame}[fragile]\frametitle{Sam Altman}

Some notes on the deferred life plan:

\begin{itemize}

\item One of the problems with the deferred life plan is that these people are usually not that committed to the plan in the first place. E.g. if you want to go build a rocket, just go build a rocket. You can’t say you are going to do something like ‘I am going to build a \$100 million crypto fund before I start my rocket company.’ Usually, this person doesn’t build either of these ventures. 
\item Another problem with the deferred life plan is that people like investors and employees can sense that you aren’t authentic or committed to the vision of the company, so you probably won’t win their support.
\end{itemize}

{\tiny (Ref: ``Sam Altman on Choosing Projects, Creating Value, and Finding Purpose'' https://www.youtube.com/watch?v=uEl2KUZ3JWA)}
\end{frame}

%%%%%%%%%%%%%%%%%%%%%%%%%%%%%%%%%%%%%%%%%%%%%%%%%%%%%%%%%%%
\begin{frame}[fragile]\frametitle{Sam Altman}

\begin{itemize}
\item Compound yourself: As your career progresses, each unit of work you do should generate more and more results. Build on existing skills day by day.
\item Have almost too much self-belief: Full conviction of the efforts, like Elon Musk is sure of sending rocket to Mars.
\item Learn to think independently: Thinking from first principles and trying to generate new ideas is fun.
\item Get good at “sales”: Self-belief alone is not sufficient—you also have to be able to convince other people of what you believe. Genuinely believe in what you’re selling.
\item Make it easy to take risks: Most people overestimate risk and underestimate reward. It’s often easier to take risks early in your career; you don’t have much to lose, and you potentially have a lot to gain. Keeping your life cheap and flexible.
\end{itemize}

{\tiny (Ref: ``How To Be Successful'' https://blog.samaltman.com/how-to-be-successful)}
\end{frame}


%%%%%%%%%%%%%%%%%%%%%%%%%%%%%%%%%%%%%%%%%%%%%%%%%%%%%%%%%%%
\begin{frame}[fragile]\frametitle{Sam Altman}

\begin{itemize}
\item Focus: Most people waste most of their time on stuff that doesn’t matter.
\item Work hard: Many hours. You can get to about the 90th percentile in your field by working either smart or hard, but for 99th percentile, need both. Extreme people get extreme results.
\item Be bold: Let yourself grow more ambitious, and don’t be afraid to work on what you really want to work on.
\item Be willful: Be persistent long enough to give a chance for luck to go their way.
\end{itemize}

{\tiny (Ref: ``How To Be Successful'' https://blog.samaltman.com/how-to-be-successful)}
\end{frame}



%%%%%%%%%%%%%%%%%%%%%%%%%%%%%%%%%%%%%%%%%%%%%%%%%%%%%%%%%%%
\begin{frame}[fragile]\frametitle{Sam Altman}

\begin{itemize}
\item Be hard to compete with: You should not be doing what a fresher can do (in 1/10th salary). 
\item Build a network:  Be overly generous with sharing the upside; it will come back to you 10x. 
\item You get rich by owning things: This can be a piece of a business, real estate, natural resource, intellectual property, or other similar things.
\item Be internally driven
\end{itemize}

{\tiny (Ref: ``How To Be Successful'' https://blog.samaltman.com/how-to-be-successful)}
\end{frame}